% Options for packages loaded elsewhere
\PassOptionsToPackage{unicode}{hyperref}
\PassOptionsToPackage{hyphens}{url}
\PassOptionsToPackage{dvipsnames,svgnames,x11names}{xcolor}
%
\documentclass[
  letterpaper,
  DIV=11,
  numbers=noendperiod]{scrartcl}

\usepackage{amsmath,amssymb}
\usepackage{iftex}
\ifPDFTeX
  \usepackage[T1]{fontenc}
  \usepackage[utf8]{inputenc}
  \usepackage{textcomp} % provide euro and other symbols
\else % if luatex or xetex
  \usepackage{unicode-math}
  \defaultfontfeatures{Scale=MatchLowercase}
  \defaultfontfeatures[\rmfamily]{Ligatures=TeX,Scale=1}
\fi
\usepackage{lmodern}
\ifPDFTeX\else  
    % xetex/luatex font selection
\fi
% Use upquote if available, for straight quotes in verbatim environments
\IfFileExists{upquote.sty}{\usepackage{upquote}}{}
\IfFileExists{microtype.sty}{% use microtype if available
  \usepackage[]{microtype}
  \UseMicrotypeSet[protrusion]{basicmath} % disable protrusion for tt fonts
}{}
\makeatletter
\@ifundefined{KOMAClassName}{% if non-KOMA class
  \IfFileExists{parskip.sty}{%
    \usepackage{parskip}
  }{% else
    \setlength{\parindent}{0pt}
    \setlength{\parskip}{6pt plus 2pt minus 1pt}}
}{% if KOMA class
  \KOMAoptions{parskip=half}}
\makeatother
\usepackage{xcolor}
\setlength{\emergencystretch}{3em} % prevent overfull lines
\setcounter{secnumdepth}{5}
% Make \paragraph and \subparagraph free-standing
\ifx\paragraph\undefined\else
  \let\oldparagraph\paragraph
  \renewcommand{\paragraph}[1]{\oldparagraph{#1}\mbox{}}
\fi
\ifx\subparagraph\undefined\else
  \let\oldsubparagraph\subparagraph
  \renewcommand{\subparagraph}[1]{\oldsubparagraph{#1}\mbox{}}
\fi

\usepackage{color}
\usepackage{fancyvrb}
\newcommand{\VerbBar}{|}
\newcommand{\VERB}{\Verb[commandchars=\\\{\}]}
\DefineVerbatimEnvironment{Highlighting}{Verbatim}{commandchars=\\\{\}}
% Add ',fontsize=\small' for more characters per line
\usepackage{framed}
\definecolor{shadecolor}{RGB}{241,243,245}
\newenvironment{Shaded}{\begin{snugshade}}{\end{snugshade}}
\newcommand{\AlertTok}[1]{\textcolor[rgb]{0.68,0.00,0.00}{#1}}
\newcommand{\AnnotationTok}[1]{\textcolor[rgb]{0.37,0.37,0.37}{#1}}
\newcommand{\AttributeTok}[1]{\textcolor[rgb]{0.40,0.45,0.13}{#1}}
\newcommand{\BaseNTok}[1]{\textcolor[rgb]{0.68,0.00,0.00}{#1}}
\newcommand{\BuiltInTok}[1]{\textcolor[rgb]{0.00,0.23,0.31}{#1}}
\newcommand{\CharTok}[1]{\textcolor[rgb]{0.13,0.47,0.30}{#1}}
\newcommand{\CommentTok}[1]{\textcolor[rgb]{0.37,0.37,0.37}{#1}}
\newcommand{\CommentVarTok}[1]{\textcolor[rgb]{0.37,0.37,0.37}{\textit{#1}}}
\newcommand{\ConstantTok}[1]{\textcolor[rgb]{0.56,0.35,0.01}{#1}}
\newcommand{\ControlFlowTok}[1]{\textcolor[rgb]{0.00,0.23,0.31}{#1}}
\newcommand{\DataTypeTok}[1]{\textcolor[rgb]{0.68,0.00,0.00}{#1}}
\newcommand{\DecValTok}[1]{\textcolor[rgb]{0.68,0.00,0.00}{#1}}
\newcommand{\DocumentationTok}[1]{\textcolor[rgb]{0.37,0.37,0.37}{\textit{#1}}}
\newcommand{\ErrorTok}[1]{\textcolor[rgb]{0.68,0.00,0.00}{#1}}
\newcommand{\ExtensionTok}[1]{\textcolor[rgb]{0.00,0.23,0.31}{#1}}
\newcommand{\FloatTok}[1]{\textcolor[rgb]{0.68,0.00,0.00}{#1}}
\newcommand{\FunctionTok}[1]{\textcolor[rgb]{0.28,0.35,0.67}{#1}}
\newcommand{\ImportTok}[1]{\textcolor[rgb]{0.00,0.46,0.62}{#1}}
\newcommand{\InformationTok}[1]{\textcolor[rgb]{0.37,0.37,0.37}{#1}}
\newcommand{\KeywordTok}[1]{\textcolor[rgb]{0.00,0.23,0.31}{#1}}
\newcommand{\NormalTok}[1]{\textcolor[rgb]{0.00,0.23,0.31}{#1}}
\newcommand{\OperatorTok}[1]{\textcolor[rgb]{0.37,0.37,0.37}{#1}}
\newcommand{\OtherTok}[1]{\textcolor[rgb]{0.00,0.23,0.31}{#1}}
\newcommand{\PreprocessorTok}[1]{\textcolor[rgb]{0.68,0.00,0.00}{#1}}
\newcommand{\RegionMarkerTok}[1]{\textcolor[rgb]{0.00,0.23,0.31}{#1}}
\newcommand{\SpecialCharTok}[1]{\textcolor[rgb]{0.37,0.37,0.37}{#1}}
\newcommand{\SpecialStringTok}[1]{\textcolor[rgb]{0.13,0.47,0.30}{#1}}
\newcommand{\StringTok}[1]{\textcolor[rgb]{0.13,0.47,0.30}{#1}}
\newcommand{\VariableTok}[1]{\textcolor[rgb]{0.07,0.07,0.07}{#1}}
\newcommand{\VerbatimStringTok}[1]{\textcolor[rgb]{0.13,0.47,0.30}{#1}}
\newcommand{\WarningTok}[1]{\textcolor[rgb]{0.37,0.37,0.37}{\textit{#1}}}

\providecommand{\tightlist}{%
  \setlength{\itemsep}{0pt}\setlength{\parskip}{0pt}}\usepackage{longtable,booktabs,array}
\usepackage{calc} % for calculating minipage widths
% Correct order of tables after \paragraph or \subparagraph
\usepackage{etoolbox}
\makeatletter
\patchcmd\longtable{\par}{\if@noskipsec\mbox{}\fi\par}{}{}
\makeatother
% Allow footnotes in longtable head/foot
\IfFileExists{footnotehyper.sty}{\usepackage{footnotehyper}}{\usepackage{footnote}}
\makesavenoteenv{longtable}
\usepackage{graphicx}
\makeatletter
\def\maxwidth{\ifdim\Gin@nat@width>\linewidth\linewidth\else\Gin@nat@width\fi}
\def\maxheight{\ifdim\Gin@nat@height>\textheight\textheight\else\Gin@nat@height\fi}
\makeatother
% Scale images if necessary, so that they will not overflow the page
% margins by default, and it is still possible to overwrite the defaults
% using explicit options in \includegraphics[width, height, ...]{}
\setkeys{Gin}{width=\maxwidth,height=\maxheight,keepaspectratio}
% Set default figure placement to htbp
\makeatletter
\def\fps@figure{htbp}
\makeatother

\KOMAoption{captions}{tableheading}
\makeatletter
\@ifpackageloaded{caption}{}{\usepackage{caption}}
\AtBeginDocument{%
\ifdefined\contentsname
  \renewcommand*\contentsname{Table of contents}
\else
  \newcommand\contentsname{Table of contents}
\fi
\ifdefined\listfigurename
  \renewcommand*\listfigurename{List of Figures}
\else
  \newcommand\listfigurename{List of Figures}
\fi
\ifdefined\listtablename
  \renewcommand*\listtablename{List of Tables}
\else
  \newcommand\listtablename{List of Tables}
\fi
\ifdefined\figurename
  \renewcommand*\figurename{Figure}
\else
  \newcommand\figurename{Figure}
\fi
\ifdefined\tablename
  \renewcommand*\tablename{Table}
\else
  \newcommand\tablename{Table}
\fi
}
\@ifpackageloaded{float}{}{\usepackage{float}}
\floatstyle{ruled}
\@ifundefined{c@chapter}{\newfloat{codelisting}{h}{lop}}{\newfloat{codelisting}{h}{lop}[chapter]}
\floatname{codelisting}{Stan

Program}
\newcommand*\listoflistings{\listof{codelisting}{List of Listings}}
\makeatother
\makeatletter
\makeatother
\makeatletter
\@ifpackageloaded{caption}{}{\usepackage{caption}}
\@ifpackageloaded{subcaption}{}{\usepackage{subcaption}}
\makeatother
\ifLuaTeX
  \usepackage{selnolig}  % disable illegal ligatures
\fi
\usepackage{bookmark}

\IfFileExists{xurl.sty}{\usepackage{xurl}}{} % add URL line breaks if available
\urlstyle{same} % disable monospaced font for URLs
\hypersetup{
  pdftitle={Bae's Theorem},
  pdfauthor={Michael Betancourt},
  colorlinks=true,
  linkcolor={blue},
  filecolor={Maroon},
  citecolor={Blue},
  urlcolor={Blue},
  pdfcreator={LaTeX via pandoc}}

\title{Bae's Theorem}
\author{Michael Betancourt}
\date{December 2024}

\begin{document}
\maketitle

\renewcommand*\contentsname{Table of contents}
{
\hypersetup{linkcolor=}
\setcounter{tocdepth}{3}
\tableofcontents
}
In 2006 Netflix announced the infamous
\href{https://web.archive.org/web/20200510213032/https://www.netflixprize.com/assets/rules.pdf}{Netflix
Prize}. The competition challenged anyone to use a data set of customer
movie reviews to inform predictions for a second, held-out set of
customer movie reviews. Superficially the Netflix challenge was a mixed
success, although from a broader perspective it demonstrated the perils
of poorly-chosen metrics for predictive performance and the subtleties
of data privacy. Wikipedia summarizes the
\href{https://en.wikipedia.org/wiki/Netflix_Prize}{history} well.

Beyond the Netflix Prize itself the corresponding data set is a nice
example of some of the problems that can arise in a wide range of
practical applications. For instance not only are customer preferences
limited to five star ratings but also the interpretation of those
ratings are ambiguous and typically not consistent across customers.
Some customers are generous with their five star ratings while some are
meager with not only their five star ratings but also their four star
and sometimes even three star ratings.

In addition to the idiosyncratic rating scales any analysis of this data
also has to contend with idiosyncratic customer preferences. Because not
every customer will agree on the quality of a given movie we have to
decide whether we want to try to learn an aggregate preference across
the entire population or the individual customer preferences.

In this chapter I develop a Bayesian analysis of a subset of the Netflix
training data set, not in an attempt to win the Netflix Prize decades
too late but rather to demonstrate some strategies for approaching these
analysis challenges.

Importantly this analysis will not be the first time that Netflix has
been associated with Bayesian inference. In 2016 Amy Hogan
(@alittlestats) presented her influential
\href{https://twitter.com/alittlestats/status/664923862853922820}{Bae's
Theorem}, \[
p( \text{chill} \mid \text{Netflix} )
=
\frac{ p( \text{Netflix} \mid \text{chill} ) \, p( \text{chill} ) }
{ p( \text{Netflix} ) }.
\]

\section{Setup}\label{setup}

As always we begin by setting up our local \texttt{R} environment.

\begin{Shaded}
\begin{Highlighting}[]
\FunctionTok{par}\NormalTok{(}\AttributeTok{family=}\StringTok{"serif"}\NormalTok{, }\AttributeTok{las=}\DecValTok{1}\NormalTok{, }\AttributeTok{bty=}\StringTok{"l"}\NormalTok{,}
    \AttributeTok{cex.axis=}\DecValTok{1}\NormalTok{, }\AttributeTok{cex.lab=}\DecValTok{1}\NormalTok{, }\AttributeTok{cex.main=}\DecValTok{1}\NormalTok{,}
    \AttributeTok{xaxs=}\StringTok{"i"}\NormalTok{, }\AttributeTok{yaxs=}\StringTok{"i"}\NormalTok{, }\AttributeTok{mar =} \FunctionTok{c}\NormalTok{(}\DecValTok{5}\NormalTok{, }\DecValTok{5}\NormalTok{, }\DecValTok{3}\NormalTok{, }\DecValTok{5}\NormalTok{))}
\end{Highlighting}
\end{Shaded}

\begin{Shaded}
\begin{Highlighting}[]
\FunctionTok{library}\NormalTok{(rstan)}
\FunctionTok{rstan\_options}\NormalTok{(}\AttributeTok{auto\_write =} \ConstantTok{TRUE}\NormalTok{)            }\CommentTok{\# Cache compiled Stan programs}
\FunctionTok{options}\NormalTok{(}\AttributeTok{mc.cores =}\NormalTok{ parallel}\SpecialCharTok{::}\FunctionTok{detectCores}\NormalTok{()) }\CommentTok{\# Parallelize chains}
\NormalTok{parallel}\SpecialCharTok{:::}\FunctionTok{setDefaultClusterOptions}\NormalTok{(}\AttributeTok{setup\_strategy =} \StringTok{"sequential"}\NormalTok{)}
\end{Highlighting}
\end{Shaded}

\begin{Shaded}
\begin{Highlighting}[]
\NormalTok{util }\OtherTok{\textless{}{-}} \FunctionTok{new.env}\NormalTok{()}
\FunctionTok{source}\NormalTok{(}\StringTok{\textquotesingle{}mcmc\_analysis\_tools\_rstan.R\textquotesingle{}}\NormalTok{, }\AttributeTok{local=}\NormalTok{util)}
\FunctionTok{source}\NormalTok{(}\StringTok{\textquotesingle{}mcmc\_visualization\_tools.R\textquotesingle{}}\NormalTok{, }\AttributeTok{local=}\NormalTok{util)}
\end{Highlighting}
\end{Shaded}

\section{Data Exploration}\label{data-exploration}

The full Netflix Prize training data set consisted of 100,480,507
customer-movie pairs, each accompanied by an ordinal rating between one
and five ``stars'', with one being the worst rating and five being the
best. The observed ratings spanned 480,189 anonymized customers and
17,770 movies.

To allow for a more manageable demonstration I reduced the full data set
by considering only the first 1000 movies and then randomly subsampling
100 customers and 200 movies with probabilities proportional to the
total number of ratings. This left 2,415 total ratings.

Finally to facilitate model implementations, and add another layer of
anonymizing obfuscating, I relabeled the selected customers and movies
with contiguous indices.

\begin{Shaded}
\begin{Highlighting}[]
\NormalTok{data }\OtherTok{\textless{}{-}} \FunctionTok{read\_rdump}\NormalTok{(}\StringTok{\textquotesingle{}data/ratings.data.R\textquotesingle{}}\NormalTok{)}

\FunctionTok{cat}\NormalTok{(}\FunctionTok{sprintf}\NormalTok{(}\StringTok{"\%s Customers"}\NormalTok{, data}\SpecialCharTok{$}\NormalTok{N\_customers))}
\end{Highlighting}
\end{Shaded}

\begin{verbatim}
100 Customers
\end{verbatim}

\begin{Shaded}
\begin{Highlighting}[]
\FunctionTok{cat}\NormalTok{(}\FunctionTok{sprintf}\NormalTok{(}\StringTok{"\%s Movies"}\NormalTok{, data}\SpecialCharTok{$}\NormalTok{N\_movies))}
\end{Highlighting}
\end{Shaded}

\begin{verbatim}
200 Movies
\end{verbatim}

\begin{Shaded}
\begin{Highlighting}[]
\FunctionTok{cat}\NormalTok{(}\FunctionTok{sprintf}\NormalTok{(}\StringTok{"\%s Total Ratings"}\NormalTok{, data}\SpecialCharTok{$}\NormalTok{N\_ratings))}
\end{Highlighting}
\end{Shaded}

\begin{verbatim}
2415 Total Ratings
\end{verbatim}

Despite the data subsampling favoring customers with more ratings, most
of the selected customers rated only a few movies. Overall the training
data set is relatively sparse, with only a few customers contributing
most of the ratings.

\begin{Shaded}
\begin{Highlighting}[]
\FunctionTok{par}\NormalTok{(}\AttributeTok{mfrow=}\FunctionTok{c}\NormalTok{(}\DecValTok{1}\NormalTok{, }\DecValTok{1}\NormalTok{), }\AttributeTok{mar=}\FunctionTok{c}\NormalTok{(}\DecValTok{5}\NormalTok{, }\DecValTok{5}\NormalTok{, }\DecValTok{2}\NormalTok{, }\DecValTok{1}\NormalTok{))}

\NormalTok{util}\SpecialCharTok{$}\FunctionTok{plot\_line\_hist}\NormalTok{(}\FunctionTok{table}\NormalTok{(data}\SpecialCharTok{$}\NormalTok{customer\_idxs),}
                    \SpecialCharTok{{-}}\FloatTok{0.5}\NormalTok{, }\FloatTok{95.5}\NormalTok{, }\DecValTok{5}\NormalTok{,}
                    \AttributeTok{xlab=}\StringTok{"Number of Ratings Per Customer"}\NormalTok{)}
\end{Highlighting}
\end{Shaded}

\includegraphics{ratings_files/figure-pdf/unnamed-chunk-6-1.pdf}

Similarly most of the selected movies have only a few ratings.

\begin{Shaded}
\begin{Highlighting}[]
\FunctionTok{par}\NormalTok{(}\AttributeTok{mfrow=}\FunctionTok{c}\NormalTok{(}\DecValTok{1}\NormalTok{, }\DecValTok{1}\NormalTok{), }\AttributeTok{mar=}\FunctionTok{c}\NormalTok{(}\DecValTok{5}\NormalTok{, }\DecValTok{5}\NormalTok{, }\DecValTok{2}\NormalTok{, }\DecValTok{1}\NormalTok{))}

\NormalTok{util}\SpecialCharTok{$}\FunctionTok{plot\_line\_hist}\NormalTok{(}\FunctionTok{table}\NormalTok{(data}\SpecialCharTok{$}\NormalTok{movie\_idxs),}
                    \SpecialCharTok{{-}}\FloatTok{0.5}\NormalTok{, }\FloatTok{55.5}\NormalTok{, }\DecValTok{2}\NormalTok{,}
                    \AttributeTok{xlab=}\StringTok{"Number of Ratings Per Movie"}\NormalTok{)}
\end{Highlighting}
\end{Shaded}

\begin{verbatim}
Warning in check_bin_containment(bin_min, bin_max, values): 2 values (1.0%)
fell above the binning.
\end{verbatim}

\includegraphics{ratings_files/figure-pdf/unnamed-chunk-7-1.pdf}

This sparsity is even more evident if we visualize the customer-movie
pairings.

\begin{Shaded}
\begin{Highlighting}[]
\NormalTok{xs }\OtherTok{\textless{}{-}} \FunctionTok{seq}\NormalTok{(}\DecValTok{1}\NormalTok{, data}\SpecialCharTok{$}\NormalTok{N\_movies, }\DecValTok{1}\NormalTok{)}
\NormalTok{ys }\OtherTok{\textless{}{-}} \FunctionTok{seq}\NormalTok{(}\DecValTok{1}\NormalTok{, data}\SpecialCharTok{$}\NormalTok{N\_customers, }\DecValTok{1}\NormalTok{)}
\NormalTok{zs }\OtherTok{\textless{}{-}} \FunctionTok{matrix}\NormalTok{(}\DecValTok{0}\NormalTok{, }\AttributeTok{nrow=}\NormalTok{data}\SpecialCharTok{$}\NormalTok{N\_movies, }\AttributeTok{ncol=}\NormalTok{data}\SpecialCharTok{$}\NormalTok{N\_customers)}

\ControlFlowTok{for}\NormalTok{ (n }\ControlFlowTok{in} \DecValTok{1}\SpecialCharTok{:}\NormalTok{data}\SpecialCharTok{$}\NormalTok{N\_ratings) \{}
\NormalTok{  zs[data}\SpecialCharTok{$}\NormalTok{movie\_idxs[n], data}\SpecialCharTok{$}\NormalTok{customer\_idxs[n]] }\OtherTok{\textless{}{-}} \DecValTok{1}
\NormalTok{\}}

\FunctionTok{par}\NormalTok{(}\AttributeTok{mfrow=}\FunctionTok{c}\NormalTok{(}\DecValTok{1}\NormalTok{, }\DecValTok{1}\NormalTok{), }\AttributeTok{mar =} \FunctionTok{c}\NormalTok{(}\DecValTok{5}\NormalTok{, }\DecValTok{5}\NormalTok{, }\DecValTok{1}\NormalTok{, }\DecValTok{1}\NormalTok{))}

\FunctionTok{image}\NormalTok{(xs, ys, zs, }\AttributeTok{col=}\FunctionTok{c}\NormalTok{(}\StringTok{"white"}\NormalTok{, util}\SpecialCharTok{$}\NormalTok{c\_dark\_teal),}
      \AttributeTok{xlab=}\StringTok{"Movie"}\NormalTok{, }\AttributeTok{ylab=}\StringTok{"Customer"}\NormalTok{)}
\end{Highlighting}
\end{Shaded}

\includegraphics{ratings_files/figure-pdf/unnamed-chunk-8-1.pdf}

The observed ratings are slightly biased towards large values, with far
more four star ratings than two star ratings. From the data alone,
however, we cannot determine whether or not this is because most movies
that had been rated were relatively good or because most customers were
just generous with their ratings.

\begin{Shaded}
\begin{Highlighting}[]
\FunctionTok{par}\NormalTok{(}\AttributeTok{mfrow=}\FunctionTok{c}\NormalTok{(}\DecValTok{1}\NormalTok{, }\DecValTok{1}\NormalTok{), }\AttributeTok{mar=}\FunctionTok{c}\NormalTok{(}\DecValTok{5}\NormalTok{, }\DecValTok{5}\NormalTok{, }\DecValTok{2}\NormalTok{, }\DecValTok{1}\NormalTok{))}

\NormalTok{util}\SpecialCharTok{$}\FunctionTok{plot\_line\_hist}\NormalTok{(data}\SpecialCharTok{$}\NormalTok{ratings,}
                    \SpecialCharTok{{-}}\FloatTok{0.5}\NormalTok{, }\FloatTok{6.5}\NormalTok{, }\DecValTok{1}\NormalTok{, }\AttributeTok{xlab=}\StringTok{"Ratings"}\NormalTok{)}
\end{Highlighting}
\end{Shaded}

\includegraphics{ratings_files/figure-pdf/unnamed-chunk-9-1.pdf}

That said there is substantial heterogeneity in the rating behavior
across customers. Customer 70, for example, gave many high ratings while
Customer 23 gave many low ratings.

\begin{Shaded}
\begin{Highlighting}[]
\FunctionTok{par}\NormalTok{(}\AttributeTok{mfrow=}\FunctionTok{c}\NormalTok{(}\DecValTok{2}\NormalTok{, }\DecValTok{3}\NormalTok{), }\AttributeTok{mar=}\FunctionTok{c}\NormalTok{(}\DecValTok{5}\NormalTok{, }\DecValTok{5}\NormalTok{, }\DecValTok{1}\NormalTok{, }\DecValTok{1}\NormalTok{))}

\ControlFlowTok{for}\NormalTok{ (c }\ControlFlowTok{in} \FunctionTok{c}\NormalTok{(}\DecValTok{7}\NormalTok{, }\DecValTok{23}\NormalTok{, }\DecValTok{40}\NormalTok{, }\DecValTok{70}\NormalTok{, }\DecValTok{77}\NormalTok{, }\DecValTok{100}\NormalTok{)) \{}
\NormalTok{  util}\SpecialCharTok{$}\FunctionTok{plot\_line\_hist}\NormalTok{(data}\SpecialCharTok{$}\NormalTok{ratings[data}\SpecialCharTok{$}\NormalTok{customer\_idxs }\SpecialCharTok{==}\NormalTok{ c],}
                      \SpecialCharTok{{-}}\FloatTok{0.5}\NormalTok{, }\FloatTok{6.5}\NormalTok{, }\DecValTok{1}\NormalTok{,}
                      \AttributeTok{xlab=}\StringTok{"Rating"}\NormalTok{, }\AttributeTok{main=}\FunctionTok{paste}\NormalTok{(}\StringTok{\textquotesingle{}Customer\textquotesingle{}}\NormalTok{, c))}
\NormalTok{\}}
\end{Highlighting}
\end{Shaded}

\includegraphics{ratings_files/figure-pdf/unnamed-chunk-10-1.pdf}

Similarly we see strong variation in the observed ratings across movies.
Movies 117 and 180, for example, are particularly well-reviewed.

\begin{Shaded}
\begin{Highlighting}[]
\FunctionTok{par}\NormalTok{(}\AttributeTok{mfrow=}\FunctionTok{c}\NormalTok{(}\DecValTok{2}\NormalTok{, }\DecValTok{3}\NormalTok{), }\AttributeTok{mar=}\FunctionTok{c}\NormalTok{(}\DecValTok{5}\NormalTok{, }\DecValTok{5}\NormalTok{, }\DecValTok{1}\NormalTok{, }\DecValTok{1}\NormalTok{))}

\ControlFlowTok{for}\NormalTok{ (m }\ControlFlowTok{in} \FunctionTok{c}\NormalTok{(}\DecValTok{33}\NormalTok{, }\DecValTok{53}\NormalTok{, }\DecValTok{61}\NormalTok{, }\DecValTok{80}\NormalTok{, }\DecValTok{117}\NormalTok{, }\DecValTok{180}\NormalTok{)) \{}
\NormalTok{  util}\SpecialCharTok{$}\FunctionTok{plot\_line\_hist}\NormalTok{(data}\SpecialCharTok{$}\NormalTok{ratings[data}\SpecialCharTok{$}\NormalTok{movie\_idxs }\SpecialCharTok{==}\NormalTok{ m],}
                      \SpecialCharTok{{-}}\FloatTok{0.5}\NormalTok{, }\FloatTok{6.5}\NormalTok{, }\DecValTok{1}\NormalTok{,}
                      \AttributeTok{xlab=}\StringTok{"Rating"}\NormalTok{, }\AttributeTok{main=}\FunctionTok{paste}\NormalTok{(}\StringTok{\textquotesingle{}Movie\textquotesingle{}}\NormalTok{, m))}
\NormalTok{\}}
\end{Highlighting}
\end{Shaded}

\includegraphics{ratings_files/figure-pdf/unnamed-chunk-11-1.pdf}

When critiquing any model of these ratings we want to be able to
interrogate this variation in rating behavior across customers and
movies. Even with the subsampled data, however, visualizing a histogram
of ratings for each customer and movie would be far too ungainly. A more
scalable, albeit less informative, approach is to compute a scalar
summary of the ratings within each group, and then construct a histogram
of those \textbf{stratified} summaries.

The only problem here is identifying useful summary statistics. Because
ordinal spaces are not equipped with a distinguished metric empirical
moments are ill-defined; in particular given any order-preserving map \[
f : \{ 1, 2, 3, 4, 5 \} \rightarrow \mathbb{R}
\] we can construct a \emph{distinct} empirical mean, \[
\hat{mu}_{f} = \frac{1}{N} \sum_{n = 1}^{N} f(r_{n}),
\] and corresponding higher-order empirical moments.

That said, most of these empirical moments share similar qualitative
behaviors. If a stratified histogram is peaked towards central values,
for example, then most empirical means will end up somewhere near that
peak. Even if empirical moments capture different behaviors, however,
they can still be useful for comparing observed and posterior predictive
behaviors.

For this case study I'll assume a metric with equal unit distances
between neighboring ordinal elements, with the resulting empirical mean
\[
\hat{mu}_{f} = \frac{1}{N} \sum_{n = 1}^{N} r_{n}.
\] We can then stratify this mean by customers and movies and histogram
the resulting ensemble of summaries.

\begin{Shaded}
\begin{Highlighting}[]
\FunctionTok{par}\NormalTok{(}\AttributeTok{mfrow=}\FunctionTok{c}\NormalTok{(}\DecValTok{1}\NormalTok{, }\DecValTok{2}\NormalTok{), }\AttributeTok{mar=}\FunctionTok{c}\NormalTok{(}\DecValTok{5}\NormalTok{, }\DecValTok{5}\NormalTok{, }\DecValTok{2}\NormalTok{, }\DecValTok{1}\NormalTok{))}

\NormalTok{mean\_rating\_customer }\OtherTok{\textless{}{-}}
  \FunctionTok{sapply}\NormalTok{(}\DecValTok{1}\SpecialCharTok{:}\NormalTok{data}\SpecialCharTok{$}\NormalTok{N\_customers,}
         \ControlFlowTok{function}\NormalTok{(c) }\FunctionTok{mean}\NormalTok{(data}\SpecialCharTok{$}\NormalTok{ratings[data}\SpecialCharTok{$}\NormalTok{customer\_idxs }\SpecialCharTok{==}\NormalTok{ c]))}
\NormalTok{util}\SpecialCharTok{$}\FunctionTok{plot\_line\_hist}\NormalTok{(mean\_rating\_customer,}
                    \DecValTok{0}\NormalTok{, }\DecValTok{6}\NormalTok{, }\DecValTok{1}\NormalTok{,}
                    \AttributeTok{xlab=}\StringTok{"Customer{-}wise Average Ratings"}\NormalTok{)}


\NormalTok{mean\_rating\_movie }\OtherTok{\textless{}{-}}
  \FunctionTok{sapply}\NormalTok{(}\DecValTok{1}\SpecialCharTok{:}\NormalTok{data}\SpecialCharTok{$}\NormalTok{N\_movies,}
         \ControlFlowTok{function}\NormalTok{(m) }\FunctionTok{mean}\NormalTok{(data}\SpecialCharTok{$}\NormalTok{ratings[data}\SpecialCharTok{$}\NormalTok{movie\_idxs }\SpecialCharTok{==}\NormalTok{ m]))}
\NormalTok{util}\SpecialCharTok{$}\FunctionTok{plot\_line\_hist}\NormalTok{(mean\_rating\_movie,}
                    \DecValTok{0}\NormalTok{, }\DecValTok{6}\NormalTok{, }\FloatTok{0.5}\NormalTok{,}
                    \AttributeTok{xlab=}\StringTok{"Movie{-}wise Average Ratings"}\NormalTok{)}
\end{Highlighting}
\end{Shaded}

\includegraphics{ratings_files/figure-pdf/unnamed-chunk-12-1.pdf}

Of course an empirical mean captures only some of the rating behavior
within each strata. We can capture more with a summary that is sensitive
to the dispersion of ratings in each strata, such as the empirical
variance or empirical entropy. Note one nice advantage of the empirical
entropy relative to the empirical variance is that it does not require a
choice of metric over the ordinal values.

Here let's go with the empirical variance based on the same assumptions
as our empirical mean, or rather a modified empirical variance that
defaults to zero when a strata consists of only one value.

\begin{Shaded}
\begin{Highlighting}[]
\NormalTok{safe\_var }\OtherTok{\textless{}{-}} \ControlFlowTok{function}\NormalTok{(vals) \{}
  \ControlFlowTok{if}\NormalTok{ (}\FunctionTok{length}\NormalTok{(vals) }\SpecialCharTok{==} \DecValTok{1}\NormalTok{)}
\NormalTok{    (}\DecValTok{0}\NormalTok{)}
  \ControlFlowTok{else}
\NormalTok{    (}\FunctionTok{var}\NormalTok{(vals))}
\NormalTok{\}}
\end{Highlighting}
\end{Shaded}

\begin{Shaded}
\begin{Highlighting}[]
\FunctionTok{par}\NormalTok{(}\AttributeTok{mfrow=}\FunctionTok{c}\NormalTok{(}\DecValTok{1}\NormalTok{, }\DecValTok{2}\NormalTok{), }\AttributeTok{mar=}\FunctionTok{c}\NormalTok{(}\DecValTok{5}\NormalTok{, }\DecValTok{5}\NormalTok{, }\DecValTok{2}\NormalTok{, }\DecValTok{1}\NormalTok{))}

\NormalTok{var\_rating\_customer }\OtherTok{\textless{}{-}}
  \FunctionTok{sapply}\NormalTok{(}\DecValTok{1}\SpecialCharTok{:}\NormalTok{data}\SpecialCharTok{$}\NormalTok{N\_customers,}
         \ControlFlowTok{function}\NormalTok{(c) }\FunctionTok{safe\_var}\NormalTok{(data}\SpecialCharTok{$}\NormalTok{ratings[data}\SpecialCharTok{$}\NormalTok{customer\_idxs }\SpecialCharTok{==}\NormalTok{ c]))}
\NormalTok{util}\SpecialCharTok{$}\FunctionTok{plot\_line\_hist}\NormalTok{(var\_rating\_customer,}
                    \DecValTok{0}\NormalTok{, }\DecValTok{5}\NormalTok{, }\FloatTok{0.5}\NormalTok{,}
                    \AttributeTok{xlab=}\StringTok{"Customer{-}wise Rating Variances"}\NormalTok{)}

\NormalTok{var\_rating\_movie }\OtherTok{\textless{}{-}}
  \FunctionTok{sapply}\NormalTok{(}\DecValTok{1}\SpecialCharTok{:}\NormalTok{data}\SpecialCharTok{$}\NormalTok{N\_movies,}
         \ControlFlowTok{function}\NormalTok{(m) }\FunctionTok{safe\_var}\NormalTok{(data}\SpecialCharTok{$}\NormalTok{ratings[data}\SpecialCharTok{$}\NormalTok{movie\_idxs }\SpecialCharTok{==}\NormalTok{ m]))}
\NormalTok{util}\SpecialCharTok{$}\FunctionTok{plot\_line\_hist}\NormalTok{(var\_rating\_movie,}
                    \DecValTok{0}\NormalTok{, }\DecValTok{5}\NormalTok{, }\FloatTok{0.5}\NormalTok{,}
                    \AttributeTok{xlab=}\StringTok{"Movie{-}wise Rating Variances"}\NormalTok{)}
\end{Highlighting}
\end{Shaded}

\includegraphics{ratings_files/figure-pdf/unnamed-chunk-14-1.pdf}

The main limitation with these stratified summary statistics is that
they are sensitive to only the marginal variation across movies and
customers. In other words they are sensitive to heterogeneity across
movies or customers but not heterogeneity across movies and customers
\emph{at the same time}. Unfortunately because each customer-movie pair
has at most one rating, and most have no ratings, we can't just stratify
summary statistics by both customer and movie.

One potential compromise is to construct empirical covariances for each
pair of customers or movies. For example given our assumed metric the
empirical covariance between two movies \(m_{1}\) and \(m_{2}\) is
defined by \[
\hat{\rho}_{m_{1} m_{2}}
=
\hat{\rho}_{m_{2} m_{1}}
=
\frac{1}{N_{C} - 1} \sum_{c = 1}^{N_{C}}
(r_{c m_{1}} - \hat{\mu}_{m_{1}}) \cdot
(r_{c m_{2}} - \hat{\mu}_{m_{2}})
\] where \(N_{C}\) is the number of customers, \(r_{c m}\) is the rating
given to movie \(m\) by customer \(c\), and \[
\hat{\mu}_{m}
=
\frac{1}{N_{C}} \sum_{c = 1}^{N_{C}} r_{c m}.
\]

The immediate issue is that, because not every customer rates every
movie, many of the \(r_{c m}\) in these sums will be undefined. All we
can do here is limit the sums to the customers who have rated both
movies.

More formally let \(\mathsf{c}_{m}\) denote the set of customers who
have rated movie \(m\) and \(N_{C, m}\) denote the number of elements in
that set. Similarly let \(\mathsf{c}_{m_{1} m_{2}}\) denote the set of
customers who have rated \emph{both} movies \(m_{1}\) and \(m_{2}\) with
\(N_{C, m_{1} m_{2}}\) the number of elements in that set. Then we can
define \[
\hat{\rho}_{m_{1} m_{2}}
=
\hat{\rho}_{m_{2} m_{1}}
=
\frac{1}{N_{C, m_{1} m_{2}} - 1} \sum_{c \in \mathsf{c}_{m_{1} m_{2}}}
(r_{c m_{1}} - \hat{\mu}_{m_{1}}) \cdot
(r_{c m_{2}} - \hat{\mu}_{m_{2}})
\] with \[
\hat{\mu}_{m}
=
\frac{1}{N_{C, m}} \sum_{c \in \mathsf{c}_{m}} r_{c m}
\] the movie-wise empirical means that we've already constructed.

All of this said given the relative sparsity of the observed ratings the
set \(\mathsf{c}_{m_{1} m_{2}}\) will be empty for most pairs of movies.
Even fewer pairs of movies will have the \(N_{C, m_{1} m_{2}} > 1\)
needed for \(\hat{\rho}_{m_{1} m_{2}}\) to be well-defined, let alone
\(N_{C, m_{1} m_{2}}\) large enough for \(\hat{\rho}_{m_{1} m_{2}}\) to
provide an informative summary.

We can avoid any ill-defined or poorly informative empirical covariances
by including only those movie pairs with \(N_{C, m_{1} m_{2}}\)
sufficiently large enough in the final histogram. This also has the
added benefit of reducing the total number of summaries that we have to
bin into the final histogram visualization. Here I will require
\(N_{C, m_{1} m_{2}} > 7\).

Now that we've carefully laid out the math the implementation is
relatively straightforward. First we loop over the observed ratings
twice, incrementing the partial sums for each pair of movies when
appropriate. This gives us \[
\hat{\Sigma}_{m_{1} m_{2}}
=
\sum_{c \in \mathsf{c}_{m_{1} m_{2}}}
(r_{c m_{1}} - \hat{\mu}_{m_{1}}) \cdot
(r_{c m_{2}} - \hat{\mu}_{m_{2}})
\] and \[
N_{C, m_{1} m_{2}}
=
\sum_{c \in \mathsf{c}_{m_{1} m_{2}}} 1.
\]

\begin{Shaded}
\begin{Highlighting}[]
\NormalTok{covar\_rating\_movie }\OtherTok{\textless{}{-}} \FunctionTok{matrix}\NormalTok{(}\DecValTok{0}\NormalTok{,}
                             \AttributeTok{nrow=}\NormalTok{data}\SpecialCharTok{$}\NormalTok{N\_movies,}
                             \AttributeTok{ncol=}\NormalTok{data}\SpecialCharTok{$}\NormalTok{N\_movies)}
\NormalTok{movie\_pair\_counts }\OtherTok{\textless{}{-}} \FunctionTok{matrix}\NormalTok{(}\DecValTok{0}\NormalTok{,}
                            \AttributeTok{nrow=}\NormalTok{data}\SpecialCharTok{$}\NormalTok{N\_movies,}
                            \AttributeTok{ncol=}\NormalTok{data}\SpecialCharTok{$}\NormalTok{N\_movies)}

\ControlFlowTok{for}\NormalTok{ (n1 }\ControlFlowTok{in} \DecValTok{1}\SpecialCharTok{:}\NormalTok{data}\SpecialCharTok{$}\NormalTok{N\_ratings) \{}
  \ControlFlowTok{for}\NormalTok{ (n2 }\ControlFlowTok{in} \DecValTok{1}\SpecialCharTok{:}\NormalTok{data}\SpecialCharTok{$}\NormalTok{N\_ratings) \{}
    \ControlFlowTok{if}\NormalTok{ (data}\SpecialCharTok{$}\NormalTok{customer\_idxs[n1] }\SpecialCharTok{==}\NormalTok{ data}\SpecialCharTok{$}\NormalTok{customer\_idxs[n2]) \{}
\NormalTok{      m1 }\OtherTok{\textless{}{-}}\NormalTok{ data}\SpecialCharTok{$}\NormalTok{movie\_idxs[n1]}
\NormalTok{      m2 }\OtherTok{\textless{}{-}}\NormalTok{ data}\SpecialCharTok{$}\NormalTok{movie\_idxs[n2]}
\NormalTok{      y }\OtherTok{\textless{}{-}}\NormalTok{ (data}\SpecialCharTok{$}\NormalTok{ratings[n1] }\SpecialCharTok{{-}}\NormalTok{ mean\_rating\_movie[m1]) }\SpecialCharTok{*}
\NormalTok{           (data}\SpecialCharTok{$}\NormalTok{ratings[n2] }\SpecialCharTok{{-}}\NormalTok{ mean\_rating\_movie[m2])}
\NormalTok{      covar\_rating\_movie[m1, m2] }\OtherTok{\textless{}{-}}\NormalTok{ covar\_rating\_movie[m1, m2] }\SpecialCharTok{+}\NormalTok{ y}
\NormalTok{      covar\_rating\_movie[m2, m1] }\OtherTok{\textless{}{-}}\NormalTok{ covar\_rating\_movie[m2, m1] }\SpecialCharTok{+}\NormalTok{ y}
\NormalTok{      movie\_pair\_counts[m1, m2] }\OtherTok{\textless{}{-}}\NormalTok{ movie\_pair\_counts[m1, m2] }\SpecialCharTok{+} \DecValTok{1}
\NormalTok{      movie\_pair\_counts[m2, m1] }\OtherTok{\textless{}{-}}\NormalTok{ movie\_pair\_counts[m2, m1] }\SpecialCharTok{+} \DecValTok{1}
\NormalTok{    \}}
\NormalTok{  \}}
\NormalTok{\}}
\end{Highlighting}
\end{Shaded}

Next we compute \[
\hat{\rho}_{m_{1} m_{2}}
=
\frac{ \hat{\Sigma}_{m_{1} m_{2}} }{ N_{C, m_{1} m_{2}} - 1 }
\] for each pair of movies where \(N_{C, m_{1} m_{2}}\) is larger than
\(7\).

\begin{Shaded}
\begin{Highlighting}[]
\NormalTok{m\_pairs }\OtherTok{\textless{}{-}} \FunctionTok{list}\NormalTok{()}
\NormalTok{covar\_rating\_movie\_filt }\OtherTok{\textless{}{-}} \FunctionTok{c}\NormalTok{()}

\ControlFlowTok{for}\NormalTok{ (m1 }\ControlFlowTok{in} \DecValTok{2}\SpecialCharTok{:}\NormalTok{data}\SpecialCharTok{$}\NormalTok{N\_movies) \{}
  \ControlFlowTok{for}\NormalTok{ (m2 }\ControlFlowTok{in} \DecValTok{1}\SpecialCharTok{:}\NormalTok{(m1 }\SpecialCharTok{{-}} \DecValTok{1}\NormalTok{)) \{}
    \ControlFlowTok{if}\NormalTok{ (movie\_pair\_counts[m1, m2] }\SpecialCharTok{\textgreater{}} \DecValTok{7}\NormalTok{) \{}
\NormalTok{      m\_pairs[[}\FunctionTok{length}\NormalTok{(m\_pairs) }\SpecialCharTok{+} \DecValTok{1}\NormalTok{]] }\OtherTok{\textless{}{-}} \FunctionTok{c}\NormalTok{(m1, m2)}
\NormalTok{      covar\_rating\_movie\_filt }\OtherTok{\textless{}{-}} \FunctionTok{c}\NormalTok{(covar\_rating\_movie\_filt,}
\NormalTok{                                   covar\_rating\_movie[m1, m2] }\SpecialCharTok{/}
\NormalTok{                                     (movie\_pair\_counts[m1, m2] }\SpecialCharTok{{-}} \DecValTok{1}\NormalTok{))}
\NormalTok{    \}}
\NormalTok{  \}}
\NormalTok{\}}
\end{Highlighting}
\end{Shaded}

Finally we bin these values into a histogram that visualizes the range
of these partial empirical covariance behaviors.

\begin{Shaded}
\begin{Highlighting}[]
\FunctionTok{par}\NormalTok{(}\AttributeTok{mfrow=}\FunctionTok{c}\NormalTok{(}\DecValTok{1}\NormalTok{, }\DecValTok{1}\NormalTok{), }\AttributeTok{mar=}\FunctionTok{c}\NormalTok{(}\DecValTok{5}\NormalTok{, }\DecValTok{5}\NormalTok{, }\DecValTok{2}\NormalTok{, }\DecValTok{1}\NormalTok{))}

\NormalTok{util}\SpecialCharTok{$}\FunctionTok{plot\_line\_hist}\NormalTok{(covar\_rating\_movie\_filt,}
                    \SpecialCharTok{{-}}\DecValTok{4}\NormalTok{, }\DecValTok{4}\NormalTok{, }\FloatTok{0.25}\NormalTok{,}
                    \AttributeTok{xlab=}\StringTok{"Movie{-}wise Rating Covariances"}\NormalTok{)}
\end{Highlighting}
\end{Shaded}

\includegraphics{ratings_files/figure-pdf/unnamed-chunk-17-1.pdf}

In order to construct posterior retrodictive checks later on we will
need to select the posterior predictive values for these same selected
movie pairs. We might as well construct the appropriate variable names
now and have them ready.

\begin{Shaded}
\begin{Highlighting}[]
\NormalTok{covar\_rating\_movie\_filt\_names }\OtherTok{\textless{}{-}}
  \FunctionTok{sapply}\NormalTok{(m\_pairs,}
         \ControlFlowTok{function}\NormalTok{(p) }\FunctionTok{paste0}\NormalTok{(}\StringTok{\textquotesingle{}covar\_rating\_movie\_pred[\textquotesingle{}}\NormalTok{,}
\NormalTok{                            p[}\DecValTok{1}\NormalTok{], }\StringTok{\textquotesingle{},\textquotesingle{}}\NormalTok{, p[}\DecValTok{2}\NormalTok{], }\StringTok{\textquotesingle{}]\textquotesingle{}}\NormalTok{))}
\end{Highlighting}
\end{Shaded}

All of this said I think that there is still a lot of opportunity for
better summary statistics in applications like these, such as summaries
that are more compatible with the structure of an ordinal space and
don't require the assumption of an arbitrary metric and summaries that
better capture couplings between different strata.

\section{Homogeneous Customer Model}\label{homogeneous-customer-model}

Now that we've familiarized ourselves with the data we can make our
first attempt at modeling the data generating process that, well,
generated it. Our models will be built on a foundation of
\href{https://betanalpha.github.io/assets/chapters_html/ordinal_modeling.html}{ordinal
pairwise comparison modeling techniques}.

Given a latent logistic probability density function we will use cut
points to derive baseline ordinal probabilities for each star rating.
This baseline will not be tied to any particular movie but rather a
hypothetical default movie implied by the configuration of an induced
Dirichlet prior model.

Next we will assume that movies systematically shift these baseline
probabilities to lower or higher ratings depending on their quality.
This is implemented with affinity parameters for each movie that shift
the latent logistic probability density function, and hence the derived
ordinal probabilities, based on customer preference. Initially we will
assume that the rating behavior is homogeneous across customers so that
we need only a single set of cut points to model all of the data.

Lastly assuming that our domain expertise about consumer preferences is
exchangeable we will model the movie affinity parameters hierarchically.
To avoid degeneracy in the customer-movie comparisons we will need to
anchor the population location of this hierarchical model to zero, the
same anchor location used in the induced Dirichlet prior model. Given
the sparsity of observed ratings we'll implement this hierarchical model
with a monolithic non-centered parameterization.

\begin{codelisting}

\caption{\texttt{model1.stan}}

\begin{Shaded}
\begin{Highlighting}[]
\KeywordTok{functions}\NormalTok{ \{}
  \CommentTok{// Log probability density function over cut point}
  \CommentTok{// induced by a Dirichlet probability density function}
  \CommentTok{// over baseline probabilities and latent logistic}
  \CommentTok{// density function.}
  \DataTypeTok{real}\NormalTok{ induced\_dirichlet\_lpdf(}\DataTypeTok{vector}\NormalTok{ c, }\DataTypeTok{vector}\NormalTok{ alpha) \{}
    \DataTypeTok{int}\NormalTok{ K = num\_elements(c) + }\DecValTok{1}\NormalTok{;}
    \DataTypeTok{vector}\NormalTok{[K {-} }\DecValTok{1}\NormalTok{] Pi = inv\_logit(c);}
    \DataTypeTok{vector}\NormalTok{[K] p = append\_row(Pi, [}\DecValTok{1}\NormalTok{]\textquotesingle{}) {-} append\_row([}\DecValTok{0}\NormalTok{]\textquotesingle{}, Pi);}

    \CommentTok{// Log Jacobian correction}
    \DataTypeTok{real}\NormalTok{ logJ = }\DecValTok{0}\NormalTok{;}
    \ControlFlowTok{for}\NormalTok{ (k }\ControlFlowTok{in} \DecValTok{1}\NormalTok{:(K {-} }\DecValTok{1}\NormalTok{)) \{}
      \ControlFlowTok{if}\NormalTok{ (c[k] \textgreater{}= }\DecValTok{0}\NormalTok{)}
\NormalTok{        logJ += {-}c[k] {-} }\DecValTok{2}\NormalTok{ * log(}\DecValTok{1}\NormalTok{ + exp({-}c[k]));}
      \ControlFlowTok{else}
\NormalTok{        logJ += +c[k] {-} }\DecValTok{2}\NormalTok{ * log(}\DecValTok{1}\NormalTok{ + exp(+c[k]));}
\NormalTok{    \}}

    \ControlFlowTok{return}\NormalTok{ dirichlet\_lpdf(p | alpha) + logJ;}
\NormalTok{  \}}
\NormalTok{\}}

\KeywordTok{data}\NormalTok{ \{}
  \DataTypeTok{int}\NormalTok{\textless{}}\KeywordTok{lower}\NormalTok{=}\DecValTok{1}\NormalTok{\textgreater{} N\_ratings;}
  \DataTypeTok{array}\NormalTok{[N\_ratings] }\DataTypeTok{int}\NormalTok{\textless{}}\KeywordTok{lower}\NormalTok{=}\DecValTok{1}\NormalTok{, }\KeywordTok{upper}\NormalTok{=}\DecValTok{5}\NormalTok{\textgreater{} ratings;}

  \DataTypeTok{int}\NormalTok{\textless{}}\KeywordTok{lower}\NormalTok{=}\DecValTok{1}\NormalTok{\textgreater{} N\_customers;}
  \DataTypeTok{array}\NormalTok{[N\_ratings] }\DataTypeTok{int}\NormalTok{\textless{}}\KeywordTok{lower}\NormalTok{=}\DecValTok{1}\NormalTok{, }\KeywordTok{upper}\NormalTok{=N\_customers\textgreater{} customer\_idxs;}

  \DataTypeTok{int}\NormalTok{\textless{}}\KeywordTok{lower}\NormalTok{=}\DecValTok{1}\NormalTok{\textgreater{} N\_movies;}
  \DataTypeTok{array}\NormalTok{[N\_ratings] }\DataTypeTok{int}\NormalTok{\textless{}}\KeywordTok{lower}\NormalTok{=}\DecValTok{1}\NormalTok{, }\KeywordTok{upper}\NormalTok{=N\_movies\textgreater{} movie\_idxs;}
\NormalTok{\}}

\KeywordTok{parameters}\NormalTok{ \{}
  \DataTypeTok{vector}\NormalTok{[N\_movies] gamma\_ncp; }\CommentTok{// Non{-}centered movie affinities}
  \DataTypeTok{real}\NormalTok{\textless{}}\KeywordTok{lower}\NormalTok{=}\DecValTok{0}\NormalTok{\textgreater{} tau\_gamma;    }\CommentTok{// Movie affinity population scale}

  \DataTypeTok{ordered}\NormalTok{[}\DecValTok{4}\NormalTok{] cut\_points;      }\CommentTok{// Customer rating cut points}
\NormalTok{\}}

\KeywordTok{transformed parameters}\NormalTok{ \{}
  \CommentTok{// Centered movie affinities}
  \DataTypeTok{vector}\NormalTok{[N\_movies] gamma = tau\_gamma * gamma\_ncp;}
\NormalTok{\}}

\KeywordTok{model}\NormalTok{ \{}
  \CommentTok{// Prior model}
\NormalTok{  gamma\_ncp \textasciitilde{} normal(}\DecValTok{0}\NormalTok{, }\DecValTok{1}\NormalTok{);}
\NormalTok{  tau\_gamma \textasciitilde{} normal(}\DecValTok{0}\NormalTok{, }\DecValTok{5}\NormalTok{ / }\FloatTok{2.57}\NormalTok{);}

\NormalTok{  cut\_points \textasciitilde{} induced\_dirichlet(rep\_vector(}\DecValTok{1}\NormalTok{, }\DecValTok{5}\NormalTok{));}

  \CommentTok{// Observational model}
\NormalTok{  ratings \textasciitilde{} ordered\_logistic(gamma[movie\_idxs], cut\_points);}
\NormalTok{\}}

\KeywordTok{generated quantities}\NormalTok{ \{}
  \DataTypeTok{array}\NormalTok{[N\_ratings] }\DataTypeTok{int}\NormalTok{\textless{}}\KeywordTok{lower}\NormalTok{=}\DecValTok{1}\NormalTok{, }\KeywordTok{upper}\NormalTok{=}\DecValTok{5}\NormalTok{\textgreater{} rating\_pred;}

  \DataTypeTok{array}\NormalTok{[N\_customers] }\DataTypeTok{real}\NormalTok{ mean\_rating\_customer\_pred}
\NormalTok{    = rep\_array(}\DecValTok{0}\NormalTok{, N\_customers);}
  \DataTypeTok{array}\NormalTok{[N\_customers] }\DataTypeTok{real}\NormalTok{ var\_rating\_customer\_pred}
\NormalTok{    = rep\_array(}\DecValTok{0}\NormalTok{, N\_customers);}

  \DataTypeTok{array}\NormalTok{[N\_movies] }\DataTypeTok{real}\NormalTok{ mean\_rating\_movie\_pred = rep\_array(}\DecValTok{0}\NormalTok{, N\_movies);}
  \DataTypeTok{array}\NormalTok{[N\_movies] }\DataTypeTok{real}\NormalTok{ var\_rating\_movie\_pred = rep\_array(}\DecValTok{0}\NormalTok{, N\_movies);}

  \DataTypeTok{matrix}\NormalTok{[N\_movies, N\_movies] covar\_rating\_movie\_pred;}

\NormalTok{  \{}
    \DataTypeTok{array}\NormalTok{[N\_customers] }\DataTypeTok{real}\NormalTok{ C = rep\_array(}\DecValTok{0}\NormalTok{, N\_customers);}
    \DataTypeTok{array}\NormalTok{[N\_movies] }\DataTypeTok{real}\NormalTok{ M = rep\_array(}\DecValTok{0}\NormalTok{, N\_movies);}

    \ControlFlowTok{for}\NormalTok{ (n }\ControlFlowTok{in} \DecValTok{1}\NormalTok{:N\_ratings) \{}
      \DataTypeTok{real}\NormalTok{ delta = }\DecValTok{0}\NormalTok{;}
      \DataTypeTok{int}\NormalTok{ c = customer\_idxs[n];}
      \DataTypeTok{int}\NormalTok{ m = movie\_idxs[n];}

\NormalTok{      rating\_pred[n] = ordered\_logistic\_rng(gamma[m], cut\_points);}

\NormalTok{      C[c] += }\DecValTok{1}\NormalTok{;}
\NormalTok{      delta = rating\_pred[n] {-} mean\_rating\_customer\_pred[c];}
\NormalTok{      mean\_rating\_customer\_pred[c] += delta / C[c];}
\NormalTok{      var\_rating\_customer\_pred[c]}
\NormalTok{        += delta * (rating\_pred[n] {-} mean\_rating\_customer\_pred[c]);}

\NormalTok{      M[m] += }\DecValTok{1}\NormalTok{;}
\NormalTok{      delta = rating\_pred[n] {-} mean\_rating\_movie\_pred[m];}
\NormalTok{      mean\_rating\_movie\_pred[m] += delta / M[m];}
\NormalTok{      var\_rating\_movie\_pred[m]}
\NormalTok{        += delta * (rating\_pred[n] {-} mean\_rating\_movie\_pred[m]);}
\NormalTok{    \}}

    \ControlFlowTok{for}\NormalTok{ (c }\ControlFlowTok{in} \DecValTok{1}\NormalTok{:N\_customers) \{}
      \ControlFlowTok{if}\NormalTok{ (C[c] \textgreater{} }\DecValTok{1}\NormalTok{)}
\NormalTok{        var\_rating\_customer\_pred[c] /= C[c] {-} }\DecValTok{1}\NormalTok{;}
      \ControlFlowTok{else}
\NormalTok{        var\_rating\_customer\_pred[c] = }\DecValTok{0}\NormalTok{;}
\NormalTok{    \}}
    \ControlFlowTok{for}\NormalTok{ (m }\ControlFlowTok{in} \DecValTok{1}\NormalTok{:N\_movies) \{}
      \ControlFlowTok{if}\NormalTok{ (M[m] \textgreater{} }\DecValTok{1}\NormalTok{)}
\NormalTok{        var\_rating\_movie\_pred[m] /= M[m] {-} }\DecValTok{1}\NormalTok{;}
      \ControlFlowTok{else}
\NormalTok{        var\_rating\_movie\_pred[m] = }\DecValTok{0}\NormalTok{;}
\NormalTok{    \}}
\NormalTok{  \}}

\NormalTok{  \{}
    \DataTypeTok{matrix}\NormalTok{[N\_movies, N\_movies] counts;}

    \ControlFlowTok{for}\NormalTok{ (m1 }\ControlFlowTok{in} \DecValTok{1}\NormalTok{:N\_movies) \{}
      \ControlFlowTok{for}\NormalTok{ (m2 }\ControlFlowTok{in} \DecValTok{1}\NormalTok{:N\_movies) \{}
\NormalTok{        counts[m1, m2] = }\DecValTok{0}\NormalTok{;}
\NormalTok{        covar\_rating\_movie\_pred[m1, m2] = }\DecValTok{0}\NormalTok{;}
\NormalTok{      \}}
\NormalTok{    \}}

    \ControlFlowTok{for}\NormalTok{ (n1 }\ControlFlowTok{in} \DecValTok{1}\NormalTok{:N\_ratings) \{}
      \ControlFlowTok{for}\NormalTok{ (n2 }\ControlFlowTok{in} \DecValTok{1}\NormalTok{:N\_ratings) \{}
        \ControlFlowTok{if}\NormalTok{ (customer\_idxs[n1] == customer\_idxs[n2]) \{}
          \DataTypeTok{int}\NormalTok{ m1 = movie\_idxs[n1];}
          \DataTypeTok{int}\NormalTok{ m2 = movie\_idxs[n2];}
          \DataTypeTok{real}\NormalTok{ y =   (ratings[n1] {-} mean\_rating\_movie\_pred[m1])}
\NormalTok{                   * (ratings[n2] {-} mean\_rating\_movie\_pred[m2]);}
\NormalTok{          covar\_rating\_movie\_pred[m1, m2] += y;}
\NormalTok{          covar\_rating\_movie\_pred[m2, m1] += y;}
\NormalTok{          counts[m1, m2] += }\DecValTok{1}\NormalTok{;}
\NormalTok{          counts[m2, m1] += }\DecValTok{1}\NormalTok{;}
\NormalTok{        \}}
\NormalTok{      \}}
\NormalTok{    \}}

    \ControlFlowTok{for}\NormalTok{ (m1 }\ControlFlowTok{in} \DecValTok{1}\NormalTok{:N\_movies) \{}
      \ControlFlowTok{for}\NormalTok{ (m2 }\ControlFlowTok{in} \DecValTok{1}\NormalTok{:N\_movies) \{}
        \ControlFlowTok{if}\NormalTok{ (counts[m1, m2] \textgreater{} }\DecValTok{1}\NormalTok{)}
\NormalTok{          covar\_rating\_movie\_pred[m1, m2] /= counts[m1, m2] {-} }\DecValTok{1}\NormalTok{;}
\NormalTok{      \}}
\NormalTok{    \}}
\NormalTok{  \}}
\NormalTok{\}}
\end{Highlighting}
\end{Shaded}

\end{codelisting}

\begin{Shaded}
\begin{Highlighting}[]
\NormalTok{fit }\OtherTok{\textless{}{-}} \FunctionTok{stan}\NormalTok{(}\AttributeTok{file=}\StringTok{"stan\_programs/model1.stan"}\NormalTok{,}
            \AttributeTok{data=}\NormalTok{data, }\AttributeTok{seed=}\DecValTok{8438338}\NormalTok{,}
            \AttributeTok{warmup=}\DecValTok{1000}\NormalTok{, }\AttributeTok{iter=}\DecValTok{2024}\NormalTok{, }\AttributeTok{refresh=}\DecValTok{0}\NormalTok{)}
\end{Highlighting}
\end{Shaded}

Despite the initial model complexity there are no diagnostic issues
indicating suspect computation.

\begin{Shaded}
\begin{Highlighting}[]
\NormalTok{diagnostics }\OtherTok{\textless{}{-}}\NormalTok{ util}\SpecialCharTok{$}\FunctionTok{extract\_hmc\_diagnostics}\NormalTok{(fit)}
\NormalTok{util}\SpecialCharTok{$}\FunctionTok{check\_all\_hmc\_diagnostics}\NormalTok{(diagnostics)}
\end{Highlighting}
\end{Shaded}

\begin{verbatim}
  All Hamiltonian Monte Carlo diagnostics are consistent with reliable
Markov chain Monte Carlo.
\end{verbatim}

\begin{Shaded}
\begin{Highlighting}[]
\NormalTok{samples1 }\OtherTok{\textless{}{-}}\NormalTok{ util}\SpecialCharTok{$}\FunctionTok{extract\_expectand\_vals}\NormalTok{(fit)}
\NormalTok{base\_samples }\OtherTok{\textless{}{-}}\NormalTok{ util}\SpecialCharTok{$}\FunctionTok{filter\_expectands}\NormalTok{(samples1,}
                                       \FunctionTok{c}\NormalTok{(}\StringTok{\textquotesingle{}gamma\_ncp\textquotesingle{}}\NormalTok{,}
                                         \StringTok{\textquotesingle{}tau\_gamma\textquotesingle{}}\NormalTok{,}
                                         \StringTok{\textquotesingle{}cut\_points\textquotesingle{}}\NormalTok{),}
                                       \AttributeTok{check\_arrays=}\ConstantTok{TRUE}\NormalTok{)}
\NormalTok{util}\SpecialCharTok{$}\FunctionTok{check\_all\_expectand\_diagnostics}\NormalTok{(base\_samples)}
\end{Highlighting}
\end{Shaded}

\begin{verbatim}
All expectands checked appear to be behaving well enough for reliable
Markov chain Monte Carlo estimation.
\end{verbatim}

Consequently we're ready to investigate this model's retrodictive
performance.

The model appears to be flexible enough to capture the behavior of the
aggregate ratings.

\begin{Shaded}
\begin{Highlighting}[]
\FunctionTok{par}\NormalTok{(}\AttributeTok{mfrow=}\FunctionTok{c}\NormalTok{(}\DecValTok{1}\NormalTok{, }\DecValTok{1}\NormalTok{), }\AttributeTok{mar=}\FunctionTok{c}\NormalTok{(}\DecValTok{5}\NormalTok{, }\DecValTok{5}\NormalTok{, }\DecValTok{1}\NormalTok{, }\DecValTok{1}\NormalTok{))}

\NormalTok{util}\SpecialCharTok{$}\FunctionTok{plot\_hist\_quantiles}\NormalTok{(samples1, }\StringTok{\textquotesingle{}rating\_pred\textquotesingle{}}\NormalTok{, }\SpecialCharTok{{-}}\FloatTok{0.5}\NormalTok{, }\FloatTok{6.5}\NormalTok{, }\DecValTok{1}\NormalTok{,}
                         \AttributeTok{baseline\_values=}\NormalTok{data}\SpecialCharTok{$}\NormalTok{ratings,}
                         \AttributeTok{xlab=}\StringTok{"All Ratings"}\NormalTok{)}
\end{Highlighting}
\end{Shaded}

\includegraphics{ratings_files/figure-pdf/unnamed-chunk-21-1.pdf}

On the other hand the retrodictive performance is much worse if we look
at individual customer behaviors. In particular there is much more
heterogeneity in the observed data than what the model can reproduce,
which isn't surprising given that we explicitly assumed homogeneous
customer behavior.

\begin{Shaded}
\begin{Highlighting}[]
\FunctionTok{par}\NormalTok{(}\AttributeTok{mfrow=}\FunctionTok{c}\NormalTok{(}\DecValTok{2}\NormalTok{, }\DecValTok{3}\NormalTok{), }\AttributeTok{mar=}\FunctionTok{c}\NormalTok{(}\DecValTok{5}\NormalTok{, }\DecValTok{5}\NormalTok{, }\DecValTok{1}\NormalTok{, }\DecValTok{1}\NormalTok{))}

\ControlFlowTok{for}\NormalTok{ (c }\ControlFlowTok{in} \FunctionTok{c}\NormalTok{(}\DecValTok{7}\NormalTok{, }\DecValTok{23}\NormalTok{, }\DecValTok{40}\NormalTok{, }\DecValTok{70}\NormalTok{, }\DecValTok{77}\NormalTok{, }\DecValTok{100}\NormalTok{)) \{}
\NormalTok{  names }\OtherTok{\textless{}{-}} \FunctionTok{sapply}\NormalTok{(}\FunctionTok{which}\NormalTok{(data}\SpecialCharTok{$}\NormalTok{customer\_idxs }\SpecialCharTok{==}\NormalTok{ c),}
                  \ControlFlowTok{function}\NormalTok{(n) }\FunctionTok{paste0}\NormalTok{(}\StringTok{\textquotesingle{}rating\_pred[\textquotesingle{}}\NormalTok{, n, }\StringTok{\textquotesingle{}]\textquotesingle{}}\NormalTok{))}
\NormalTok{  filtered\_samples }\OtherTok{\textless{}{-}}\NormalTok{ util}\SpecialCharTok{$}\FunctionTok{filter\_expectands}\NormalTok{(samples1, names)}

\NormalTok{  customer\_ratings }\OtherTok{\textless{}{-}}\NormalTok{ data}\SpecialCharTok{$}\NormalTok{ratings[data}\SpecialCharTok{$}\NormalTok{customer\_idxs }\SpecialCharTok{==}\NormalTok{ c]}
\NormalTok{  util}\SpecialCharTok{$}\FunctionTok{plot\_hist\_quantiles}\NormalTok{(filtered\_samples, }\StringTok{\textquotesingle{}rating\_pred\textquotesingle{}}\NormalTok{,}
                           \SpecialCharTok{{-}}\FloatTok{0.5}\NormalTok{, }\FloatTok{6.5}\NormalTok{, }\DecValTok{1}\NormalTok{,}
                           \AttributeTok{baseline\_values=}\NormalTok{customer\_ratings,}
                           \AttributeTok{xlab=}\StringTok{"Ratings"}\NormalTok{,}
                           \AttributeTok{main=}\FunctionTok{paste}\NormalTok{(}\StringTok{\textquotesingle{}Customer\textquotesingle{}}\NormalTok{, c))}
\NormalTok{\}}
\end{Highlighting}
\end{Shaded}

\includegraphics{ratings_files/figure-pdf/unnamed-chunk-22-1.pdf}

On the other hand the model doesn't seem to have a problem capturing the
heterogeneity in the observed ratings stratified across movies, at least
for this quick spot check.

\begin{Shaded}
\begin{Highlighting}[]
\FunctionTok{par}\NormalTok{(}\AttributeTok{mfrow=}\FunctionTok{c}\NormalTok{(}\DecValTok{2}\NormalTok{, }\DecValTok{3}\NormalTok{), }\AttributeTok{mar=}\FunctionTok{c}\NormalTok{(}\DecValTok{5}\NormalTok{, }\DecValTok{5}\NormalTok{, }\DecValTok{1}\NormalTok{, }\DecValTok{1}\NormalTok{))}

\ControlFlowTok{for}\NormalTok{ (m }\ControlFlowTok{in} \FunctionTok{c}\NormalTok{(}\DecValTok{33}\NormalTok{, }\DecValTok{53}\NormalTok{, }\DecValTok{61}\NormalTok{, }\DecValTok{80}\NormalTok{, }\DecValTok{117}\NormalTok{, }\DecValTok{180}\NormalTok{)) \{}
\NormalTok{  names }\OtherTok{\textless{}{-}} \FunctionTok{sapply}\NormalTok{(}\FunctionTok{which}\NormalTok{(data}\SpecialCharTok{$}\NormalTok{movie\_idxs }\SpecialCharTok{==}\NormalTok{ m),}
                  \ControlFlowTok{function}\NormalTok{(n) }\FunctionTok{paste0}\NormalTok{(}\StringTok{\textquotesingle{}rating\_pred[\textquotesingle{}}\NormalTok{, n, }\StringTok{\textquotesingle{}]\textquotesingle{}}\NormalTok{))}
\NormalTok{  filtered\_samples }\OtherTok{\textless{}{-}}\NormalTok{ util}\SpecialCharTok{$}\FunctionTok{filter\_expectands}\NormalTok{(samples1, names)}

\NormalTok{  movie\_ratings }\OtherTok{\textless{}{-}}\NormalTok{ data}\SpecialCharTok{$}\NormalTok{ratings[data}\SpecialCharTok{$}\NormalTok{movie\_idxs }\SpecialCharTok{==}\NormalTok{ m]}
\NormalTok{  util}\SpecialCharTok{$}\FunctionTok{plot\_hist\_quantiles}\NormalTok{(filtered\_samples, }\StringTok{\textquotesingle{}rating\_pred\textquotesingle{}}\NormalTok{,}
                           \SpecialCharTok{{-}}\FloatTok{0.5}\NormalTok{, }\FloatTok{6.5}\NormalTok{, }\DecValTok{1}\NormalTok{,}
                           \AttributeTok{baseline\_values=}\NormalTok{movie\_ratings,}
                           \AttributeTok{xlab=}\StringTok{"Ratings"}\NormalTok{,}
                           \AttributeTok{main=}\FunctionTok{paste}\NormalTok{(}\StringTok{\textquotesingle{}Movie\textquotesingle{}}\NormalTok{, m))}
\NormalTok{\}}
\end{Highlighting}
\end{Shaded}

\includegraphics{ratings_files/figure-pdf/unnamed-chunk-23-1.pdf}

To better investigate the variation in retrodictive performance,
however, we need to look beyond just a few customers and movies. We can
examine the behavior of all customers and movies at the same time with
histograms of the stratified summary statistics we discussed above.

Here we see that the posterior predictive behavior of the customer-wise
empirical means are more narrowly distributed than what we see in the
observed data. At the same time the posterior predictive customer-wise
empirical variances concentrate at larger values than the observed
customer-wise empirical variances.

\begin{Shaded}
\begin{Highlighting}[]
\FunctionTok{par}\NormalTok{(}\AttributeTok{mfrow=}\FunctionTok{c}\NormalTok{(}\DecValTok{2}\NormalTok{, }\DecValTok{2}\NormalTok{), }\AttributeTok{mar=}\FunctionTok{c}\NormalTok{(}\DecValTok{5}\NormalTok{, }\DecValTok{5}\NormalTok{, }\DecValTok{1}\NormalTok{, }\DecValTok{1}\NormalTok{))}

\NormalTok{util}\SpecialCharTok{$}\FunctionTok{plot\_hist\_quantiles}\NormalTok{(samples1, }\StringTok{\textquotesingle{}mean\_rating\_customer\_pred\textquotesingle{}}\NormalTok{,}
                         \DecValTok{0}\NormalTok{, }\DecValTok{6}\NormalTok{, }\FloatTok{0.5}\NormalTok{,}
                         \AttributeTok{baseline\_values=}\NormalTok{mean\_rating\_customer,}
                         \AttributeTok{xlab=}\StringTok{"Customer{-}wise Average Ratings"}\NormalTok{)}

\NormalTok{util}\SpecialCharTok{$}\FunctionTok{plot\_hist\_quantiles}\NormalTok{(samples1, }\StringTok{\textquotesingle{}mean\_rating\_movie\_pred\textquotesingle{}}\NormalTok{,}
                         \DecValTok{0}\NormalTok{, }\DecValTok{6}\NormalTok{, }\FloatTok{0.6}\NormalTok{,}
                         \AttributeTok{baseline\_values=}\NormalTok{mean\_rating\_movie,}
                         \AttributeTok{xlab=}\StringTok{"Movie{-}wise Average Ratings"}\NormalTok{)}

\NormalTok{util}\SpecialCharTok{$}\FunctionTok{plot\_hist\_quantiles}\NormalTok{(samples1, }\StringTok{\textquotesingle{}var\_rating\_customer\_pred\textquotesingle{}}\NormalTok{,}
                         \DecValTok{0}\NormalTok{, }\DecValTok{7}\NormalTok{, }\FloatTok{0.5}\NormalTok{,}
                         \AttributeTok{baseline\_values=}\NormalTok{var\_rating\_customer,}
                         \AttributeTok{xlab=}\StringTok{"Customer{-}wise Rating Variances"}\NormalTok{)}
\end{Highlighting}
\end{Shaded}

\begin{verbatim}
Warning in check_bin_containment(bin_min, bin_max, collapsed_values,
"predictive value"): 239 predictive values (0.1%) fell above the binning.
\end{verbatim}

\begin{Shaded}
\begin{Highlighting}[]
\NormalTok{util}\SpecialCharTok{$}\FunctionTok{plot\_hist\_quantiles}\NormalTok{(samples1, }\StringTok{\textquotesingle{}var\_rating\_movie\_pred\textquotesingle{}}\NormalTok{,}
                         \DecValTok{0}\NormalTok{, }\DecValTok{7}\NormalTok{, }\FloatTok{0.5}\NormalTok{,}
                         \AttributeTok{baseline\_values=}\NormalTok{var\_rating\_movie,}
                         \AttributeTok{xlab=}\StringTok{"Movie{-}wise Rating Variances"}\NormalTok{)}
\end{Highlighting}
\end{Shaded}

\begin{verbatim}
Warning in check_bin_containment(bin_min, bin_max, collapsed_values,
"predictive value"): 1287 predictive values (0.2%) fell above the binning.
\end{verbatim}

\includegraphics{ratings_files/figure-pdf/unnamed-chunk-24-1.pdf}

Finally the collection of selected movie empirical covariances appears
to be more heavy-tailed in the observed data relative to the posterior
predictions.

\begin{Shaded}
\begin{Highlighting}[]
\FunctionTok{par}\NormalTok{(}\AttributeTok{mfrow=}\FunctionTok{c}\NormalTok{(}\DecValTok{1}\NormalTok{, }\DecValTok{1}\NormalTok{), }\AttributeTok{mar=}\FunctionTok{c}\NormalTok{(}\DecValTok{5}\NormalTok{, }\DecValTok{5}\NormalTok{, }\DecValTok{1}\NormalTok{, }\DecValTok{1}\NormalTok{))}

\NormalTok{filtered\_samples }\OtherTok{\textless{}{-}}
\NormalTok{  util}\SpecialCharTok{$}\FunctionTok{filter\_expectands}\NormalTok{(samples1,}
\NormalTok{                         covar\_rating\_movie\_filt\_names)}

\NormalTok{util}\SpecialCharTok{$}\FunctionTok{plot\_hist\_quantiles}\NormalTok{(filtered\_samples, }\StringTok{\textquotesingle{}covar\_rating\_movie\_pred\textquotesingle{}}\NormalTok{,}
                         \SpecialCharTok{{-}}\FloatTok{4.25}\NormalTok{, }\FloatTok{4.25}\NormalTok{, }\FloatTok{0.25}\NormalTok{,}
                         \AttributeTok{baseline\_values=}\NormalTok{covar\_rating\_movie\_filt,}
                         \AttributeTok{xlab=}\StringTok{"Filtered Movie{-}wise Rating Covariances"}\NormalTok{)}
\end{Highlighting}
\end{Shaded}

\begin{verbatim}
Warning in check_bin_containment(bin_min, bin_max, collapsed_values,
"predictive value"): 273 predictive values (0.0%) fell below the binning.
\end{verbatim}

\begin{verbatim}
Warning in check_bin_containment(bin_min, bin_max, collapsed_values,
"predictive value"): 2329 predictive values (0.0%) fell above the binning.
\end{verbatim}

\includegraphics{ratings_files/figure-pdf/unnamed-chunk-25-1.pdf}

\section{Independent, Heterogeneous Customer
Model}\label{independent-heterogeneous-customer-model}

All of our retrodictive checks tell a consistent story -- customers do
not rate movies the same way as each other. Fortunately it's
straightforward to model each customer's idiosyncratic behavior by
allowing them with their own set of cut points, and hence baseline
rating probabilities.

\begin{codelisting}

\caption{\texttt{model2.stan}}

\begin{Shaded}
\begin{Highlighting}[]
\KeywordTok{functions}\NormalTok{ \{}
  \CommentTok{// Log probability density function over cut point}
  \CommentTok{// induced by a Dirichlet probability density function}
  \CommentTok{// over baseline probabilities and latent logistic}
  \CommentTok{// density function.}
  \DataTypeTok{real}\NormalTok{ induced\_dirichlet\_lpdf(}\DataTypeTok{vector}\NormalTok{ c, }\DataTypeTok{vector}\NormalTok{ alpha) \{}
    \DataTypeTok{int}\NormalTok{ K = num\_elements(c) + }\DecValTok{1}\NormalTok{;}
    \DataTypeTok{vector}\NormalTok{[K {-} }\DecValTok{1}\NormalTok{] Pi = inv\_logit(c);}
    \DataTypeTok{vector}\NormalTok{[K] p = append\_row(Pi, [}\DecValTok{1}\NormalTok{]\textquotesingle{}) {-} append\_row([}\DecValTok{0}\NormalTok{]\textquotesingle{}, Pi);}

    \CommentTok{// Log Jacobian correction}
    \DataTypeTok{real}\NormalTok{ logJ = }\DecValTok{0}\NormalTok{;}
    \ControlFlowTok{for}\NormalTok{ (k }\ControlFlowTok{in} \DecValTok{1}\NormalTok{:(K {-} }\DecValTok{1}\NormalTok{)) \{}
      \ControlFlowTok{if}\NormalTok{ (c[k] \textgreater{}= }\DecValTok{0}\NormalTok{)}
\NormalTok{        logJ += {-}c[k] {-} }\DecValTok{2}\NormalTok{ * log(}\DecValTok{1}\NormalTok{ + exp({-}c[k]));}
      \ControlFlowTok{else}
\NormalTok{        logJ += +c[k] {-} }\DecValTok{2}\NormalTok{ * log(}\DecValTok{1}\NormalTok{ + exp(+c[k]));}
\NormalTok{    \}}

    \ControlFlowTok{return}\NormalTok{ dirichlet\_lpdf(p | alpha) + logJ;}
\NormalTok{  \}}
\NormalTok{\}}

\KeywordTok{data}\NormalTok{ \{}
  \DataTypeTok{int}\NormalTok{\textless{}}\KeywordTok{lower}\NormalTok{=}\DecValTok{1}\NormalTok{\textgreater{} N\_ratings;}
  \DataTypeTok{array}\NormalTok{[N\_ratings] }\DataTypeTok{int}\NormalTok{\textless{}}\KeywordTok{lower}\NormalTok{=}\DecValTok{1}\NormalTok{, }\KeywordTok{upper}\NormalTok{=}\DecValTok{5}\NormalTok{\textgreater{} ratings;}

  \DataTypeTok{int}\NormalTok{\textless{}}\KeywordTok{lower}\NormalTok{=}\DecValTok{1}\NormalTok{\textgreater{} N\_customers;}
  \DataTypeTok{array}\NormalTok{[N\_ratings] }\DataTypeTok{int}\NormalTok{\textless{}}\KeywordTok{lower}\NormalTok{=}\DecValTok{1}\NormalTok{, }\KeywordTok{upper}\NormalTok{=N\_customers\textgreater{} customer\_idxs;}

  \DataTypeTok{int}\NormalTok{\textless{}}\KeywordTok{lower}\NormalTok{=}\DecValTok{1}\NormalTok{\textgreater{} N\_movies;}
  \DataTypeTok{array}\NormalTok{[N\_ratings] }\DataTypeTok{int}\NormalTok{\textless{}}\KeywordTok{lower}\NormalTok{=}\DecValTok{1}\NormalTok{, }\KeywordTok{upper}\NormalTok{=N\_movies\textgreater{} movie\_idxs;}
\NormalTok{\}}

\KeywordTok{parameters}\NormalTok{ \{}
  \DataTypeTok{vector}\NormalTok{[N\_movies] gamma\_ncp; }\CommentTok{// Non{-}centered movie qualities}
  \DataTypeTok{real}\NormalTok{\textless{}}\KeywordTok{lower}\NormalTok{=}\DecValTok{0}\NormalTok{\textgreater{} tau\_gamma;    }\CommentTok{// Movie quality population scale}

  \DataTypeTok{array}\NormalTok{[N\_customers] }\DataTypeTok{ordered}\NormalTok{[}\DecValTok{4}\NormalTok{] cut\_points; }\CommentTok{// Customer rating cut points}
\NormalTok{\}}

\KeywordTok{transformed parameters}\NormalTok{ \{}
  \DataTypeTok{vector}\NormalTok{[N\_movies] gamma = tau\_gamma * gamma\_ncp;}
\NormalTok{\}}

\KeywordTok{model}\NormalTok{ \{}
  \CommentTok{// Prior model}
\NormalTok{  gamma\_ncp \textasciitilde{} normal(}\DecValTok{0}\NormalTok{, }\DecValTok{1}\NormalTok{);}
\NormalTok{  tau\_gamma \textasciitilde{} normal(}\DecValTok{0}\NormalTok{, }\DecValTok{5}\NormalTok{ / }\FloatTok{2.57}\NormalTok{);}

  \ControlFlowTok{for}\NormalTok{ (c }\ControlFlowTok{in} \DecValTok{1}\NormalTok{:N\_customers)}
\NormalTok{    cut\_points[c] \textasciitilde{} induced\_dirichlet(rep\_vector(}\DecValTok{1}\NormalTok{, }\DecValTok{5}\NormalTok{));}

  \CommentTok{// Observational model}
  \ControlFlowTok{for}\NormalTok{ (n }\ControlFlowTok{in} \DecValTok{1}\NormalTok{:N\_ratings) \{}
    \DataTypeTok{int}\NormalTok{ c = customer\_idxs[n];}
    \DataTypeTok{int}\NormalTok{ m = movie\_idxs[n];}
\NormalTok{    ratings[n] \textasciitilde{} ordered\_logistic(gamma[m], cut\_points[c]);}
\NormalTok{  \}}
\NormalTok{\}}

\KeywordTok{generated quantities}\NormalTok{ \{}
  \DataTypeTok{array}\NormalTok{[N\_ratings] }\DataTypeTok{int}\NormalTok{\textless{}}\KeywordTok{lower}\NormalTok{=}\DecValTok{1}\NormalTok{, }\KeywordTok{upper}\NormalTok{=}\DecValTok{5}\NormalTok{\textgreater{} rating\_pred;}

  \DataTypeTok{array}\NormalTok{[N\_customers] }\DataTypeTok{real}\NormalTok{ mean\_rating\_customer\_pred}
\NormalTok{    = rep\_array(}\DecValTok{0}\NormalTok{, N\_customers);}
  \DataTypeTok{array}\NormalTok{[N\_customers] }\DataTypeTok{real}\NormalTok{ var\_rating\_customer\_pred}
\NormalTok{    = rep\_array(}\DecValTok{0}\NormalTok{, N\_customers);}

  \DataTypeTok{array}\NormalTok{[N\_movies] }\DataTypeTok{real}\NormalTok{ mean\_rating\_movie\_pred = rep\_array(}\DecValTok{0}\NormalTok{, N\_movies);}
  \DataTypeTok{array}\NormalTok{[N\_movies] }\DataTypeTok{real}\NormalTok{ var\_rating\_movie\_pred = rep\_array(}\DecValTok{0}\NormalTok{, N\_movies);}

  \DataTypeTok{matrix}\NormalTok{[N\_movies, N\_movies] covar\_rating\_movie\_pred;}

\NormalTok{  \{}
    \DataTypeTok{array}\NormalTok{[N\_customers] }\DataTypeTok{real}\NormalTok{ C = rep\_array(}\DecValTok{0}\NormalTok{, N\_customers);}
    \DataTypeTok{array}\NormalTok{[N\_movies] }\DataTypeTok{real}\NormalTok{ M = rep\_array(}\DecValTok{0}\NormalTok{, N\_movies);}

    \ControlFlowTok{for}\NormalTok{ (n }\ControlFlowTok{in} \DecValTok{1}\NormalTok{:N\_ratings) \{}
      \DataTypeTok{real}\NormalTok{ delta = }\DecValTok{0}\NormalTok{;}
      \DataTypeTok{int}\NormalTok{ c = customer\_idxs[n];}
      \DataTypeTok{int}\NormalTok{ m = movie\_idxs[n];}

\NormalTok{      rating\_pred[n] = ordered\_logistic\_rng(gamma[m], cut\_points[c]);}

\NormalTok{      C[c] += }\DecValTok{1}\NormalTok{;}
\NormalTok{      delta = rating\_pred[n] {-} mean\_rating\_customer\_pred[c];}
\NormalTok{      mean\_rating\_customer\_pred[c] += delta / C[c];}
\NormalTok{      var\_rating\_customer\_pred[c]}
\NormalTok{        += delta * (rating\_pred[n] {-} mean\_rating\_customer\_pred[c]);}

\NormalTok{      M[m] += }\DecValTok{1}\NormalTok{;}
\NormalTok{      delta = rating\_pred[n] {-} mean\_rating\_movie\_pred[m];}
\NormalTok{      mean\_rating\_movie\_pred[m] += delta / M[m];}
\NormalTok{      var\_rating\_movie\_pred[m]}
\NormalTok{        += delta * (rating\_pred[n] {-} mean\_rating\_movie\_pred[m]);}
\NormalTok{    \}}

    \ControlFlowTok{for}\NormalTok{ (c }\ControlFlowTok{in} \DecValTok{1}\NormalTok{:N\_customers) \{}
      \ControlFlowTok{if}\NormalTok{ (C[c] \textgreater{} }\DecValTok{1}\NormalTok{)}
\NormalTok{        var\_rating\_customer\_pred[c] /= C[c] {-} }\DecValTok{1}\NormalTok{;}
      \ControlFlowTok{else}
\NormalTok{        var\_rating\_customer\_pred[c] = }\DecValTok{0}\NormalTok{;}
\NormalTok{    \}}
    \ControlFlowTok{for}\NormalTok{ (m }\ControlFlowTok{in} \DecValTok{1}\NormalTok{:N\_movies) \{}
      \ControlFlowTok{if}\NormalTok{ (M[m] \textgreater{} }\DecValTok{1}\NormalTok{)}
\NormalTok{        var\_rating\_movie\_pred[m] /= M[m] {-} }\DecValTok{1}\NormalTok{;}
      \ControlFlowTok{else}
\NormalTok{        var\_rating\_movie\_pred[m] = }\DecValTok{0}\NormalTok{;}
\NormalTok{    \}}
\NormalTok{  \}}

\NormalTok{  \{}
    \DataTypeTok{matrix}\NormalTok{[N\_movies, N\_movies] counts;}

    \ControlFlowTok{for}\NormalTok{ (m1 }\ControlFlowTok{in} \DecValTok{1}\NormalTok{:N\_movies) \{}
      \ControlFlowTok{for}\NormalTok{ (m2 }\ControlFlowTok{in} \DecValTok{1}\NormalTok{:N\_movies) \{}
\NormalTok{        counts[m1, m2] = }\DecValTok{0}\NormalTok{;}
\NormalTok{        covar\_rating\_movie\_pred[m1, m2] = }\DecValTok{0}\NormalTok{;}
\NormalTok{      \}}
\NormalTok{    \}}

    \ControlFlowTok{for}\NormalTok{ (n1 }\ControlFlowTok{in} \DecValTok{1}\NormalTok{:N\_ratings) \{}
      \ControlFlowTok{for}\NormalTok{ (n2 }\ControlFlowTok{in} \DecValTok{1}\NormalTok{:N\_ratings) \{}
        \ControlFlowTok{if}\NormalTok{ (customer\_idxs[n1] == customer\_idxs[n2]) \{}
          \DataTypeTok{int}\NormalTok{ m1 = movie\_idxs[n1];}
          \DataTypeTok{int}\NormalTok{ m2 = movie\_idxs[n2];}
          \DataTypeTok{real}\NormalTok{ y =   (ratings[n1] {-} mean\_rating\_movie\_pred[m1])}
\NormalTok{                   * (ratings[n2] {-} mean\_rating\_movie\_pred[m2]);}
\NormalTok{          covar\_rating\_movie\_pred[m1, m2] += y;}
\NormalTok{          covar\_rating\_movie\_pred[m2, m1] += y;}
\NormalTok{          counts[m1, m2] += }\DecValTok{1}\NormalTok{;}
\NormalTok{          counts[m2, m1] += }\DecValTok{1}\NormalTok{;}
\NormalTok{        \}}
\NormalTok{      \}}
\NormalTok{    \}}

    \ControlFlowTok{for}\NormalTok{ (m1 }\ControlFlowTok{in} \DecValTok{1}\NormalTok{:N\_movies) \{}
      \ControlFlowTok{for}\NormalTok{ (m2 }\ControlFlowTok{in} \DecValTok{1}\NormalTok{:N\_movies) \{}
        \ControlFlowTok{if}\NormalTok{ (counts[m1, m2] \textgreater{} }\DecValTok{1}\NormalTok{)}
\NormalTok{          covar\_rating\_movie\_pred[m1, m2] /= counts[m1, m2] {-} }\DecValTok{1}\NormalTok{;}
\NormalTok{      \}}
\NormalTok{    \}}
\NormalTok{  \}}
\NormalTok{\}}
\end{Highlighting}
\end{Shaded}

\end{codelisting}

\begin{Shaded}
\begin{Highlighting}[]
\NormalTok{fit }\OtherTok{\textless{}{-}} \FunctionTok{stan}\NormalTok{(}\AttributeTok{file=}\StringTok{"stan\_programs/model2.stan"}\NormalTok{,}
            \AttributeTok{data=}\NormalTok{data, }\AttributeTok{seed=}\DecValTok{8438330}\NormalTok{,}
            \AttributeTok{warmup=}\DecValTok{1000}\NormalTok{, }\AttributeTok{iter=}\DecValTok{2024}\NormalTok{, }\AttributeTok{refresh=}\DecValTok{0}\NormalTok{)}
\end{Highlighting}
\end{Shaded}

Frustratingly, the computation suffers from a few stray divergences.

\begin{Shaded}
\begin{Highlighting}[]
\NormalTok{diagnostics }\OtherTok{\textless{}{-}}\NormalTok{ util}\SpecialCharTok{$}\FunctionTok{extract\_hmc\_diagnostics}\NormalTok{(fit)}
\NormalTok{util}\SpecialCharTok{$}\FunctionTok{check\_all\_hmc\_diagnostics}\NormalTok{(diagnostics)}
\end{Highlighting}
\end{Shaded}

\begin{verbatim}
  Chain 4: 2 of 1024 transitions (0.2%) diverged.

  Divergent Hamiltonian transitions result from unstable numerical
trajectories.  These instabilities are often due to degenerate target
geometry, especially "pinches".  If there are only a small number of
divergences then running with adept_delta larger than 0.801 may reduce
the instabilities at the cost of more expensive Hamiltonian
transitions.
\end{verbatim}

\begin{Shaded}
\begin{Highlighting}[]
\NormalTok{samples2 }\OtherTok{\textless{}{-}}\NormalTok{ util}\SpecialCharTok{$}\FunctionTok{extract\_expectand\_vals}\NormalTok{(fit)}
\NormalTok{base\_samples }\OtherTok{\textless{}{-}}\NormalTok{ util}\SpecialCharTok{$}\FunctionTok{filter\_expectands}\NormalTok{(samples2,}
                                       \FunctionTok{c}\NormalTok{(}\StringTok{\textquotesingle{}gamma\_ncp\textquotesingle{}}\NormalTok{,}
                                         \StringTok{\textquotesingle{}tau\_gamma\textquotesingle{}}\NormalTok{,}
                                         \StringTok{\textquotesingle{}cut\_points\textquotesingle{}}\NormalTok{),}
                                       \AttributeTok{check\_arrays=}\ConstantTok{TRUE}\NormalTok{)}
\NormalTok{util}\SpecialCharTok{$}\FunctionTok{check\_all\_expectand\_diagnostics}\NormalTok{(base\_samples)}
\end{Highlighting}
\end{Shaded}

\begin{verbatim}
All expectands checked appear to be behaving well enough for reliable
Markov chain Monte Carlo estimation.
\end{verbatim}

One possibility is that the replicated cut points are messing with the
hierarchical geometry of the movie affinities. That said the movie
affinities most susceptible to degenerate behavior in a non-centered
parameterization are those with the most ratings, and those do not
exhibit any obvious geometric pathologies.

\begin{Shaded}
\begin{Highlighting}[]
\NormalTok{idxs }\OtherTok{\textless{}{-}} \FunctionTok{as.numeric}\NormalTok{(}\FunctionTok{names}\NormalTok{(}\FunctionTok{tail}\NormalTok{(}\FunctionTok{sort}\NormalTok{(}\FunctionTok{table}\NormalTok{(data}\SpecialCharTok{$}\NormalTok{movie\_idxs)), }\DecValTok{9}\NormalTok{)))}
\NormalTok{names }\OtherTok{\textless{}{-}} \FunctionTok{sapply}\NormalTok{(idxs, }\ControlFlowTok{function}\NormalTok{(m) }\FunctionTok{paste0}\NormalTok{(}\StringTok{\textquotesingle{}gamma[\textquotesingle{}}\NormalTok{, m, }\StringTok{\textquotesingle{}]\textquotesingle{}}\NormalTok{))}
\NormalTok{util}\SpecialCharTok{$}\FunctionTok{plot\_div\_pairs}\NormalTok{(names, }\StringTok{\textquotesingle{}tau\_gamma\textquotesingle{}}\NormalTok{, samples2, diagnostics)}
\end{Highlighting}
\end{Shaded}

\includegraphics{ratings_files/figure-pdf/unnamed-chunk-28-1.pdf}

At this point we could spend some time investigating for any degenerate
behavior between the cut points that could be causing problems, or even
between some cut points and some movie affinities. Given the small
number of divergences, however, let's just run again with a less
aggressive step size adaptation and cope with the increased
computational cost. We can always come back to this investigation later
if this ends up being our final model.

\begin{Shaded}
\begin{Highlighting}[]
\NormalTok{fit }\OtherTok{\textless{}{-}} \FunctionTok{stan}\NormalTok{(}\AttributeTok{file=}\StringTok{"stan\_programs/model2.stan"}\NormalTok{,}
            \AttributeTok{data=}\NormalTok{data, }\AttributeTok{seed=}\DecValTok{8438330}\NormalTok{,}
            \AttributeTok{warmup=}\DecValTok{1000}\NormalTok{, }\AttributeTok{iter=}\DecValTok{2024}\NormalTok{, }\AttributeTok{refresh=}\DecValTok{0}\NormalTok{,}
            \AttributeTok{control=}\FunctionTok{list}\NormalTok{(}\StringTok{\textquotesingle{}adapt\_delta\textquotesingle{}}\OtherTok{=}\FloatTok{0.9}\NormalTok{))}
\end{Highlighting}
\end{Shaded}

Fortunately that seems to have done the trick and now our diagnostics
are squeaky clean.

\begin{Shaded}
\begin{Highlighting}[]
\NormalTok{diagnostics }\OtherTok{\textless{}{-}}\NormalTok{ util}\SpecialCharTok{$}\FunctionTok{extract\_hmc\_diagnostics}\NormalTok{(fit)}
\NormalTok{util}\SpecialCharTok{$}\FunctionTok{check\_all\_hmc\_diagnostics}\NormalTok{(diagnostics)}
\end{Highlighting}
\end{Shaded}

\begin{verbatim}
  All Hamiltonian Monte Carlo diagnostics are consistent with reliable
Markov chain Monte Carlo.
\end{verbatim}

\begin{Shaded}
\begin{Highlighting}[]
\NormalTok{samples2 }\OtherTok{\textless{}{-}}\NormalTok{ util}\SpecialCharTok{$}\FunctionTok{extract\_expectand\_vals}\NormalTok{(fit)}
\NormalTok{base\_samples }\OtherTok{\textless{}{-}}\NormalTok{ util}\SpecialCharTok{$}\FunctionTok{filter\_expectands}\NormalTok{(samples2,}
                                       \FunctionTok{c}\NormalTok{(}\StringTok{\textquotesingle{}gamma\_ncp\textquotesingle{}}\NormalTok{,}
                                         \StringTok{\textquotesingle{}tau\_gamma\textquotesingle{}}\NormalTok{,}
                                         \StringTok{\textquotesingle{}cut\_points\textquotesingle{}}\NormalTok{),}
                                       \AttributeTok{check\_arrays=}\ConstantTok{TRUE}\NormalTok{)}
\NormalTok{util}\SpecialCharTok{$}\FunctionTok{check\_all\_expectand\_diagnostics}\NormalTok{(base\_samples)}
\end{Highlighting}
\end{Shaded}

\begin{verbatim}
All expectands checked appear to be behaving well enough for reliable
Markov chain Monte Carlo estimation.
\end{verbatim}

Has our retrodictive performance improved?

Interestingly the behavior of the aggregate ratings isn't quite as
consistent between the observed data and posterior predictions as it was
before. That said the increased retrodictive tension isn't necessarily
large enough to be a concern yet.

\begin{Shaded}
\begin{Highlighting}[]
\FunctionTok{par}\NormalTok{(}\AttributeTok{mfrow=}\FunctionTok{c}\NormalTok{(}\DecValTok{1}\NormalTok{, }\DecValTok{1}\NormalTok{), }\AttributeTok{mar=}\FunctionTok{c}\NormalTok{(}\DecValTok{5}\NormalTok{, }\DecValTok{5}\NormalTok{, }\DecValTok{1}\NormalTok{, }\DecValTok{1}\NormalTok{))}

\NormalTok{util}\SpecialCharTok{$}\FunctionTok{plot\_hist\_quantiles}\NormalTok{(samples2, }\StringTok{\textquotesingle{}rating\_pred\textquotesingle{}}\NormalTok{, }\SpecialCharTok{{-}}\FloatTok{0.5}\NormalTok{, }\FloatTok{6.5}\NormalTok{, }\DecValTok{1}\NormalTok{,}
                         \AttributeTok{baseline\_values=}\NormalTok{data}\SpecialCharTok{$}\NormalTok{ratings,}
                         \AttributeTok{xlab=}\StringTok{"All Ratings"}\NormalTok{)}
\end{Highlighting}
\end{Shaded}

\includegraphics{ratings_files/figure-pdf/unnamed-chunk-31-1.pdf}

More importantly the retrodictive performance for the customers that
we've spot checked is much better.

\begin{Shaded}
\begin{Highlighting}[]
\FunctionTok{par}\NormalTok{(}\AttributeTok{mfrow=}\FunctionTok{c}\NormalTok{(}\DecValTok{2}\NormalTok{, }\DecValTok{3}\NormalTok{), }\AttributeTok{mar=}\FunctionTok{c}\NormalTok{(}\DecValTok{5}\NormalTok{, }\DecValTok{5}\NormalTok{, }\DecValTok{1}\NormalTok{, }\DecValTok{1}\NormalTok{))}

\ControlFlowTok{for}\NormalTok{ (c }\ControlFlowTok{in} \FunctionTok{c}\NormalTok{(}\DecValTok{7}\NormalTok{, }\DecValTok{23}\NormalTok{, }\DecValTok{40}\NormalTok{, }\DecValTok{70}\NormalTok{, }\DecValTok{77}\NormalTok{, }\DecValTok{100}\NormalTok{)) \{}
\NormalTok{  names }\OtherTok{\textless{}{-}} \FunctionTok{sapply}\NormalTok{(}\FunctionTok{which}\NormalTok{(data}\SpecialCharTok{$}\NormalTok{customer\_idxs }\SpecialCharTok{==}\NormalTok{ c),}
                  \ControlFlowTok{function}\NormalTok{(n) }\FunctionTok{paste0}\NormalTok{(}\StringTok{\textquotesingle{}rating\_pred[\textquotesingle{}}\NormalTok{, n, }\StringTok{\textquotesingle{}]\textquotesingle{}}\NormalTok{))}
\NormalTok{  filtered\_samples }\OtherTok{\textless{}{-}}\NormalTok{ util}\SpecialCharTok{$}\FunctionTok{filter\_expectands}\NormalTok{(samples2, names)}

\NormalTok{  customer\_ratings }\OtherTok{\textless{}{-}}\NormalTok{ data}\SpecialCharTok{$}\NormalTok{ratings[data}\SpecialCharTok{$}\NormalTok{customer\_idxs }\SpecialCharTok{==}\NormalTok{ c]}
\NormalTok{  util}\SpecialCharTok{$}\FunctionTok{plot\_hist\_quantiles}\NormalTok{(filtered\_samples, }\StringTok{\textquotesingle{}rating\_pred\textquotesingle{}}\NormalTok{,}
                           \SpecialCharTok{{-}}\FloatTok{0.5}\NormalTok{, }\FloatTok{6.5}\NormalTok{, }\DecValTok{1}\NormalTok{,}
                           \AttributeTok{baseline\_values=}\NormalTok{customer\_ratings,}
                           \AttributeTok{xlab=}\StringTok{"Ratings"}\NormalTok{,}
                           \AttributeTok{main=}\FunctionTok{paste}\NormalTok{(}\StringTok{\textquotesingle{}Customer\textquotesingle{}}\NormalTok{, c))}
\NormalTok{\}}
\end{Highlighting}
\end{Shaded}

\includegraphics{ratings_files/figure-pdf/unnamed-chunk-32-1.pdf}

This improvement in the customer-wise behaviors hasn't come at any cost
to the movie-wise retrodictive performance, at least for this limited
spot check.

\begin{Shaded}
\begin{Highlighting}[]
\FunctionTok{par}\NormalTok{(}\AttributeTok{mfrow=}\FunctionTok{c}\NormalTok{(}\DecValTok{2}\NormalTok{, }\DecValTok{3}\NormalTok{), }\AttributeTok{mar=}\FunctionTok{c}\NormalTok{(}\DecValTok{5}\NormalTok{, }\DecValTok{5}\NormalTok{, }\DecValTok{1}\NormalTok{, }\DecValTok{1}\NormalTok{))}

\ControlFlowTok{for}\NormalTok{ (m }\ControlFlowTok{in} \FunctionTok{c}\NormalTok{(}\DecValTok{33}\NormalTok{, }\DecValTok{53}\NormalTok{, }\DecValTok{61}\NormalTok{, }\DecValTok{80}\NormalTok{, }\DecValTok{117}\NormalTok{, }\DecValTok{180}\NormalTok{)) \{}
\NormalTok{  names }\OtherTok{\textless{}{-}} \FunctionTok{sapply}\NormalTok{(}\FunctionTok{which}\NormalTok{(data}\SpecialCharTok{$}\NormalTok{movie\_idxs }\SpecialCharTok{==}\NormalTok{ m),}
                  \ControlFlowTok{function}\NormalTok{(n) }\FunctionTok{paste0}\NormalTok{(}\StringTok{\textquotesingle{}rating\_pred[\textquotesingle{}}\NormalTok{, n, }\StringTok{\textquotesingle{}]\textquotesingle{}}\NormalTok{))}
\NormalTok{  filtered\_samples }\OtherTok{\textless{}{-}}\NormalTok{ util}\SpecialCharTok{$}\FunctionTok{filter\_expectands}\NormalTok{(samples2, names)}

\NormalTok{  movie\_ratings }\OtherTok{\textless{}{-}}\NormalTok{ data}\SpecialCharTok{$}\NormalTok{ratings[data}\SpecialCharTok{$}\NormalTok{movie\_idxs }\SpecialCharTok{==}\NormalTok{ m]}
\NormalTok{  util}\SpecialCharTok{$}\FunctionTok{plot\_hist\_quantiles}\NormalTok{(filtered\_samples, }\StringTok{\textquotesingle{}rating\_pred\textquotesingle{}}\NormalTok{,}
                           \SpecialCharTok{{-}}\FloatTok{0.5}\NormalTok{, }\FloatTok{6.5}\NormalTok{, }\DecValTok{1}\NormalTok{,}
                           \AttributeTok{baseline\_values=}\NormalTok{movie\_ratings,}
                           \AttributeTok{xlab=}\StringTok{"Ratings"}\NormalTok{,}
                           \AttributeTok{main=}\FunctionTok{paste}\NormalTok{(}\StringTok{\textquotesingle{}Movie\textquotesingle{}}\NormalTok{, m))}
\NormalTok{\}}
\end{Highlighting}
\end{Shaded}

\includegraphics{ratings_files/figure-pdf/unnamed-chunk-33-1.pdf}

Do the histograms of stratified summary statistics tell a similar story?
The retrodictive tension in the customer-wise empirical means and
empirical variances has decreased, although some remains in the
empirical variances.

\begin{Shaded}
\begin{Highlighting}[]
\FunctionTok{par}\NormalTok{(}\AttributeTok{mfrow=}\FunctionTok{c}\NormalTok{(}\DecValTok{2}\NormalTok{, }\DecValTok{2}\NormalTok{), }\AttributeTok{mar=}\FunctionTok{c}\NormalTok{(}\DecValTok{5}\NormalTok{, }\DecValTok{5}\NormalTok{, }\DecValTok{1}\NormalTok{, }\DecValTok{1}\NormalTok{))}

\NormalTok{util}\SpecialCharTok{$}\FunctionTok{plot\_hist\_quantiles}\NormalTok{(samples2, }\StringTok{\textquotesingle{}mean\_rating\_customer\_pred\textquotesingle{}}\NormalTok{,}
                         \DecValTok{0}\NormalTok{, }\DecValTok{6}\NormalTok{, }\FloatTok{0.5}\NormalTok{,}
                         \AttributeTok{baseline\_values=}\NormalTok{mean\_rating\_customer,}
                         \AttributeTok{xlab=}\StringTok{"Customer{-}wise Average Ratings"}\NormalTok{)}

\NormalTok{util}\SpecialCharTok{$}\FunctionTok{plot\_hist\_quantiles}\NormalTok{(samples2, }\StringTok{\textquotesingle{}mean\_rating\_movie\_pred\textquotesingle{}}\NormalTok{,}
                         \DecValTok{0}\NormalTok{, }\DecValTok{6}\NormalTok{, }\FloatTok{0.6}\NormalTok{,}
                         \AttributeTok{baseline\_values=}\NormalTok{mean\_rating\_movie,}
                         \AttributeTok{xlab=}\StringTok{"Movie{-}wise Average Ratings"}\NormalTok{)}

\NormalTok{util}\SpecialCharTok{$}\FunctionTok{plot\_hist\_quantiles}\NormalTok{(samples2, }\StringTok{\textquotesingle{}var\_rating\_customer\_pred\textquotesingle{}}\NormalTok{,}
                         \DecValTok{0}\NormalTok{, }\DecValTok{7}\NormalTok{, }\FloatTok{0.5}\NormalTok{,}
                         \AttributeTok{baseline\_values=}\NormalTok{var\_rating\_customer,}
                         \AttributeTok{xlab=}\StringTok{"Customer{-}wise Rating Variances"}\NormalTok{)}
\end{Highlighting}
\end{Shaded}

\begin{verbatim}
Warning in check_bin_containment(bin_min, bin_max, collapsed_values,
"predictive value"): 400 predictive values (0.1%) fell above the binning.
\end{verbatim}

\begin{Shaded}
\begin{Highlighting}[]
\NormalTok{util}\SpecialCharTok{$}\FunctionTok{plot\_hist\_quantiles}\NormalTok{(samples2, }\StringTok{\textquotesingle{}var\_rating\_movie\_pred\textquotesingle{}}\NormalTok{,}
                         \DecValTok{0}\NormalTok{, }\DecValTok{7}\NormalTok{, }\FloatTok{0.5}\NormalTok{,}
                         \AttributeTok{baseline\_values=}\NormalTok{var\_rating\_movie,}
                         \AttributeTok{xlab=}\StringTok{"Movie{-}wise Rating Variances"}\NormalTok{)}
\end{Highlighting}
\end{Shaded}

\begin{verbatim}
Warning in check_bin_containment(bin_min, bin_max, collapsed_values,
"predictive value"): 2589 predictive values (0.3%) fell above the binning.
\end{verbatim}

\includegraphics{ratings_files/figure-pdf/unnamed-chunk-34-1.pdf}

Moreover the retrodictive tension in the collection of selected movie
empirical covariances remains.

\begin{Shaded}
\begin{Highlighting}[]
\FunctionTok{par}\NormalTok{(}\AttributeTok{mfrow=}\FunctionTok{c}\NormalTok{(}\DecValTok{1}\NormalTok{, }\DecValTok{1}\NormalTok{), }\AttributeTok{mar=}\FunctionTok{c}\NormalTok{(}\DecValTok{5}\NormalTok{, }\DecValTok{5}\NormalTok{, }\DecValTok{1}\NormalTok{, }\DecValTok{1}\NormalTok{))}

\NormalTok{filtered\_samples }\OtherTok{\textless{}{-}}
\NormalTok{  util}\SpecialCharTok{$}\FunctionTok{filter\_expectands}\NormalTok{(samples2,}
\NormalTok{                         covar\_rating\_movie\_filt\_names)}

\NormalTok{util}\SpecialCharTok{$}\FunctionTok{plot\_hist\_quantiles}\NormalTok{(filtered\_samples, }\StringTok{\textquotesingle{}covar\_rating\_movie\_pred\textquotesingle{}}\NormalTok{,}
                         \SpecialCharTok{{-}}\FloatTok{4.25}\NormalTok{, }\FloatTok{4.25}\NormalTok{, }\FloatTok{0.25}\NormalTok{,}
                         \AttributeTok{baseline\_values=}\NormalTok{covar\_rating\_movie\_filt,}
                         \AttributeTok{xlab=}\StringTok{"Filtered Movie{-}wise Rating Covariances"}\NormalTok{)}
\end{Highlighting}
\end{Shaded}

\begin{verbatim}
Warning in check_bin_containment(bin_min, bin_max, collapsed_values,
"predictive value"): 370 predictive values (0.0%) fell below the binning.
\end{verbatim}

\begin{verbatim}
Warning in check_bin_containment(bin_min, bin_max, collapsed_values,
"predictive value"): 248 predictive values (0.0%) fell above the binning.
\end{verbatim}

\includegraphics{ratings_files/figure-pdf/unnamed-chunk-35-1.pdf}

If we were content with these mild retrodictive tensions then we would
move on to exploring the posterior inferences themselves. For example we
could visualize the marginal posterior inferences for the interior cut
points of each customer.

\begin{Shaded}
\begin{Highlighting}[]
\FunctionTok{par}\NormalTok{(}\AttributeTok{mfrow=}\FunctionTok{c}\NormalTok{(}\DecValTok{4}\NormalTok{, }\DecValTok{1}\NormalTok{), }\AttributeTok{mar=}\FunctionTok{c}\NormalTok{(}\DecValTok{5}\NormalTok{, }\DecValTok{5}\NormalTok{, }\DecValTok{1}\NormalTok{, }\DecValTok{1}\NormalTok{))}

\ControlFlowTok{for}\NormalTok{ (k }\ControlFlowTok{in} \DecValTok{1}\SpecialCharTok{:}\DecValTok{4}\NormalTok{) \{}
\NormalTok{  names }\OtherTok{\textless{}{-}} \FunctionTok{sapply}\NormalTok{(}\DecValTok{1}\SpecialCharTok{:}\NormalTok{data}\SpecialCharTok{$}\NormalTok{N\_customers,}
                  \ControlFlowTok{function}\NormalTok{(r) }\FunctionTok{paste0}\NormalTok{(}\StringTok{\textquotesingle{}cut\_points[\textquotesingle{}}\NormalTok{, r, }\StringTok{\textquotesingle{},\textquotesingle{}}\NormalTok{, k, }\StringTok{\textquotesingle{}]\textquotesingle{}}\NormalTok{))}
\NormalTok{  util}\SpecialCharTok{$}\FunctionTok{plot\_disc\_pushforward\_quantiles}\NormalTok{(samples2, names,}
                                       \AttributeTok{xlab=}\StringTok{"Customer"}\NormalTok{,}
                                       \AttributeTok{display\_ylim=}\FunctionTok{c}\NormalTok{(}\SpecialCharTok{{-}}\DecValTok{6}\NormalTok{, }\DecValTok{6}\NormalTok{),}
                                       \AttributeTok{ylab=}\FunctionTok{paste0}\NormalTok{(}\StringTok{\textquotesingle{}cut\_point[\textquotesingle{}}\NormalTok{, k, }\StringTok{\textquotesingle{}]\textquotesingle{}}\NormalTok{))}
\NormalTok{\}}
\end{Highlighting}
\end{Shaded}

\includegraphics{ratings_files/figure-pdf/unnamed-chunk-36-1.pdf}

The interior cut points are a bit more interpretable when visualized
together, especially when comparing different customers. For example to
compare Customer 23 and Customer 70 we might overlay the marginal
posterior visualizations of the four interior cut points of each
customer in adjacent plots.

\begin{Shaded}
\begin{Highlighting}[]
\FunctionTok{par}\NormalTok{(}\AttributeTok{mfrow=}\FunctionTok{c}\NormalTok{(}\DecValTok{2}\NormalTok{, }\DecValTok{1}\NormalTok{), }\AttributeTok{mar=}\FunctionTok{c}\NormalTok{(}\DecValTok{5}\NormalTok{, }\DecValTok{5}\NormalTok{, }\DecValTok{1}\NormalTok{, }\DecValTok{1}\NormalTok{))}

\NormalTok{cols }\OtherTok{\textless{}{-}} \FunctionTok{c}\NormalTok{(util}\SpecialCharTok{$}\NormalTok{c\_mid, util}\SpecialCharTok{$}\NormalTok{c\_mid\_highlight,}
\NormalTok{          util}\SpecialCharTok{$}\NormalTok{c\_dark, util}\SpecialCharTok{$}\NormalTok{c\_dark\_highlight)}

\NormalTok{c }\OtherTok{\textless{}{-}} \DecValTok{23}

\NormalTok{k }\OtherTok{\textless{}{-}} \DecValTok{1}
\NormalTok{name }\OtherTok{\textless{}{-}}\FunctionTok{paste0}\NormalTok{(}\StringTok{\textquotesingle{}cut\_points[\textquotesingle{}}\NormalTok{, c, }\StringTok{\textquotesingle{},\textquotesingle{}}\NormalTok{, k, }\StringTok{\textquotesingle{}]\textquotesingle{}}\NormalTok{)}
\NormalTok{util}\SpecialCharTok{$}\FunctionTok{plot\_expectand\_pushforward}\NormalTok{(samples2[[name]],}
                                \DecValTok{50}\NormalTok{, }\AttributeTok{flim=}\FunctionTok{c}\NormalTok{(}\SpecialCharTok{{-}}\DecValTok{9}\NormalTok{, }\DecValTok{9}\NormalTok{), }\AttributeTok{ylim=}\FunctionTok{c}\NormalTok{(}\DecValTok{0}\NormalTok{, }\DecValTok{2}\NormalTok{),}
                                \AttributeTok{col=}\NormalTok{cols[k],}
                                \AttributeTok{display\_name=}\StringTok{\textquotesingle{}Interior Cut Points\textquotesingle{}}\NormalTok{,}
                                \AttributeTok{main=}\FunctionTok{paste}\NormalTok{(}\StringTok{\textquotesingle{}Customer\textquotesingle{}}\NormalTok{, c))}


\ControlFlowTok{for}\NormalTok{ (k }\ControlFlowTok{in} \DecValTok{2}\SpecialCharTok{:}\DecValTok{4}\NormalTok{) \{}
\NormalTok{  name }\OtherTok{\textless{}{-}}\FunctionTok{paste0}\NormalTok{(}\StringTok{\textquotesingle{}cut\_points[\textquotesingle{}}\NormalTok{, c, }\StringTok{\textquotesingle{},\textquotesingle{}}\NormalTok{, k, }\StringTok{\textquotesingle{}]\textquotesingle{}}\NormalTok{)}
\NormalTok{  util}\SpecialCharTok{$}\FunctionTok{plot\_expectand\_pushforward}\NormalTok{(samples2[[name]],}
                                  \DecValTok{50}\NormalTok{, }\AttributeTok{flim=}\FunctionTok{c}\NormalTok{(}\SpecialCharTok{{-}}\DecValTok{9}\NormalTok{, }\DecValTok{9}\NormalTok{),}
                                  \AttributeTok{col=}\NormalTok{cols[k], }\AttributeTok{border=}\StringTok{"\#BBBBBB88"}\NormalTok{,}
                                  \AttributeTok{add=}\ConstantTok{TRUE}\NormalTok{)}
\NormalTok{\}}

\FunctionTok{text}\NormalTok{(}\DecValTok{0}\NormalTok{, }\FloatTok{1.65}\NormalTok{, }\StringTok{"cut\_points[1]"}\NormalTok{, }\AttributeTok{col=}\NormalTok{util}\SpecialCharTok{$}\NormalTok{c\_mid)}
\FunctionTok{text}\NormalTok{(}\DecValTok{2}\NormalTok{, }\FloatTok{1.4}\NormalTok{, }\StringTok{"cut\_points[2]"}\NormalTok{, }\AttributeTok{col=}\NormalTok{util}\SpecialCharTok{$}\NormalTok{c\_mid\_highlight)}
\FunctionTok{text}\NormalTok{(}\DecValTok{4}\NormalTok{, }\FloatTok{1.1}\NormalTok{, }\StringTok{"cut\_points[3]"}\NormalTok{, }\AttributeTok{col=}\NormalTok{util}\SpecialCharTok{$}\NormalTok{c\_dark)}
\FunctionTok{text}\NormalTok{(}\DecValTok{6}\NormalTok{, }\FloatTok{0.5}\NormalTok{, }\StringTok{"cut\_points[4]"}\NormalTok{, }\AttributeTok{col=}\NormalTok{util}\SpecialCharTok{$}\NormalTok{c\_dark\_highlight)}

\NormalTok{c }\OtherTok{\textless{}{-}} \DecValTok{70}

\NormalTok{k }\OtherTok{\textless{}{-}} \DecValTok{1}
\NormalTok{name }\OtherTok{\textless{}{-}}\FunctionTok{paste0}\NormalTok{(}\StringTok{\textquotesingle{}cut\_points[\textquotesingle{}}\NormalTok{, c, }\StringTok{\textquotesingle{},\textquotesingle{}}\NormalTok{, k, }\StringTok{\textquotesingle{}]\textquotesingle{}}\NormalTok{)}
\NormalTok{util}\SpecialCharTok{$}\FunctionTok{plot\_expectand\_pushforward}\NormalTok{(samples2[[name]],}
                                \DecValTok{50}\NormalTok{, }\AttributeTok{flim=}\FunctionTok{c}\NormalTok{(}\SpecialCharTok{{-}}\DecValTok{9}\NormalTok{, }\DecValTok{9}\NormalTok{), }\AttributeTok{ylim=}\FunctionTok{c}\NormalTok{(}\DecValTok{0}\NormalTok{, }\DecValTok{2}\NormalTok{),}
                                \AttributeTok{col=}\NormalTok{cols[k],}
                                \AttributeTok{display\_name=}\StringTok{\textquotesingle{}Interior Cut Points\textquotesingle{}}\NormalTok{,}
                                \AttributeTok{main=}\FunctionTok{paste}\NormalTok{(}\StringTok{\textquotesingle{}Customer\textquotesingle{}}\NormalTok{, c))}
\end{Highlighting}
\end{Shaded}

\begin{verbatim}
Warning in util$plot_expectand_pushforward(samples2[[name]], 50, flim = c(-9, :
8 values (0.2%) fell below the histogram binning.
\end{verbatim}

\begin{verbatim}
Warning in util$plot_expectand_pushforward(samples2[[name]], 50, flim = c(-9, :
0 values (0.0%) fell above the histogram binning.
\end{verbatim}

\begin{Shaded}
\begin{Highlighting}[]
\ControlFlowTok{for}\NormalTok{ (k }\ControlFlowTok{in} \DecValTok{2}\SpecialCharTok{:}\DecValTok{4}\NormalTok{) \{}
\NormalTok{  name }\OtherTok{\textless{}{-}}\FunctionTok{paste0}\NormalTok{(}\StringTok{\textquotesingle{}cut\_points[\textquotesingle{}}\NormalTok{, c, }\StringTok{\textquotesingle{},\textquotesingle{}}\NormalTok{, k, }\StringTok{\textquotesingle{}]\textquotesingle{}}\NormalTok{)}
\NormalTok{  util}\SpecialCharTok{$}\FunctionTok{plot\_expectand\_pushforward}\NormalTok{(samples2[[name]],}
                                  \DecValTok{50}\NormalTok{, }\AttributeTok{flim=}\FunctionTok{c}\NormalTok{(}\SpecialCharTok{{-}}\DecValTok{9}\NormalTok{, }\DecValTok{9}\NormalTok{),}
                                  \AttributeTok{col=}\NormalTok{cols[k], }\AttributeTok{border=}\StringTok{"\#BBBBBB88"}\NormalTok{,}
                                  \AttributeTok{add=}\ConstantTok{TRUE}\NormalTok{)}
\NormalTok{\}}

\FunctionTok{text}\NormalTok{(}\SpecialCharTok{{-}}\FloatTok{5.75}\NormalTok{, }\FloatTok{0.35}\NormalTok{, }\StringTok{"cut\_points[1]"}\NormalTok{, }\AttributeTok{col=}\NormalTok{util}\SpecialCharTok{$}\NormalTok{c\_mid)}
\FunctionTok{text}\NormalTok{(}\SpecialCharTok{{-}}\DecValTok{3}\NormalTok{, }\FloatTok{0.85}\NormalTok{, }\StringTok{"cut\_points[2]"}\NormalTok{, }\AttributeTok{col=}\NormalTok{util}\SpecialCharTok{$}\NormalTok{c\_mid\_highlight)}
\FunctionTok{text}\NormalTok{(}\SpecialCharTok{{-}}\DecValTok{2}\NormalTok{, }\FloatTok{1.1}\NormalTok{, }\StringTok{"cut\_points[3]"}\NormalTok{, }\AttributeTok{col=}\NormalTok{util}\SpecialCharTok{$}\NormalTok{c\_dark)}
\FunctionTok{text}\NormalTok{(}\FloatTok{1.0}\NormalTok{, }\FloatTok{1.25}\NormalTok{, }\StringTok{"cut\_points[4]"}\NormalTok{, }\AttributeTok{col=}\NormalTok{util}\SpecialCharTok{$}\NormalTok{c\_dark\_highlight)}
\end{Highlighting}
\end{Shaded}

\includegraphics{ratings_files/figure-pdf/unnamed-chunk-37-1.pdf}

This allow us to see that Customer 23 is pretty stingy; a movie affinity
needs to be pretty large in order for the probability of a large rating
to become non-negligible. On the other hand Customer 70 is much more
generous; they are likely to give even a mediocre movie a high rating.

At the same time we can investigate the inferred affinities for each
movie.

\begin{Shaded}
\begin{Highlighting}[]
\FunctionTok{par}\NormalTok{(}\AttributeTok{mfrow=}\FunctionTok{c}\NormalTok{(}\DecValTok{2}\NormalTok{, }\DecValTok{1}\NormalTok{), }\AttributeTok{mar=}\FunctionTok{c}\NormalTok{(}\DecValTok{5}\NormalTok{, }\DecValTok{5}\NormalTok{, }\DecValTok{1}\NormalTok{, }\DecValTok{1}\NormalTok{))}

\NormalTok{util}\SpecialCharTok{$}\FunctionTok{plot\_expectand\_pushforward}\NormalTok{(samples2[[}\StringTok{\textquotesingle{}tau\_gamma\textquotesingle{}}\NormalTok{]],}
                                \DecValTok{25}\NormalTok{, }\AttributeTok{flim=}\FunctionTok{c}\NormalTok{(}\DecValTok{0}\NormalTok{, }\DecValTok{1}\NormalTok{),}
                                \AttributeTok{display\_name=}\StringTok{\textquotesingle{}tau\_gamma\textquotesingle{}}\NormalTok{)}

\NormalTok{names }\OtherTok{\textless{}{-}} \FunctionTok{sapply}\NormalTok{(}\DecValTok{1}\SpecialCharTok{:}\NormalTok{data}\SpecialCharTok{$}\NormalTok{N\_movies,}
                \ControlFlowTok{function}\NormalTok{(m) }\FunctionTok{paste0}\NormalTok{(}\StringTok{\textquotesingle{}gamma[\textquotesingle{}}\NormalTok{, m, }\StringTok{\textquotesingle{}]\textquotesingle{}}\NormalTok{))}
\NormalTok{util}\SpecialCharTok{$}\FunctionTok{plot\_disc\_pushforward\_quantiles}\NormalTok{(samples2, names,}
                                     \AttributeTok{xlab=}\StringTok{"Movie"}\NormalTok{,}
                                     \AttributeTok{ylab=}\StringTok{"Affinity"}\NormalTok{)}
\end{Highlighting}
\end{Shaded}

\includegraphics{ratings_files/figure-pdf/unnamed-chunk-38-1.pdf}

While there is a lot of uncertainty, not surprising given the limited
data, we can definitely see some trends. For example the concentration
of the \texttt{tau\_gamma} marginal posterior distribution away from
zero indicates substantial variation in the movie affinities. Moreover
despite the uncertainties we can clearly differentiate the best movies
from the worst movies. We'll do even more with these movie affinity
inferences in the next section.

\section{Hierarchical Customer Model}\label{hierarchical-customer-model}

At this point we need to address a stark asymmetry in the last model.
Although we're accounting for heterogeneity in both customer and movie
behaviors, we're modeling only the latter hierarchically. We don't have
any domain expertise that obstructs the exchangeability of the customers
so why don't we model the individual interior cut points hierarchically
as well? All we need is an appropriate multivariate population model.

Conveniently the induced Dirichlet prior naturally composes with the
hyper Dirichlet population model that I discussed in my
\href{https://betanalpha.github.io/assets/chapters_html/die_fairness.html}{die
fairness case study}. This makes for a natural interior cut point
population model that pools each customer's baseline ratings towards a
common behavior.

Note that this is not the most general hierarchical model that we might
consider. This model assumes that the heterogeneity in the interior cut
points is independent of the heterogeneity in the movie affinities; more
generally those heterogeneities could be coupled together. That said I
think that it is pretty reasonable to assume that how each customer
translates their preferences into a movie rating is independent of those
particular preferences.

\begin{codelisting}

\caption{\texttt{model3.stan}}

\begin{Shaded}
\begin{Highlighting}[]
\KeywordTok{functions}\NormalTok{ \{}
  \CommentTok{// Log probability density function over cut point}
  \CommentTok{// induced by a Dirichlet probability density function}
  \CommentTok{// over baseline probabilities and latent logistic}
  \CommentTok{// density function.}
  \DataTypeTok{real}\NormalTok{ induced\_dirichlet\_lpdf(}\DataTypeTok{vector}\NormalTok{ c, }\DataTypeTok{vector}\NormalTok{ alpha) \{}
    \DataTypeTok{int}\NormalTok{ K = num\_elements(c) + }\DecValTok{1}\NormalTok{;}
    \DataTypeTok{vector}\NormalTok{[K {-} }\DecValTok{1}\NormalTok{] Pi = inv\_logit(c);}
    \DataTypeTok{vector}\NormalTok{[K] p = append\_row(Pi, [}\DecValTok{1}\NormalTok{]\textquotesingle{}) {-} append\_row([}\DecValTok{0}\NormalTok{]\textquotesingle{}, Pi);}

    \CommentTok{// Log Jacobian correction}
    \DataTypeTok{real}\NormalTok{ logJ = }\DecValTok{0}\NormalTok{;}
    \ControlFlowTok{for}\NormalTok{ (k }\ControlFlowTok{in} \DecValTok{1}\NormalTok{:(K {-} }\DecValTok{1}\NormalTok{)) \{}
      \ControlFlowTok{if}\NormalTok{ (c[k] \textgreater{}= }\DecValTok{0}\NormalTok{)}
\NormalTok{        logJ += {-}c[k] {-} }\DecValTok{2}\NormalTok{ * log(}\DecValTok{1}\NormalTok{ + exp({-}c[k]));}
      \ControlFlowTok{else}
\NormalTok{        logJ += +c[k] {-} }\DecValTok{2}\NormalTok{ * log(}\DecValTok{1}\NormalTok{ + exp(+c[k]));}
\NormalTok{    \}}

    \ControlFlowTok{return}\NormalTok{ dirichlet\_lpdf(p | alpha) + logJ;}
\NormalTok{  \}}
\NormalTok{\}}

\KeywordTok{data}\NormalTok{ \{}
  \DataTypeTok{int}\NormalTok{\textless{}}\KeywordTok{lower}\NormalTok{=}\DecValTok{1}\NormalTok{\textgreater{} N\_ratings;}
  \DataTypeTok{array}\NormalTok{[N\_ratings] }\DataTypeTok{int}\NormalTok{\textless{}}\KeywordTok{lower}\NormalTok{=}\DecValTok{1}\NormalTok{, }\KeywordTok{upper}\NormalTok{=}\DecValTok{5}\NormalTok{\textgreater{} ratings;}

  \DataTypeTok{int}\NormalTok{\textless{}}\KeywordTok{lower}\NormalTok{=}\DecValTok{1}\NormalTok{\textgreater{} N\_customers;}
  \DataTypeTok{array}\NormalTok{[N\_ratings] }\DataTypeTok{int}\NormalTok{\textless{}}\KeywordTok{lower}\NormalTok{=}\DecValTok{1}\NormalTok{, }\KeywordTok{upper}\NormalTok{=N\_customers\textgreater{} customer\_idxs;}

  \DataTypeTok{int}\NormalTok{\textless{}}\KeywordTok{lower}\NormalTok{=}\DecValTok{1}\NormalTok{\textgreater{} N\_movies;}
  \DataTypeTok{array}\NormalTok{[N\_ratings] }\DataTypeTok{int}\NormalTok{\textless{}}\KeywordTok{lower}\NormalTok{=}\DecValTok{1}\NormalTok{, }\KeywordTok{upper}\NormalTok{=N\_movies\textgreater{} movie\_idxs;}
\NormalTok{\}}

\KeywordTok{parameters}\NormalTok{ \{}
  \DataTypeTok{vector}\NormalTok{[N\_movies] gamma\_ncp; }\CommentTok{// Non{-}centered movie affinities}
  \DataTypeTok{real}\NormalTok{\textless{}}\KeywordTok{lower}\NormalTok{=}\DecValTok{0}\NormalTok{\textgreater{} tau\_gamma;    }\CommentTok{// Movie affinity population scale}

  \DataTypeTok{array}\NormalTok{[N\_customers] }\DataTypeTok{ordered}\NormalTok{[}\DecValTok{4}\NormalTok{] cut\_points; }\CommentTok{// Customer rating cut points}

  \DataTypeTok{simplex}\NormalTok{[}\DecValTok{5}\NormalTok{] mu\_q;     }\CommentTok{// Rating simplex population location}
  \DataTypeTok{real}\NormalTok{\textless{}}\KeywordTok{lower}\NormalTok{=}\DecValTok{0}\NormalTok{\textgreater{} tau\_q; }\CommentTok{// Rating simplex population scale}
\NormalTok{\}}

\KeywordTok{transformed parameters}\NormalTok{ \{}
  \CommentTok{// Centered movie affinities}
  \DataTypeTok{vector}\NormalTok{[N\_movies] gamma = tau\_gamma * gamma\_ncp;}
\NormalTok{\}}

\KeywordTok{model}\NormalTok{ \{}
  \DataTypeTok{vector}\NormalTok{[}\DecValTok{5}\NormalTok{] ones = rep\_vector(}\DecValTok{1}\NormalTok{, }\DecValTok{5}\NormalTok{);}
  \DataTypeTok{vector}\NormalTok{[}\DecValTok{5}\NormalTok{] alpha = mu\_q / tau\_q + ones;}

  \CommentTok{// Prior model}
\NormalTok{  gamma\_ncp \textasciitilde{} normal(}\DecValTok{0}\NormalTok{, }\DecValTok{1}\NormalTok{);}
\NormalTok{  tau\_gamma \textasciitilde{} normal(}\DecValTok{0}\NormalTok{, }\DecValTok{5}\NormalTok{ / }\FloatTok{2.57}\NormalTok{);}

\NormalTok{  mu\_q \textasciitilde{} dirichlet(}\DecValTok{5}\NormalTok{ * ones);}
\NormalTok{  tau\_q \textasciitilde{} normal(}\DecValTok{0}\NormalTok{, }\DecValTok{1}\NormalTok{);}

  \ControlFlowTok{for}\NormalTok{ (c }\ControlFlowTok{in} \DecValTok{1}\NormalTok{:N\_customers)}
\NormalTok{    cut\_points[c] \textasciitilde{} induced\_dirichlet(alpha);}

  \CommentTok{// Observational model}
  \ControlFlowTok{for}\NormalTok{ (n }\ControlFlowTok{in} \DecValTok{1}\NormalTok{:N\_ratings) \{}
    \DataTypeTok{int}\NormalTok{ c = customer\_idxs[n];}
    \DataTypeTok{int}\NormalTok{ m = movie\_idxs[n];}
\NormalTok{    ratings[n] \textasciitilde{} ordered\_logistic(gamma[m], cut\_points[c]);}
\NormalTok{  \}}
\NormalTok{\}}

\KeywordTok{generated quantities}\NormalTok{ \{}
  \DataTypeTok{array}\NormalTok{[N\_ratings] }\DataTypeTok{int}\NormalTok{\textless{}}\KeywordTok{lower}\NormalTok{=}\DecValTok{1}\NormalTok{, }\KeywordTok{upper}\NormalTok{=}\DecValTok{5}\NormalTok{\textgreater{} rating\_pred;}

  \DataTypeTok{array}\NormalTok{[N\_customers] }\DataTypeTok{real}\NormalTok{ mean\_rating\_customer\_pred}
\NormalTok{    = rep\_array(}\DecValTok{0}\NormalTok{, N\_customers);}
  \DataTypeTok{array}\NormalTok{[N\_customers] }\DataTypeTok{real}\NormalTok{ var\_rating\_customer\_pred}
\NormalTok{    = rep\_array(}\DecValTok{0}\NormalTok{, N\_customers);}

  \DataTypeTok{array}\NormalTok{[N\_movies] }\DataTypeTok{real}\NormalTok{ mean\_rating\_movie\_pred = rep\_array(}\DecValTok{0}\NormalTok{, N\_movies);}
  \DataTypeTok{array}\NormalTok{[N\_movies] }\DataTypeTok{real}\NormalTok{ var\_rating\_movie\_pred = rep\_array(}\DecValTok{0}\NormalTok{, N\_movies);}

  \DataTypeTok{matrix}\NormalTok{[N\_movies, N\_movies] covar\_rating\_movie\_pred;}

\NormalTok{  \{}
    \DataTypeTok{array}\NormalTok{[N\_customers] }\DataTypeTok{real}\NormalTok{ C = rep\_array(}\DecValTok{0}\NormalTok{, N\_customers);}
    \DataTypeTok{array}\NormalTok{[N\_movies] }\DataTypeTok{real}\NormalTok{ M = rep\_array(}\DecValTok{0}\NormalTok{, N\_movies);}

    \ControlFlowTok{for}\NormalTok{ (n }\ControlFlowTok{in} \DecValTok{1}\NormalTok{:N\_ratings) \{}
      \DataTypeTok{real}\NormalTok{ delta = }\DecValTok{0}\NormalTok{;}
      \DataTypeTok{int}\NormalTok{ c = customer\_idxs[n];}
      \DataTypeTok{int}\NormalTok{ m = movie\_idxs[n];}

\NormalTok{      rating\_pred[n] = ordered\_logistic\_rng(gamma[m], cut\_points[c]);}

\NormalTok{      C[c] += }\DecValTok{1}\NormalTok{;}
\NormalTok{      delta = rating\_pred[n] {-} mean\_rating\_customer\_pred[c];}
\NormalTok{      mean\_rating\_customer\_pred[c] += delta / C[c];}
\NormalTok{      var\_rating\_customer\_pred[c]}
\NormalTok{        += delta * (rating\_pred[n] {-} mean\_rating\_customer\_pred[c]);}

\NormalTok{      M[m] += }\DecValTok{1}\NormalTok{;}
\NormalTok{      delta = rating\_pred[n] {-} mean\_rating\_movie\_pred[m];}
\NormalTok{      mean\_rating\_movie\_pred[m] += delta / M[m];}
\NormalTok{      var\_rating\_movie\_pred[m]}
\NormalTok{        += delta * (rating\_pred[n] {-} mean\_rating\_movie\_pred[m]);}
\NormalTok{    \}}

    \ControlFlowTok{for}\NormalTok{ (c }\ControlFlowTok{in} \DecValTok{1}\NormalTok{:N\_customers) \{}
      \ControlFlowTok{if}\NormalTok{ (C[c] \textgreater{} }\DecValTok{1}\NormalTok{)}
\NormalTok{        var\_rating\_customer\_pred[c] /= C[c] {-} }\DecValTok{1}\NormalTok{;}
      \ControlFlowTok{else}
\NormalTok{        var\_rating\_customer\_pred[c] = }\DecValTok{0}\NormalTok{;}
\NormalTok{    \}}
    \ControlFlowTok{for}\NormalTok{ (m }\ControlFlowTok{in} \DecValTok{1}\NormalTok{:N\_movies) \{}
      \ControlFlowTok{if}\NormalTok{ (M[m] \textgreater{} }\DecValTok{1}\NormalTok{)}
\NormalTok{        var\_rating\_movie\_pred[m] /= M[m] {-} }\DecValTok{1}\NormalTok{;}
      \ControlFlowTok{else}
\NormalTok{        var\_rating\_movie\_pred[m] = }\DecValTok{0}\NormalTok{;}
\NormalTok{    \}}
\NormalTok{  \}}

\NormalTok{  \{}
    \DataTypeTok{matrix}\NormalTok{[N\_movies, N\_movies] counts;}

    \ControlFlowTok{for}\NormalTok{ (m1 }\ControlFlowTok{in} \DecValTok{1}\NormalTok{:N\_movies) \{}
      \ControlFlowTok{for}\NormalTok{ (m2 }\ControlFlowTok{in} \DecValTok{1}\NormalTok{:N\_movies) \{}
\NormalTok{        counts[m1, m2] = }\DecValTok{0}\NormalTok{;}
\NormalTok{        covar\_rating\_movie\_pred[m1, m2] = }\DecValTok{0}\NormalTok{;}
\NormalTok{      \}}
\NormalTok{    \}}

    \ControlFlowTok{for}\NormalTok{ (n1 }\ControlFlowTok{in} \DecValTok{1}\NormalTok{:N\_ratings) \{}
      \ControlFlowTok{for}\NormalTok{ (n2 }\ControlFlowTok{in} \DecValTok{1}\NormalTok{:N\_ratings) \{}
        \ControlFlowTok{if}\NormalTok{ (customer\_idxs[n1] == customer\_idxs[n2]) \{}
          \DataTypeTok{int}\NormalTok{ m1 = movie\_idxs[n1];}
          \DataTypeTok{int}\NormalTok{ m2 = movie\_idxs[n2];}
          \DataTypeTok{real}\NormalTok{ y =   (ratings[n1] {-} mean\_rating\_movie\_pred[m1])}
\NormalTok{                   * (ratings[n2] {-} mean\_rating\_movie\_pred[m2]);}
\NormalTok{          covar\_rating\_movie\_pred[m1, m2] += y;}
\NormalTok{          covar\_rating\_movie\_pred[m2, m1] += y;}
\NormalTok{          counts[m1, m2] += }\DecValTok{1}\NormalTok{;}
\NormalTok{          counts[m2, m1] += }\DecValTok{1}\NormalTok{;}
\NormalTok{        \}}
\NormalTok{      \}}
\NormalTok{    \}}

    \ControlFlowTok{for}\NormalTok{ (m1 }\ControlFlowTok{in} \DecValTok{1}\NormalTok{:N\_movies) \{}
      \ControlFlowTok{for}\NormalTok{ (m2 }\ControlFlowTok{in} \DecValTok{1}\NormalTok{:N\_movies) \{}
        \ControlFlowTok{if}\NormalTok{ (counts[m1, m2] \textgreater{} }\DecValTok{1}\NormalTok{)}
\NormalTok{          covar\_rating\_movie\_pred[m1, m2] /= counts[m1, m2] {-} }\DecValTok{1}\NormalTok{;}
\NormalTok{      \}}
\NormalTok{    \}}
\NormalTok{  \}}
\NormalTok{\}}
\end{Highlighting}
\end{Shaded}

\end{codelisting}

\begin{Shaded}
\begin{Highlighting}[]
\NormalTok{fit }\OtherTok{\textless{}{-}} \FunctionTok{stan}\NormalTok{(}\AttributeTok{file=}\StringTok{"stan\_programs/model3.stan"}\NormalTok{,}
            \AttributeTok{data=}\NormalTok{data, }\AttributeTok{seed=}\DecValTok{8438338}\NormalTok{,}
            \AttributeTok{warmup=}\DecValTok{1000}\NormalTok{, }\AttributeTok{iter=}\DecValTok{2024}\NormalTok{, }\AttributeTok{refresh=}\DecValTok{0}\NormalTok{)}
\end{Highlighting}
\end{Shaded}

The hierarchical model over the interior cut points has already proved
useful -- the mild computational issues that we had considered in the
last model fit have vanished.

\begin{Shaded}
\begin{Highlighting}[]
\NormalTok{diagnostics }\OtherTok{\textless{}{-}}\NormalTok{ util}\SpecialCharTok{$}\FunctionTok{extract\_hmc\_diagnostics}\NormalTok{(fit)}
\NormalTok{util}\SpecialCharTok{$}\FunctionTok{check\_all\_hmc\_diagnostics}\NormalTok{(diagnostics)}
\end{Highlighting}
\end{Shaded}

\begin{verbatim}
  All Hamiltonian Monte Carlo diagnostics are consistent with reliable
Markov chain Monte Carlo.
\end{verbatim}

\begin{Shaded}
\begin{Highlighting}[]
\NormalTok{samples3 }\OtherTok{\textless{}{-}}\NormalTok{ util}\SpecialCharTok{$}\FunctionTok{extract\_expectand\_vals}\NormalTok{(fit)}
\NormalTok{base\_samples }\OtherTok{\textless{}{-}}\NormalTok{ util}\SpecialCharTok{$}\FunctionTok{filter\_expectands}\NormalTok{(samples3,}
                                       \FunctionTok{c}\NormalTok{(}\StringTok{\textquotesingle{}gamma\_ncp\textquotesingle{}}\NormalTok{,}
                                         \StringTok{\textquotesingle{}tau\_gamma\textquotesingle{}}\NormalTok{,}
                                         \StringTok{\textquotesingle{}cut\_points\textquotesingle{}}\NormalTok{,}
                                         \StringTok{\textquotesingle{}mu\_q\textquotesingle{}}\NormalTok{, }\StringTok{\textquotesingle{}tau\_q\textquotesingle{}}\NormalTok{),}
                                       \AttributeTok{check\_arrays=}\ConstantTok{TRUE}\NormalTok{)}
\NormalTok{util}\SpecialCharTok{$}\FunctionTok{check\_all\_expectand\_diagnostics}\NormalTok{(base\_samples)}
\end{Highlighting}
\end{Shaded}

\begin{verbatim}
All expectands checked appear to be behaving well enough for reliable
Markov chain Monte Carlo estimation.
\end{verbatim}

Overall the retrodictive performance is a little bit better than in the
previous model. In particular the agreements in the aggregate ratings
histogram and stratified empirical covariances histogram have improved
slightly. On the other hand the retrodictive tension in the stratified
empirical variances stubbornly persists. We shouldn't expect too much
performance, however, given that the hierarchical structure doesn't add
any flexibility to the customer behaviors beyond what we had already
included.

\begin{Shaded}
\begin{Highlighting}[]
\FunctionTok{par}\NormalTok{(}\AttributeTok{mfrow=}\FunctionTok{c}\NormalTok{(}\DecValTok{1}\NormalTok{, }\DecValTok{1}\NormalTok{), }\AttributeTok{mar=}\FunctionTok{c}\NormalTok{(}\DecValTok{5}\NormalTok{, }\DecValTok{5}\NormalTok{, }\DecValTok{1}\NormalTok{, }\DecValTok{1}\NormalTok{))}

\NormalTok{util}\SpecialCharTok{$}\FunctionTok{plot\_hist\_quantiles}\NormalTok{(samples3, }\StringTok{\textquotesingle{}rating\_pred\textquotesingle{}}\NormalTok{, }\SpecialCharTok{{-}}\FloatTok{0.5}\NormalTok{, }\FloatTok{6.5}\NormalTok{, }\DecValTok{1}\NormalTok{,}
                         \AttributeTok{baseline\_values=}\NormalTok{data}\SpecialCharTok{$}\NormalTok{ratings,}
                         \AttributeTok{xlab=}\StringTok{"All Ratings"}\NormalTok{)}
\end{Highlighting}
\end{Shaded}

\includegraphics{ratings_files/figure-pdf/unnamed-chunk-41-1.pdf}

\begin{Shaded}
\begin{Highlighting}[]
\FunctionTok{par}\NormalTok{(}\AttributeTok{mfrow=}\FunctionTok{c}\NormalTok{(}\DecValTok{2}\NormalTok{, }\DecValTok{3}\NormalTok{), }\AttributeTok{mar=}\FunctionTok{c}\NormalTok{(}\DecValTok{5}\NormalTok{, }\DecValTok{5}\NormalTok{, }\DecValTok{1}\NormalTok{, }\DecValTok{1}\NormalTok{))}

\ControlFlowTok{for}\NormalTok{ (c }\ControlFlowTok{in} \FunctionTok{c}\NormalTok{(}\DecValTok{7}\NormalTok{, }\DecValTok{23}\NormalTok{, }\DecValTok{40}\NormalTok{, }\DecValTok{70}\NormalTok{, }\DecValTok{77}\NormalTok{, }\DecValTok{100}\NormalTok{)) \{}
\NormalTok{  names }\OtherTok{\textless{}{-}} \FunctionTok{sapply}\NormalTok{(}\FunctionTok{which}\NormalTok{(data}\SpecialCharTok{$}\NormalTok{customer\_idxs }\SpecialCharTok{==}\NormalTok{ c),}
                  \ControlFlowTok{function}\NormalTok{(n) }\FunctionTok{paste0}\NormalTok{(}\StringTok{\textquotesingle{}rating\_pred[\textquotesingle{}}\NormalTok{, n, }\StringTok{\textquotesingle{}]\textquotesingle{}}\NormalTok{))}
\NormalTok{  filtered\_samples }\OtherTok{\textless{}{-}}\NormalTok{ util}\SpecialCharTok{$}\FunctionTok{filter\_expectands}\NormalTok{(samples3, names)}

\NormalTok{  customer\_ratings }\OtherTok{\textless{}{-}}\NormalTok{ data}\SpecialCharTok{$}\NormalTok{ratings[data}\SpecialCharTok{$}\NormalTok{customer\_idxs }\SpecialCharTok{==}\NormalTok{ c]}
\NormalTok{  util}\SpecialCharTok{$}\FunctionTok{plot\_hist\_quantiles}\NormalTok{(filtered\_samples, }\StringTok{\textquotesingle{}rating\_pred\textquotesingle{}}\NormalTok{,}
                           \SpecialCharTok{{-}}\FloatTok{0.5}\NormalTok{, }\FloatTok{6.5}\NormalTok{, }\DecValTok{1}\NormalTok{,}
                           \AttributeTok{baseline\_values=}\NormalTok{customer\_ratings,}
                           \AttributeTok{xlab=}\StringTok{"Ratings"}\NormalTok{,}
                           \AttributeTok{main=}\FunctionTok{paste}\NormalTok{(}\StringTok{\textquotesingle{}Customer\textquotesingle{}}\NormalTok{, c))}
\NormalTok{\}}
\end{Highlighting}
\end{Shaded}

\includegraphics{ratings_files/figure-pdf/unnamed-chunk-42-1.pdf}

\begin{Shaded}
\begin{Highlighting}[]
\FunctionTok{par}\NormalTok{(}\AttributeTok{mfrow=}\FunctionTok{c}\NormalTok{(}\DecValTok{2}\NormalTok{, }\DecValTok{3}\NormalTok{), }\AttributeTok{mar=}\FunctionTok{c}\NormalTok{(}\DecValTok{5}\NormalTok{, }\DecValTok{5}\NormalTok{, }\DecValTok{1}\NormalTok{, }\DecValTok{1}\NormalTok{))}

\ControlFlowTok{for}\NormalTok{ (m }\ControlFlowTok{in} \FunctionTok{c}\NormalTok{(}\DecValTok{33}\NormalTok{, }\DecValTok{53}\NormalTok{, }\DecValTok{61}\NormalTok{, }\DecValTok{80}\NormalTok{, }\DecValTok{117}\NormalTok{, }\DecValTok{180}\NormalTok{)) \{}
\NormalTok{  names }\OtherTok{\textless{}{-}} \FunctionTok{sapply}\NormalTok{(}\FunctionTok{which}\NormalTok{(data}\SpecialCharTok{$}\NormalTok{movie\_idxs }\SpecialCharTok{==}\NormalTok{ m),}
                  \ControlFlowTok{function}\NormalTok{(n) }\FunctionTok{paste0}\NormalTok{(}\StringTok{\textquotesingle{}rating\_pred[\textquotesingle{}}\NormalTok{, n, }\StringTok{\textquotesingle{}]\textquotesingle{}}\NormalTok{))}
\NormalTok{  filtered\_samples }\OtherTok{\textless{}{-}}\NormalTok{ util}\SpecialCharTok{$}\FunctionTok{filter\_expectands}\NormalTok{(samples3, names)}

\NormalTok{  movie\_ratings }\OtherTok{\textless{}{-}}\NormalTok{ data}\SpecialCharTok{$}\NormalTok{ratings[data}\SpecialCharTok{$}\NormalTok{movie\_idxs }\SpecialCharTok{==}\NormalTok{ m]}
\NormalTok{  util}\SpecialCharTok{$}\FunctionTok{plot\_hist\_quantiles}\NormalTok{(filtered\_samples, }\StringTok{\textquotesingle{}rating\_pred\textquotesingle{}}\NormalTok{,}
                           \SpecialCharTok{{-}}\FloatTok{0.5}\NormalTok{, }\FloatTok{6.5}\NormalTok{, }\DecValTok{1}\NormalTok{,}
                           \AttributeTok{baseline\_values=}\NormalTok{movie\_ratings,}
                           \AttributeTok{xlab=}\StringTok{"Ratings"}\NormalTok{,}
                           \AttributeTok{main=}\FunctionTok{paste}\NormalTok{(}\StringTok{\textquotesingle{}Movie\textquotesingle{}}\NormalTok{, m))}
\NormalTok{\}}
\end{Highlighting}
\end{Shaded}

\includegraphics{ratings_files/figure-pdf/unnamed-chunk-43-1.pdf}

\begin{Shaded}
\begin{Highlighting}[]
\FunctionTok{par}\NormalTok{(}\AttributeTok{mfrow=}\FunctionTok{c}\NormalTok{(}\DecValTok{2}\NormalTok{, }\DecValTok{2}\NormalTok{), }\AttributeTok{mar=}\FunctionTok{c}\NormalTok{(}\DecValTok{5}\NormalTok{, }\DecValTok{5}\NormalTok{, }\DecValTok{1}\NormalTok{, }\DecValTok{1}\NormalTok{))}

\NormalTok{util}\SpecialCharTok{$}\FunctionTok{plot\_hist\_quantiles}\NormalTok{(samples3, }\StringTok{\textquotesingle{}mean\_rating\_customer\_pred\textquotesingle{}}\NormalTok{,}
                         \DecValTok{0}\NormalTok{, }\DecValTok{6}\NormalTok{, }\FloatTok{0.5}\NormalTok{,}
                         \AttributeTok{baseline\_values=}\NormalTok{mean\_rating\_customer,}
                         \AttributeTok{xlab=}\StringTok{"Customer{-}wise Average Ratings"}\NormalTok{)}

\NormalTok{util}\SpecialCharTok{$}\FunctionTok{plot\_hist\_quantiles}\NormalTok{(samples3, }\StringTok{\textquotesingle{}mean\_rating\_movie\_pred\textquotesingle{}}\NormalTok{,}
                         \DecValTok{0}\NormalTok{, }\DecValTok{6}\NormalTok{, }\FloatTok{0.6}\NormalTok{,}
                         \AttributeTok{baseline\_values=}\NormalTok{mean\_rating\_movie,}
                         \AttributeTok{xlab=}\StringTok{"Movie{-}wise Average Ratings"}\NormalTok{)}

\NormalTok{util}\SpecialCharTok{$}\FunctionTok{plot\_hist\_quantiles}\NormalTok{(samples3, }\StringTok{\textquotesingle{}var\_rating\_customer\_pred\textquotesingle{}}\NormalTok{,}
                         \DecValTok{0}\NormalTok{, }\DecValTok{7}\NormalTok{, }\FloatTok{0.5}\NormalTok{,}
                         \AttributeTok{baseline\_values=}\NormalTok{var\_rating\_customer,}
                         \AttributeTok{xlab=}\StringTok{"Customer{-}wise Rating Variances"}\NormalTok{)}
\end{Highlighting}
\end{Shaded}

\begin{verbatim}
Warning in check_bin_containment(bin_min, bin_max, collapsed_values,
"predictive value"): 227 predictive values (0.1%) fell above the binning.
\end{verbatim}

\begin{Shaded}
\begin{Highlighting}[]
\NormalTok{util}\SpecialCharTok{$}\FunctionTok{plot\_hist\_quantiles}\NormalTok{(samples3, }\StringTok{\textquotesingle{}var\_rating\_movie\_pred\textquotesingle{}}\NormalTok{,}
                         \DecValTok{0}\NormalTok{, }\DecValTok{7}\NormalTok{, }\FloatTok{0.5}\NormalTok{,}
                         \AttributeTok{baseline\_values=}\NormalTok{var\_rating\_movie,}
                         \AttributeTok{xlab=}\StringTok{"Movie{-}wise Rating Variances"}\NormalTok{)}
\end{Highlighting}
\end{Shaded}

\begin{verbatim}
Warning in check_bin_containment(bin_min, bin_max, collapsed_values,
"predictive value"): 2264 predictive values (0.3%) fell above the binning.
\end{verbatim}

\includegraphics{ratings_files/figure-pdf/unnamed-chunk-44-1.pdf}

\begin{Shaded}
\begin{Highlighting}[]
\FunctionTok{par}\NormalTok{(}\AttributeTok{mfrow=}\FunctionTok{c}\NormalTok{(}\DecValTok{1}\NormalTok{, }\DecValTok{1}\NormalTok{), }\AttributeTok{mar=}\FunctionTok{c}\NormalTok{(}\DecValTok{5}\NormalTok{, }\DecValTok{5}\NormalTok{, }\DecValTok{1}\NormalTok{, }\DecValTok{1}\NormalTok{))}

\NormalTok{filtered\_samples }\OtherTok{\textless{}{-}}
\NormalTok{  util}\SpecialCharTok{$}\FunctionTok{filter\_expectands}\NormalTok{(samples3,}
\NormalTok{                         covar\_rating\_movie\_filt\_names)}

\NormalTok{util}\SpecialCharTok{$}\FunctionTok{plot\_hist\_quantiles}\NormalTok{(filtered\_samples, }\StringTok{\textquotesingle{}covar\_rating\_movie\_pred\textquotesingle{}}\NormalTok{,}
                         \SpecialCharTok{{-}}\FloatTok{4.25}\NormalTok{, }\FloatTok{4.25}\NormalTok{, }\FloatTok{0.25}\NormalTok{,}
                         \AttributeTok{baseline\_values=}\NormalTok{covar\_rating\_movie\_filt,}
                         \AttributeTok{xlab=}\StringTok{"Filtered Movie{-}wise Rating Covariances"}\NormalTok{)}
\end{Highlighting}
\end{Shaded}

\begin{verbatim}
Warning in check_bin_containment(bin_min, bin_max, collapsed_values,
"predictive value"): 355 predictive values (0.0%) fell below the binning.
\end{verbatim}

\begin{verbatim}
Warning in check_bin_containment(bin_min, bin_max, collapsed_values,
"predictive value"): 432 predictive values (0.0%) fell above the binning.
\end{verbatim}

\includegraphics{ratings_files/figure-pdf/unnamed-chunk-45-1.pdf}

Assuming that we are satisfied with this retrodictive performance there
many ways that we can explore and utilize the associated inferences.

For example with the new hierarchical structure we can investigate what
we learned about not only each individual customer but also the
population of customers. The marginal posterior distribution for
\texttt{tau\_q} suggests a small but non-negligible heterogeneity in the
interior cut points. Moreover the inferences for the baseline
probabilities \texttt{mu\_q} suggest that customers are relatively
optimistic, with four star ratings being more probable than tree star
ratings for a neutral movie with zero affinity.

\begin{Shaded}
\begin{Highlighting}[]
\FunctionTok{par}\NormalTok{(}\AttributeTok{mfrow=}\FunctionTok{c}\NormalTok{(}\DecValTok{2}\NormalTok{, }\DecValTok{1}\NormalTok{), }\AttributeTok{mar=}\FunctionTok{c}\NormalTok{(}\DecValTok{5}\NormalTok{, }\DecValTok{5}\NormalTok{, }\DecValTok{1}\NormalTok{, }\DecValTok{1}\NormalTok{))}

\NormalTok{util}\SpecialCharTok{$}\FunctionTok{plot\_expectand\_pushforward}\NormalTok{(samples3[[}\StringTok{\textquotesingle{}tau\_q\textquotesingle{}}\NormalTok{]],}
                                \DecValTok{25}\NormalTok{, }\AttributeTok{flim=}\FunctionTok{c}\NormalTok{(}\DecValTok{0}\NormalTok{, }\FloatTok{0.3}\NormalTok{),}
                                \AttributeTok{display\_name=}\StringTok{\textquotesingle{}tau\_q\textquotesingle{}}\NormalTok{)}

\NormalTok{names }\OtherTok{\textless{}{-}} \FunctionTok{sapply}\NormalTok{(}\DecValTok{1}\SpecialCharTok{:}\DecValTok{5}\NormalTok{, }\ControlFlowTok{function}\NormalTok{(k) }\FunctionTok{paste0}\NormalTok{(}\StringTok{\textquotesingle{}mu\_q[\textquotesingle{}}\NormalTok{, k, }\StringTok{\textquotesingle{}]\textquotesingle{}}\NormalTok{))}
\NormalTok{util}\SpecialCharTok{$}\FunctionTok{plot\_disc\_pushforward\_quantiles}\NormalTok{(samples3, names,}
                                     \AttributeTok{xlab=}\StringTok{"Rating"}\NormalTok{,}
                                     \AttributeTok{ylab=}\StringTok{"Baseline Rating Probability"}\NormalTok{)}
\end{Highlighting}
\end{Shaded}

\includegraphics{ratings_files/figure-pdf/unnamed-chunk-46-1.pdf}

The regularizing influence of the hierarchical model is strongest for
those customers with the fewest ratings. For example Customer 42 has
only three observed ratings, and the hierarchical inferences for their
interior cut points shift and narrow pretty substantially relative to
the inferences from the previous model.

\begin{Shaded}
\begin{Highlighting}[]
\NormalTok{c }\OtherTok{\textless{}{-}} \DecValTok{42}
\FunctionTok{cat}\NormalTok{(}\FunctionTok{sprintf}\NormalTok{(}\StringTok{"Customer \%s: \%s ratings"}\NormalTok{,}
\NormalTok{            c, }\FunctionTok{table}\NormalTok{(data}\SpecialCharTok{$}\NormalTok{customer\_idxs)[c]))}
\end{Highlighting}
\end{Shaded}

\begin{verbatim}
Customer 42: 3 ratings
\end{verbatim}

\begin{Shaded}
\begin{Highlighting}[]
\FunctionTok{par}\NormalTok{(}\AttributeTok{mfrow=}\FunctionTok{c}\NormalTok{(}\DecValTok{2}\NormalTok{, }\DecValTok{2}\NormalTok{), }\AttributeTok{mar=}\FunctionTok{c}\NormalTok{(}\DecValTok{5}\NormalTok{, }\DecValTok{5}\NormalTok{, }\DecValTok{1}\NormalTok{, }\DecValTok{1}\NormalTok{))}

\NormalTok{lab2\_xs }\OtherTok{\textless{}{-}} \FunctionTok{c}\NormalTok{(}\FloatTok{1.5}\NormalTok{, }\DecValTok{2}\NormalTok{, }\SpecialCharTok{{-}}\DecValTok{4}\NormalTok{, }\SpecialCharTok{{-}}\DecValTok{3}\NormalTok{)}
\NormalTok{lab2\_ys }\OtherTok{\textless{}{-}} \FunctionTok{c}\NormalTok{(}\FloatTok{0.15}\NormalTok{, }\FloatTok{0.25}\NormalTok{, }\FloatTok{0.35}\NormalTok{, }\FloatTok{0.35}\NormalTok{)}

\NormalTok{lab3\_xs }\OtherTok{\textless{}{-}} \FunctionTok{c}\NormalTok{(}\SpecialCharTok{{-}}\DecValTok{6}\NormalTok{, }\SpecialCharTok{{-}}\DecValTok{5}\NormalTok{, }\DecValTok{3}\NormalTok{, }\FloatTok{3.5}\NormalTok{)}
\NormalTok{lab3\_ys }\OtherTok{\textless{}{-}} \FunctionTok{c}\NormalTok{(}\FloatTok{0.25}\NormalTok{, }\FloatTok{0.4}\NormalTok{, }\FloatTok{0.5}\NormalTok{, }\FloatTok{0.6}\NormalTok{)}

\ControlFlowTok{for}\NormalTok{ (k }\ControlFlowTok{in} \DecValTok{1}\SpecialCharTok{:}\DecValTok{4}\NormalTok{) \{}
\NormalTok{  name }\OtherTok{\textless{}{-}} \FunctionTok{paste0}\NormalTok{(}\StringTok{\textquotesingle{}cut\_points[\textquotesingle{}}\NormalTok{, c, }\StringTok{\textquotesingle{},\textquotesingle{}}\NormalTok{, k, }\StringTok{\textquotesingle{}]\textquotesingle{}}\NormalTok{)}
\NormalTok{  util}\SpecialCharTok{$}\FunctionTok{plot\_expectand\_pushforward}\NormalTok{(samples2[[name]],}
                                  \DecValTok{50}\NormalTok{, }\AttributeTok{flim=}\FunctionTok{c}\NormalTok{(}\SpecialCharTok{{-}}\DecValTok{10}\NormalTok{, }\DecValTok{5}\NormalTok{), }\AttributeTok{ylim=}\FunctionTok{c}\NormalTok{(}\DecValTok{0}\NormalTok{, }\FloatTok{1.0}\NormalTok{),}
                                  \AttributeTok{col=}\NormalTok{util}\SpecialCharTok{$}\NormalTok{c\_light,}
                                  \AttributeTok{display\_name=}\StringTok{\textquotesingle{}Interior Cut Points\textquotesingle{}}\NormalTok{,}
                                  \AttributeTok{main=}\FunctionTok{paste}\NormalTok{(}\StringTok{\textquotesingle{}Customer\textquotesingle{}}\NormalTok{, c))}

\NormalTok{  name }\OtherTok{\textless{}{-}} \FunctionTok{paste0}\NormalTok{(}\StringTok{\textquotesingle{}cut\_points[\textquotesingle{}}\NormalTok{, c, }\StringTok{\textquotesingle{},\textquotesingle{}}\NormalTok{, k, }\StringTok{\textquotesingle{}]\textquotesingle{}}\NormalTok{)}
\NormalTok{  util}\SpecialCharTok{$}\FunctionTok{plot\_expectand\_pushforward}\NormalTok{(samples3[[name]],}
                                  \DecValTok{50}\NormalTok{, }\AttributeTok{flim=}\FunctionTok{c}\NormalTok{(}\SpecialCharTok{{-}}\DecValTok{10}\NormalTok{, }\DecValTok{5}\NormalTok{),}
                                  \AttributeTok{col=}\NormalTok{util}\SpecialCharTok{$}\NormalTok{c\_dark, }\AttributeTok{border=}\StringTok{"\#BBBBBB88"}\NormalTok{,}
                                  \AttributeTok{add=}\ConstantTok{TRUE}\NormalTok{)}

  \FunctionTok{text}\NormalTok{(lab2\_xs[k], lab2\_ys[k], }\StringTok{"Model 2"}\NormalTok{, }\AttributeTok{col=}\NormalTok{util}\SpecialCharTok{$}\NormalTok{c\_light)}
  \FunctionTok{text}\NormalTok{(lab3\_xs[k], lab3\_ys[k], }\StringTok{"Model 3"}\NormalTok{, }\AttributeTok{col=}\NormalTok{util}\SpecialCharTok{$}\NormalTok{c\_dark)}
\NormalTok{\}}
\end{Highlighting}
\end{Shaded}

\begin{verbatim}
Warning in util$plot_expectand_pushforward(samples2[[name]], 50, flim = c(-10,
: 1 value (0.0%) fell below the histogram binning.
\end{verbatim}

\begin{verbatim}
Warning in util$plot_expectand_pushforward(samples2[[name]], 50, flim = c(-10,
: 0 value (0.0%) fell above the histogram binning.
\end{verbatim}

\begin{verbatim}
Warning in util$plot_expectand_pushforward(samples3[[name]], 50, flim = c(-10,
: 1 value (0.0%) fell below the histogram binning.
\end{verbatim}

\begin{verbatim}
Warning in util$plot_expectand_pushforward(samples3[[name]], 50, flim = c(-10,
: 0 value (0.0%) fell above the histogram binning.
\end{verbatim}

\includegraphics{ratings_files/figure-pdf/unnamed-chunk-48-1.pdf}

We can also see the impact of the hierarchical model if we examine the
interior cut point inferences for the two models across all customers.
The more uncertain customer inferences in the previous model, especially
for the first and last cut points, are pulled towards the other customer
inferences.

\begin{Shaded}
\begin{Highlighting}[]
\FunctionTok{par}\NormalTok{(}\AttributeTok{mfrow=}\FunctionTok{c}\NormalTok{(}\DecValTok{4}\NormalTok{, }\DecValTok{1}\NormalTok{), }\AttributeTok{mar=}\FunctionTok{c}\NormalTok{(}\DecValTok{5}\NormalTok{, }\DecValTok{5}\NormalTok{, }\DecValTok{1}\NormalTok{, }\DecValTok{1}\NormalTok{))}

\ControlFlowTok{for}\NormalTok{ (k }\ControlFlowTok{in} \DecValTok{1}\SpecialCharTok{:}\DecValTok{4}\NormalTok{) \{}
\NormalTok{  names }\OtherTok{\textless{}{-}} \FunctionTok{sapply}\NormalTok{(}\DecValTok{1}\SpecialCharTok{:}\NormalTok{data}\SpecialCharTok{$}\NormalTok{N\_customers,}
                  \ControlFlowTok{function}\NormalTok{(c) }\FunctionTok{paste0}\NormalTok{(}\StringTok{\textquotesingle{}cut\_points[\textquotesingle{}}\NormalTok{, c, }\StringTok{\textquotesingle{},\textquotesingle{}}\NormalTok{, k, }\StringTok{\textquotesingle{}]\textquotesingle{}}\NormalTok{))}

\NormalTok{  yname }\OtherTok{\textless{}{-}} \FunctionTok{paste0}\NormalTok{(}\StringTok{\textquotesingle{}cut\_points[\textquotesingle{}}\NormalTok{, k, }\StringTok{\textquotesingle{}]\textquotesingle{}}\NormalTok{)}
\NormalTok{  util}\SpecialCharTok{$}\FunctionTok{plot\_disc\_pushforward\_quantiles}\NormalTok{(samples3, names,}
                                       \AttributeTok{xlab=}\StringTok{"customer"}\NormalTok{,}
                                       \AttributeTok{display\_ylim=}\FunctionTok{c}\NormalTok{(}\SpecialCharTok{{-}}\DecValTok{6}\NormalTok{, }\DecValTok{6}\NormalTok{),}
                                       \AttributeTok{ylab=}\NormalTok{yname)}
\NormalTok{\}}
\end{Highlighting}
\end{Shaded}

\includegraphics{ratings_files/figure-pdf/unnamed-chunk-49-1.pdf}

The hierarchical influence, however, doesn't change the qualitative
details. For example Customer 23 is still stingy with their ratings
while Customer 70 is still generous.

\begin{Shaded}
\begin{Highlighting}[]
\FunctionTok{par}\NormalTok{(}\AttributeTok{mfrow=}\FunctionTok{c}\NormalTok{(}\DecValTok{2}\NormalTok{, }\DecValTok{1}\NormalTok{), }\AttributeTok{mar=}\FunctionTok{c}\NormalTok{(}\DecValTok{5}\NormalTok{, }\DecValTok{5}\NormalTok{, }\DecValTok{1}\NormalTok{, }\DecValTok{1}\NormalTok{))}

\NormalTok{cols }\OtherTok{\textless{}{-}} \FunctionTok{c}\NormalTok{(util}\SpecialCharTok{$}\NormalTok{c\_mid, util}\SpecialCharTok{$}\NormalTok{c\_mid\_highlight,}
\NormalTok{          util}\SpecialCharTok{$}\NormalTok{c\_dark, util}\SpecialCharTok{$}\NormalTok{c\_dark\_highlight)}

\NormalTok{c }\OtherTok{\textless{}{-}} \DecValTok{23}

\NormalTok{k }\OtherTok{\textless{}{-}} \DecValTok{1}
\NormalTok{name }\OtherTok{\textless{}{-}}\FunctionTok{paste0}\NormalTok{(}\StringTok{\textquotesingle{}cut\_points[\textquotesingle{}}\NormalTok{, c, }\StringTok{\textquotesingle{},\textquotesingle{}}\NormalTok{, k, }\StringTok{\textquotesingle{}]\textquotesingle{}}\NormalTok{)}
\NormalTok{util}\SpecialCharTok{$}\FunctionTok{plot\_expectand\_pushforward}\NormalTok{(samples3[[name]],}
                                \DecValTok{50}\NormalTok{, }\AttributeTok{flim=}\FunctionTok{c}\NormalTok{(}\SpecialCharTok{{-}}\DecValTok{9}\NormalTok{, }\DecValTok{9}\NormalTok{), }\AttributeTok{ylim=}\FunctionTok{c}\NormalTok{(}\DecValTok{0}\NormalTok{, }\DecValTok{2}\NormalTok{),}
                                \AttributeTok{col=}\NormalTok{cols[k],}
                                \AttributeTok{display\_name=}\StringTok{\textquotesingle{}Interior Cut Points\textquotesingle{}}\NormalTok{,}
                                \AttributeTok{main=}\FunctionTok{paste}\NormalTok{(}\StringTok{\textquotesingle{}Customer\textquotesingle{}}\NormalTok{, c))}


\ControlFlowTok{for}\NormalTok{ (k }\ControlFlowTok{in} \DecValTok{2}\SpecialCharTok{:}\DecValTok{4}\NormalTok{) \{}
\NormalTok{  name }\OtherTok{\textless{}{-}}\FunctionTok{paste0}\NormalTok{(}\StringTok{\textquotesingle{}cut\_points[\textquotesingle{}}\NormalTok{, c, }\StringTok{\textquotesingle{},\textquotesingle{}}\NormalTok{, k, }\StringTok{\textquotesingle{}]\textquotesingle{}}\NormalTok{)}
\NormalTok{  util}\SpecialCharTok{$}\FunctionTok{plot\_expectand\_pushforward}\NormalTok{(samples3[[name]],}
                                  \DecValTok{50}\NormalTok{, }\AttributeTok{flim=}\FunctionTok{c}\NormalTok{(}\SpecialCharTok{{-}}\DecValTok{9}\NormalTok{, }\DecValTok{9}\NormalTok{),}
                                  \AttributeTok{col=}\NormalTok{cols[k], }\AttributeTok{border=}\StringTok{"\#BBBBBB88"}\NormalTok{,}
                                  \AttributeTok{add=}\ConstantTok{TRUE}\NormalTok{)}
\NormalTok{\}}

\FunctionTok{text}\NormalTok{(}\SpecialCharTok{{-}}\DecValTok{1}\NormalTok{, }\FloatTok{1.65}\NormalTok{, }\StringTok{"cut\_points[1]"}\NormalTok{, }\AttributeTok{col=}\NormalTok{util}\SpecialCharTok{$}\NormalTok{c\_mid)}
\FunctionTok{text}\NormalTok{(}\DecValTok{2}\NormalTok{, }\FloatTok{1.55}\NormalTok{, }\StringTok{"cut\_points[2]"}\NormalTok{, }\AttributeTok{col=}\NormalTok{util}\SpecialCharTok{$}\NormalTok{c\_mid\_highlight)}
\FunctionTok{text}\NormalTok{(}\FloatTok{4.25}\NormalTok{, }\DecValTok{1}\NormalTok{, }\StringTok{"cut\_points[3]"}\NormalTok{, }\AttributeTok{col=}\NormalTok{util}\SpecialCharTok{$}\NormalTok{c\_dark)}
\FunctionTok{text}\NormalTok{(}\FloatTok{6.25}\NormalTok{, }\FloatTok{0.5}\NormalTok{, }\StringTok{"cut\_points[4]"}\NormalTok{, }\AttributeTok{col=}\NormalTok{util}\SpecialCharTok{$}\NormalTok{c\_dark\_highlight)}

\NormalTok{c }\OtherTok{\textless{}{-}} \DecValTok{70}

\NormalTok{k }\OtherTok{\textless{}{-}} \DecValTok{1}
\NormalTok{name }\OtherTok{\textless{}{-}}\FunctionTok{paste0}\NormalTok{(}\StringTok{\textquotesingle{}cut\_points[\textquotesingle{}}\NormalTok{, c, }\StringTok{\textquotesingle{},\textquotesingle{}}\NormalTok{, k, }\StringTok{\textquotesingle{}]\textquotesingle{}}\NormalTok{)}
\NormalTok{util}\SpecialCharTok{$}\FunctionTok{plot\_expectand\_pushforward}\NormalTok{(samples3[[name]],}
                                \DecValTok{50}\NormalTok{, }\AttributeTok{flim=}\FunctionTok{c}\NormalTok{(}\SpecialCharTok{{-}}\DecValTok{9}\NormalTok{, }\DecValTok{9}\NormalTok{), }\AttributeTok{ylim=}\FunctionTok{c}\NormalTok{(}\DecValTok{0}\NormalTok{, }\DecValTok{2}\NormalTok{),}
                                \AttributeTok{col=}\NormalTok{cols[k],}
                                \AttributeTok{display\_name=}\StringTok{\textquotesingle{}Interior Cut Points\textquotesingle{}}\NormalTok{,}
                                \AttributeTok{main=}\FunctionTok{paste}\NormalTok{(}\StringTok{\textquotesingle{}Customer\textquotesingle{}}\NormalTok{, c))}
\end{Highlighting}
\end{Shaded}

\begin{verbatim}
Warning in util$plot_expectand_pushforward(samples3[[name]], 50, flim = c(-9, :
11 values (0.3%) fell below the histogram binning.
\end{verbatim}

\begin{verbatim}
Warning in util$plot_expectand_pushforward(samples3[[name]], 50, flim = c(-9, :
0 values (0.0%) fell above the histogram binning.
\end{verbatim}

\begin{Shaded}
\begin{Highlighting}[]
\ControlFlowTok{for}\NormalTok{ (k }\ControlFlowTok{in} \DecValTok{2}\SpecialCharTok{:}\DecValTok{4}\NormalTok{) \{}
\NormalTok{  name }\OtherTok{\textless{}{-}}\FunctionTok{paste0}\NormalTok{(}\StringTok{\textquotesingle{}cut\_points[\textquotesingle{}}\NormalTok{, c, }\StringTok{\textquotesingle{},\textquotesingle{}}\NormalTok{, k, }\StringTok{\textquotesingle{}]\textquotesingle{}}\NormalTok{)}
\NormalTok{  util}\SpecialCharTok{$}\FunctionTok{plot\_expectand\_pushforward}\NormalTok{(samples3[[name]],}
                                  \DecValTok{50}\NormalTok{, }\AttributeTok{flim=}\FunctionTok{c}\NormalTok{(}\SpecialCharTok{{-}}\DecValTok{9}\NormalTok{, }\DecValTok{9}\NormalTok{),}
                                  \AttributeTok{col=}\NormalTok{cols[k], }\AttributeTok{border=}\StringTok{"\#BBBBBB88"}\NormalTok{,}
                                  \AttributeTok{add=}\ConstantTok{TRUE}\NormalTok{)}
\NormalTok{\}}

\FunctionTok{text}\NormalTok{(}\SpecialCharTok{{-}}\FloatTok{5.75}\NormalTok{, }\FloatTok{0.4}\NormalTok{, }\StringTok{"cut\_points[1]"}\NormalTok{, }\AttributeTok{col=}\NormalTok{util}\SpecialCharTok{$}\NormalTok{c\_mid)}
\FunctionTok{text}\NormalTok{(}\SpecialCharTok{{-}}\DecValTok{3}\NormalTok{, }\FloatTok{0.9}\NormalTok{, }\StringTok{"cut\_points[2]"}\NormalTok{, }\AttributeTok{col=}\NormalTok{util}\SpecialCharTok{$}\NormalTok{c\_mid\_highlight)}
\FunctionTok{text}\NormalTok{(}\SpecialCharTok{{-}}\DecValTok{2}\NormalTok{, }\FloatTok{1.2}\NormalTok{, }\StringTok{"cut\_points[3]"}\NormalTok{, }\AttributeTok{col=}\NormalTok{util}\SpecialCharTok{$}\NormalTok{c\_dark)}
\FunctionTok{text}\NormalTok{(}\FloatTok{1.0}\NormalTok{, }\FloatTok{1.4}\NormalTok{, }\StringTok{"cut\_points[4]"}\NormalTok{, }\AttributeTok{col=}\NormalTok{util}\SpecialCharTok{$}\NormalTok{c\_dark\_highlight)}
\end{Highlighting}
\end{Shaded}

\includegraphics{ratings_files/figure-pdf/unnamed-chunk-50-1.pdf}

Of course we can still investigate the behavior of individual movies and
the hierarchical population of movies. The hierarchical regularization
of the interior cut points allows the observed ratings to slightly
better inform the movie affinities.

\begin{Shaded}
\begin{Highlighting}[]
\FunctionTok{par}\NormalTok{(}\AttributeTok{mfrow=}\FunctionTok{c}\NormalTok{(}\DecValTok{2}\NormalTok{, }\DecValTok{1}\NormalTok{), }\AttributeTok{mar=}\FunctionTok{c}\NormalTok{(}\DecValTok{5}\NormalTok{, }\DecValTok{5}\NormalTok{, }\DecValTok{1}\NormalTok{, }\DecValTok{1}\NormalTok{))}

\NormalTok{util}\SpecialCharTok{$}\FunctionTok{plot\_expectand\_pushforward}\NormalTok{(samples3[[}\StringTok{\textquotesingle{}tau\_gamma\textquotesingle{}}\NormalTok{]],}
                                \DecValTok{50}\NormalTok{, }\AttributeTok{flim=}\FunctionTok{c}\NormalTok{(}\DecValTok{0}\NormalTok{, }\DecValTok{1}\NormalTok{),}
                                \AttributeTok{display\_name=}\StringTok{\textquotesingle{}tau\_gamma\textquotesingle{}}\NormalTok{)}

\NormalTok{names }\OtherTok{\textless{}{-}} \FunctionTok{sapply}\NormalTok{(}\DecValTok{1}\SpecialCharTok{:}\NormalTok{data}\SpecialCharTok{$}\NormalTok{N\_movies,}
                \ControlFlowTok{function}\NormalTok{(m) }\FunctionTok{paste0}\NormalTok{(}\StringTok{\textquotesingle{}gamma[\textquotesingle{}}\NormalTok{, m, }\StringTok{\textquotesingle{}]\textquotesingle{}}\NormalTok{))}
\NormalTok{util}\SpecialCharTok{$}\FunctionTok{plot\_disc\_pushforward\_quantiles}\NormalTok{(samples3, names,}
                                     \AttributeTok{xlab=}\StringTok{"Movie"}\NormalTok{,}
                                     \AttributeTok{ylab=}\StringTok{"Affinity"}\NormalTok{)}
\end{Highlighting}
\end{Shaded}

\includegraphics{ratings_files/figure-pdf/unnamed-chunk-51-1.pdf}

Now let's go one step further than we did in the previous section and
use these movie inferences to rank the movies by their expected
affinities. This is just one heuristic for ranking items based on their
inferred qualities, but one that has the advantage of being relatively
fast to compute.

\begin{Shaded}
\begin{Highlighting}[]
\NormalTok{expected\_affinity }\OtherTok{\textless{}{-}} \ControlFlowTok{function}\NormalTok{(m) \{}
\NormalTok{  util}\SpecialCharTok{$}\FunctionTok{ensemble\_mcmc\_est}\NormalTok{(samples3[[}\FunctionTok{paste0}\NormalTok{(}\StringTok{\textquotesingle{}gamma[\textquotesingle{}}\NormalTok{, m, }\StringTok{\textquotesingle{}]\textquotesingle{}}\NormalTok{)]])[}\DecValTok{1}\NormalTok{]}
\NormalTok{\}}

\NormalTok{expected\_affinities }\OtherTok{\textless{}{-}} \FunctionTok{sapply}\NormalTok{(}\DecValTok{1}\SpecialCharTok{:}\NormalTok{data}\SpecialCharTok{$}\NormalTok{N\_movies,}
                             \ControlFlowTok{function}\NormalTok{(m) }\FunctionTok{expected\_affinity}\NormalTok{(m))}

\NormalTok{post\_mean\_ordering }\OtherTok{\textless{}{-}} \FunctionTok{sort}\NormalTok{(expected\_affinities, }\AttributeTok{index.return=}\ConstantTok{TRUE}\NormalTok{)}\SpecialCharTok{$}\NormalTok{ix}
\end{Highlighting}
\end{Shaded}

We can then use this ranking to select the five worst movies for this
particular set of customers.

\begin{Shaded}
\begin{Highlighting}[]
\FunctionTok{print}\NormalTok{(}\FunctionTok{data.frame}\NormalTok{(}\StringTok{"Rank"}\OtherTok{=}\DecValTok{200}\SpecialCharTok{:}\DecValTok{196}\NormalTok{,}
                 \StringTok{"Movie"}\OtherTok{=}\FunctionTok{head}\NormalTok{(post\_mean\_ordering, }\DecValTok{5}\NormalTok{)),}
      \AttributeTok{row.names=}\ConstantTok{FALSE}\NormalTok{)}
\end{Highlighting}
\end{Shaded}

\begin{verbatim}
 Rank Movie
  200    31
  199   159
  198    13
  197    40
  196   175
\end{verbatim}

To be a bit less pessimistic we could also consider the five best movies
for this particular set of customers.

\begin{Shaded}
\begin{Highlighting}[]
\FunctionTok{print}\NormalTok{(}\FunctionTok{data.frame}\NormalTok{(}\StringTok{"Rank"}\OtherTok{=}\DecValTok{5}\SpecialCharTok{:}\DecValTok{1}\NormalTok{,}
                 \StringTok{"Movie"}\OtherTok{=}\FunctionTok{tail}\NormalTok{(post\_mean\_ordering, }\DecValTok{5}\NormalTok{)),}
      \AttributeTok{row.names=}\ConstantTok{FALSE}\NormalTok{)}
\end{Highlighting}
\end{Shaded}

\begin{verbatim}
 Rank Movie
    5   193
    4    33
    3   117
    2    44
    1   180
\end{verbatim}

We also have a variety of ways to make inferential comparisons between
two movies at a time. For example we could just overlay the marginal
posterior distributions for each movie affinity.

\begin{Shaded}
\begin{Highlighting}[]
\FunctionTok{par}\NormalTok{(}\AttributeTok{mfrow=}\FunctionTok{c}\NormalTok{(}\DecValTok{1}\NormalTok{, }\DecValTok{1}\NormalTok{), }\AttributeTok{mar=}\FunctionTok{c}\NormalTok{(}\DecValTok{5}\NormalTok{, }\DecValTok{5}\NormalTok{, }\DecValTok{1}\NormalTok{, }\DecValTok{1}\NormalTok{))}

\NormalTok{m1 }\OtherTok{\textless{}{-}} \FunctionTok{head}\NormalTok{(post\_mean\_ordering, }\DecValTok{1}\NormalTok{)}
\NormalTok{name }\OtherTok{\textless{}{-}}\FunctionTok{paste0}\NormalTok{(}\StringTok{\textquotesingle{}gamma[\textquotesingle{}}\NormalTok{, m1, }\StringTok{\textquotesingle{}]\textquotesingle{}}\NormalTok{)}
\NormalTok{util}\SpecialCharTok{$}\FunctionTok{plot\_expectand\_pushforward}\NormalTok{(samples3[[name]],}
                                \DecValTok{50}\NormalTok{, }\AttributeTok{flim=}\FunctionTok{c}\NormalTok{(}\SpecialCharTok{{-}}\DecValTok{3}\NormalTok{, }\DecValTok{3}\NormalTok{),}
                                \AttributeTok{ylim=}\FunctionTok{c}\NormalTok{(}\DecValTok{0}\NormalTok{, }\FloatTok{1.35}\NormalTok{),}
                                \AttributeTok{col=}\NormalTok{util}\SpecialCharTok{$}\NormalTok{c\_mid,}
                                \AttributeTok{display\_name=}\StringTok{\textquotesingle{}Affinity\textquotesingle{}}\NormalTok{)}
\FunctionTok{text}\NormalTok{(}\SpecialCharTok{{-}}\FloatTok{1.25}\NormalTok{, }\FloatTok{1.3}\NormalTok{, }\FunctionTok{paste}\NormalTok{(}\StringTok{\textquotesingle{}Movie\textquotesingle{}}\NormalTok{, m1), }\AttributeTok{col=}\NormalTok{util}\SpecialCharTok{$}\NormalTok{c\_mid)}

\NormalTok{m2 }\OtherTok{\textless{}{-}} \FunctionTok{tail}\NormalTok{(post\_mean\_ordering, }\DecValTok{1}\NormalTok{)}
\NormalTok{name }\OtherTok{\textless{}{-}}\FunctionTok{paste0}\NormalTok{(}\StringTok{\textquotesingle{}gamma[\textquotesingle{}}\NormalTok{, m2, }\StringTok{\textquotesingle{}]\textquotesingle{}}\NormalTok{)}
\NormalTok{util}\SpecialCharTok{$}\FunctionTok{plot\_expectand\_pushforward}\NormalTok{(samples3[[name]],}
                                \DecValTok{50}\NormalTok{, }\AttributeTok{flim=}\FunctionTok{c}\NormalTok{(}\SpecialCharTok{{-}}\DecValTok{3}\NormalTok{, }\DecValTok{3}\NormalTok{),}
                                \AttributeTok{col=}\NormalTok{util}\SpecialCharTok{$}\NormalTok{c\_dark,}
                                \AttributeTok{border=}\StringTok{"\#BBBBBB88"}\NormalTok{,}
                                \AttributeTok{add=}\ConstantTok{TRUE}\NormalTok{)}
\FunctionTok{text}\NormalTok{(}\FloatTok{1.25}\NormalTok{, }\FloatTok{1.3}\NormalTok{, }\FunctionTok{paste}\NormalTok{(}\StringTok{\textquotesingle{}Movie\textquotesingle{}}\NormalTok{, m2), }\AttributeTok{col=}\NormalTok{util}\SpecialCharTok{$}\NormalTok{c\_dark)}
\end{Highlighting}
\end{Shaded}

\includegraphics{ratings_files/figure-pdf/unnamed-chunk-55-1.pdf}

Even better we can directly compute the probability that one movie
affinity is larger than the other. Here there is little ambiguity
whether or not the top ranked movie is actually better than the worst
ranked movie.

\begin{Shaded}
\begin{Highlighting}[]
\NormalTok{var\_repl }\OtherTok{\textless{}{-}} \FunctionTok{list}\NormalTok{(}\StringTok{\textquotesingle{}g1\textquotesingle{}} \OtherTok{=} \FunctionTok{paste0}\NormalTok{(}\StringTok{\textquotesingle{}gamma[\textquotesingle{}}\NormalTok{, m1,}\StringTok{\textquotesingle{}]\textquotesingle{}}\NormalTok{),}
                 \StringTok{\textquotesingle{}g2\textquotesingle{}} \OtherTok{=} \FunctionTok{paste0}\NormalTok{(}\StringTok{\textquotesingle{}gamma[\textquotesingle{}}\NormalTok{, m2,}\StringTok{\textquotesingle{}]\textquotesingle{}}\NormalTok{))}

\NormalTok{p\_est }\OtherTok{\textless{}{-}}
\NormalTok{  util}\SpecialCharTok{$}\FunctionTok{implicit\_subset\_prob}\NormalTok{(samples3,}
                            \ControlFlowTok{function}\NormalTok{(g1, g2) g1 }\SpecialCharTok{\textless{}}\NormalTok{ g2,}
\NormalTok{                            var\_repl)}

\NormalTok{format\_string }\OtherTok{\textless{}{-}} \FunctionTok{paste0}\NormalTok{(}\StringTok{"Posterior probability that movie \%i affinity "}\NormalTok{,}
                        \StringTok{"\textgreater{} movie \%i affinity = \%.3f +/{-} \%.3f."}\NormalTok{)}
\FunctionTok{cat}\NormalTok{(}\FunctionTok{sprintf}\NormalTok{(format\_string, m1, m2, p\_est[}\DecValTok{1}\NormalTok{], }\DecValTok{2} \SpecialCharTok{*}\NormalTok{ p\_est[}\DecValTok{2}\NormalTok{]))}
\end{Highlighting}
\end{Shaded}

\begin{verbatim}
Posterior probability that movie 31 affinity > movie 180 affinity = 1.000 +/- 0.000.
\end{verbatim}

\section{Hierarchical Customer Model With Heterogeneous
Affinities}\label{hierarchical-customer-model-with-heterogeneous-affinities}

There are two main limitations with the last model. Firstly there is the
mild retrodictive tension in a few of the summary statistics. Secondly
it makes the strong assumption that once the different interpretations
of ratings are taken into account all customers have the same opinion
about each movie. This for example prevents us from making personalized
recommendations to each customer.

Incorporating individual tastes into the model would inform much more
nuanced inferences and predictions. It could even resolve some of the
retrodictive tension. The immediate challenge with trying to infer
personal movie preferences, however, is that customers rate only a very
small proportion of the available movies.

\begin{Shaded}
\begin{Highlighting}[]
\NormalTok{xs }\OtherTok{\textless{}{-}} \FunctionTok{seq}\NormalTok{(}\DecValTok{1}\NormalTok{, data}\SpecialCharTok{$}\NormalTok{N\_movies, }\DecValTok{1}\NormalTok{)}
\NormalTok{ys }\OtherTok{\textless{}{-}} \FunctionTok{seq}\NormalTok{(}\DecValTok{1}\NormalTok{, data}\SpecialCharTok{$}\NormalTok{N\_customers, }\DecValTok{1}\NormalTok{)}
\NormalTok{zs }\OtherTok{\textless{}{-}} \FunctionTok{matrix}\NormalTok{(}\DecValTok{0}\NormalTok{, }\AttributeTok{nrow=}\NormalTok{data}\SpecialCharTok{$}\NormalTok{N\_movies, }\AttributeTok{ncol=}\NormalTok{data}\SpecialCharTok{$}\NormalTok{N\_customers)}

\ControlFlowTok{for}\NormalTok{ (n }\ControlFlowTok{in} \DecValTok{1}\SpecialCharTok{:}\NormalTok{data}\SpecialCharTok{$}\NormalTok{N\_ratings) \{}
\NormalTok{  zs[data}\SpecialCharTok{$}\NormalTok{movie\_idxs[n], data}\SpecialCharTok{$}\NormalTok{customer\_idxs[n]] }\OtherTok{\textless{}{-}} \DecValTok{1}
\NormalTok{\}}

\FunctionTok{par}\NormalTok{(}\AttributeTok{mfrow=}\FunctionTok{c}\NormalTok{(}\DecValTok{1}\NormalTok{, }\DecValTok{1}\NormalTok{), }\AttributeTok{mar =} \FunctionTok{c}\NormalTok{(}\DecValTok{5}\NormalTok{, }\DecValTok{5}\NormalTok{, }\DecValTok{1}\NormalTok{, }\DecValTok{1}\NormalTok{))}

\FunctionTok{image}\NormalTok{(xs, ys, zs, }\AttributeTok{col=}\FunctionTok{c}\NormalTok{(}\StringTok{"white"}\NormalTok{, util}\SpecialCharTok{$}\NormalTok{c\_dark\_teal),}
      \AttributeTok{xlab=}\StringTok{"Movie"}\NormalTok{, }\AttributeTok{ylab=}\StringTok{"customer"}\NormalTok{)}
\end{Highlighting}
\end{Shaded}

\includegraphics{ratings_files/figure-pdf/unnamed-chunk-57-1.pdf}

In the machine learning literature the problem of filling in unobserved
comparisons, like customer-movie ratings in this application, is often
known as ``matrix completion'' in analogy to filling in the missing
cells in this pairwise comparison matrix.

Because most of the ratings are unobserved the only way to inform
individual customer preferences for each movie is to pool correlations
in each movie affinities across customers. For example we might model
each customers' movie affinities as common baseline affinities plus
individual deviations, \[
\boldsymbol{\gamma}_{c}
=
\boldsymbol{\gamma}_{0} + \boldsymbol{\delta}_{c}.
\] We can then pool these individual deviations together with a
multivariate normal population model, \[
p(\boldsymbol{\delta}_{c})
=
\text{multi-normal}( \mathbf{0}, \boldsymbol{\Sigma} )
\] with \[
\boldsymbol{\Sigma}
=
\boldsymbol{\tau}_{\gamma}^{T} \cdot
\boldsymbol{\Phi}_{\gamma} \cdot
\boldsymbol{\tau}_{\gamma}.
\] Whatever we learn about the population baseline
\(\boldsymbol{\gamma}_{0}\), the population scales
\(\boldsymbol{\tau}_{\gamma}\), and the population correlations
\(\boldsymbol{\Phi}_{\gamma}\) fill in whatever elements of any
particular customer affinity vector \(\boldsymbol{\gamma}_{c}\) that
might be unobserved.

Because our domain expertise about the movies is still exchangeable we
can model the baseline movie affinities hierarchically as well, \[
p( \gamma_{0, m} ) = \text{normal}(0, \tau_{\gamma_{0}} ).
\] We're now modeling the interior cut points, the baseline movie
affinities, \emph{and} the individual customer affinities hierarchically
all at the same time. Pretty neat.

Given the sparsity of observed ratings we'll implement the normal and
multivariate normal hierarchical models with non-centered
parameterizations.

\begin{codelisting}

\caption{\texttt{model4.stan}}

\begin{Shaded}
\begin{Highlighting}[]
\KeywordTok{functions}\NormalTok{ \{}
  \CommentTok{// Log probability density function over cut point}
  \CommentTok{// induced by a Dirichlet probability density function}
  \CommentTok{// over baseline probabilities and latent logistic}
  \CommentTok{// density function.}
  \DataTypeTok{real}\NormalTok{ induced\_dirichlet\_lpdf(}\DataTypeTok{vector}\NormalTok{ c, }\DataTypeTok{vector}\NormalTok{ alpha) \{}
    \DataTypeTok{int}\NormalTok{ K = num\_elements(c) + }\DecValTok{1}\NormalTok{;}
    \DataTypeTok{vector}\NormalTok{[K {-} }\DecValTok{1}\NormalTok{] Pi = inv\_logit(c);}
    \DataTypeTok{vector}\NormalTok{[K] p = append\_row(Pi, [}\DecValTok{1}\NormalTok{]\textquotesingle{}) {-} append\_row([}\DecValTok{0}\NormalTok{]\textquotesingle{}, Pi);}

    \CommentTok{// Log Jacobian correction}
    \DataTypeTok{real}\NormalTok{ logJ = }\DecValTok{0}\NormalTok{;}
    \ControlFlowTok{for}\NormalTok{ (k }\ControlFlowTok{in} \DecValTok{1}\NormalTok{:(K {-} }\DecValTok{1}\NormalTok{)) \{}
      \ControlFlowTok{if}\NormalTok{ (c[k] \textgreater{}= }\DecValTok{0}\NormalTok{)}
\NormalTok{        logJ += {-}c[k] {-} }\DecValTok{2}\NormalTok{ * log(}\DecValTok{1}\NormalTok{ + exp({-}c[k]));}
      \ControlFlowTok{else}
\NormalTok{        logJ += +c[k] {-} }\DecValTok{2}\NormalTok{ * log(}\DecValTok{1}\NormalTok{ + exp(+c[k]));}
\NormalTok{    \}}

    \ControlFlowTok{return}\NormalTok{ dirichlet\_lpdf(p | alpha) + logJ;}
\NormalTok{  \}}
\NormalTok{\}}
\KeywordTok{data}\NormalTok{ \{}
  \DataTypeTok{int}\NormalTok{\textless{}}\KeywordTok{lower}\NormalTok{=}\DecValTok{1}\NormalTok{\textgreater{} N\_ratings;}
  \DataTypeTok{array}\NormalTok{[N\_ratings] }\DataTypeTok{int}\NormalTok{\textless{}}\KeywordTok{lower}\NormalTok{=}\DecValTok{1}\NormalTok{, }\KeywordTok{upper}\NormalTok{=}\DecValTok{5}\NormalTok{\textgreater{} ratings;}

  \DataTypeTok{int}\NormalTok{\textless{}}\KeywordTok{lower}\NormalTok{=}\DecValTok{1}\NormalTok{\textgreater{} N\_customers;}
  \DataTypeTok{array}\NormalTok{[N\_ratings] }\DataTypeTok{int}\NormalTok{\textless{}}\KeywordTok{lower}\NormalTok{=}\DecValTok{1}\NormalTok{, }\KeywordTok{upper}\NormalTok{=N\_customers\textgreater{} customer\_idxs;}

  \DataTypeTok{int}\NormalTok{\textless{}}\KeywordTok{lower}\NormalTok{=}\DecValTok{1}\NormalTok{\textgreater{} N\_movies;}
  \DataTypeTok{array}\NormalTok{[N\_ratings] }\DataTypeTok{int}\NormalTok{\textless{}}\KeywordTok{lower}\NormalTok{=}\DecValTok{1}\NormalTok{, }\KeywordTok{upper}\NormalTok{=N\_movies\textgreater{} movie\_idxs;}
\NormalTok{\}}

\KeywordTok{parameters}\NormalTok{ \{}
  \CommentTok{// Baseline movie affinities population model}
  \DataTypeTok{vector}\NormalTok{[N\_movies] gamma0\_ncp;}
  \DataTypeTok{real}\NormalTok{\textless{}}\KeywordTok{lower}\NormalTok{=}\DecValTok{0}\NormalTok{\textgreater{} tau\_gamma0;}

  \CommentTok{// Individual customer affinity population model}
  \DataTypeTok{array}\NormalTok{[N\_customers] }\DataTypeTok{vector}\NormalTok{[N\_movies] delta\_gamma\_ncp;}
  \DataTypeTok{vector}\NormalTok{\textless{}}\KeywordTok{lower}\NormalTok{=}\DecValTok{0}\NormalTok{\textgreater{}[N\_movies] tau\_delta\_gamma;}
  \DataTypeTok{cholesky\_factor\_corr}\NormalTok{[N\_movies] L\_delta\_gamma;}

  \DataTypeTok{array}\NormalTok{[N\_customers] }\DataTypeTok{ordered}\NormalTok{[}\DecValTok{4}\NormalTok{] cut\_points; }\CommentTok{// Customer rating cut points}

  \DataTypeTok{simplex}\NormalTok{[}\DecValTok{5}\NormalTok{] mu\_q;     }\CommentTok{// Rating simplex population location}
  \DataTypeTok{real}\NormalTok{\textless{}}\KeywordTok{lower}\NormalTok{=}\DecValTok{0}\NormalTok{\textgreater{} tau\_q; }\CommentTok{// Rating simplex population scale}
\NormalTok{\}}

\KeywordTok{transformed parameters}\NormalTok{ \{}
  \CommentTok{// Centered baseline movie affinities}
  \DataTypeTok{vector}\NormalTok{[N\_movies] gamma0 = tau\_gamma0 * gamma0\_ncp;}

  \CommentTok{// Centered customer movie affinities}
  \DataTypeTok{array}\NormalTok{[N\_customers] }\DataTypeTok{vector}\NormalTok{[N\_movies] delta\_gamma;}
\NormalTok{  \{}
    \DataTypeTok{matrix}\NormalTok{[N\_movies, N\_movies] L\_cov}
\NormalTok{      = diag\_pre\_multiply(tau\_delta\_gamma, L\_delta\_gamma);}
    \ControlFlowTok{for}\NormalTok{ (c }\ControlFlowTok{in} \DecValTok{1}\NormalTok{:N\_customers)}
\NormalTok{      delta\_gamma[c] = L\_cov * delta\_gamma\_ncp[c];}
\NormalTok{  \}}
\NormalTok{\}}

\KeywordTok{model}\NormalTok{ \{}
  \DataTypeTok{vector}\NormalTok{[}\DecValTok{5}\NormalTok{] ones = rep\_vector(}\DecValTok{1}\NormalTok{, }\DecValTok{5}\NormalTok{);}
  \DataTypeTok{vector}\NormalTok{[}\DecValTok{5}\NormalTok{] alpha = mu\_q / tau\_q + ones;}

  \CommentTok{// Prior model}
\NormalTok{  gamma0\_ncp \textasciitilde{} normal(}\DecValTok{0}\NormalTok{, }\DecValTok{1}\NormalTok{);}
\NormalTok{  tau\_gamma0 \textasciitilde{} normal(}\DecValTok{0}\NormalTok{, }\DecValTok{5}\NormalTok{ / }\FloatTok{2.57}\NormalTok{);}

  \ControlFlowTok{for}\NormalTok{ (r }\ControlFlowTok{in} \DecValTok{1}\NormalTok{:N\_customers)}
\NormalTok{    delta\_gamma\_ncp[r] \textasciitilde{} normal(}\DecValTok{0}\NormalTok{, }\DecValTok{1}\NormalTok{);}
\NormalTok{  tau\_delta\_gamma \textasciitilde{} normal(}\DecValTok{0}\NormalTok{, }\DecValTok{5}\NormalTok{ / }\FloatTok{2.57}\NormalTok{);}
\NormalTok{  L\_delta\_gamma \textasciitilde{} lkj\_corr\_cholesky(}\FloatTok{5.0}\NormalTok{ * sqrt(N\_movies));}

  \ControlFlowTok{for}\NormalTok{ (c }\ControlFlowTok{in} \DecValTok{1}\NormalTok{:N\_customers)}
\NormalTok{    cut\_points[c] \textasciitilde{} induced\_dirichlet(alpha);}
\NormalTok{  mu\_q \textasciitilde{} dirichlet(}\DecValTok{5}\NormalTok{ * ones);}
\NormalTok{  tau\_q \textasciitilde{} normal(}\DecValTok{0}\NormalTok{, }\DecValTok{1}\NormalTok{);}

  \CommentTok{// Observational model}
  \ControlFlowTok{for}\NormalTok{ (n }\ControlFlowTok{in} \DecValTok{1}\NormalTok{:N\_ratings) \{}
    \DataTypeTok{int}\NormalTok{ c = customer\_idxs[n];}
    \DataTypeTok{int}\NormalTok{ m = movie\_idxs[n];}
\NormalTok{    ratings[n] \textasciitilde{} ordered\_logistic(gamma0[m] + delta\_gamma[c][m],}
\NormalTok{                                  cut\_points[c]);}
\NormalTok{  \}}
\NormalTok{\}}

\KeywordTok{generated quantities}\NormalTok{ \{}
  \DataTypeTok{matrix}\NormalTok{[N\_movies, N\_movies] Phi;}

  \DataTypeTok{array}\NormalTok{[N\_ratings] }\DataTypeTok{int}\NormalTok{\textless{}}\KeywordTok{lower}\NormalTok{=}\DecValTok{1}\NormalTok{, }\KeywordTok{upper}\NormalTok{=}\DecValTok{5}\NormalTok{\textgreater{} rating\_pred;}

  \DataTypeTok{array}\NormalTok{[N\_customers] }\DataTypeTok{real}\NormalTok{ mean\_rating\_customer\_pred}
\NormalTok{    = rep\_array(}\DecValTok{0}\NormalTok{, N\_customers);}
  \DataTypeTok{array}\NormalTok{[N\_customers] }\DataTypeTok{real}\NormalTok{ var\_rating\_customer\_pred}
\NormalTok{    = rep\_array(}\DecValTok{0}\NormalTok{, N\_customers);}

  \DataTypeTok{array}\NormalTok{[N\_movies] }\DataTypeTok{real}\NormalTok{ mean\_rating\_movie\_pred = rep\_array(}\DecValTok{0}\NormalTok{, N\_movies);}
  \DataTypeTok{array}\NormalTok{[N\_movies] }\DataTypeTok{real}\NormalTok{ var\_rating\_movie\_pred = rep\_array(}\DecValTok{0}\NormalTok{, N\_movies);}

  \DataTypeTok{matrix}\NormalTok{[N\_movies, N\_movies] covar\_rating\_movie\_pred;}

\NormalTok{  \{}
    \DataTypeTok{matrix}\NormalTok{[N\_movies, N\_movies] L\_cov}
\NormalTok{      = diag\_pre\_multiply(tau\_delta\_gamma, L\_delta\_gamma);}
\NormalTok{    Phi = L\_cov * L\_cov\textquotesingle{};}
\NormalTok{  \}}

\NormalTok{  \{}
    \DataTypeTok{array}\NormalTok{[N\_customers] }\DataTypeTok{real}\NormalTok{ C = rep\_array(}\DecValTok{0}\NormalTok{, N\_customers);}
    \DataTypeTok{array}\NormalTok{[N\_movies] }\DataTypeTok{real}\NormalTok{ M = rep\_array(}\DecValTok{0}\NormalTok{, N\_movies);}

    \ControlFlowTok{for}\NormalTok{ (n }\ControlFlowTok{in} \DecValTok{1}\NormalTok{:N\_ratings) \{}
      \DataTypeTok{real}\NormalTok{ delta = }\DecValTok{0}\NormalTok{;}
      \DataTypeTok{int}\NormalTok{ c = customer\_idxs[n];}
      \DataTypeTok{int}\NormalTok{ m = movie\_idxs[n];}

\NormalTok{      rating\_pred[n]}
\NormalTok{        = ordered\_logistic\_rng(gamma0[m] + delta\_gamma[c][m],}
\NormalTok{                               cut\_points[c]);}

\NormalTok{      C[c] += }\DecValTok{1}\NormalTok{;}
\NormalTok{      delta = rating\_pred[n] {-} mean\_rating\_customer\_pred[c];}
\NormalTok{      mean\_rating\_customer\_pred[c] += delta / C[c];}
\NormalTok{      var\_rating\_customer\_pred[c]}
\NormalTok{        += delta * (rating\_pred[n] {-} mean\_rating\_customer\_pred[c]);}

\NormalTok{      M[m] += }\DecValTok{1}\NormalTok{;}
\NormalTok{      delta = rating\_pred[n] {-} mean\_rating\_movie\_pred[m];}
\NormalTok{      mean\_rating\_movie\_pred[m] += delta / M[m];}
\NormalTok{      var\_rating\_movie\_pred[m]}
\NormalTok{        += delta * (rating\_pred[n] {-} mean\_rating\_movie\_pred[m]);}
\NormalTok{    \}}

    \ControlFlowTok{for}\NormalTok{ (c }\ControlFlowTok{in} \DecValTok{1}\NormalTok{:N\_customers) \{}
      \ControlFlowTok{if}\NormalTok{ (C[c] \textgreater{} }\DecValTok{1}\NormalTok{)}
\NormalTok{        var\_rating\_customer\_pred[c] /= C[c] {-} }\DecValTok{1}\NormalTok{;}
      \ControlFlowTok{else}
\NormalTok{        var\_rating\_customer\_pred[c] = }\DecValTok{0}\NormalTok{;}
\NormalTok{    \}}
    \ControlFlowTok{for}\NormalTok{ (m }\ControlFlowTok{in} \DecValTok{1}\NormalTok{:N\_movies) \{}
      \ControlFlowTok{if}\NormalTok{ (M[m] \textgreater{} }\DecValTok{1}\NormalTok{)}
\NormalTok{        var\_rating\_movie\_pred[m] /= M[m] {-} }\DecValTok{1}\NormalTok{;}
      \ControlFlowTok{else}
\NormalTok{        var\_rating\_movie\_pred[m] = }\DecValTok{0}\NormalTok{;}
\NormalTok{    \}}
\NormalTok{  \}}

\NormalTok{  \{}
    \DataTypeTok{matrix}\NormalTok{[N\_movies, N\_movies] counts;}

    \ControlFlowTok{for}\NormalTok{ (m1 }\ControlFlowTok{in} \DecValTok{1}\NormalTok{:N\_movies) \{}
      \ControlFlowTok{for}\NormalTok{ (m2 }\ControlFlowTok{in} \DecValTok{1}\NormalTok{:N\_movies) \{}
\NormalTok{        counts[m1, m2] = }\DecValTok{0}\NormalTok{;}
\NormalTok{        covar\_rating\_movie\_pred[m1, m2] = }\DecValTok{0}\NormalTok{;}
\NormalTok{      \}}
\NormalTok{    \}}

    \ControlFlowTok{for}\NormalTok{ (n1 }\ControlFlowTok{in} \DecValTok{1}\NormalTok{:N\_ratings) \{}
      \ControlFlowTok{for}\NormalTok{ (n2 }\ControlFlowTok{in} \DecValTok{1}\NormalTok{:N\_ratings) \{}
        \ControlFlowTok{if}\NormalTok{ (customer\_idxs[n1] == customer\_idxs[n2]) \{}
          \DataTypeTok{int}\NormalTok{ m1 = movie\_idxs[n1];}
          \DataTypeTok{int}\NormalTok{ m2 = movie\_idxs[n2];}
          \DataTypeTok{real}\NormalTok{ y =   (ratings[n1] {-} mean\_rating\_movie\_pred[m1])}
\NormalTok{                   * (ratings[n2] {-} mean\_rating\_movie\_pred[m2]);}
\NormalTok{          covar\_rating\_movie\_pred[m1, m2] += y;}
\NormalTok{          covar\_rating\_movie\_pred[m2, m1] += y;}
\NormalTok{          counts[m1, m2] += }\DecValTok{1}\NormalTok{;}
\NormalTok{          counts[m2, m1] += }\DecValTok{1}\NormalTok{;}
\NormalTok{        \}}
\NormalTok{      \}}
\NormalTok{    \}}

    \ControlFlowTok{for}\NormalTok{ (m1 }\ControlFlowTok{in} \DecValTok{1}\NormalTok{:N\_movies) \{}
      \ControlFlowTok{for}\NormalTok{ (m2 }\ControlFlowTok{in} \DecValTok{1}\NormalTok{:N\_movies) \{}
        \ControlFlowTok{if}\NormalTok{ (counts[m1, m2] \textgreater{} }\DecValTok{1}\NormalTok{)}
\NormalTok{          covar\_rating\_movie\_pred[m1, m2] /= counts[m1, m2] {-} }\DecValTok{1}\NormalTok{;}
\NormalTok{      \}}
\NormalTok{    \}}
\NormalTok{  \}}
\NormalTok{\}}
\end{Highlighting}
\end{Shaded}

\end{codelisting}

\begin{Shaded}
\begin{Highlighting}[]
\NormalTok{fit }\OtherTok{\textless{}{-}} \FunctionTok{stan}\NormalTok{(}\AttributeTok{file=}\StringTok{"stan\_programs/model4.stan"}\NormalTok{,}
            \AttributeTok{data=}\NormalTok{data, }\AttributeTok{seed=}\DecValTok{8438338}\NormalTok{, }\AttributeTok{init=}\DecValTok{0}\NormalTok{,}
            \AttributeTok{warmup=}\DecValTok{1000}\NormalTok{, }\AttributeTok{iter=}\DecValTok{2024}\NormalTok{, }\AttributeTok{refresh=}\DecValTok{0}\NormalTok{)}
\end{Highlighting}
\end{Shaded}

People complain about frustration with diagnostic warnings, but how bad
can they be if we don't see serious warnings for this complex model with
40,706 degrees of freedom? Turns out Hamiltonian Monte Carlo is pretty
good!

\begin{Shaded}
\begin{Highlighting}[]
\NormalTok{diagnostics }\OtherTok{\textless{}{-}}\NormalTok{ util}\SpecialCharTok{$}\FunctionTok{extract\_hmc\_diagnostics}\NormalTok{(fit)}
\NormalTok{util}\SpecialCharTok{$}\FunctionTok{check\_all\_hmc\_diagnostics}\NormalTok{(diagnostics)}
\end{Highlighting}
\end{Shaded}

\begin{verbatim}
  All Hamiltonian Monte Carlo diagnostics are consistent with reliable
Markov chain Monte Carlo.
\end{verbatim}

\begin{Shaded}
\begin{Highlighting}[]
\NormalTok{samples4 }\OtherTok{\textless{}{-}}\NormalTok{ util}\SpecialCharTok{$}\FunctionTok{extract\_expectand\_vals}\NormalTok{(fit)}
\NormalTok{base\_samples }\OtherTok{\textless{}{-}}\NormalTok{ util}\SpecialCharTok{$}\FunctionTok{filter\_expectands}\NormalTok{(samples4,}
                                       \FunctionTok{c}\NormalTok{(}\StringTok{\textquotesingle{}gamma0\_ncp\textquotesingle{}}\NormalTok{,}
                                         \StringTok{\textquotesingle{}tau\_gamma0\textquotesingle{}}\NormalTok{,}
                                         \StringTok{\textquotesingle{}delta\_gamma\_ncp\textquotesingle{}}\NormalTok{,}
                                         \StringTok{\textquotesingle{}tau\_delta\_gamma\textquotesingle{}}\NormalTok{,}
                                         \StringTok{\textquotesingle{}L\_delta\_gamma\textquotesingle{}}\NormalTok{,}
                                         \StringTok{\textquotesingle{}cut\_points\textquotesingle{}}\NormalTok{,}
                                         \StringTok{\textquotesingle{}mu\_q\textquotesingle{}}\NormalTok{, }\StringTok{\textquotesingle{}tau\_q\textquotesingle{}}\NormalTok{),}
                                       \AttributeTok{check\_arrays=}\ConstantTok{TRUE}\NormalTok{)}
\NormalTok{util}\SpecialCharTok{$}\FunctionTok{check\_all\_expectand\_diagnostics}\NormalTok{(base\_samples,}
                                     \AttributeTok{exclude\_zvar=}\ConstantTok{TRUE}\NormalTok{)}
\end{Highlighting}
\end{Shaded}

\begin{verbatim}
tau_delta_gamma[97]:
  Chain 1: hat{ESS} (39.565) is smaller than desired (100).


Small empirical effective sample sizes result in imprecise Markov chain
Monte Carlo estimators.
\end{verbatim}

Unfortunately all of this added sophisticated doesn't actually seem to
improve the retrodictive performance much. Even the retrodictive tension
in the empirical covariances, which should be particularly sensitive to
added flexibility in the new model, is similar to what we saw with the
previous model. One possibility is that the multivariate normal
population model is isn't sufficiently heavy-tailed to accommodate the
more extreme tastes of the customers.

\begin{Shaded}
\begin{Highlighting}[]
\FunctionTok{par}\NormalTok{(}\AttributeTok{mfrow=}\FunctionTok{c}\NormalTok{(}\DecValTok{1}\NormalTok{, }\DecValTok{1}\NormalTok{), }\AttributeTok{mar=}\FunctionTok{c}\NormalTok{(}\DecValTok{5}\NormalTok{, }\DecValTok{5}\NormalTok{, }\DecValTok{1}\NormalTok{, }\DecValTok{1}\NormalTok{))}

\NormalTok{util}\SpecialCharTok{$}\FunctionTok{plot\_hist\_quantiles}\NormalTok{(samples4, }\StringTok{\textquotesingle{}rating\_pred\textquotesingle{}}\NormalTok{, }\SpecialCharTok{{-}}\FloatTok{0.5}\NormalTok{, }\FloatTok{6.5}\NormalTok{, }\DecValTok{1}\NormalTok{,}
                         \AttributeTok{baseline\_values=}\NormalTok{data}\SpecialCharTok{$}\NormalTok{ratings,}
                         \AttributeTok{xlab=}\StringTok{"All Ratings"}\NormalTok{)}
\end{Highlighting}
\end{Shaded}

\includegraphics{ratings_files/figure-pdf/unnamed-chunk-60-1.pdf}

\begin{Shaded}
\begin{Highlighting}[]
\FunctionTok{par}\NormalTok{(}\AttributeTok{mfrow=}\FunctionTok{c}\NormalTok{(}\DecValTok{2}\NormalTok{, }\DecValTok{3}\NormalTok{), }\AttributeTok{mar=}\FunctionTok{c}\NormalTok{(}\DecValTok{5}\NormalTok{, }\DecValTok{5}\NormalTok{, }\DecValTok{1}\NormalTok{, }\DecValTok{1}\NormalTok{))}

\ControlFlowTok{for}\NormalTok{ (c }\ControlFlowTok{in} \FunctionTok{c}\NormalTok{(}\DecValTok{7}\NormalTok{, }\DecValTok{23}\NormalTok{, }\DecValTok{40}\NormalTok{, }\DecValTok{70}\NormalTok{, }\DecValTok{77}\NormalTok{, }\DecValTok{100}\NormalTok{)) \{}
\NormalTok{  names }\OtherTok{\textless{}{-}} \FunctionTok{sapply}\NormalTok{(}\FunctionTok{which}\NormalTok{(data}\SpecialCharTok{$}\NormalTok{customer\_idxs }\SpecialCharTok{==}\NormalTok{ c),}
                  \ControlFlowTok{function}\NormalTok{(n) }\FunctionTok{paste0}\NormalTok{(}\StringTok{\textquotesingle{}rating\_pred[\textquotesingle{}}\NormalTok{, n, }\StringTok{\textquotesingle{}]\textquotesingle{}}\NormalTok{))}
\NormalTok{  filtered\_samples }\OtherTok{\textless{}{-}}\NormalTok{ util}\SpecialCharTok{$}\FunctionTok{filter\_expectands}\NormalTok{(samples4, names)}

\NormalTok{  customer\_ratings }\OtherTok{\textless{}{-}}\NormalTok{ data}\SpecialCharTok{$}\NormalTok{ratings[data}\SpecialCharTok{$}\NormalTok{customer\_idxs }\SpecialCharTok{==}\NormalTok{ c]}
\NormalTok{  util}\SpecialCharTok{$}\FunctionTok{plot\_hist\_quantiles}\NormalTok{(filtered\_samples, }\StringTok{\textquotesingle{}rating\_pred\textquotesingle{}}\NormalTok{,}
                           \SpecialCharTok{{-}}\FloatTok{0.5}\NormalTok{, }\FloatTok{6.5}\NormalTok{, }\DecValTok{1}\NormalTok{,}
                           \AttributeTok{baseline\_values=}\NormalTok{customer\_ratings,}
                           \AttributeTok{xlab=}\StringTok{"Ratings"}\NormalTok{,}
                           \AttributeTok{main=}\FunctionTok{paste}\NormalTok{(}\StringTok{\textquotesingle{}Customer\textquotesingle{}}\NormalTok{, c))}
\NormalTok{\}}
\end{Highlighting}
\end{Shaded}

\includegraphics{ratings_files/figure-pdf/unnamed-chunk-61-1.pdf}

\begin{Shaded}
\begin{Highlighting}[]
\FunctionTok{par}\NormalTok{(}\AttributeTok{mfrow=}\FunctionTok{c}\NormalTok{(}\DecValTok{2}\NormalTok{, }\DecValTok{3}\NormalTok{), }\AttributeTok{mar=}\FunctionTok{c}\NormalTok{(}\DecValTok{5}\NormalTok{, }\DecValTok{5}\NormalTok{, }\DecValTok{1}\NormalTok{, }\DecValTok{1}\NormalTok{))}

\ControlFlowTok{for}\NormalTok{ (m }\ControlFlowTok{in} \FunctionTok{c}\NormalTok{(}\DecValTok{33}\NormalTok{, }\DecValTok{53}\NormalTok{, }\DecValTok{61}\NormalTok{, }\DecValTok{80}\NormalTok{, }\DecValTok{117}\NormalTok{, }\DecValTok{180}\NormalTok{)) \{}
\NormalTok{  names }\OtherTok{\textless{}{-}} \FunctionTok{sapply}\NormalTok{(}\FunctionTok{which}\NormalTok{(data}\SpecialCharTok{$}\NormalTok{movie\_idxs }\SpecialCharTok{==}\NormalTok{ m),}
                  \ControlFlowTok{function}\NormalTok{(n) }\FunctionTok{paste0}\NormalTok{(}\StringTok{\textquotesingle{}rating\_pred[\textquotesingle{}}\NormalTok{, n, }\StringTok{\textquotesingle{}]\textquotesingle{}}\NormalTok{))}
\NormalTok{  filtered\_samples }\OtherTok{\textless{}{-}}\NormalTok{ util}\SpecialCharTok{$}\FunctionTok{filter\_expectands}\NormalTok{(samples4, names)}

\NormalTok{  movie\_ratings }\OtherTok{\textless{}{-}}\NormalTok{ data}\SpecialCharTok{$}\NormalTok{ratings[data}\SpecialCharTok{$}\NormalTok{movie\_idxs }\SpecialCharTok{==}\NormalTok{ m]}
\NormalTok{  util}\SpecialCharTok{$}\FunctionTok{plot\_hist\_quantiles}\NormalTok{(filtered\_samples, }\StringTok{\textquotesingle{}rating\_pred\textquotesingle{}}\NormalTok{,}
                           \SpecialCharTok{{-}}\FloatTok{0.5}\NormalTok{, }\FloatTok{6.5}\NormalTok{, }\DecValTok{1}\NormalTok{,}
                           \AttributeTok{baseline\_values=}\NormalTok{movie\_ratings,}
                           \AttributeTok{xlab=}\StringTok{"Ratings"}\NormalTok{,}
                           \AttributeTok{main=}\FunctionTok{paste}\NormalTok{(}\StringTok{\textquotesingle{}Movie\textquotesingle{}}\NormalTok{, m))}
\NormalTok{\}}
\end{Highlighting}
\end{Shaded}

\includegraphics{ratings_files/figure-pdf/unnamed-chunk-62-1.pdf}

\begin{Shaded}
\begin{Highlighting}[]
\FunctionTok{par}\NormalTok{(}\AttributeTok{mfrow=}\FunctionTok{c}\NormalTok{(}\DecValTok{2}\NormalTok{, }\DecValTok{2}\NormalTok{), }\AttributeTok{mar=}\FunctionTok{c}\NormalTok{(}\DecValTok{5}\NormalTok{, }\DecValTok{5}\NormalTok{, }\DecValTok{1}\NormalTok{, }\DecValTok{1}\NormalTok{))}

\NormalTok{util}\SpecialCharTok{$}\FunctionTok{plot\_hist\_quantiles}\NormalTok{(samples4, }\StringTok{\textquotesingle{}mean\_rating\_customer\_pred\textquotesingle{}}\NormalTok{,}
                         \DecValTok{0}\NormalTok{, }\DecValTok{6}\NormalTok{, }\FloatTok{0.5}\NormalTok{,}
                         \AttributeTok{baseline\_values=}\NormalTok{mean\_rating\_customer,}
                         \AttributeTok{xlab=}\StringTok{"Customer{-}wise Average Ratings"}\NormalTok{)}

\NormalTok{util}\SpecialCharTok{$}\FunctionTok{plot\_hist\_quantiles}\NormalTok{(samples4, }\StringTok{\textquotesingle{}mean\_rating\_movie\_pred\textquotesingle{}}\NormalTok{,}
                         \DecValTok{0}\NormalTok{, }\DecValTok{6}\NormalTok{, }\FloatTok{0.6}\NormalTok{,}
                         \AttributeTok{baseline\_values=}\NormalTok{mean\_rating\_movie,}
                         \AttributeTok{xlab=}\StringTok{"Movie{-}wise Average Ratings"}\NormalTok{)}

\NormalTok{util}\SpecialCharTok{$}\FunctionTok{plot\_hist\_quantiles}\NormalTok{(samples4, }\StringTok{\textquotesingle{}var\_rating\_customer\_pred\textquotesingle{}}\NormalTok{,}
                         \DecValTok{0}\NormalTok{, }\DecValTok{7}\NormalTok{, }\FloatTok{0.5}\NormalTok{,}
                         \AttributeTok{baseline\_values=}\NormalTok{var\_rating\_customer,}
                         \AttributeTok{xlab=}\StringTok{"Customer{-}wise Rating Variances"}\NormalTok{)}
\end{Highlighting}
\end{Shaded}

\begin{verbatim}
Warning in check_bin_containment(bin_min, bin_max, collapsed_values,
"predictive value"): 307 predictive values (0.1%) fell above the binning.
\end{verbatim}

\begin{Shaded}
\begin{Highlighting}[]
\NormalTok{util}\SpecialCharTok{$}\FunctionTok{plot\_hist\_quantiles}\NormalTok{(samples4, }\StringTok{\textquotesingle{}var\_rating\_movie\_pred\textquotesingle{}}\NormalTok{,}
                         \DecValTok{0}\NormalTok{, }\DecValTok{7}\NormalTok{, }\FloatTok{0.5}\NormalTok{,}
                         \AttributeTok{baseline\_values=}\NormalTok{var\_rating\_movie,}
                         \AttributeTok{xlab=}\StringTok{"Movie{-}wise Rating Variances"}\NormalTok{)}
\end{Highlighting}
\end{Shaded}

\begin{verbatim}
Warning in check_bin_containment(bin_min, bin_max, collapsed_values,
"predictive value"): 2321 predictive values (0.3%) fell above the binning.
\end{verbatim}

\includegraphics{ratings_files/figure-pdf/unnamed-chunk-63-1.pdf}

\begin{Shaded}
\begin{Highlighting}[]
\FunctionTok{par}\NormalTok{(}\AttributeTok{mfrow=}\FunctionTok{c}\NormalTok{(}\DecValTok{1}\NormalTok{, }\DecValTok{1}\NormalTok{), }\AttributeTok{mar=}\FunctionTok{c}\NormalTok{(}\DecValTok{5}\NormalTok{, }\DecValTok{5}\NormalTok{, }\DecValTok{1}\NormalTok{, }\DecValTok{1}\NormalTok{))}

\NormalTok{filtered\_samples }\OtherTok{\textless{}{-}}
\NormalTok{  util}\SpecialCharTok{$}\FunctionTok{filter\_expectands}\NormalTok{(samples4,}
\NormalTok{                         covar\_rating\_movie\_filt\_names)}

\NormalTok{util}\SpecialCharTok{$}\FunctionTok{plot\_hist\_quantiles}\NormalTok{(filtered\_samples, }\StringTok{\textquotesingle{}covar\_rating\_movie\_pred\textquotesingle{}}\NormalTok{,}
                         \SpecialCharTok{{-}}\FloatTok{4.25}\NormalTok{, }\FloatTok{4.25}\NormalTok{, }\FloatTok{0.25}\NormalTok{,}
                         \AttributeTok{baseline\_values=}\NormalTok{covar\_rating\_movie\_filt,}
                         \AttributeTok{xlab=}\StringTok{"Filtered Movie{-}wise Rating Covariances"}\NormalTok{)}
\end{Highlighting}
\end{Shaded}

\begin{verbatim}
Warning in check_bin_containment(bin_min, bin_max, collapsed_values,
"predictive value"): 113 predictive values (0.0%) fell below the binning.
\end{verbatim}

\begin{verbatim}
Warning in check_bin_containment(bin_min, bin_max, collapsed_values,
"predictive value"): 95 predictive values (0.0%) fell above the binning.
\end{verbatim}

\includegraphics{ratings_files/figure-pdf/unnamed-chunk-64-1.pdf}

The lack of any substantial improvements in the retrodictive performance
suggests that we might not have included enough data to really resolve
individual customer preferences quite yet. We can directly quantify how
well we can resolve individual customer preferences by examining our
posterior inferences.

Posterior inferences for the interior cut point population behaviors are
mostly consistent with the previous model, although the baseline rating
probabilities do slightly shift down to be more centered around 3. This
suggest that the previous model may have been contorting itself a bit to
account for the variation in customer tastes.

\begin{Shaded}
\begin{Highlighting}[]
\FunctionTok{par}\NormalTok{(}\AttributeTok{mfrow=}\FunctionTok{c}\NormalTok{(}\DecValTok{1}\NormalTok{, }\DecValTok{1}\NormalTok{), }\AttributeTok{mar=}\FunctionTok{c}\NormalTok{(}\DecValTok{5}\NormalTok{, }\DecValTok{5}\NormalTok{, }\DecValTok{1}\NormalTok{, }\DecValTok{1}\NormalTok{))}

\NormalTok{util}\SpecialCharTok{$}\FunctionTok{plot\_expectand\_pushforward}\NormalTok{(samples3[[}\StringTok{\textquotesingle{}tau\_q\textquotesingle{}}\NormalTok{]],}
                                \DecValTok{25}\NormalTok{, }\AttributeTok{flim=}\FunctionTok{c}\NormalTok{(}\DecValTok{0}\NormalTok{, }\FloatTok{0.3}\NormalTok{),}
                                \AttributeTok{col=}\NormalTok{util}\SpecialCharTok{$}\NormalTok{c\_light,}
                                \AttributeTok{display\_name=}\StringTok{\textquotesingle{}tau\_q\textquotesingle{}}\NormalTok{)}
\FunctionTok{text}\NormalTok{(}\FloatTok{0.05}\NormalTok{, }\DecValTok{10}\NormalTok{, }\StringTok{"Model 2"}\NormalTok{, }\AttributeTok{col=}\NormalTok{util}\SpecialCharTok{$}\NormalTok{c\_light)}

\NormalTok{util}\SpecialCharTok{$}\FunctionTok{plot\_expectand\_pushforward}\NormalTok{(samples4[[}\StringTok{\textquotesingle{}tau\_q\textquotesingle{}}\NormalTok{]],}
                                \DecValTok{25}\NormalTok{, }\AttributeTok{flim=}\FunctionTok{c}\NormalTok{(}\DecValTok{0}\NormalTok{, }\FloatTok{0.3}\NormalTok{),}
                                \AttributeTok{col=}\NormalTok{util}\SpecialCharTok{$}\NormalTok{c\_dark,}
                                \AttributeTok{border=}\StringTok{"\#BBBBBB88"}\NormalTok{,}
                                \AttributeTok{add=}\ConstantTok{TRUE}\NormalTok{)}
\FunctionTok{text}\NormalTok{(}\FloatTok{0.2}\NormalTok{, }\DecValTok{10}\NormalTok{, }\StringTok{"Model 3"}\NormalTok{, }\AttributeTok{col=}\NormalTok{util}\SpecialCharTok{$}\NormalTok{c\_dark)}
\end{Highlighting}
\end{Shaded}

\includegraphics{ratings_files/figure-pdf/unnamed-chunk-65-1.pdf}

\begin{Shaded}
\begin{Highlighting}[]
\FunctionTok{par}\NormalTok{(}\AttributeTok{mfrow=}\FunctionTok{c}\NormalTok{(}\DecValTok{2}\NormalTok{, }\DecValTok{1}\NormalTok{), }\AttributeTok{mar=}\FunctionTok{c}\NormalTok{(}\DecValTok{5}\NormalTok{, }\DecValTok{5}\NormalTok{, }\DecValTok{1}\NormalTok{, }\DecValTok{1}\NormalTok{))}

\NormalTok{names }\OtherTok{\textless{}{-}} \FunctionTok{sapply}\NormalTok{(}\DecValTok{1}\SpecialCharTok{:}\DecValTok{5}\NormalTok{, }\ControlFlowTok{function}\NormalTok{(k) }\FunctionTok{paste0}\NormalTok{(}\StringTok{\textquotesingle{}mu\_q[\textquotesingle{}}\NormalTok{, k, }\StringTok{\textquotesingle{}]\textquotesingle{}}\NormalTok{))}
\NormalTok{util}\SpecialCharTok{$}\FunctionTok{plot\_disc\_pushforward\_quantiles}\NormalTok{(samples3, names,}
                                     \AttributeTok{xlab=}\StringTok{"Rating"}\NormalTok{,}
                                     \AttributeTok{ylab=}\StringTok{"Baseline Rating Probability"}\NormalTok{,}
                                     \AttributeTok{main=}\StringTok{"Model 3"}\NormalTok{)}

\NormalTok{names }\OtherTok{\textless{}{-}} \FunctionTok{sapply}\NormalTok{(}\DecValTok{1}\SpecialCharTok{:}\DecValTok{5}\NormalTok{, }\ControlFlowTok{function}\NormalTok{(k) }\FunctionTok{paste0}\NormalTok{(}\StringTok{\textquotesingle{}mu\_q[\textquotesingle{}}\NormalTok{, k, }\StringTok{\textquotesingle{}]\textquotesingle{}}\NormalTok{))}
\NormalTok{util}\SpecialCharTok{$}\FunctionTok{plot\_disc\_pushforward\_quantiles}\NormalTok{(samples4, names,}
                                     \AttributeTok{xlab=}\StringTok{"Rating"}\NormalTok{,}
                                     \AttributeTok{ylab=}\StringTok{"Baseline Rating Probability"}\NormalTok{,}
                                     \AttributeTok{main=}\StringTok{"Model 4"}\NormalTok{)}
\end{Highlighting}
\end{Shaded}

\includegraphics{ratings_files/figure-pdf/unnamed-chunk-66-1.pdf}

Similarly some of the individual customer interior cut points change
slightly. For example the cut points for Customer 23 shift a bit towards
larger values.

\begin{Shaded}
\begin{Highlighting}[]
\FunctionTok{par}\NormalTok{(}\AttributeTok{mfrow=}\FunctionTok{c}\NormalTok{(}\DecValTok{2}\NormalTok{, }\DecValTok{2}\NormalTok{), }\AttributeTok{mar=}\FunctionTok{c}\NormalTok{(}\DecValTok{5}\NormalTok{, }\DecValTok{5}\NormalTok{, }\DecValTok{1}\NormalTok{, }\DecValTok{1}\NormalTok{))}

\NormalTok{c }\OtherTok{\textless{}{-}} \DecValTok{23}

\NormalTok{lab3\_xs }\OtherTok{\textless{}{-}} \FunctionTok{c}\NormalTok{(}\DecValTok{2}\NormalTok{, }\SpecialCharTok{{-}}\FloatTok{0.5}\NormalTok{, }\DecValTok{0}\NormalTok{, }\DecValTok{1}\NormalTok{)}
\NormalTok{lab3\_ys }\OtherTok{\textless{}{-}} \FunctionTok{c}\NormalTok{(}\FloatTok{1.75}\NormalTok{, }\FloatTok{0.5}\NormalTok{, }\FloatTok{0.5}\NormalTok{, }\FloatTok{0.25}\NormalTok{)}

\NormalTok{lab4\_xs }\OtherTok{\textless{}{-}} \FunctionTok{c}\NormalTok{(}\DecValTok{2}\NormalTok{, }\DecValTok{4}\NormalTok{, }\FloatTok{5.5}\NormalTok{, }\DecValTok{8}\NormalTok{)}
\NormalTok{lab4\_ys }\OtherTok{\textless{}{-}} \FunctionTok{c}\NormalTok{(}\FloatTok{0.5}\NormalTok{, }\FloatTok{0.5}\NormalTok{, }\FloatTok{0.5}\NormalTok{, }\FloatTok{0.25}\NormalTok{)}

\ControlFlowTok{for}\NormalTok{ (k }\ControlFlowTok{in} \DecValTok{1}\SpecialCharTok{:}\DecValTok{4}\NormalTok{) \{}
\NormalTok{  name }\OtherTok{\textless{}{-}}\FunctionTok{paste0}\NormalTok{(}\StringTok{\textquotesingle{}cut\_points[\textquotesingle{}}\NormalTok{, c, }\StringTok{\textquotesingle{},\textquotesingle{}}\NormalTok{, k, }\StringTok{\textquotesingle{}]\textquotesingle{}}\NormalTok{)}
\NormalTok{  util}\SpecialCharTok{$}\FunctionTok{plot\_expectand\_pushforward}\NormalTok{(samples3[[name]],}
                                  \DecValTok{40}\NormalTok{, }\AttributeTok{flim=}\FunctionTok{c}\NormalTok{(}\SpecialCharTok{{-}}\DecValTok{2}\NormalTok{, }\DecValTok{10}\NormalTok{),}
                                  \AttributeTok{col=}\NormalTok{util}\SpecialCharTok{$}\NormalTok{c\_light,}
                                  \AttributeTok{display\_name=}\NormalTok{name)}
\NormalTok{  util}\SpecialCharTok{$}\FunctionTok{plot\_expectand\_pushforward}\NormalTok{(samples4[[name]],}
                                  \DecValTok{40}\NormalTok{, }\AttributeTok{flim=}\FunctionTok{c}\NormalTok{(}\SpecialCharTok{{-}}\DecValTok{2}\NormalTok{, }\DecValTok{10}\NormalTok{),}
                                  \AttributeTok{col=}\NormalTok{util}\SpecialCharTok{$}\NormalTok{c\_dark,}
                                  \AttributeTok{border=}\StringTok{"\#BBBBBB88"}\NormalTok{,}
                                  \AttributeTok{add=}\ConstantTok{TRUE}\NormalTok{)}

  \FunctionTok{text}\NormalTok{(lab3\_xs[k], lab3\_ys[k], }\StringTok{"Model 3"}\NormalTok{, }\AttributeTok{col=}\NormalTok{util}\SpecialCharTok{$}\NormalTok{c\_light)}
  \FunctionTok{text}\NormalTok{(lab4\_xs[k], lab4\_ys[k], }\StringTok{"Model 4"}\NormalTok{, }\AttributeTok{col=}\NormalTok{util}\SpecialCharTok{$}\NormalTok{c\_dark)}
\NormalTok{\}}
\end{Highlighting}
\end{Shaded}

\includegraphics{ratings_files/figure-pdf/unnamed-chunk-67-1.pdf}

The baseline movie affinities emulate the universal movie preferences in
the previous model and indeed inferences for them are similar, if
slightly more heterogeneous.

\begin{Shaded}
\begin{Highlighting}[]
\FunctionTok{par}\NormalTok{(}\AttributeTok{mfrow=}\FunctionTok{c}\NormalTok{(}\DecValTok{2}\NormalTok{, }\DecValTok{2}\NormalTok{), }\AttributeTok{mar=}\FunctionTok{c}\NormalTok{(}\DecValTok{5}\NormalTok{, }\DecValTok{5}\NormalTok{, }\DecValTok{1}\NormalTok{, }\DecValTok{1}\NormalTok{))}

\NormalTok{util}\SpecialCharTok{$}\FunctionTok{plot\_expectand\_pushforward}\NormalTok{(samples3[[}\StringTok{\textquotesingle{}tau\_gamma\textquotesingle{}}\NormalTok{]],}
                                \DecValTok{25}\NormalTok{, }\AttributeTok{flim=}\FunctionTok{c}\NormalTok{(}\DecValTok{0}\NormalTok{, }\FloatTok{1.25}\NormalTok{),}
                                \AttributeTok{display\_name=}\StringTok{"tau\_gamma"}\NormalTok{,}
                                \AttributeTok{main=}\StringTok{"Model 3"}\NormalTok{)}

\NormalTok{names }\OtherTok{\textless{}{-}} \FunctionTok{sapply}\NormalTok{(}\DecValTok{1}\SpecialCharTok{:}\NormalTok{data}\SpecialCharTok{$}\NormalTok{N\_movies,}
                \ControlFlowTok{function}\NormalTok{(m) }\FunctionTok{paste0}\NormalTok{(}\StringTok{\textquotesingle{}gamma[\textquotesingle{}}\NormalTok{, m, }\StringTok{\textquotesingle{}]\textquotesingle{}}\NormalTok{))}
\NormalTok{util}\SpecialCharTok{$}\FunctionTok{plot\_disc\_pushforward\_quantiles}\NormalTok{(samples3, names,}
                                     \AttributeTok{xlab=}\StringTok{"Movie"}\NormalTok{,}
                                     \AttributeTok{ylab=}\StringTok{"Affinities"}\NormalTok{,}
                                     \AttributeTok{main=}\StringTok{"Model 3"}\NormalTok{)}


\NormalTok{util}\SpecialCharTok{$}\FunctionTok{plot\_expectand\_pushforward}\NormalTok{(samples4[[}\StringTok{\textquotesingle{}tau\_gamma0\textquotesingle{}}\NormalTok{]],}
                                \DecValTok{25}\NormalTok{, }\AttributeTok{flim=}\FunctionTok{c}\NormalTok{(}\DecValTok{0}\NormalTok{, }\FloatTok{1.25}\NormalTok{),}
                                \AttributeTok{display\_name=}\StringTok{"tau\_gamma0"}\NormalTok{,}
                                \AttributeTok{main=}\StringTok{"Model 4"}\NormalTok{)}

\NormalTok{names }\OtherTok{\textless{}{-}} \FunctionTok{sapply}\NormalTok{(}\DecValTok{1}\SpecialCharTok{:}\NormalTok{data}\SpecialCharTok{$}\NormalTok{N\_movies,}
                \ControlFlowTok{function}\NormalTok{(m) }\FunctionTok{paste0}\NormalTok{(}\StringTok{\textquotesingle{}gamma0[\textquotesingle{}}\NormalTok{, m, }\StringTok{\textquotesingle{}]\textquotesingle{}}\NormalTok{))}
\NormalTok{util}\SpecialCharTok{$}\FunctionTok{plot\_disc\_pushforward\_quantiles}\NormalTok{(samples4, names,}
                                     \AttributeTok{xlab=}\StringTok{"Movie"}\NormalTok{,}
                                     \AttributeTok{ylab=}\StringTok{"Baseline Affinities"}\NormalTok{,}
                                     \AttributeTok{main=}\StringTok{"Model 4"}\NormalTok{)}
\end{Highlighting}
\end{Shaded}

\includegraphics{ratings_files/figure-pdf/unnamed-chunk-68-1.pdf}

Now, however, we can investigate the preferences idiosyncratic to each
customer. For example the individual movie affinity scales quantify how
much the customers disagree about a particular movie.

\begin{Shaded}
\begin{Highlighting}[]
\FunctionTok{par}\NormalTok{(}\AttributeTok{mfrow=}\FunctionTok{c}\NormalTok{(}\DecValTok{1}\NormalTok{, }\DecValTok{1}\NormalTok{), }\AttributeTok{mar=}\FunctionTok{c}\NormalTok{(}\DecValTok{5}\NormalTok{, }\DecValTok{5}\NormalTok{, }\DecValTok{1}\NormalTok{, }\DecValTok{1}\NormalTok{))}

\NormalTok{names }\OtherTok{\textless{}{-}} \FunctionTok{sapply}\NormalTok{(}\DecValTok{1}\SpecialCharTok{:}\NormalTok{data}\SpecialCharTok{$}\NormalTok{N\_movies,}
                \ControlFlowTok{function}\NormalTok{(m) }\FunctionTok{paste0}\NormalTok{(}\StringTok{\textquotesingle{}tau\_delta\_gamma[\textquotesingle{}}\NormalTok{, m, }\StringTok{\textquotesingle{}]\textquotesingle{}}\NormalTok{))}
\NormalTok{util}\SpecialCharTok{$}\FunctionTok{plot\_disc\_pushforward\_quantiles}\NormalTok{(samples4, names,}
                                     \AttributeTok{xlab=}\StringTok{"Movie"}\NormalTok{,}
                                     \AttributeTok{ylab=}\StringTok{"Affinity Variation Scales"}\NormalTok{)}
\end{Highlighting}
\end{Shaded}

\includegraphics{ratings_files/figure-pdf/unnamed-chunk-69-1.pdf}

Overall there is a lot of uncertainty, with the inferences for most of
the movie affinity scales concentrating around zero. That said a few
stand out. For example the posterior inferences for
\(\tau_{\gamma, 159}\) are starting to pull away from zero, suggesting
that customers tend to disagree about the quality of this movie more
than usual.

\begin{Shaded}
\begin{Highlighting}[]
\FunctionTok{par}\NormalTok{(}\AttributeTok{mfrow=}\FunctionTok{c}\NormalTok{(}\DecValTok{1}\NormalTok{, }\DecValTok{1}\NormalTok{), }\AttributeTok{mar=}\FunctionTok{c}\NormalTok{(}\DecValTok{5}\NormalTok{, }\DecValTok{5}\NormalTok{, }\DecValTok{1}\NormalTok{, }\DecValTok{1}\NormalTok{))}

\NormalTok{m }\OtherTok{\textless{}{-}} \DecValTok{159}
\NormalTok{name }\OtherTok{\textless{}{-}} \FunctionTok{paste0}\NormalTok{(}\StringTok{\textquotesingle{}tau\_delta\_gamma[\textquotesingle{}}\NormalTok{, m, }\StringTok{\textquotesingle{}]\textquotesingle{}}\NormalTok{)}
\NormalTok{util}\SpecialCharTok{$}\FunctionTok{plot\_expectand\_pushforward}\NormalTok{(samples4[[name]],}
                                \DecValTok{25}\NormalTok{, }\AttributeTok{flim=}\FunctionTok{c}\NormalTok{(}\DecValTok{0}\NormalTok{, }\DecValTok{10}\NormalTok{),}
                                \AttributeTok{display\_name=}\NormalTok{name)}
\end{Highlighting}
\end{Shaded}

\includegraphics{ratings_files/figure-pdf/unnamed-chunk-70-1.pdf}

The inferred correlations in the multivariate normal population model
allow us to inform predictions for how a customer would react to unrated
movies given the movies they have rated. While most of the correlations
are consistent with zero, here approximated by how much of their
marginal posterior probability concentrates on values below 0.05, a few
are consistent with larger values.

Note that I've had to break out some custom, heavily-optimized code here
to calculate the 19,900 posterior probabilities reasonably efficiently.

\begin{Shaded}
\begin{Highlighting}[]
\NormalTok{apply\_pushforward\_expectation }\OtherTok{\textless{}{-}} \ControlFlowTok{function}\NormalTok{(expectand\_vals\_list,}
\NormalTok{                                          pushforward\_expectand,}
\NormalTok{                                          input\_names) \{}
\NormalTok{  arg\_name }\OtherTok{\textless{}{-}} \FunctionTok{formalArgs}\NormalTok{(pushforward\_expectand)}

\NormalTok{  C }\OtherTok{\textless{}{-}} \FunctionTok{dim}\NormalTok{(expectand\_vals\_list[[}\DecValTok{1}\NormalTok{]])[}\DecValTok{1}\NormalTok{]}
\NormalTok{  S }\OtherTok{\textless{}{-}} \FunctionTok{dim}\NormalTok{(expectand\_vals\_list[[}\DecValTok{1}\NormalTok{]])[}\DecValTok{2}\NormalTok{]}

\NormalTok{  expectand\_vals\_env }\OtherTok{\textless{}{-}} \FunctionTok{as.environment}\NormalTok{(expectand\_vals\_list)}
\NormalTok{  access\_val }\OtherTok{\textless{}{-}} \ControlFlowTok{function}\NormalTok{(name) \{}
\NormalTok{    expectand\_vals\_env[[name]][c, s]}
\NormalTok{  \}}

\NormalTok{  I }\OtherTok{\textless{}{-}} \FunctionTok{length}\NormalTok{(input\_names)}
\NormalTok{  psh\_fwd\_exp\_vals }\OtherTok{\textless{}{-}} \FunctionTok{as.list}\NormalTok{(}\FunctionTok{rep}\NormalTok{(}\ConstantTok{NA}\NormalTok{, I))}
\NormalTok{  pushforward\_vals }\OtherTok{\textless{}{-}} \FunctionTok{matrix}\NormalTok{(}\ConstantTok{NA}\NormalTok{, }\AttributeTok{nrow=}\NormalTok{C, }\AttributeTok{ncol=}\NormalTok{S)}

  \ControlFlowTok{for}\NormalTok{ (i }\ControlFlowTok{in} \DecValTok{1}\SpecialCharTok{:}\NormalTok{I) \{}
    \ControlFlowTok{for}\NormalTok{ (c }\ControlFlowTok{in} \DecValTok{1}\SpecialCharTok{:}\NormalTok{C) \{}
      \ControlFlowTok{for}\NormalTok{ (s }\ControlFlowTok{in} \DecValTok{1}\SpecialCharTok{:}\NormalTok{S) \{}
\NormalTok{        pushforward\_vals[c, s] }\OtherTok{\textless{}{-}}
          \FunctionTok{as.numeric}\NormalTok{(}\FunctionTok{do.call}\NormalTok{(pushforward\_expectand,}
                             \FunctionTok{setNames}\NormalTok{(}\FunctionTok{list}\NormalTok{(}\FunctionTok{access\_val}\NormalTok{(input\_names[[i]])),}
\NormalTok{                                      arg\_name)))}
\NormalTok{      \}}
\NormalTok{    \}}
\NormalTok{    psh\_fwd\_exp\_vals[[i]] }\OtherTok{\textless{}{-}}\NormalTok{ util}\SpecialCharTok{$}\FunctionTok{ensemble\_mcmc\_est}\NormalTok{(pushforward\_vals)[}\DecValTok{1}\NormalTok{]}
\NormalTok{  \}}
  \FunctionTok{return}\NormalTok{(}\FunctionTok{as.numeric}\NormalTok{(psh\_fwd\_exp\_vals))}
\NormalTok{\}}
\end{Highlighting}
\end{Shaded}

\begin{Shaded}
\begin{Highlighting}[]
\NormalTok{M }\OtherTok{\textless{}{-}}\NormalTok{ data}\SpecialCharTok{$}\NormalTok{N\_movies }\SpecialCharTok{*}\NormalTok{ (data}\SpecialCharTok{$}\NormalTok{N\_movies }\SpecialCharTok{{-}} \DecValTok{1}\NormalTok{) }\SpecialCharTok{/} \DecValTok{2}
\NormalTok{input\_names }\OtherTok{\textless{}{-}} \FunctionTok{as.list}\NormalTok{(}\FunctionTok{rep}\NormalTok{(}\ConstantTok{NA}\NormalTok{, M))}

\NormalTok{idx }\OtherTok{\textless{}{-}} \DecValTok{1}
\ControlFlowTok{for}\NormalTok{ (m1 }\ControlFlowTok{in} \DecValTok{2}\SpecialCharTok{:}\NormalTok{data}\SpecialCharTok{$}\NormalTok{N\_movies) \{}
  \ControlFlowTok{for}\NormalTok{ (m2 }\ControlFlowTok{in} \DecValTok{1}\SpecialCharTok{:}\NormalTok{(m1 }\SpecialCharTok{{-}} \DecValTok{1}\NormalTok{)) \{}
\NormalTok{    input\_names[[idx]] }\OtherTok{\textless{}{-}} \FunctionTok{paste0}\NormalTok{(}\StringTok{\textquotesingle{}Phi[\textquotesingle{}}\NormalTok{, m1, }\StringTok{\textquotesingle{},\textquotesingle{}}\NormalTok{, m2, }\StringTok{\textquotesingle{}]\textquotesingle{}}\NormalTok{)}
\NormalTok{    idx }\OtherTok{\textless{}{-}}\NormalTok{ idx }\SpecialCharTok{+} \DecValTok{1}
\NormalTok{  \}}
\NormalTok{\}}

\NormalTok{corr\_probs }\OtherTok{\textless{}{-}} \FunctionTok{apply\_pushforward\_expectation}\NormalTok{(samples4,}
                                            \ControlFlowTok{function}\NormalTok{(phi) phi }\SpecialCharTok{\textless{}} \FloatTok{0.05}\NormalTok{,}
\NormalTok{                                            input\_names)}

\FunctionTok{par}\NormalTok{(}\AttributeTok{mfrow=}\FunctionTok{c}\NormalTok{(}\DecValTok{1}\NormalTok{, }\DecValTok{1}\NormalTok{), }\AttributeTok{mar=}\FunctionTok{c}\NormalTok{(}\DecValTok{5}\NormalTok{, }\DecValTok{5}\NormalTok{, }\DecValTok{2}\NormalTok{, }\DecValTok{1}\NormalTok{))}

\NormalTok{util}\SpecialCharTok{$}\FunctionTok{plot\_line\_hist}\NormalTok{(corr\_probs, }\DecValTok{0}\NormalTok{, }\DecValTok{1}\NormalTok{, }\FloatTok{0.01}\NormalTok{, }\AttributeTok{col=}\NormalTok{util}\SpecialCharTok{$}\NormalTok{c\_dark,}
                    \AttributeTok{xlab=}\StringTok{"Posterior Probability Phi[m1,m2] \textless{} 0.05"}\NormalTok{)}
\end{Highlighting}
\end{Shaded}

\includegraphics{ratings_files/figure-pdf/unnamed-chunk-72-1.pdf}

That said it's more practical to investigate the consequence of these
correlations. In particular we can look at the movie affinities for each
customer by adding together the common baselines with their individual
preferences. Here we'll consider Customer 23.

\begin{Shaded}
\begin{Highlighting}[]
\FunctionTok{par}\NormalTok{(}\AttributeTok{mfrow=}\FunctionTok{c}\NormalTok{(}\DecValTok{2}\NormalTok{, }\DecValTok{1}\NormalTok{), }\AttributeTok{mar=}\FunctionTok{c}\NormalTok{(}\DecValTok{5}\NormalTok{, }\DecValTok{5}\NormalTok{, }\DecValTok{1}\NormalTok{, }\DecValTok{1}\NormalTok{))}

\NormalTok{c }\OtherTok{\textless{}{-}} \DecValTok{23}

\NormalTok{names }\OtherTok{\textless{}{-}} \FunctionTok{sapply}\NormalTok{(}\DecValTok{1}\SpecialCharTok{:}\NormalTok{data}\SpecialCharTok{$}\NormalTok{N\_movies,}
                \ControlFlowTok{function}\NormalTok{(m) }\FunctionTok{paste0}\NormalTok{(}\StringTok{\textquotesingle{}gamma0[\textquotesingle{}}\NormalTok{, m, }\StringTok{\textquotesingle{}]\textquotesingle{}}\NormalTok{))}
\NormalTok{util}\SpecialCharTok{$}\FunctionTok{plot\_disc\_pushforward\_quantiles}\NormalTok{(samples4, names,}
                                     \AttributeTok{xlab=}\StringTok{"Movie"}\NormalTok{,}
                                     \AttributeTok{ylab=}\StringTok{"Baseline Affinity"}\NormalTok{,}
                                     \AttributeTok{main=}\StringTok{"Baseline"}\NormalTok{)}

\NormalTok{names }\OtherTok{\textless{}{-}} \FunctionTok{sapply}\NormalTok{(}\DecValTok{1}\SpecialCharTok{:}\NormalTok{data}\SpecialCharTok{$}\NormalTok{N\_movies,}
                \ControlFlowTok{function}\NormalTok{(m) }\FunctionTok{paste0}\NormalTok{(}\StringTok{\textquotesingle{}delta\_gamma[\textquotesingle{}}\NormalTok{, c, }\StringTok{\textquotesingle{},\textquotesingle{}}\NormalTok{, m, }\StringTok{\textquotesingle{}]\textquotesingle{}}\NormalTok{))}
\NormalTok{util}\SpecialCharTok{$}\FunctionTok{plot\_disc\_pushforward\_quantiles}\NormalTok{(samples4, names,}
                                     \AttributeTok{xlab=}\StringTok{"Movie"}\NormalTok{,}
                                     \AttributeTok{ylab=}\StringTok{"Change in Affinity"}\NormalTok{,}
                                     \AttributeTok{main=}\FunctionTok{paste0}\NormalTok{(}\StringTok{\textquotesingle{}Customer\textquotesingle{}}\NormalTok{, c))}
\end{Highlighting}
\end{Shaded}

\includegraphics{ratings_files/figure-pdf/unnamed-chunk-73-1.pdf}

\begin{Shaded}
\begin{Highlighting}[]
\NormalTok{expectands }\OtherTok{\textless{}{-}} \FunctionTok{sapply}\NormalTok{(}\DecValTok{1}\SpecialCharTok{:}\NormalTok{data}\SpecialCharTok{$}\NormalTok{N\_movies,}
                     \ControlFlowTok{function}\NormalTok{(m)}
                       \FunctionTok{local}\NormalTok{(\{ idx }\OtherTok{=}\NormalTok{ m; }\ControlFlowTok{function}\NormalTok{(x1, x2)}
\NormalTok{                                        x1[idx] }\SpecialCharTok{+}\NormalTok{ x2[idx] \}) )}
\FunctionTok{names}\NormalTok{(expectands) }\OtherTok{\textless{}{-}} \FunctionTok{sapply}\NormalTok{(}\DecValTok{1}\SpecialCharTok{:}\NormalTok{data}\SpecialCharTok{$}\NormalTok{N\_movies,}
                            \ControlFlowTok{function}\NormalTok{(m)}
                            \FunctionTok{paste0}\NormalTok{(}\StringTok{\textquotesingle{}gamma[\textquotesingle{}}\NormalTok{, c, }\StringTok{\textquotesingle{},\textquotesingle{}}\NormalTok{, m, }\StringTok{\textquotesingle{}]\textquotesingle{}}\NormalTok{))}

\NormalTok{var\_repl }\OtherTok{\textless{}{-}} \FunctionTok{list}\NormalTok{(}\StringTok{\textquotesingle{}x1\textquotesingle{}}\OtherTok{=}\FunctionTok{array}\NormalTok{(}\FunctionTok{sapply}\NormalTok{(}\DecValTok{1}\SpecialCharTok{:}\NormalTok{data}\SpecialCharTok{$}\NormalTok{N\_movies,}
                                   \ControlFlowTok{function}\NormalTok{(m)}
                                   \FunctionTok{paste0}\NormalTok{(}\StringTok{\textquotesingle{}gamma0[\textquotesingle{}}\NormalTok{, m, }\StringTok{\textquotesingle{}]\textquotesingle{}}\NormalTok{))),}
                 \StringTok{\textquotesingle{}x2\textquotesingle{}}\OtherTok{=}\FunctionTok{array}\NormalTok{(}\FunctionTok{sapply}\NormalTok{(}\DecValTok{1}\SpecialCharTok{:}\NormalTok{data}\SpecialCharTok{$}\NormalTok{N\_movies,}
                                   \ControlFlowTok{function}\NormalTok{(m)}
                                   \FunctionTok{paste0}\NormalTok{(}\StringTok{\textquotesingle{}delta\_gamma[\textquotesingle{}}\NormalTok{, c,}
                                          \StringTok{\textquotesingle{},\textquotesingle{}}\NormalTok{, m, }\StringTok{\textquotesingle{}]\textquotesingle{}}\NormalTok{))))}

\NormalTok{affinity\_samples }\OtherTok{\textless{}{-}}
\NormalTok{  util}\SpecialCharTok{$}\FunctionTok{eval\_expectand\_pushforwards}\NormalTok{(samples4,}
\NormalTok{                                   expectands,}
\NormalTok{                                   var\_repl)}
\end{Highlighting}
\end{Shaded}

\begin{Shaded}
\begin{Highlighting}[]
\FunctionTok{par}\NormalTok{(}\AttributeTok{mfrow=}\FunctionTok{c}\NormalTok{(}\DecValTok{1}\NormalTok{, }\DecValTok{1}\NormalTok{), }\AttributeTok{mar=}\FunctionTok{c}\NormalTok{(}\DecValTok{5}\NormalTok{, }\DecValTok{5}\NormalTok{, }\DecValTok{1}\NormalTok{, }\DecValTok{1}\NormalTok{))}

\NormalTok{names }\OtherTok{\textless{}{-}} \FunctionTok{sapply}\NormalTok{(}\DecValTok{1}\SpecialCharTok{:}\NormalTok{data}\SpecialCharTok{$}\NormalTok{N\_movies,}
                \ControlFlowTok{function}\NormalTok{(m) }\FunctionTok{paste0}\NormalTok{(}\StringTok{\textquotesingle{}gamma[\textquotesingle{}}\NormalTok{, c, }\StringTok{\textquotesingle{},\textquotesingle{}}\NormalTok{, m, }\StringTok{\textquotesingle{}]\textquotesingle{}}\NormalTok{))}

\NormalTok{util}\SpecialCharTok{$}\FunctionTok{plot\_disc\_pushforward\_quantiles}\NormalTok{(affinity\_samples, names,}
                                     \AttributeTok{xlab=}\StringTok{"Movie"}\NormalTok{,}
                                     \AttributeTok{ylab=}\StringTok{"Affinity"}\NormalTok{,}
                                     \AttributeTok{main=}\FunctionTok{paste}\NormalTok{(}\StringTok{\textquotesingle{}Customer\textquotesingle{}}\NormalTok{, c))}
\end{Highlighting}
\end{Shaded}

\includegraphics{ratings_files/figure-pdf/unnamed-chunk-75-1.pdf}

Even better we can separately visualize the movie affinities that are
directly informed by observed ratings and those informed by only the
multivariate normal hierarchical model. Note that the affinities for the
movies that Customer 23 did not rate are not only much more uncertain
but also much more uniform.

\begin{Shaded}
\begin{Highlighting}[]
\NormalTok{rated\_movie\_idxs }\OtherTok{\textless{}{-}}\NormalTok{ data}\SpecialCharTok{$}\NormalTok{movie\_idxs[data}\SpecialCharTok{$}\NormalTok{customer\_idxs }\SpecialCharTok{==}\NormalTok{ c]}
\NormalTok{unrated\_movie\_idxs }\OtherTok{\textless{}{-}} \FunctionTok{setdiff}\NormalTok{(}\DecValTok{1}\SpecialCharTok{:}\NormalTok{data}\SpecialCharTok{$}\NormalTok{N\_movies, rated\_movie\_idxs)}

\FunctionTok{par}\NormalTok{(}\AttributeTok{mfrow=}\FunctionTok{c}\NormalTok{(}\DecValTok{2}\NormalTok{, }\DecValTok{1}\NormalTok{), }\AttributeTok{mar=}\FunctionTok{c}\NormalTok{(}\DecValTok{5}\NormalTok{, }\DecValTok{5}\NormalTok{, }\DecValTok{1}\NormalTok{, }\DecValTok{1}\NormalTok{))}

\NormalTok{names }\OtherTok{\textless{}{-}} \FunctionTok{sapply}\NormalTok{(}\DecValTok{1}\SpecialCharTok{:}\NormalTok{data}\SpecialCharTok{$}\NormalTok{N\_movies,}
                \ControlFlowTok{function}\NormalTok{(m) }\FunctionTok{paste0}\NormalTok{(}\StringTok{\textquotesingle{}gamma[\textquotesingle{}}\NormalTok{, c, }\StringTok{\textquotesingle{},\textquotesingle{}}\NormalTok{, m, }\StringTok{\textquotesingle{}]\textquotesingle{}}\NormalTok{))}
\NormalTok{util}\SpecialCharTok{$}\FunctionTok{plot\_disc\_pushforward\_quantiles}\NormalTok{(affinity\_samples, names,}
                                     \AttributeTok{xlab=}\StringTok{"Rated Movie"}\NormalTok{,}
                                     \AttributeTok{ylab=}\StringTok{"Customer Affinity"}\NormalTok{,}
                                     \AttributeTok{main=}\FunctionTok{paste0}\NormalTok{(}\StringTok{\textquotesingle{}Customer\textquotesingle{}}\NormalTok{, c))}
\ControlFlowTok{for}\NormalTok{ (m }\ControlFlowTok{in}\NormalTok{ unrated\_movie\_idxs) \{}
  \FunctionTok{polygon}\NormalTok{(}\FunctionTok{c}\NormalTok{(m }\SpecialCharTok{{-}} \FloatTok{0.5}\NormalTok{, m }\SpecialCharTok{+} \FloatTok{0.5}\NormalTok{, m }\SpecialCharTok{+} \FloatTok{0.5}\NormalTok{, m}\SpecialCharTok{{-}} \FloatTok{0.5}\NormalTok{),}
          \FunctionTok{c}\NormalTok{(}\SpecialCharTok{{-}}\FloatTok{4.75}\NormalTok{, }\SpecialCharTok{{-}}\FloatTok{4.75}\NormalTok{, }\FloatTok{4.75}\NormalTok{, }\FloatTok{4.75}\NormalTok{), }\AttributeTok{col=}\StringTok{"white"}\NormalTok{, }\AttributeTok{border=}\ConstantTok{NA}\NormalTok{)}
\NormalTok{\}}

\NormalTok{util}\SpecialCharTok{$}\FunctionTok{plot\_disc\_pushforward\_quantiles}\NormalTok{(affinity\_samples, names,}
                                     \AttributeTok{xlab=}\StringTok{"Unrated Movie"}\NormalTok{,}
                                     \AttributeTok{ylab=}\StringTok{"Customer Affinity"}\NormalTok{,}
                                     \AttributeTok{main=}\FunctionTok{paste0}\NormalTok{(}\StringTok{\textquotesingle{}Customer\textquotesingle{}}\NormalTok{, c))}
\ControlFlowTok{for}\NormalTok{ (m }\ControlFlowTok{in}\NormalTok{ rated\_movie\_idxs) \{}
  \FunctionTok{polygon}\NormalTok{(}\FunctionTok{c}\NormalTok{(m }\SpecialCharTok{{-}} \FloatTok{0.5}\NormalTok{, m }\SpecialCharTok{+} \FloatTok{0.5}\NormalTok{, m }\SpecialCharTok{+} \FloatTok{0.5}\NormalTok{, m}\SpecialCharTok{{-}} \FloatTok{0.5}\NormalTok{),}
          \FunctionTok{c}\NormalTok{(}\SpecialCharTok{{-}}\FloatTok{4.75}\NormalTok{, }\SpecialCharTok{{-}}\FloatTok{4.75}\NormalTok{, }\FloatTok{4.75}\NormalTok{, }\FloatTok{4.75}\NormalTok{), }\AttributeTok{col=}\StringTok{"white"}\NormalTok{, }\AttributeTok{border=}\ConstantTok{NA}\NormalTok{)}
\NormalTok{\}}
\end{Highlighting}
\end{Shaded}

\includegraphics{ratings_files/figure-pdf/unnamed-chunk-76-1.pdf}

The immediate benefit of modeling individual preferences is that we can
now make movie recommendations bespoke to Customer 23.

\begin{Shaded}
\begin{Highlighting}[]
\NormalTok{expected\_affinity }\OtherTok{\textless{}{-}} \ControlFlowTok{function}\NormalTok{(m) \{}
\NormalTok{  name }\OtherTok{\textless{}{-}} \FunctionTok{paste0}\NormalTok{(}\StringTok{\textquotesingle{}gamma[\textquotesingle{}}\NormalTok{, c, }\StringTok{\textquotesingle{},\textquotesingle{}}\NormalTok{, m, }\StringTok{\textquotesingle{}]\textquotesingle{}}\NormalTok{)}
\NormalTok{  util}\SpecialCharTok{$}\FunctionTok{ensemble\_mcmc\_est}\NormalTok{(affinity\_samples[[name]])[}\DecValTok{1}\NormalTok{]}
\NormalTok{\}}

\NormalTok{expected\_affinities }\OtherTok{\textless{}{-}} \FunctionTok{sapply}\NormalTok{(}\DecValTok{1}\SpecialCharTok{:}\NormalTok{data}\SpecialCharTok{$}\NormalTok{N\_movies,}
                             \ControlFlowTok{function}\NormalTok{(m) }\FunctionTok{expected\_affinity}\NormalTok{(m))}

\NormalTok{post\_mean\_ordering }\OtherTok{\textless{}{-}} \FunctionTok{sort}\NormalTok{(expected\_affinities, }\AttributeTok{index.return=}\ConstantTok{TRUE}\NormalTok{)}\SpecialCharTok{$}\NormalTok{ix}
\end{Highlighting}
\end{Shaded}

\begin{Shaded}
\begin{Highlighting}[]
\FunctionTok{par}\NormalTok{(}\AttributeTok{mfrow=}\FunctionTok{c}\NormalTok{(}\DecValTok{1}\NormalTok{, }\DecValTok{1}\NormalTok{), }\AttributeTok{mar=}\FunctionTok{c}\NormalTok{(}\DecValTok{5}\NormalTok{, }\DecValTok{5}\NormalTok{, }\DecValTok{1}\NormalTok{, }\DecValTok{1}\NormalTok{))}

\NormalTok{names }\OtherTok{\textless{}{-}} \FunctionTok{sapply}\NormalTok{(post\_mean\_ordering,}
                \ControlFlowTok{function}\NormalTok{(m) }\FunctionTok{paste0}\NormalTok{(}\StringTok{\textquotesingle{}gamma[\textquotesingle{}}\NormalTok{, c, }\StringTok{\textquotesingle{},\textquotesingle{}}\NormalTok{, m, }\StringTok{\textquotesingle{}]\textquotesingle{}}\NormalTok{))}

\NormalTok{xname }\OtherTok{\textless{}{-}} \StringTok{"Movies Ordered by Expected Affinity"}
\NormalTok{util}\SpecialCharTok{$}\FunctionTok{plot\_disc\_pushforward\_quantiles}\NormalTok{(affinity\_samples, names,}
                                     \AttributeTok{xlab=}\NormalTok{xname,}
                                     \AttributeTok{ylab=}\StringTok{"Affinity"}\NormalTok{)}
\end{Highlighting}
\end{Shaded}

\includegraphics{ratings_files/figure-pdf/unnamed-chunk-78-1.pdf}

From this we can infer what movies we think Customer 23 will like the
least.

\begin{Shaded}
\begin{Highlighting}[]
\FunctionTok{print}\NormalTok{(}\FunctionTok{data.frame}\NormalTok{(}\StringTok{"Rank"}\OtherTok{=}\DecValTok{200}\SpecialCharTok{:}\DecValTok{196}\NormalTok{,}
                 \StringTok{"Movie"}\OtherTok{=}\FunctionTok{head}\NormalTok{(post\_mean\_ordering, }\DecValTok{5}\NormalTok{)),}
      \AttributeTok{row.names=}\ConstantTok{FALSE}\NormalTok{)}
\end{Highlighting}
\end{Shaded}

\begin{verbatim}
 Rank Movie
  200    40
  199    13
  198    78
  197    55
  196   119
\end{verbatim}

As well as what movies we think they will like the most.

\begin{Shaded}
\begin{Highlighting}[]
\FunctionTok{print}\NormalTok{(}\FunctionTok{data.frame}\NormalTok{(}\StringTok{"Rank"}\OtherTok{=}\DecValTok{5}\SpecialCharTok{:}\DecValTok{1}\NormalTok{,}
                 \StringTok{"Movie"}\OtherTok{=}\FunctionTok{tail}\NormalTok{(post\_mean\_ordering, }\DecValTok{5}\NormalTok{)),}
      \AttributeTok{row.names=}\ConstantTok{FALSE}\NormalTok{)}
\end{Highlighting}
\end{Shaded}

\begin{verbatim}
 Rank Movie
    5   167
    4    61
    3   156
    2    23
    1    97
\end{verbatim}

Of course there's not much utility in recommending a customer a movie
that they've already seen. A much more useful recommendation is for
movies that they haven't yet seen but might enjoy.

Here let's assume that a movie has been unrated by a customer only if
the customer has not yet seen it. Consequently the recommendation task
comes down to inferring the unrated movies with the highest affinities
for Customer 23.

\begin{Shaded}
\begin{Highlighting}[]
\NormalTok{expected\_affinities }\OtherTok{\textless{}{-}} \FunctionTok{sapply}\NormalTok{(unrated\_movie\_idxs,}
                              \ControlFlowTok{function}\NormalTok{(m) }\FunctionTok{expected\_affinity}\NormalTok{(m))}

\NormalTok{post\_mean\_ordering }\OtherTok{\textless{}{-}} \FunctionTok{sort}\NormalTok{(expected\_affinities, }\AttributeTok{index.return=}\ConstantTok{TRUE}\NormalTok{)}\SpecialCharTok{$}\NormalTok{ix}
\end{Highlighting}
\end{Shaded}

We can finally present a list of the top new movies to recommend to
Customer 23.

\begin{Shaded}
\begin{Highlighting}[]
\FunctionTok{print}\NormalTok{(}\FunctionTok{data.frame}\NormalTok{(}\StringTok{"Rank"}\OtherTok{=}\DecValTok{10}\SpecialCharTok{:}\DecValTok{1}\NormalTok{,}
                 \StringTok{"Movie"}\OtherTok{=}\FunctionTok{tail}\NormalTok{(unrated\_movie\_idxs[post\_mean\_ordering], }\DecValTok{10}\NormalTok{)),}
      \AttributeTok{row.names=}\ConstantTok{FALSE}\NormalTok{)}
\end{Highlighting}
\end{Shaded}

\begin{verbatim}
 Rank Movie
   10    72
    9   186
    8    15
    7   155
    6    86
    5   161
    4   107
    3    43
    2    87
    1   193
\end{verbatim}

All of this said we should have only mild confidence in these
recommendations given the large uncertainties.

\begin{Shaded}
\begin{Highlighting}[]
\FunctionTok{par}\NormalTok{(}\AttributeTok{mfrow=}\FunctionTok{c}\NormalTok{(}\DecValTok{1}\NormalTok{, }\DecValTok{1}\NormalTok{), }\AttributeTok{mar=}\FunctionTok{c}\NormalTok{(}\DecValTok{5}\NormalTok{, }\DecValTok{5}\NormalTok{, }\DecValTok{1}\NormalTok{, }\DecValTok{1}\NormalTok{))}

\NormalTok{names }\OtherTok{\textless{}{-}} \FunctionTok{sapply}\NormalTok{(unrated\_movie\_idxs[post\_mean\_ordering],}
                \ControlFlowTok{function}\NormalTok{(m) }\FunctionTok{paste0}\NormalTok{(}\StringTok{\textquotesingle{}gamma[\textquotesingle{}}\NormalTok{, c, }\StringTok{\textquotesingle{},\textquotesingle{}}\NormalTok{, m, }\StringTok{\textquotesingle{}]\textquotesingle{}}\NormalTok{))}

\NormalTok{xname }\OtherTok{\textless{}{-}} \StringTok{"Unrated Movies Ordered by Expected Affinity"}
\NormalTok{util}\SpecialCharTok{$}\FunctionTok{plot\_disc\_pushforward\_quantiles}\NormalTok{(affinity\_samples, names,}
                                     \AttributeTok{xlab=}\NormalTok{xname,}
                                     \AttributeTok{ylab=}\StringTok{"Affinity"}\NormalTok{)}
\end{Highlighting}
\end{Shaded}

\includegraphics{ratings_files/figure-pdf/unnamed-chunk-83-1.pdf}

One subtlety with recommendations is that in most applications we cannot
evaluate their performance directly. For example absent any additional
interrogation of Customer 23 the only indication of how much they agree
with one our recommendations is how well they rate the movie in the
future.

Fortunately we can use our inferences to predict not only movie
recommendations but also how we think Customer 23 would rate them. This
would allow us to for example compare predicted rankings to actual
rankings.

\begin{Shaded}
\begin{Highlighting}[]
\NormalTok{movie\_idx }\OtherTok{\textless{}{-}} \FunctionTok{tail}\NormalTok{(unrated\_movie\_idxs[post\_mean\_ordering], }\DecValTok{1}\NormalTok{)}

\NormalTok{logistic }\OtherTok{\textless{}{-}} \ControlFlowTok{function}\NormalTok{(x) \{}
  \ControlFlowTok{if}\NormalTok{ (x }\SpecialCharTok{\textgreater{}} \DecValTok{0}\NormalTok{) \{}
    \DecValTok{1} \SpecialCharTok{/}\NormalTok{ (}\DecValTok{1} \SpecialCharTok{+} \FunctionTok{exp}\NormalTok{(}\SpecialCharTok{{-}}\NormalTok{x))}
\NormalTok{  \} }\ControlFlowTok{else}\NormalTok{ \{}
\NormalTok{    e }\OtherTok{\textless{}{-}} \FunctionTok{exp}\NormalTok{(x)}
\NormalTok{    e }\SpecialCharTok{/}\NormalTok{ (}\DecValTok{1} \SpecialCharTok{+}\NormalTok{ e)}
\NormalTok{  \}}
\NormalTok{\}}

\NormalTok{expectands }\OtherTok{\textless{}{-}} \FunctionTok{list}\NormalTok{(}\ControlFlowTok{function}\NormalTok{(c, gamma) }\DecValTok{1} \SpecialCharTok{{-}} \FunctionTok{logistic}\NormalTok{(gamma }\SpecialCharTok{{-}}\NormalTok{ c[}\DecValTok{1}\NormalTok{]),}
                   \ControlFlowTok{function}\NormalTok{(c, gamma) }\FunctionTok{logistic}\NormalTok{(gamma }\SpecialCharTok{{-}}\NormalTok{ c[}\DecValTok{1}\NormalTok{]) }\SpecialCharTok{{-}}
                                      \FunctionTok{logistic}\NormalTok{(gamma }\SpecialCharTok{{-}}\NormalTok{ c[}\DecValTok{2}\NormalTok{]),}
                   \ControlFlowTok{function}\NormalTok{(c, gamma) }\FunctionTok{logistic}\NormalTok{(gamma }\SpecialCharTok{{-}}\NormalTok{ c[}\DecValTok{2}\NormalTok{]) }\SpecialCharTok{{-}}
                                      \FunctionTok{logistic}\NormalTok{(gamma }\SpecialCharTok{{-}}\NormalTok{ c[}\DecValTok{3}\NormalTok{]),}
                   \ControlFlowTok{function}\NormalTok{(c, gamma) }\FunctionTok{logistic}\NormalTok{(gamma }\SpecialCharTok{{-}}\NormalTok{ c[}\DecValTok{3}\NormalTok{]) }\SpecialCharTok{{-}}
                                      \FunctionTok{logistic}\NormalTok{(gamma }\SpecialCharTok{{-}}\NormalTok{ c[}\DecValTok{4}\NormalTok{]),}
                   \ControlFlowTok{function}\NormalTok{(c, gamma) }\FunctionTok{logistic}\NormalTok{(gamma }\SpecialCharTok{{-}}\NormalTok{ c[}\DecValTok{4}\NormalTok{]))}
\FunctionTok{names}\NormalTok{(expectands) }\OtherTok{\textless{}{-}} \FunctionTok{c}\NormalTok{(}\StringTok{\textquotesingle{}p[1]\textquotesingle{}}\NormalTok{, }\StringTok{\textquotesingle{}p[2]\textquotesingle{}}\NormalTok{, }\StringTok{\textquotesingle{}p[3]\textquotesingle{}}\NormalTok{, }\StringTok{\textquotesingle{}p[4]\textquotesingle{}}\NormalTok{, }\StringTok{\textquotesingle{}p[5]\textquotesingle{}}\NormalTok{)}

\NormalTok{var\_repl }\OtherTok{\textless{}{-}} \FunctionTok{list}\NormalTok{(}\StringTok{\textquotesingle{}c\textquotesingle{}}\OtherTok{=}\FunctionTok{array}\NormalTok{(}\FunctionTok{sapply}\NormalTok{(}\DecValTok{1}\SpecialCharTok{:}\DecValTok{4}\NormalTok{,}
                           \ControlFlowTok{function}\NormalTok{(k)}
                           \FunctionTok{paste0}\NormalTok{(}\StringTok{\textquotesingle{}cut\_points[\textquotesingle{}}\NormalTok{, c, }\StringTok{\textquotesingle{},\textquotesingle{}}\NormalTok{, k, }\StringTok{\textquotesingle{}]\textquotesingle{}}\NormalTok{))),}
                 \StringTok{\textquotesingle{}gamma\textquotesingle{}}\OtherTok{=}\FunctionTok{paste0}\NormalTok{(}\StringTok{\textquotesingle{}gamma[\textquotesingle{}}\NormalTok{, c, }\StringTok{\textquotesingle{},\textquotesingle{}}\NormalTok{, movie\_idx, }\StringTok{\textquotesingle{}]\textquotesingle{}}\NormalTok{))}

\ControlFlowTok{for}\NormalTok{ (k }\ControlFlowTok{in} \DecValTok{1}\SpecialCharTok{:}\DecValTok{4}\NormalTok{) \{}
\NormalTok{  name }\OtherTok{\textless{}{-}} \FunctionTok{paste0}\NormalTok{(}\StringTok{\textquotesingle{}cut\_points[\textquotesingle{}}\NormalTok{, c, }\StringTok{\textquotesingle{},\textquotesingle{}}\NormalTok{, k, }\StringTok{\textquotesingle{}]\textquotesingle{}}\NormalTok{)}
\NormalTok{  affinity\_samples[[name]] }\OtherTok{\textless{}{-}}\NormalTok{ samples4[[name]]}
\NormalTok{\}}

\NormalTok{prob\_samples }\OtherTok{\textless{}{-}}\NormalTok{util}\SpecialCharTok{$}\FunctionTok{eval\_expectand\_pushforwards}\NormalTok{(affinity\_samples,}
\NormalTok{                                                expectands,}
\NormalTok{                                                var\_repl)}
\end{Highlighting}
\end{Shaded}

\begin{Shaded}
\begin{Highlighting}[]
\FunctionTok{par}\NormalTok{(}\AttributeTok{mfrow=}\FunctionTok{c}\NormalTok{(}\DecValTok{1}\NormalTok{, }\DecValTok{1}\NormalTok{), }\AttributeTok{mar=}\FunctionTok{c}\NormalTok{(}\DecValTok{5}\NormalTok{, }\DecValTok{5}\NormalTok{, }\DecValTok{1}\NormalTok{, }\DecValTok{1}\NormalTok{))}

\NormalTok{util}\SpecialCharTok{$}\FunctionTok{plot\_disc\_pushforward\_quantiles}\NormalTok{(prob\_samples, }\FunctionTok{names}\NormalTok{(expectands),}
                                     \AttributeTok{xlab=}\StringTok{"Rating"}\NormalTok{,}
                                     \AttributeTok{ylab=}\StringTok{"Posterior Probability"}\NormalTok{)}
\end{Highlighting}
\end{Shaded}

\includegraphics{ratings_files/figure-pdf/unnamed-chunk-85-1.pdf}

Interestingly we don't actually predict a particularly high rating for
our top recommendation. In hindsight, however, this shouldn't be
unexpected given how austere Customer 23 is with their stars.

\begin{Shaded}
\begin{Highlighting}[]
\FunctionTok{par}\NormalTok{(}\AttributeTok{mfrow=}\FunctionTok{c}\NormalTok{(}\DecValTok{1}\NormalTok{, }\DecValTok{1}\NormalTok{), }\AttributeTok{mar=}\FunctionTok{c}\NormalTok{(}\DecValTok{5}\NormalTok{, }\DecValTok{5}\NormalTok{, }\DecValTok{1}\NormalTok{, }\DecValTok{1}\NormalTok{))}

\NormalTok{util}\SpecialCharTok{$}\FunctionTok{plot\_line\_hist}\NormalTok{(data}\SpecialCharTok{$}\NormalTok{ratings[data}\SpecialCharTok{$}\NormalTok{customer\_idxs }\SpecialCharTok{==}\NormalTok{ c],}
                    \SpecialCharTok{{-}}\FloatTok{0.5}\NormalTok{, }\FloatTok{6.5}\NormalTok{, }\DecValTok{1}\NormalTok{,}
                    \AttributeTok{xlab=}\StringTok{"Rating"}\NormalTok{, }\AttributeTok{main=}\FunctionTok{paste}\NormalTok{(}\StringTok{\textquotesingle{}customer\textquotesingle{}}\NormalTok{, c))}
\end{Highlighting}
\end{Shaded}

\includegraphics{ratings_files/figure-pdf/unnamed-chunk-86-1.pdf}

Another benefit of this hierarchical approach is that we are not limited
to making inferences and predictions for existing customers. In
particular we can also make inferences and predictions for new customers
by sampling new interior cut points and movie affinities from the
respective hierarchical population models. With the dearth of observed
ratings these predictions won't be all that precise, but at the same
time that uncertainty prevents us from making overly confident claims.

\section{Computational
Considerations}\label{computational-considerations}

I want to emphasize that this case study is first and foremost a
demonstrative analysis. In particular I reduced the data not for any
statistical reason by rather to ensure that the models would not take
too long to run. Ultimately the final model took about three hours to
run on my laptop which wasn't too onerous, especially given the total
number of parameters.

That said I do think it is useful to at least consider what the
different priorities might be for a more realistic analysis where a
specific inferential goal would be driving the amount of data to include
and different computational resources might be available. What
\emph{would} it take to speed up the fit of the final model or scale it
up to a larger data set?

Recall that the overall cost of running Hamiltonian Monte Carlo can
roughly be decomposed into the number of iterations, the number of model
evaluations per iteration, and the cost of each model evaluation. The
number of model evaluations per iteration is driven by the posterior
geometry and how hard the Hamiltonian Monte Carlo algorithm has to work
to explore it. For a fixed data set the two main ways that we can reduce
computation is to improve the posterior geometry or speed up the model
evaluations.

When working with hierarchical models we need to be considerate of the
potentially problematic geometries to which they are prone. In this case
study we seemed to do okay with a monolithic non-centered
parameterization for the normal and multivariate normal hierarchies, but
we could possibly improve the geometry by non-centering the parameters
corresponding to more prolific movies and customers. Before considering
that, however, we can estimate the potential for improvement by
examining the length of the numerical Hamiltonian trajectories, and
hence how many model evaluations were needed per iteration, in our last
fit.

\begin{Shaded}
\begin{Highlighting}[]
\NormalTok{util}\SpecialCharTok{$}\FunctionTok{plot\_num\_leapfrogs\_by\_chain}\NormalTok{(diagnostics)}
\end{Highlighting}
\end{Shaded}

\includegraphics{ratings_files/figure-pdf/unnamed-chunk-87-1.pdf}

Despite the complexity of the final model the numerical Hamiltonian
trajectories weren't all that long. Even in an ideal case we can't do do
much better than ten or so leapfrog steps per trajectory; consequently
the maximum possible speed up that we could get from improving the
geometry here would be less than an order of magnitude! That's not
trivial but it suggests that the computational cost is not being
dominated by the number of model evaluations but rather the cost of each
model evaluation itself.

So how can we speed up the model evaluations? One immediate strategy is
parallelization, especially if we're working with computers blessed with
lots and lots of threads. For example we can, at least in theory,
parallelize the many matrix-vector products that are required in the
\texttt{transformed\ parameters} block. That said achieving these
potential speed ups in practice is always frustrated by the subtle
input/output costs of passing all of the needed information to each
thread and back in each model evaluation.

Either approach to speeding up the fitting of the final model will be
challenging, requiring careful investigations and implementations and
offering no guarantee of success. Even worse neither of these strategies
will really be able to compete with the quadratic cost of evaluating the
model, both in terms of \(N_{\text{customer}} \cdot N_{\text{movie}}\)
and \(N_{\text{movie}}^{2}\), if we attempt to add more customers and/or
movies. For example scaling up from 200 movies to 2000 movies, still
only a fraction of the total data set and an even more negligible part
of the full data a company like Netflix would have available, would
require a 100 fold increase in the cost of evaluating the model. Even if
the posterior geometry doesn't get any worst that pushes three hours to
over a full day of computation.

In practice we can fight quadratic scaling only so far. Ultimately the
problem is that the final model always has to compare every movie to
every other movie. Consequently the most effective scaling strategy is
often to limit the number of movies to which each movie is compared.
More formally we need to introduce an appropriate sparsity structure on
the movies so that most of the \(N_{\text{movie}}^{2}\) covariances are
zero.

Many methods attempt to learn a sparsity structure consistent with the
observed data, dynamically turning off covariances that end up too
small. This, however, is an outrageously difficult learning problem and
approximate results tend to be fragile without unreasonable amounts of
data. We can usually do much better by taking advantage of our domain
expertise to motivate appropriate sparsity structures directly.

For example we could first group movies into genres before modeling
common baseline affinities, correlated deviations across genres, and
perhaps even correlated deviations for each movie within each genre.
This effectively introduces a block-diagonal structure to the full
covariance matrix which scales much more efficiently without sacrificing
\emph{all} of the correlations that can help inform predictions for
unrated movies.

Given the sparsity of the observed ratings we can learn only so much.
Consequently we might as well build a more restricted model of
meaningful behaviors that we have a hope of resolving than attempting to
learn details about which we just don't have enough information.

\section{Conclusion}\label{conclusion}

In this case study I developed a relatively sophisticated analysis of
consumer feedback that accounts for not only how each customer
interprets the possible ordinal ratings in different ways but also the
variation in their preferences. To learn anything about these behaviors
in spite of the sparsity of the data we had to leverage our domain
expertise and some formidable modeling techniques.

If anything I hope that this case study demonstrates how powerful
hierarchical modeling techniques can be when used carefully. In
particular to achieve our final inferences we used our domain expertise
to sketch out the data generating process first, and only then
considered opportunities for heterogeneous behaviors.

By starting with the broad features of the data generating process we
established an explicit context that made is easier to identify not only
what behaviors were heterogeneous but also what heterogeneous behaviors
might be coupled together. Moreover the structure of those behaviors
motivated appropriate population models. In this way we were able to
develop an elaborate model with multiple, multivariate hierarchies
without becoming to overwhelmed in the process.

Hierarchical modeling is so much more than one-dimensional normal
population models!

\section*{Acknowledgements}\label{acknowledgements}
\addcontentsline{toc}{section}{Acknowledgements}

I thank jd for helpful comments.

A very special thanks to everyone supporting me on Patreon: Adam
Fleischhacker, Adriano Yoshino, Alejandro Navarro-Martínez, Alessandro
Varacca, Alex D, Alexander Noll, Andrea Serafino, Andrew Mascioli,
Andrew Rouillard, Andrew Vigotsky, Ara Winter, Austin Rochford, Avraham
Adler, Ben Matthews, Ben Swallow, Benoit Essiambre, Bertrand Wilden,
boot, Bradley Kolb, Brendan Galdo, Bryan Chang, Brynjolfur Gauti
Jónsson, Cameron Smith, Canaan Breiss, Cat Shark, CG, Charles Naylor,
Chase Dwelle, Chris Jones, Christopher Mehrvarzi, Colin Carroll, Colin
McAuliffe, Damien Mannion, dan mackinlay, Dan W Joyce, Dan Waxman, Dan
Weitzenfeld, Daniel Edward Marthaler, Daniel Saunders, Darshan Pandit,
Darthmaluus , David Galley, David Wurtz, Doug Rivers, Dr.~Jobo, Dr.~Omri
Har Shemesh, Dylan Maher, Ed Cashin, Edgar Merkle, Eli Witus, Eric
LaMotte, Ero Carrera, Eugene O'Friel, Felipe González, Fergus Chadwick,
Finn Lindgren, Francesco Corona, Geoff Rollins, Guilherme Marthe, Håkan
Johansson, Hamed Bastan-Hagh, haubur, Hector Munoz, Henri Wallen, hs,
Hugo Botha, Ian, Ian Costley, idontgetoutmuch, Ignacio Vera, Ilaria
Prosdocimi, Isaac Vock, Isidor Belic, jacob pine, Jair Andrade, James C,
James Hodgson, James Wade, Janek Berger, Jarrett Byrnes, Jason Martin,
Jason Pekos, Jason Wong, jd, Jeff Burnett, Jeff Dotson, Jeff Helzner,
Jeffrey Erlich, Jerry Lin , Jessica Graves, Joe Sloan, John Flournoy,
Jonathan H. Morgan, Jonathon Vallejo, Joran Jongerling, Josh Knecht,
June, Justin Bois, Kádár András, Karim Naguib, Karim Osman, Kristian
Gårdhus Wichmann, Lars Barquist, lizzie , LOU ODETTE, Luís F, Marcel
Lüthi, Marek Kwiatkowski, Mariana Carmona, Mark Donoghoe, Markus P.,
Márton Vaitkus, Matthew, Matthew Kay, Matthew Mulvahill, Matthieu LEROY,
Mattia Arsendi, Matěj Kolouch Grabovský, Maurits van der Meer, Max,
Michael Colaresi, Michael DeWitt, Michael Dillon, Michael Lerner, Mick
Cooney, Mike Lawrence, MisterMentat , N Sanders, N.S. , Name, Nathaniel
Burbank, Nic Fishman, Nicholas Clark, Nicholas Cowie, Nick S, Ole
Rogeberg, Oliver Crook, Olivier Ma, Patrick Kelley, Patrick Boehnke, Pau
Pereira Batlle, Pete St.~Marie, Peter Johnson, Pieter van den Berg ,
ptr, quasar, Ramiro Barrantes Reynolds, Raúl Peralta Lozada, Ravin
Kumar, Rémi , Rex Ha, Riccardo Fusaroli, Richard Nerland, Robert Frost,
Robert Goldman, Robert kohn, Robin Taylor, Ryan Gan, Ryan Grossman, Ryan
Kelly, S Hong, Sean Wilson, Sergiy Protsiv, Seth Axen, shira, Simon
Duane, Simon Lilburn, Simone Sebben, sssz, Stefan Lorenz, Stephen
Lienhard, Steve Harris, Stew Watts, Stone Chen, Susan Holmes, Svilup,
Tao Ye, Tate Tunstall, Tatsuo Okubo, Teresa Ortiz, Theodore Dasher,
Thomas Siegert, Thomas Vladeck, Tobychev , Tony Wuersch, Tyler Burch,
Virginia Fisher, Vladimir Markov, Wil Yegelwel, Will Farr, Will Lowe,
Will Wen, woejozney, yolhaj , yureq , Zach A, Zad Rafi, and Zhengchen
Cai.

\section*{License}\label{license}
\addcontentsline{toc}{section}{License}

A repository containing all of the files used to generate this chapter
is available on
\href{https://github.com/betanalpha/quarto_case_studies/tree/main/ratings}{GitHub}.

The code in this case study is copyrighted by Michael Betancourt and
licensed under the new BSD (3-clause) license:

\url{https://opensource.org/licenses/BSD-3-Clause}

The text and figures in this chapter are copyrighted by Michael
Betancourt and licensed under the CC BY-NC 4.0 license:

\url{https://creativecommons.org/licenses/by-nc/4.0/}

\section*{Original Computing
Environment}\label{original-computing-environment}
\addcontentsline{toc}{section}{Original Computing Environment}

\begin{Shaded}
\begin{Highlighting}[]
\FunctionTok{writeLines}\NormalTok{(}\FunctionTok{readLines}\NormalTok{(}\FunctionTok{file.path}\NormalTok{(}\FunctionTok{Sys.getenv}\NormalTok{(}\StringTok{"HOME"}\NormalTok{), }\StringTok{".R/Makevars"}\NormalTok{)))}
\end{Highlighting}
\end{Shaded}

\begin{verbatim}
CC=clang

CXXFLAGS=-O3 -mtune=native -march=native -Wno-unused-variable -Wno-unused-function -Wno-macro-redefined -Wno-unneeded-internal-declaration
CXX=clang++ -arch x86_64 -ftemplate-depth-256

CXX14FLAGS=-O3 -mtune=native -march=native -Wno-unused-variable -Wno-unused-function -Wno-macro-redefined -Wno-unneeded-internal-declaration -Wno-unknown-pragmas
CXX14=clang++ -arch x86_64 -ftemplate-depth-256
\end{verbatim}

\begin{Shaded}
\begin{Highlighting}[]
\FunctionTok{sessionInfo}\NormalTok{()}
\end{Highlighting}
\end{Shaded}

\begin{verbatim}
R version 4.3.2 (2023-10-31)
Platform: x86_64-apple-darwin20 (64-bit)
Running under: macOS Sonoma 14.4.1

Matrix products: default
BLAS:   /Library/Frameworks/R.framework/Versions/4.3-x86_64/Resources/lib/libRblas.0.dylib 
LAPACK: /Library/Frameworks/R.framework/Versions/4.3-x86_64/Resources/lib/libRlapack.dylib;  LAPACK version 3.11.0

locale:
[1] en_US.UTF-8/en_US.UTF-8/en_US.UTF-8/C/en_US.UTF-8/en_US.UTF-8

time zone: America/New_York
tzcode source: internal

attached base packages:
[1] stats     graphics  grDevices utils     datasets  methods   base     

other attached packages:
[1] colormap_0.1.4     rstan_2.32.6       StanHeaders_2.32.7

loaded via a namespace (and not attached):
 [1] gtable_0.3.4       jsonlite_1.8.8     compiler_4.3.2     Rcpp_1.0.11       
 [5] parallel_4.3.2     gridExtra_2.3      scales_1.3.0       yaml_2.3.8        
 [9] fastmap_1.1.1      ggplot2_3.4.4      R6_2.5.1           curl_5.2.0        
[13] knitr_1.45         tibble_3.2.1       munsell_0.5.0      pillar_1.9.0      
[17] rlang_1.1.2        utf8_1.2.4         V8_4.4.1           inline_0.3.19     
[21] xfun_0.41          RcppParallel_5.1.7 cli_3.6.2          magrittr_2.0.3    
[25] digest_0.6.33      grid_4.3.2         lifecycle_1.0.4    vctrs_0.6.5       
[29] evaluate_0.23      glue_1.6.2         QuickJSR_1.0.8     codetools_0.2-19  
[33] stats4_4.3.2       pkgbuild_1.4.3     fansi_1.0.6        colorspace_2.1-0  
[37] rmarkdown_2.25     matrixStats_1.2.0  tools_4.3.2        loo_2.6.0         
[41] pkgconfig_2.0.3    htmltools_0.5.7   
\end{verbatim}



\end{document}
