% Options for packages loaded elsewhere
\PassOptionsToPackage{unicode}{hyperref}
\PassOptionsToPackage{hyphens}{url}
\PassOptionsToPackage{dvipsnames,svgnames,x11names}{xcolor}
%
\documentclass[
  letterpaper,
  DIV=11,
  numbers=noendperiod]{scrartcl}

\usepackage{amsmath,amssymb}
\usepackage{iftex}
\ifPDFTeX
  \usepackage[T1]{fontenc}
  \usepackage[utf8]{inputenc}
  \usepackage{textcomp} % provide euro and other symbols
\else % if luatex or xetex
  \usepackage{unicode-math}
  \defaultfontfeatures{Scale=MatchLowercase}
  \defaultfontfeatures[\rmfamily]{Ligatures=TeX,Scale=1}
\fi
\usepackage{lmodern}
\ifPDFTeX\else  
    % xetex/luatex font selection
\fi
% Use upquote if available, for straight quotes in verbatim environments
\IfFileExists{upquote.sty}{\usepackage{upquote}}{}
\IfFileExists{microtype.sty}{% use microtype if available
  \usepackage[]{microtype}
  \UseMicrotypeSet[protrusion]{basicmath} % disable protrusion for tt fonts
}{}
\makeatletter
\@ifundefined{KOMAClassName}{% if non-KOMA class
  \IfFileExists{parskip.sty}{%
    \usepackage{parskip}
  }{% else
    \setlength{\parindent}{0pt}
    \setlength{\parskip}{6pt plus 2pt minus 1pt}}
}{% if KOMA class
  \KOMAoptions{parskip=half}}
\makeatother
\usepackage{xcolor}
\setlength{\emergencystretch}{3em} % prevent overfull lines
\setcounter{secnumdepth}{5}
% Make \paragraph and \subparagraph free-standing
\ifx\paragraph\undefined\else
  \let\oldparagraph\paragraph
  \renewcommand{\paragraph}[1]{\oldparagraph{#1}\mbox{}}
\fi
\ifx\subparagraph\undefined\else
  \let\oldsubparagraph\subparagraph
  \renewcommand{\subparagraph}[1]{\oldsubparagraph{#1}\mbox{}}
\fi

\usepackage{color}
\usepackage{fancyvrb}
\newcommand{\VerbBar}{|}
\newcommand{\VERB}{\Verb[commandchars=\\\{\}]}
\DefineVerbatimEnvironment{Highlighting}{Verbatim}{commandchars=\\\{\}}
% Add ',fontsize=\small' for more characters per line
\usepackage{framed}
\definecolor{shadecolor}{RGB}{241,243,245}
\newenvironment{Shaded}{\begin{snugshade}}{\end{snugshade}}
\newcommand{\AlertTok}[1]{\textcolor[rgb]{0.68,0.00,0.00}{#1}}
\newcommand{\AnnotationTok}[1]{\textcolor[rgb]{0.37,0.37,0.37}{#1}}
\newcommand{\AttributeTok}[1]{\textcolor[rgb]{0.40,0.45,0.13}{#1}}
\newcommand{\BaseNTok}[1]{\textcolor[rgb]{0.68,0.00,0.00}{#1}}
\newcommand{\BuiltInTok}[1]{\textcolor[rgb]{0.00,0.23,0.31}{#1}}
\newcommand{\CharTok}[1]{\textcolor[rgb]{0.13,0.47,0.30}{#1}}
\newcommand{\CommentTok}[1]{\textcolor[rgb]{0.37,0.37,0.37}{#1}}
\newcommand{\CommentVarTok}[1]{\textcolor[rgb]{0.37,0.37,0.37}{\textit{#1}}}
\newcommand{\ConstantTok}[1]{\textcolor[rgb]{0.56,0.35,0.01}{#1}}
\newcommand{\ControlFlowTok}[1]{\textcolor[rgb]{0.00,0.23,0.31}{#1}}
\newcommand{\DataTypeTok}[1]{\textcolor[rgb]{0.68,0.00,0.00}{#1}}
\newcommand{\DecValTok}[1]{\textcolor[rgb]{0.68,0.00,0.00}{#1}}
\newcommand{\DocumentationTok}[1]{\textcolor[rgb]{0.37,0.37,0.37}{\textit{#1}}}
\newcommand{\ErrorTok}[1]{\textcolor[rgb]{0.68,0.00,0.00}{#1}}
\newcommand{\ExtensionTok}[1]{\textcolor[rgb]{0.00,0.23,0.31}{#1}}
\newcommand{\FloatTok}[1]{\textcolor[rgb]{0.68,0.00,0.00}{#1}}
\newcommand{\FunctionTok}[1]{\textcolor[rgb]{0.28,0.35,0.67}{#1}}
\newcommand{\ImportTok}[1]{\textcolor[rgb]{0.00,0.46,0.62}{#1}}
\newcommand{\InformationTok}[1]{\textcolor[rgb]{0.37,0.37,0.37}{#1}}
\newcommand{\KeywordTok}[1]{\textcolor[rgb]{0.00,0.23,0.31}{#1}}
\newcommand{\NormalTok}[1]{\textcolor[rgb]{0.00,0.23,0.31}{#1}}
\newcommand{\OperatorTok}[1]{\textcolor[rgb]{0.37,0.37,0.37}{#1}}
\newcommand{\OtherTok}[1]{\textcolor[rgb]{0.00,0.23,0.31}{#1}}
\newcommand{\PreprocessorTok}[1]{\textcolor[rgb]{0.68,0.00,0.00}{#1}}
\newcommand{\RegionMarkerTok}[1]{\textcolor[rgb]{0.00,0.23,0.31}{#1}}
\newcommand{\SpecialCharTok}[1]{\textcolor[rgb]{0.37,0.37,0.37}{#1}}
\newcommand{\SpecialStringTok}[1]{\textcolor[rgb]{0.13,0.47,0.30}{#1}}
\newcommand{\StringTok}[1]{\textcolor[rgb]{0.13,0.47,0.30}{#1}}
\newcommand{\VariableTok}[1]{\textcolor[rgb]{0.07,0.07,0.07}{#1}}
\newcommand{\VerbatimStringTok}[1]{\textcolor[rgb]{0.13,0.47,0.30}{#1}}
\newcommand{\WarningTok}[1]{\textcolor[rgb]{0.37,0.37,0.37}{\textit{#1}}}

\providecommand{\tightlist}{%
  \setlength{\itemsep}{0pt}\setlength{\parskip}{0pt}}\usepackage{longtable,booktabs,array}
\usepackage{calc} % for calculating minipage widths
% Correct order of tables after \paragraph or \subparagraph
\usepackage{etoolbox}
\makeatletter
\patchcmd\longtable{\par}{\if@noskipsec\mbox{}\fi\par}{}{}
\makeatother
% Allow footnotes in longtable head/foot
\IfFileExists{footnotehyper.sty}{\usepackage{footnotehyper}}{\usepackage{footnote}}
\makesavenoteenv{longtable}
\usepackage{graphicx}
\makeatletter
\def\maxwidth{\ifdim\Gin@nat@width>\linewidth\linewidth\else\Gin@nat@width\fi}
\def\maxheight{\ifdim\Gin@nat@height>\textheight\textheight\else\Gin@nat@height\fi}
\makeatother
% Scale images if necessary, so that they will not overflow the page
% margins by default, and it is still possible to overwrite the defaults
% using explicit options in \includegraphics[width, height, ...]{}
\setkeys{Gin}{width=\maxwidth,height=\maxheight,keepaspectratio}
% Set default figure placement to htbp
\makeatletter
\def\fps@figure{htbp}
\makeatother

\KOMAoption{captions}{tableheading}
\makeatletter
\@ifpackageloaded{caption}{}{\usepackage{caption}}
\AtBeginDocument{%
\ifdefined\contentsname
  \renewcommand*\contentsname{Table of contents}
\else
  \newcommand\contentsname{Table of contents}
\fi
\ifdefined\listfigurename
  \renewcommand*\listfigurename{List of Figures}
\else
  \newcommand\listfigurename{List of Figures}
\fi
\ifdefined\listtablename
  \renewcommand*\listtablename{List of Tables}
\else
  \newcommand\listtablename{List of Tables}
\fi
\ifdefined\figurename
  \renewcommand*\figurename{Figure}
\else
  \newcommand\figurename{Figure}
\fi
\ifdefined\tablename
  \renewcommand*\tablename{Table}
\else
  \newcommand\tablename{Table}
\fi
}
\@ifpackageloaded{float}{}{\usepackage{float}}
\floatstyle{ruled}
\@ifundefined{c@chapter}{\newfloat{codelisting}{h}{lop}}{\newfloat{codelisting}{h}{lop}[chapter]}
\floatname{codelisting}{Stan

Program}
\newcommand*\listoflistings{\listof{codelisting}{List of Listings}}
\makeatother
\makeatletter
\makeatother
\makeatletter
\@ifpackageloaded{caption}{}{\usepackage{caption}}
\@ifpackageloaded{subcaption}{}{\usepackage{subcaption}}
\makeatother
\ifLuaTeX
  \usepackage{selnolig}  % disable illegal ligatures
\fi
\usepackage{bookmark}

\IfFileExists{xurl.sty}{\usepackage{xurl}}{} % add URL line breaks if available
\urlstyle{same} % disable monospaced font for URLs
\hypersetup{
  pdftitle={Modeling Conception For Cancer Patients},
  pdfauthor={Michael Betancourt},
  colorlinks=true,
  linkcolor={blue},
  filecolor={Maroon},
  citecolor={Blue},
  urlcolor={Blue},
  pdfcreator={LaTeX via pandoc}}

\title{Modeling Conception For Cancer Patients}
\author{Michael Betancourt}
\date{May 2025}

\begin{document}
\maketitle

\renewcommand*\contentsname{Table of contents}
{
\hypersetup{linkcolor=}
\setcounter{tocdepth}{3}
\tableofcontents
}
Most epidemiological analyses are subject a variety of subtle
challenges. For example many health outcomes are influenced by exposures
not only directly but also indirectly; if we cannot quantify both
consistently then we will not be able to make accurate predictions for
how health will vary under the various circumstances of interest.
Moreover because these exposures will generally vary across the
individuals within any given population the corresponding health
outcomes will vary from one population to another.

In this case study we'll see how Bayesian modeling and inference can be
used to manages these difficulties and extract productive insights.

\section{Setup}\label{setup}

First and foremost we have to set up the local \texttt{R} environment.

\begin{Shaded}
\begin{Highlighting}[]
\FunctionTok{par}\NormalTok{(}\AttributeTok{family=}\StringTok{"serif"}\NormalTok{, }\AttributeTok{las=}\DecValTok{1}\NormalTok{, }\AttributeTok{bty=}\StringTok{"l"}\NormalTok{,}
    \AttributeTok{cex.axis=}\DecValTok{1}\NormalTok{, }\AttributeTok{cex.lab=}\DecValTok{1}\NormalTok{, }\AttributeTok{cex.main=}\DecValTok{1}\NormalTok{,}
    \AttributeTok{xaxs=}\StringTok{"i"}\NormalTok{, }\AttributeTok{yaxs=}\StringTok{"i"}\NormalTok{, }\AttributeTok{mar =} \FunctionTok{c}\NormalTok{(}\DecValTok{5}\NormalTok{, }\DecValTok{5}\NormalTok{, }\DecValTok{3}\NormalTok{, }\DecValTok{5}\NormalTok{))}
\end{Highlighting}
\end{Shaded}

\begin{Shaded}
\begin{Highlighting}[]
\FunctionTok{library}\NormalTok{(rstan)}
\FunctionTok{rstan\_options}\NormalTok{(}\AttributeTok{auto\_write =} \ConstantTok{TRUE}\NormalTok{)            }\CommentTok{\# Cache compiled Stan programs}
\FunctionTok{options}\NormalTok{(}\AttributeTok{mc.cores =}\NormalTok{ parallel}\SpecialCharTok{::}\FunctionTok{detectCores}\NormalTok{()) }\CommentTok{\# Parallelize chains}
\NormalTok{parallel}\SpecialCharTok{:::}\FunctionTok{setDefaultClusterOptions}\NormalTok{(}\AttributeTok{setup\_strategy =} \StringTok{"sequential"}\NormalTok{)}
\end{Highlighting}
\end{Shaded}

Next we'll load some utility functions into the local environment to
facilitate the implementation of Bayesian inference.

\begin{Shaded}
\begin{Highlighting}[]
\NormalTok{util }\OtherTok{\textless{}{-}} \FunctionTok{new.env}\NormalTok{()}
\end{Highlighting}
\end{Shaded}

First we have a suite Markov chain Monte Carlo diagnostics and
estimation tools; this code and supporting documentation are both
available on
\href{https://github.com/betanalpha/mcmc_diagnostics}{GitHub}.

\begin{Shaded}
\begin{Highlighting}[]
\FunctionTok{source}\NormalTok{(}\StringTok{\textquotesingle{}mcmc\_analysis\_tools\_rstan.R\textquotesingle{}}\NormalTok{, }\AttributeTok{local=}\NormalTok{util)}
\end{Highlighting}
\end{Shaded}

Second we have a suite of probabilistic visualization functions based on
Markov chain Monte Carlo estimation. Again the code and supporting
documentation are available on
\href{https://github.com/betanalpha/mcmc_visualization_tools}{GitHub}.

\begin{Shaded}
\begin{Highlighting}[]
\FunctionTok{source}\NormalTok{(}\StringTok{\textquotesingle{}mcmc\_visualization\_tools.R\textquotesingle{}}\NormalTok{, }\AttributeTok{local=}\NormalTok{util)}
\end{Highlighting}
\end{Shaded}

\section{Data Exploration}\label{data-exploration}

For this analysis we will be analyzing fertility across a cohort of male
patients. Each patient was recruited through a referral for fertility
preservation consultation and observed for the same twelve month period.
Many of the referrals were also coincident with a cancer diagnosis.

\begin{Shaded}
\begin{Highlighting}[]
\NormalTok{data }\OtherTok{\textless{}{-}} \FunctionTok{read\_rdump}\NormalTok{(}\StringTok{"data/conception.data.R"}\NormalTok{)}

\FunctionTok{names}\NormalTok{(data)}
\end{Highlighting}
\end{Shaded}

\begin{verbatim}
 [1] "k_trt" "y"     "K_stg" "k_art" "K_rel" "K_trt" "k_tox" "N"     "k_stg"
[10] "k_rel" "K_tox"
\end{verbatim}

The main component of the data is collection of binary observations
indicating whether or not each patient in the observed cohort conceived
a child during a particular year. Specifically \(y_{n} = 0\) indicates
that the \(n\)th patient did not conceive a child while \(y_{n} = 1\)
indicates that they conceived at least one child.

About two-thirds of the patients conceived a child.

\begin{Shaded}
\begin{Highlighting}[]
\FunctionTok{par}\NormalTok{(}\AttributeTok{mfrow=}\FunctionTok{c}\NormalTok{(}\DecValTok{1}\NormalTok{, }\DecValTok{1}\NormalTok{), }\AttributeTok{mar=}\FunctionTok{c}\NormalTok{(}\DecValTok{5}\NormalTok{, }\DecValTok{5}\NormalTok{, }\DecValTok{1}\NormalTok{, }\DecValTok{1}\NormalTok{))}

\NormalTok{util}\SpecialCharTok{$}\FunctionTok{plot\_line\_hist}\NormalTok{(data}\SpecialCharTok{$}\NormalTok{y, }\SpecialCharTok{{-}}\FloatTok{0.5}\NormalTok{, }\FloatTok{1.5}\NormalTok{, }\DecValTok{1}\NormalTok{,}
                    \AttributeTok{xlab=}\StringTok{"Observed Conception Status"}\NormalTok{)}
\end{Highlighting}
\end{Shaded}

\includegraphics{analysis_files/figure-pdf/unnamed-chunk-9-1.pdf}

We can also clarify the relative frequencies of these fertility outcomes
by normalizing the summed counts.

\begin{Shaded}
\begin{Highlighting}[]
\FunctionTok{par}\NormalTok{(}\AttributeTok{mfrow=}\FunctionTok{c}\NormalTok{(}\DecValTok{1}\NormalTok{, }\DecValTok{1}\NormalTok{), }\AttributeTok{mar=}\FunctionTok{c}\NormalTok{(}\DecValTok{5}\NormalTok{, }\DecValTok{5}\NormalTok{, }\DecValTok{1}\NormalTok{, }\DecValTok{1}\NormalTok{))}

\NormalTok{util}\SpecialCharTok{$}\FunctionTok{plot\_line\_hist}\NormalTok{(data}\SpecialCharTok{$}\NormalTok{y, }\SpecialCharTok{{-}}\FloatTok{0.5}\NormalTok{, }\FloatTok{1.5}\NormalTok{, }\DecValTok{1}\NormalTok{, }\AttributeTok{prob=}\ConstantTok{TRUE}\NormalTok{,}
                    \AttributeTok{xlab=}\StringTok{"Observed Conception Status"}\NormalTok{)}
\end{Highlighting}
\end{Shaded}

\includegraphics{analysis_files/figure-pdf/unnamed-chunk-10-1.pdf}

These fertility outcomes are complemented by clinical and demographic
information about the individual patients in the cohort (\textbf{Table
1}). Note that these are categorical variables are ordered, with a clear
notion of increasing severity. Moreover all of these variables are
indexed from \texttt{1}, even those with only two values; this is
helpful when working with \texttt{1}-indexed programming languages like
\texttt{Stan}.

\begin{Shaded}
\begin{Highlighting}[]
\FunctionTok{par}\NormalTok{(}\AttributeTok{mfrow=}\FunctionTok{c}\NormalTok{(}\DecValTok{2}\NormalTok{, }\DecValTok{2}\NormalTok{), }\AttributeTok{mar=}\FunctionTok{c}\NormalTok{(}\DecValTok{5}\NormalTok{, }\DecValTok{5}\NormalTok{, }\DecValTok{1}\NormalTok{, }\DecValTok{1}\NormalTok{))}

\NormalTok{util}\SpecialCharTok{$}\FunctionTok{plot\_line\_hist}\NormalTok{(data}\SpecialCharTok{$}\NormalTok{k\_rel, }\FloatTok{0.5}\NormalTok{, data}\SpecialCharTok{$}\NormalTok{K\_rel }\SpecialCharTok{+} \FloatTok{0.5}\NormalTok{, }\DecValTok{1}\NormalTok{,}
                    \AttributeTok{xlab=}\StringTok{"Observed Relationship Status"}\NormalTok{)}

\NormalTok{util}\SpecialCharTok{$}\FunctionTok{plot\_line\_hist}\NormalTok{(data}\SpecialCharTok{$}\NormalTok{k\_stg, }\FloatTok{0.5}\NormalTok{, data}\SpecialCharTok{$}\NormalTok{K\_stg }\SpecialCharTok{+} \FloatTok{0.5}\NormalTok{, }\DecValTok{1}\NormalTok{,}
                    \AttributeTok{xlab=}\StringTok{"Observed Cancer Stage"}\NormalTok{)}

\NormalTok{util}\SpecialCharTok{$}\FunctionTok{plot\_line\_hist}\NormalTok{(data}\SpecialCharTok{$}\NormalTok{k\_trt, }\FloatTok{0.5}\NormalTok{, data}\SpecialCharTok{$}\NormalTok{K\_trt }\SpecialCharTok{+} \FloatTok{0.5}\NormalTok{, }\DecValTok{1}\NormalTok{,}
                    \AttributeTok{xlab=}\StringTok{"Observed Treatment Status"}\NormalTok{)}

\NormalTok{util}\SpecialCharTok{$}\FunctionTok{plot\_line\_hist}\NormalTok{(data}\SpecialCharTok{$}\NormalTok{k\_tox, }\FloatTok{0.5}\NormalTok{, data}\SpecialCharTok{$}\NormalTok{K\_tox }\SpecialCharTok{+} \FloatTok{0.5}\NormalTok{, }\DecValTok{1}\NormalTok{,}
                    \AttributeTok{xlab=}\StringTok{"Observed Toxicity Status"}\NormalTok{)}
\end{Highlighting}
\end{Shaded}

\includegraphics{analysis_files/figure-pdf/unnamed-chunk-11-1.pdf}

Finally some of the patients in the observed cohort have taken advantage
of assisted reproduction technologies, or ART, which can drastically
increase fertility. Similar to the fertility outcomes, and unlike the
clinical and demographic information, ART participation is \texttt{0-1}
coded, with \texttt{0} indicating no participation and \texttt{1}
participation.

\begin{Shaded}
\begin{Highlighting}[]
\FunctionTok{par}\NormalTok{(}\AttributeTok{mfrow=}\FunctionTok{c}\NormalTok{(}\DecValTok{1}\NormalTok{, }\DecValTok{1}\NormalTok{), }\AttributeTok{mar=}\FunctionTok{c}\NormalTok{(}\DecValTok{5}\NormalTok{, }\DecValTok{5}\NormalTok{, }\DecValTok{1}\NormalTok{, }\DecValTok{1}\NormalTok{))}

\NormalTok{util}\SpecialCharTok{$}\FunctionTok{plot\_line\_hist}\NormalTok{(data}\SpecialCharTok{$}\NormalTok{k\_art, }\SpecialCharTok{{-}}\FloatTok{0.5}\NormalTok{, }\FloatTok{1.5}\NormalTok{, }\DecValTok{1}\NormalTok{,}
                    \AttributeTok{xlab=}\StringTok{"Observed ART Status"}\NormalTok{)}
\end{Highlighting}
\end{Shaded}

\includegraphics{analysis_files/figure-pdf/unnamed-chunk-12-1.pdf}

Note that these data are not actually real but rather have been
simulated from an epidemiologically-motivated true model for
demonstration purposes. Consequently these observations are a bit more
well-behaved than real data tends to be. Moreover there are no privacy
concerns.

\section{Model 1}\label{model-1}

Any epidemiological system is inherently complex. To avoid being
overwhelmed by this complexity we'll start with a relatively simple
\href{https://betanalpha.github.io/assets/case_studies/modeling_and_inference.html\#11_the_observational_process}{observational
model}.

\subsection{The Observational Model}\label{the-observational-model}

Let's assume that the conception probability is homogeneous across all
patients in the observed cohort, \[
p( y_{1}, \ldots, y_{N} \mid q_{C})
=
\prod_{n = 1}^{N} p(y_{n} \mid q_{C})
=
\prod_{n = 1}^{N} \text{Bernoulli}(y_{n} \mid q_{C}).
\]

\subsection{The Prior Model}\label{the-prior-model}

To elevate this observational model into a
\href{https://betanalpha.github.io/assets/case_studies/modeling_and_inference.html\#314_the_complete_bayesian_model}{full
Bayesian model} we need to complement it with a
\href{https://betanalpha.github.io/assets/case_studies/prior_modeling.html}{prior
model} for the conception probability.

Prior modeling in a demonstration is always tricky because few, if any,
readers will share the same domain expertise. For this analysis let's
just say that the available domain expertise is inconsistent with
conception probabilities below \(0.10\) and above \(0.95\); these values
are not outright impossible but much more extreme than the intermediate
values. I will denote this soft constraint as \[
0.10 \lessapprox q_{C} \lessapprox 0.95.
\]

Next we have to find a probability distribution over the unit interval
that is consistent with these thresholds. Conveniently the
\href{https://betanalpha.github.io/assets/case_studies/probability_densities.html\#24_the_beta_family}{beta
family} of probability density functions specifies a diversity of
candidate probability distributions over the unit interval.

To select a prior model from these candidates we need to define
consistency, although we don't have to be too previous. I like to define
consistency with tail probability conditions, \begin{align*}
\delta
&=
\pi( \, [0.00, 0.10] \, )
=
\int_{0.00}^{0.10} \mathrm{d} q_{C} \,
                   \text{beta}(q_{C} \mid \alpha, \beta)
\\
1 - \delta
&=
\pi( \, [0.95, 1.00] \, )
=
\int_{0.95}^{1.00} \mathrm{d} q_{C} \,
                   \text{beta}(q_{C} \mid \alpha, \beta).
\end{align*} with \(\delta = 0.01\).

One nice benefit of this definition of consistency is that we can
numerically solve for the parameters \(\alpha\) and \(\beta\) that
identify the compatible beta probability density function.

\begin{Shaded}
\begin{Highlighting}[]
\NormalTok{q\_low }\OtherTok{\textless{}{-}} \FloatTok{0.1}
\NormalTok{q\_high }\OtherTok{\textless{}{-}} \FloatTok{0.95}

\FunctionTok{stan}\NormalTok{(}\AttributeTok{file=}\StringTok{\textquotesingle{}stan\_programs/prior\_tune\_beta.stan\textquotesingle{}}\NormalTok{,}
     \AttributeTok{data=}\FunctionTok{list}\NormalTok{(}\StringTok{\textquotesingle{}q\_low\textquotesingle{}} \OtherTok{=}\NormalTok{ q\_low, }\StringTok{\textquotesingle{}q\_high\textquotesingle{}} \OtherTok{=}\NormalTok{ q\_high),}
     \AttributeTok{iter=}\DecValTok{1}\NormalTok{, }\AttributeTok{warmup=}\DecValTok{0}\NormalTok{, }\AttributeTok{chains=}\DecValTok{1}\NormalTok{,}
     \AttributeTok{seed=}\DecValTok{4838282}\NormalTok{, }\AttributeTok{algorithm=}\StringTok{"Fixed\_param"}\NormalTok{)}
\end{Highlighting}
\end{Shaded}

\begin{verbatim}
alpha = 2.52283
beta = 2.02117

SAMPLING FOR MODEL 'anon_model' NOW (CHAIN 1).
Chain 1: Iteration: 1 / 1 [100%]  (Sampling)
Chain 1: 
Chain 1:  Elapsed Time: 0 seconds (Warm-up)
Chain 1:                0 seconds (Sampling)
Chain 1:                0 seconds (Total)
Chain 1: 
\end{verbatim}

\begin{verbatim}
Inference for Stan model: anon_model.
1 chains, each with iter=1; warmup=0; thin=1; 
post-warmup draws per chain=1, total post-warmup draws=1.

      mean se_mean sd 2.5%  25%  50%  75% 97.5% n_eff Rhat
alpha 2.52      NA NA 2.52 2.52 2.52 2.52  2.52     0  NaN
beta  2.02      NA NA 2.02 2.02 2.02 2.02  2.02     0  NaN
lp__  0.00      NA NA 0.00 0.00 0.00 0.00  0.00     0  NaN

Samples were drawn using (diag_e) at Tue Apr 29 21:10:39 2025.
For each parameter, n_eff is a crude measure of effective sample size,
and Rhat is the potential scale reduction factor on split chains (at 
convergence, Rhat=1).
\end{verbatim}

\begin{Shaded}
\begin{Highlighting}[]
\FunctionTok{par}\NormalTok{(}\AttributeTok{mfrow=}\FunctionTok{c}\NormalTok{(}\DecValTok{1}\NormalTok{, }\DecValTok{1}\NormalTok{), }\AttributeTok{mar=}\FunctionTok{c}\NormalTok{(}\DecValTok{5}\NormalTok{, }\DecValTok{5}\NormalTok{, }\DecValTok{1}\NormalTok{, }\DecValTok{1}\NormalTok{))}

\NormalTok{qs }\OtherTok{\textless{}{-}} \FunctionTok{seq}\NormalTok{(}\DecValTok{0}\NormalTok{, }\DecValTok{1}\NormalTok{, }\FloatTok{0.001}\NormalTok{)}
\NormalTok{dens }\OtherTok{\textless{}{-}} \FunctionTok{dbeta}\NormalTok{(qs, }\FloatTok{2.5}\NormalTok{, }\FloatTok{2.0}\NormalTok{)}
\FunctionTok{plot}\NormalTok{(qs, dens, }\AttributeTok{type=}\StringTok{"l"}\NormalTok{, }\AttributeTok{col=}\NormalTok{util}\SpecialCharTok{$}\NormalTok{c\_dark, }\AttributeTok{lwd=}\DecValTok{2}\NormalTok{,}
     \AttributeTok{xlab=}\StringTok{"Conception Probability"}\NormalTok{,}
     \AttributeTok{ylab=}\StringTok{"Prior Density"}\NormalTok{, }\AttributeTok{yaxt=}\StringTok{\textquotesingle{}n\textquotesingle{}}\NormalTok{)}

\NormalTok{q98 }\OtherTok{\textless{}{-}} \FunctionTok{seq}\NormalTok{(q\_low, q\_high, }\FloatTok{0.001}\NormalTok{)}
\NormalTok{dens }\OtherTok{\textless{}{-}} \FunctionTok{dbeta}\NormalTok{(q98, }\FloatTok{2.5}\NormalTok{, }\FloatTok{2.0}\NormalTok{)}
\NormalTok{q98 }\OtherTok{\textless{}{-}} \FunctionTok{c}\NormalTok{(q98, q\_high, q\_low)}
\NormalTok{dens }\OtherTok{\textless{}{-}} \FunctionTok{c}\NormalTok{(dens, }\DecValTok{0}\NormalTok{, }\DecValTok{0}\NormalTok{)}

\FunctionTok{polygon}\NormalTok{(q98, dens, }\AttributeTok{col=}\NormalTok{util}\SpecialCharTok{$}\NormalTok{c\_dark, }\AttributeTok{border=}\ConstantTok{NA}\NormalTok{)}

\FunctionTok{abline}\NormalTok{(}\AttributeTok{v=}\NormalTok{q\_low,  }\AttributeTok{lwd=}\DecValTok{3}\NormalTok{, }\AttributeTok{lty=}\DecValTok{2}\NormalTok{, }\AttributeTok{col=}\StringTok{\textquotesingle{}\#DDDDDD\textquotesingle{}}\NormalTok{)}
\FunctionTok{abline}\NormalTok{(}\AttributeTok{v=}\NormalTok{q\_high, }\AttributeTok{lwd=}\DecValTok{3}\NormalTok{, }\AttributeTok{lty=}\DecValTok{2}\NormalTok{, }\AttributeTok{col=}\StringTok{\textquotesingle{}\#DDDDDD\textquotesingle{}}\NormalTok{)}
\end{Highlighting}
\end{Shaded}

\includegraphics{analysis_files/figure-pdf/unnamed-chunk-14-1.pdf}

\subsection{Posterior Quantification}\label{posterior-quantification}

Putting everything together we can summarize the structure of the full
Bayesian model with a
\href{https://betanalpha.github.io/assets/case_studies/probability_on_product_spaces.html\#4_Directed_Graphical_Models}{directed
graph} (Figure~\ref{fig-model1}) and implement the full Bayesian model
in a \texttt{Stan} program.

\begin{figure}

\centering{

\includegraphics[width=0.5\textwidth,height=\textheight]{figures/model1/model1.pdf}

}

\caption{\label{fig-model1}A directed graphical model visually
summarizes the narratively generative structure of our initial model.}

\end{figure}%

\begin{codelisting}

\caption{\texttt{model1.stan}}

\begin{Shaded}
\begin{Highlighting}[]
\KeywordTok{data}\NormalTok{ \{}
  \CommentTok{// Number of observations}
  \DataTypeTok{int}\NormalTok{\textless{}}\KeywordTok{lower}\NormalTok{=}\DecValTok{1}\NormalTok{\textgreater{} N;}

  \CommentTok{// Observed conception status}
  \CommentTok{// y = 0: No conception}
  \CommentTok{// y = 1: Conception}
  \DataTypeTok{array}\NormalTok{[N] }\DataTypeTok{int}\NormalTok{\textless{}}\KeywordTok{lower}\NormalTok{=}\DecValTok{0}\NormalTok{, }\KeywordTok{upper}\NormalTok{=}\DecValTok{1}\NormalTok{\textgreater{} y;}
\NormalTok{\}}

\KeywordTok{parameters}\NormalTok{ \{}
  \DataTypeTok{real}\NormalTok{\textless{}}\KeywordTok{lower}\NormalTok{=}\DecValTok{0}\NormalTok{, }\KeywordTok{upper}\NormalTok{=}\DecValTok{1}\NormalTok{\textgreater{} q\_C; }\CommentTok{// Conception probability}
\NormalTok{\}}

\KeywordTok{model}\NormalTok{ \{}
  \CommentTok{// Prior model}
  \KeywordTok{target +=}\NormalTok{ beta\_lpdf(q\_C | }\FloatTok{2.5}\NormalTok{, }\FloatTok{2.0}\NormalTok{); }\CommentTok{// 0.10 \textless{}\textasciitilde{} q\_C \textless{}\textasciitilde{} 0.95}

  \CommentTok{// Observational model}
  \KeywordTok{target +=}\NormalTok{ bernoulli\_lpmf(y | q\_C);}

  \CommentTok{// Also valid but slightly less efficient}
  \CommentTok{// for (n in 1:N) \{}
  \CommentTok{//   target += bernoulli\_lpmf(y[n] | q\_C);}
  \CommentTok{// \}}
\NormalTok{\}}

\KeywordTok{generated quantities}\NormalTok{ \{}
  \CommentTok{// Posterior predictive data}
  \DataTypeTok{array}\NormalTok{[N] }\DataTypeTok{int}\NormalTok{\textless{}}\KeywordTok{lower}\NormalTok{=}\DecValTok{0}\NormalTok{, }\KeywordTok{upper}\NormalTok{=}\DecValTok{1}\NormalTok{\textgreater{} y\_pred;}

  \ControlFlowTok{for}\NormalTok{ (n }\ControlFlowTok{in} \DecValTok{1}\NormalTok{:N) \{}
\NormalTok{    y\_pred[n] = bernoulli\_rng(q\_C);}
\NormalTok{  \}}
\NormalTok{\}}
\end{Highlighting}
\end{Shaded}

\end{codelisting}

With this Stan program we can then run Markov chain Monte Carlo to
extract information about the posterior distribution that can be used to
estimate posterior expectation values.

\begin{Shaded}
\begin{Highlighting}[]
\NormalTok{fit }\OtherTok{\textless{}{-}} \FunctionTok{stan}\NormalTok{(}\AttributeTok{file=}\StringTok{"stan\_programs/model1.stan"}\NormalTok{,}
            \AttributeTok{data=}\NormalTok{data, }\AttributeTok{seed=}\DecValTok{8438338}\NormalTok{,}
            \AttributeTok{warmup=}\DecValTok{1000}\NormalTok{, }\AttributeTok{iter=}\DecValTok{2024}\NormalTok{, }\AttributeTok{refresh=}\DecValTok{0}\NormalTok{)}
\end{Highlighting}
\end{Shaded}

Before doing anything else we have to check for any signs that the
posterior computation might be inaccurate. Fortunately there are no
diagnostic warnings suggesting any problems.

\begin{Shaded}
\begin{Highlighting}[]
\NormalTok{diagnostics }\OtherTok{\textless{}{-}}\NormalTok{ util}\SpecialCharTok{$}\FunctionTok{extract\_hmc\_diagnostics}\NormalTok{(fit)}
\NormalTok{util}\SpecialCharTok{$}\FunctionTok{check\_all\_hmc\_diagnostics}\NormalTok{(diagnostics)}
\end{Highlighting}
\end{Shaded}

\begin{verbatim}
  All Hamiltonian Monte Carlo diagnostics are consistent with reliable
Markov chain Monte Carlo.
\end{verbatim}

\begin{Shaded}
\begin{Highlighting}[]
\NormalTok{samples1 }\OtherTok{\textless{}{-}}\NormalTok{ util}\SpecialCharTok{$}\FunctionTok{extract\_expectand\_vals}\NormalTok{(fit)}
\NormalTok{base\_samples }\OtherTok{\textless{}{-}}\NormalTok{ util}\SpecialCharTok{$}\FunctionTok{filter\_expectands}\NormalTok{(samples1,}
                                       \FunctionTok{c}\NormalTok{(}\StringTok{\textquotesingle{}q\_C\textquotesingle{}}\NormalTok{))}
\NormalTok{util}\SpecialCharTok{$}\FunctionTok{check\_all\_expectand\_diagnostics}\NormalTok{(base\_samples)}
\end{Highlighting}
\end{Shaded}

\begin{verbatim}
All expectands checked appear to be behaving well enough for reliable
Markov chain Monte Carlo estimation.
\end{verbatim}

\subsection{Retrodictive Checks}\label{retrodictive-checks}

Next we need to evaluate the adequacy of our model by comparing the
behavior of the observed data to the posterior predictive distribution
with a
\href{https://betanalpha.github.io/assets/case_studies/principled_bayesian_workflow.html\#143_Posterior_Retrodiction_Checks}{posterior
retrodictive check}.

Here we'll consider a posterior retrodictive check using the histogram
summary statistic that we that we used when exploring the data.
Fortunately there is no sign of tension between the observed histogram
and the probability distribution of posterior predictive histograms

\begin{Shaded}
\begin{Highlighting}[]
\FunctionTok{par}\NormalTok{(}\AttributeTok{mfrow=}\FunctionTok{c}\NormalTok{(}\DecValTok{1}\NormalTok{, }\DecValTok{1}\NormalTok{), }\AttributeTok{mar=}\FunctionTok{c}\NormalTok{(}\DecValTok{5}\NormalTok{, }\DecValTok{5}\NormalTok{, }\DecValTok{1}\NormalTok{, }\DecValTok{1}\NormalTok{))}

\NormalTok{util}\SpecialCharTok{$}\FunctionTok{plot\_hist\_quantiles}\NormalTok{(samples1, }\StringTok{\textquotesingle{}y\_pred\textquotesingle{}}\NormalTok{, }\SpecialCharTok{{-}}\FloatTok{0.5}\NormalTok{, }\FloatTok{1.5}\NormalTok{, }\DecValTok{1}\NormalTok{,}
                         \AttributeTok{baseline\_values=}\NormalTok{data}\SpecialCharTok{$}\NormalTok{y,}
                         \AttributeTok{xlab=}\StringTok{"Observed Conception Status"}\NormalTok{)}
\end{Highlighting}
\end{Shaded}

\includegraphics{analysis_files/figure-pdf/unnamed-chunk-17-1.pdf}

\subsection{Posterior Insights}\label{posterior-insights}

Flush with confidence in our modeling assumptions, at least in the
context of the lone summary statistic we considered, we can examine the
resulting posterior inferences.

The posterior distribution concentrates on conception probabilities near
2/3, consistent with the empirical behavior. One immediate benefit of
the Bayesian approach is that the posterior distribution captures all of
the conception probabilities consistent with the observed data,
quantifying \emph{uncertainty} in our insights.

\begin{Shaded}
\begin{Highlighting}[]
\FunctionTok{par}\NormalTok{(}\AttributeTok{mfrow=}\FunctionTok{c}\NormalTok{(}\DecValTok{1}\NormalTok{, }\DecValTok{1}\NormalTok{), }\AttributeTok{mar=}\FunctionTok{c}\NormalTok{(}\DecValTok{5}\NormalTok{, }\DecValTok{5}\NormalTok{, }\DecValTok{1}\NormalTok{, }\DecValTok{1}\NormalTok{))}
\NormalTok{util}\SpecialCharTok{$}\FunctionTok{plot\_expectand\_pushforward}\NormalTok{(samples1[[}\StringTok{\textquotesingle{}q\_C\textquotesingle{}}\NormalTok{]],}
                                \DecValTok{25}\NormalTok{,}
                                \AttributeTok{display\_name=}\StringTok{"Conception Probability"}\NormalTok{)}
\end{Highlighting}
\end{Shaded}

\includegraphics{analysis_files/figure-pdf/unnamed-chunk-18-1.pdf}

When examining inferences for probability parameters it's often helpful
to plot the full range of possible values and offer as much context as
possible.

\begin{Shaded}
\begin{Highlighting}[]
\FunctionTok{par}\NormalTok{(}\AttributeTok{mfrow=}\FunctionTok{c}\NormalTok{(}\DecValTok{1}\NormalTok{, }\DecValTok{1}\NormalTok{), }\AttributeTok{mar=}\FunctionTok{c}\NormalTok{(}\DecValTok{5}\NormalTok{, }\DecValTok{5}\NormalTok{, }\DecValTok{1}\NormalTok{, }\DecValTok{1}\NormalTok{))}
\NormalTok{util}\SpecialCharTok{$}\FunctionTok{plot\_expectand\_pushforward}\NormalTok{(samples1[[}\StringTok{\textquotesingle{}q\_C\textquotesingle{}}\NormalTok{]],}
                                \DecValTok{200}\NormalTok{, }\AttributeTok{flim=}\FunctionTok{c}\NormalTok{(}\DecValTok{0}\NormalTok{, }\DecValTok{1}\NormalTok{),}
                                \AttributeTok{display\_name=}\StringTok{"Conception Probability"}\NormalTok{)}
\end{Highlighting}
\end{Shaded}

\includegraphics{analysis_files/figure-pdf/unnamed-chunk-19-1.pdf}

\section{Model 2}\label{model-2}

In practice the adequacy of a model is determined by the criteria we use
to critique it. Our initial model adequately captures the
\emph{aggregate} conception behavior across the observed cohort, but
that doesn't mean it will be able to capture finer details.

For example there's no reason why patient fertility should not vary
across most of the available clinical and demographic categories. A male
patient in a stable relationship with a female partner is more likely to
conceive than one who is not. Similarly more aggressive cancer, and the
ancillary toxicity common to most cancer treatments, is likely to reduce
fertility and hence conception probability.

Fertility should also vary with treatment, but only as a side effect of
treatment toxicity. Because we're modeling treatment toxicity directly
we don't need to consider heterogeneity across the treatment groups.

The key question for a practical analysis is \emph{not} whether or not
the variations in conception probability are zero but rather whether or
not they are large enough to manifest in the observed data. Because our
initial model assumes homogeneous conception probabilities the posterior
predictive conception behavior should be the same no matter how we
partition the patients. If the heterogeneity in fertility is strong
enough then the observed behaviors will fall outside of the posterior
predictive uncertainties, indicating the inadequacy of our initial
homogeneous model.

There are many ways to stratify retrodictive comparisons across
categories. Here I'm going to use the conditional mean summary statistic
introduced in Section 2.5 of my
\href{https://betanalpha.github.io/assets/case_studies/taylor_models.html\#25_Posterior_Retrodictive_Checks}{Taylor
modeling chapter} and implemented in my
\href{https://github.com/betanalpha/mcmc_visualization_tools/tree/main/r}{recommended
visualization tools}.

\begin{Shaded}
\begin{Highlighting}[]
\FunctionTok{par}\NormalTok{(}\AttributeTok{mfrow=}\FunctionTok{c}\NormalTok{(}\DecValTok{1}\NormalTok{, }\DecValTok{3}\NormalTok{), }\AttributeTok{mar=}\FunctionTok{c}\NormalTok{(}\DecValTok{5}\NormalTok{, }\DecValTok{5}\NormalTok{, }\DecValTok{1}\NormalTok{, }\DecValTok{1}\NormalTok{))}

\NormalTok{pred\_names }\OtherTok{\textless{}{-}} \FunctionTok{sapply}\NormalTok{(}\DecValTok{1}\SpecialCharTok{:}\NormalTok{data}\SpecialCharTok{$}\NormalTok{N, }\ControlFlowTok{function}\NormalTok{(n) }\FunctionTok{paste0}\NormalTok{(}\StringTok{\textquotesingle{}y\_pred[\textquotesingle{}}\NormalTok{, n, }\StringTok{\textquotesingle{}]\textquotesingle{}}\NormalTok{))}

\NormalTok{util}\SpecialCharTok{$}\FunctionTok{plot\_conditional\_mean\_quantiles}\NormalTok{(samples1, pred\_names, data}\SpecialCharTok{$}\NormalTok{k\_rel,}
                                     \FloatTok{0.5}\NormalTok{, data}\SpecialCharTok{$}\NormalTok{K\_rel }\SpecialCharTok{+} \FloatTok{0.5}\NormalTok{, }\DecValTok{1}\NormalTok{, data}\SpecialCharTok{$}\NormalTok{y,}
                                     \AttributeTok{xlab=}\StringTok{"Observed Relationship Status"}\NormalTok{,}
                                     \AttributeTok{ylab=}\StringTok{"Average Conception Status"}\NormalTok{)}

\NormalTok{util}\SpecialCharTok{$}\FunctionTok{plot\_conditional\_mean\_quantiles}\NormalTok{(samples1, pred\_names, data}\SpecialCharTok{$}\NormalTok{k\_stg,}
                                     \FloatTok{0.5}\NormalTok{, data}\SpecialCharTok{$}\NormalTok{K\_stg }\SpecialCharTok{+} \FloatTok{0.5}\NormalTok{, }\DecValTok{1}\NormalTok{, data}\SpecialCharTok{$}\NormalTok{y,}
                                     \AttributeTok{xlab=}\StringTok{"Observed Cancer Stage"}\NormalTok{,}
                                     \AttributeTok{ylab=}\StringTok{"Average Conception Status"}\NormalTok{)}

\NormalTok{util}\SpecialCharTok{$}\FunctionTok{plot\_conditional\_mean\_quantiles}\NormalTok{(samples1, pred\_names, data}\SpecialCharTok{$}\NormalTok{k\_tox,}
                                     \FloatTok{0.5}\NormalTok{, data}\SpecialCharTok{$}\NormalTok{K\_tox }\SpecialCharTok{+} \FloatTok{0.5}\NormalTok{, }\DecValTok{1}\NormalTok{, data}\SpecialCharTok{$}\NormalTok{y,}
                                     \AttributeTok{xlab=}\StringTok{"Observed Toxicity Status"}\NormalTok{,}
                                     \AttributeTok{ylab=}\StringTok{"Average Conception Status"}\NormalTok{)}
\end{Highlighting}
\end{Shaded}

\includegraphics{analysis_files/figure-pdf/unnamed-chunk-20-1.pdf}

Indeed we see clear disagreement between the observed and posterior
predictive behavior. This indicates that we need to incorporate
systematic variation in fertility in order to adequacy model this cohort
of patients.

\subsection{Observational Model}\label{observational-model}

One of the most productive ways to model variation across a population
is to define an interpretable \emph{baseline} and then model
\emph{deviations} around that baseline. Here we'll take our baseline to
be the subset of patients in a stable relationship, with no cancer
diagnosis, and no treatment toxicity, and use the parameter
\(q_{C_{0}}\) to model the baseline conception probability.

An immediate benefit of this choice of baseline is that fertility should
always \emph{decrease} as we move from the baseline to more extreme
patient characteristics. Increasing treatment toxicity, for instance,
should never increase fertility. Consequently we can model the
conception probability in other patient characteristics as
\emph{proportional decreases} from the baseline conception probability.

More formally for any clinical or demographic grouping we will model the
conception probability in the baseline group as \[
q = q_{C_{0}} \, \delta_{1}
\] for \(\delta_{1} = 0\), or equivalently \[
q = q_{C_{0}} \, \exp(-\alpha_{1})
\] for \(\alpha_{1} = 0\). Then we can model the conception probability
in the least extreme group beyond the baseline as \[
q = q_{C_{0}} \, \delta_{2}
\] for \(0 \le \delta_{1} \le 2\), or equivalently \[
q = q_{C_{0}} \, \exp(-\alpha_{2})
\] for \(\alpha_{2} > 0\). Similarly the conception probability in the
second least extreme group becomes \[
q = q_{C_{0}} \, \delta_{3}
\] for \(0 \le \delta_{3} < \delta_{2}\), or equivalently \[
q = q_{C_{0}} \, \exp(-\alpha_{3})
\] for \(\alpha_{3} > \alpha_{2}\). I will refer to the \(\alpha_{k}\)
for each clinical or demographic grouping as \emph{impairment
parameters}.

In order to model the variation across \(K\) different groups we need a
collection of \(K\) positive and ordered parameters that start at zero.
\[
0 = \alpha_{1} < \alpha_{2} < \ldots < \alpha_{k} < \ldots < \alpha_{K}.
\] This multivariate constraint can be tricky to maintain in practice.

Finally we repeat this construction three times to capture the variation
in fertility across each of the three patient characteristics under
consideration (Figure~\ref{fig-model2a}), \begin{align*}
q_{C, n}
&=
q_{C_{0}} \,
\exp(-\alpha_{\text{stg}, n} ) \,
\exp(-\alpha_{\text{rel}, n} ) \,
\exp(-\alpha_{\text{tox}, n} )
\\
&=
q_{C_{0}} \,
\exp(-\alpha_{\text{stg}, n}
     -\alpha_{\text{rel}, n}
     -\alpha_{\text{tox}, n} ).
\end{align*} where \begin{align*}
\alpha_{\text{stg}, n}
&= \boldsymbol{\alpha}_{\text{rel}} [ k_{\text{stg}, n} ]
\\
\alpha_{\text{rel}, n}
&= \boldsymbol{\alpha}_{\text{rel}} [ k_{\text{rel}, n} ]
\\
\alpha_{\text{tox}, n}
&= \boldsymbol{\alpha}_{\text{rel}} [ k_{\text{tox}, n} ]
\end{align*}

\begin{figure}

\centering{

\includegraphics[width=1\textwidth,height=\textheight]{figures/model2a/model2.pdf}

}

\caption{\label{fig-model2a}We model fertility heterogeneity by allowing
conception probability to vary across cancer stage, relationship status,
and treatment toxicity groups. More precisely the conception probability
for each patient \(q_{C, n}\) is modeled as a baseline conception
probability \(q_{C_{0}}\) coupled with proportional decreases depending
on group membership.}

\end{figure}%

\subsection{Prior Model}\label{prior-model}

In this new model we are no longer modeling a population-wide conception
probability but rather the conception probability for only the baseline
group of patients who are in stable relationships and have not been
diagnosed with cancer. If we have additional domain expertise about this
smaller group then we can incorporate it into a more informative prior
model.

Here let's say that our domain expertise is inconsistent with baseline
conception probabilities below \(0.5\) and above \(0.95\), \[
0.50 \lessapprox q_{C_{0}} \lessapprox 0.95.
\]

\begin{Shaded}
\begin{Highlighting}[]
\NormalTok{q\_low }\OtherTok{\textless{}{-}} \FloatTok{0.5}
\NormalTok{q\_high }\OtherTok{\textless{}{-}} \FloatTok{0.95}

\FunctionTok{stan}\NormalTok{(}\AttributeTok{file=}\StringTok{\textquotesingle{}stan\_programs/prior\_tune\_beta.stan\textquotesingle{}}\NormalTok{,}
     \AttributeTok{data=}\FunctionTok{list}\NormalTok{(}\StringTok{\textquotesingle{}q\_low\textquotesingle{}} \OtherTok{=}\NormalTok{ q\_low, }\StringTok{\textquotesingle{}q\_high\textquotesingle{}} \OtherTok{=}\NormalTok{ q\_high),}
     \AttributeTok{iter=}\DecValTok{1}\NormalTok{, }\AttributeTok{warmup=}\DecValTok{0}\NormalTok{, }\AttributeTok{chains=}\DecValTok{1}\NormalTok{,}
     \AttributeTok{seed=}\DecValTok{4838282}\NormalTok{, }\AttributeTok{algorithm=}\StringTok{"Fixed\_param"}\NormalTok{)}
\end{Highlighting}
\end{Shaded}

\begin{verbatim}
alpha = 12.6454
beta = 3.74419

SAMPLING FOR MODEL 'anon_model' NOW (CHAIN 1).
Chain 1: Iteration: 1 / 1 [100%]  (Sampling)
Chain 1: 
Chain 1:  Elapsed Time: 0 seconds (Warm-up)
Chain 1:                0 seconds (Sampling)
Chain 1:                0 seconds (Total)
Chain 1: 
\end{verbatim}

\begin{verbatim}
Inference for Stan model: anon_model.
1 chains, each with iter=1; warmup=0; thin=1; 
post-warmup draws per chain=1, total post-warmup draws=1.

       mean se_mean sd  2.5%   25%   50%   75% 97.5% n_eff Rhat
alpha 12.65      NA NA 12.65 12.65 12.65 12.65 12.65     0  NaN
beta   3.74      NA NA  3.74  3.74  3.74  3.74  3.74     0  NaN
lp__   0.00      NA NA  0.00  0.00  0.00  0.00  0.00     0  NaN

Samples were drawn using (diag_e) at Tue Apr 29 21:17:04 2025.
For each parameter, n_eff is a crude measure of effective sample size,
and Rhat is the potential scale reduction factor on split chains (at 
convergence, Rhat=1).
\end{verbatim}

\begin{Shaded}
\begin{Highlighting}[]
\FunctionTok{par}\NormalTok{(}\AttributeTok{mfrow=}\FunctionTok{c}\NormalTok{(}\DecValTok{1}\NormalTok{, }\DecValTok{1}\NormalTok{), }\AttributeTok{mar=}\FunctionTok{c}\NormalTok{(}\DecValTok{5}\NormalTok{, }\DecValTok{5}\NormalTok{, }\DecValTok{5}\NormalTok{, }\DecValTok{1}\NormalTok{))}

\NormalTok{qs }\OtherTok{\textless{}{-}} \FunctionTok{seq}\NormalTok{(}\DecValTok{0}\NormalTok{, }\DecValTok{1}\NormalTok{, }\FloatTok{0.001}\NormalTok{)}
\NormalTok{dens }\OtherTok{\textless{}{-}} \FunctionTok{dbeta}\NormalTok{(qs, }\FloatTok{12.7}\NormalTok{, }\FloatTok{3.7}\NormalTok{)}
\FunctionTok{plot}\NormalTok{(qs, dens, }\AttributeTok{type=}\StringTok{"l"}\NormalTok{, }\AttributeTok{col=}\NormalTok{util}\SpecialCharTok{$}\NormalTok{c\_dark, }\AttributeTok{lwd=}\DecValTok{2}\NormalTok{,}
     \AttributeTok{xlab=}\StringTok{"Baseline Conception Probability"}\NormalTok{,}
     \AttributeTok{ylab=}\StringTok{"Prior Density"}\NormalTok{, }\AttributeTok{yaxt=}\StringTok{\textquotesingle{}n\textquotesingle{}}\NormalTok{)}

\NormalTok{q98 }\OtherTok{\textless{}{-}} \FunctionTok{seq}\NormalTok{(q\_low, q\_high, }\FloatTok{0.001}\NormalTok{)}
\NormalTok{dens }\OtherTok{\textless{}{-}} \FunctionTok{dbeta}\NormalTok{(q98, }\FloatTok{12.7}\NormalTok{, }\FloatTok{3.7}\NormalTok{)}
\NormalTok{q98 }\OtherTok{\textless{}{-}} \FunctionTok{c}\NormalTok{(q98, q\_high, q\_low)}
\NormalTok{dens }\OtherTok{\textless{}{-}} \FunctionTok{c}\NormalTok{(dens, }\DecValTok{0}\NormalTok{, }\DecValTok{0}\NormalTok{)}

\FunctionTok{polygon}\NormalTok{(q98, dens, }\AttributeTok{col=}\NormalTok{util}\SpecialCharTok{$}\NormalTok{c\_dark, }\AttributeTok{border=}\ConstantTok{NA}\NormalTok{)}

\FunctionTok{abline}\NormalTok{(}\AttributeTok{v=}\NormalTok{q\_low,  }\AttributeTok{lwd=}\DecValTok{3}\NormalTok{, }\AttributeTok{lty=}\DecValTok{2}\NormalTok{, }\AttributeTok{col=}\StringTok{\textquotesingle{}\#DDDDDD\textquotesingle{}}\NormalTok{)}
\FunctionTok{abline}\NormalTok{(}\AttributeTok{v=}\NormalTok{q\_high, }\AttributeTok{lwd=}\DecValTok{3}\NormalTok{, }\AttributeTok{lty=}\DecValTok{2}\NormalTok{, }\AttributeTok{col=}\StringTok{\textquotesingle{}\#DDDDDD\textquotesingle{}}\NormalTok{)}
\end{Highlighting}
\end{Shaded}

\includegraphics{analysis_files/figure-pdf/unnamed-chunk-22-1.pdf}

To construct a prior model for these new fertility impairment parameters
we need to elicit any available domain expertise about the reasonable
proportional decreases across groups. For example let's say that our
domain expertise is inconsistent with any decreases below \(5\%\), \[
0.05 \lessapprox \delta \lessapprox 1.
\] This requires \begin{align*}
0.05 &\lessapprox \quad\quad \delta \;\; \lessapprox 1
\\
0.05 &\lessapprox \exp(\alpha) \lessapprox 1
\\
-\log(1) &\lessapprox \quad\quad \alpha \;\;\lessapprox -\log(0.05)
\\
0 &\lessapprox \quad\quad \alpha \;\; \lessapprox -\log(0.05).
\end{align*} We can achieve this prior containment with the half-normal
prior model \[
p( \alpha ) = \text{half-normal}( \alpha \mid 0, -\log(0.05) / 2.57)
\] that contains 99\% of the prior probability between \(0\) and
\(-\log(0.05)\).

\begin{Shaded}
\begin{Highlighting}[]
\NormalTok{q\_high }\OtherTok{\textless{}{-}} \SpecialCharTok{{-}}\FunctionTok{log}\NormalTok{(}\FloatTok{0.05}\NormalTok{)}

\FunctionTok{par}\NormalTok{(}\AttributeTok{mfrow=}\FunctionTok{c}\NormalTok{(}\DecValTok{1}\NormalTok{, }\DecValTok{1}\NormalTok{), }\AttributeTok{mar=}\FunctionTok{c}\NormalTok{(}\DecValTok{5}\NormalTok{, }\DecValTok{5}\NormalTok{, }\DecValTok{5}\NormalTok{, }\DecValTok{1}\NormalTok{))}

\NormalTok{qs }\OtherTok{\textless{}{-}} \FunctionTok{seq}\NormalTok{(}\DecValTok{0}\NormalTok{, }\FloatTok{3.5}\NormalTok{, }\FloatTok{0.001}\NormalTok{)}
\NormalTok{dens }\OtherTok{\textless{}{-}} \DecValTok{2} \SpecialCharTok{*} \FunctionTok{dnorm}\NormalTok{(qs, }\DecValTok{0}\NormalTok{, q\_high }\SpecialCharTok{/} \FloatTok{2.57}\NormalTok{ )}
\FunctionTok{plot}\NormalTok{(qs, dens, }\AttributeTok{type=}\StringTok{"l"}\NormalTok{, }\AttributeTok{col=}\NormalTok{util}\SpecialCharTok{$}\NormalTok{c\_dark, }\AttributeTok{lwd=}\DecValTok{2}\NormalTok{,}
     \AttributeTok{xlab=}\StringTok{"Conception Impairment"}\NormalTok{,}
     \AttributeTok{ylab=}\StringTok{"Prior Density"}\NormalTok{, }\AttributeTok{yaxt=}\StringTok{\textquotesingle{}n\textquotesingle{}}\NormalTok{)}

\NormalTok{q98 }\OtherTok{\textless{}{-}} \FunctionTok{seq}\NormalTok{(}\DecValTok{0}\NormalTok{, q\_high, }\FloatTok{0.001}\NormalTok{)}
\NormalTok{dens }\OtherTok{\textless{}{-}} \DecValTok{2} \SpecialCharTok{*} \FunctionTok{dnorm}\NormalTok{(q98, }\DecValTok{0}\NormalTok{, q\_high }\SpecialCharTok{/} \FloatTok{2.57}\NormalTok{ )}
\NormalTok{q98 }\OtherTok{\textless{}{-}} \FunctionTok{c}\NormalTok{(}\DecValTok{0}\NormalTok{, q98, }\SpecialCharTok{{-}}\FunctionTok{log}\NormalTok{(}\FloatTok{0.05}\NormalTok{))}
\NormalTok{dens }\OtherTok{\textless{}{-}} \FunctionTok{c}\NormalTok{(}\DecValTok{0}\NormalTok{, dens, }\DecValTok{0}\NormalTok{)}

\FunctionTok{polygon}\NormalTok{(q98, dens, }\AttributeTok{col=}\NormalTok{util}\SpecialCharTok{$}\NormalTok{c\_dark, }\AttributeTok{border=}\ConstantTok{NA}\NormalTok{)}

\FunctionTok{abline}\NormalTok{(}\AttributeTok{v=}\NormalTok{q\_high,  }\AttributeTok{lwd=}\DecValTok{3}\NormalTok{, }\AttributeTok{lty=}\DecValTok{2}\NormalTok{, }\AttributeTok{col=}\StringTok{\textquotesingle{}\#DDDDDD\textquotesingle{}}\NormalTok{)}
\end{Highlighting}
\end{Shaded}

\includegraphics{analysis_files/figure-pdf/unnamed-chunk-23-1.pdf}

\subsection{Posterior Quantification}\label{posterior-quantification-1}

The updated observational and prior models snap together into a more
elaborate full Bayesian model (Figure~\ref{fig-model2b}).

\begin{figure}

\centering{

\includegraphics[width=1\textwidth,height=\textheight]{figures/model2b/model2.pdf}

}

\caption{\label{fig-model2b}Our second model replaces the homogeneous
conception probability parameter from the first model with the varying
outputs of a deterministic function of the clinical and demographic
group memberships of each patient.}

\end{figure}%

Note that in the Stan programming language a normal log probability
density function is equivalent to a half-normal log probability density
function so long as the input variable is constrained to be positive.
Consequently we can implement half-normal prior models for the
impairment parameters using a normal probability density function.

In order to maintain the assumed constraints on the impairment
parameters for each non-baseline group, \[
0 < \alpha_{2} < \ldots < \alpha_{k} < \ldots < \alpha_{K},
\] we use the Stan programming language's \texttt{positive\_ordered}
variable type. We can then ensure the assumed constraint for all groups
including the baselines, \[
0 = \alpha_{1} < \alpha_{2} < \ldots < \alpha_{k} < \ldots < \alpha_{K},
\] by prepending the \texttt{positive\_ordered} variable with a zero.

\begin{codelisting}

\caption{\texttt{model2.stan}}

\begin{Shaded}
\begin{Highlighting}[]
\KeywordTok{data}\NormalTok{ \{}
  \CommentTok{// Number of observations}
  \DataTypeTok{int}\NormalTok{\textless{}}\KeywordTok{lower}\NormalTok{=}\DecValTok{1}\NormalTok{\textgreater{} N;}

  \CommentTok{// Relationship status}
  \CommentTok{// k = 1: Stable partner}
  \CommentTok{// k = 2: No partner}
  \DataTypeTok{int}\NormalTok{\textless{}}\KeywordTok{lower}\NormalTok{=}\DecValTok{1}\NormalTok{\textgreater{} K\_rel;}

  \CommentTok{// Cancer stage}
  \CommentTok{// k = 1: No cancer}
  \CommentTok{// k = 2: Early stage cancer}
  \CommentTok{// k = 3: Advanced stage cancer}
  \DataTypeTok{int}\NormalTok{\textless{}}\KeywordTok{lower}\NormalTok{=}\DecValTok{1}\NormalTok{\textgreater{} K\_stg;}

  \CommentTok{// Toxicity status}
  \CommentTok{// k = 1: None}
  \CommentTok{// k = 2: Low}
  \CommentTok{// k = 3: Medium}
  \CommentTok{// k = 4: High}
  \DataTypeTok{int}\NormalTok{\textless{}}\KeywordTok{lower}\NormalTok{=}\DecValTok{1}\NormalTok{\textgreater{} K\_tox;}

  \CommentTok{// Observed conception status}
  \CommentTok{// y = 0: No conception}
  \CommentTok{// y = 1: Conception}
  \DataTypeTok{array}\NormalTok{[N] }\DataTypeTok{int}\NormalTok{\textless{}}\KeywordTok{lower}\NormalTok{=}\DecValTok{0}\NormalTok{, }\KeywordTok{upper}\NormalTok{=}\DecValTok{1}\NormalTok{\textgreater{} y;}

  \CommentTok{// Observed relationship status;}
  \DataTypeTok{array}\NormalTok{[N] }\DataTypeTok{int}\NormalTok{\textless{}}\KeywordTok{lower}\NormalTok{=}\DecValTok{1}\NormalTok{, }\KeywordTok{upper}\NormalTok{=K\_rel\textgreater{} k\_rel;}

  \CommentTok{// Observed cancer stage;}
  \DataTypeTok{array}\NormalTok{[N] }\DataTypeTok{int}\NormalTok{\textless{}}\KeywordTok{lower}\NormalTok{=}\DecValTok{1}\NormalTok{, }\KeywordTok{upper}\NormalTok{=K\_stg\textgreater{} k\_stg;}

  \CommentTok{// Observed toxicity status;}
  \DataTypeTok{array}\NormalTok{[N] }\DataTypeTok{int}\NormalTok{\textless{}}\KeywordTok{lower}\NormalTok{=}\DecValTok{1}\NormalTok{, }\KeywordTok{upper}\NormalTok{=K\_tox\textgreater{} k\_tox;}
\NormalTok{\}}

\KeywordTok{parameters}\NormalTok{ \{}
  \CommentTok{// Probability of conception for baseline patients in a stable}
  \CommentTok{// relationship, no cancer, and no toxicity}
  \DataTypeTok{real}\NormalTok{\textless{}}\KeywordTok{lower}\NormalTok{=}\DecValTok{0}\NormalTok{, }\KeywordTok{upper}\NormalTok{=}\DecValTok{1}\NormalTok{\textgreater{} q\_C\_0;}

  \CommentTok{// Proportional decreases in conception probability due to}
  \CommentTok{// non{-}baseline relationship status, cancer stage, and toxicity}
  \CommentTok{// status.}
  \DataTypeTok{positive\_ordered}\NormalTok{[K\_rel {-} }\DecValTok{1}\NormalTok{] alpha\_rel;}
  \DataTypeTok{positive\_ordered}\NormalTok{[K\_stg {-} }\DecValTok{1}\NormalTok{] alpha\_stg;}
  \DataTypeTok{positive\_ordered}\NormalTok{[K\_tox {-} }\DecValTok{1}\NormalTok{] alpha\_tox;}
\NormalTok{\}}

\KeywordTok{transformed parameters}\NormalTok{ \{}
  \DataTypeTok{vector}\NormalTok{[K\_rel] alpha\_rel\_buff = append\_row([}\DecValTok{0}\NormalTok{]\textquotesingle{}, alpha\_rel);}
  \DataTypeTok{vector}\NormalTok{[K\_stg] alpha\_stg\_buff = append\_row([}\DecValTok{0}\NormalTok{]\textquotesingle{}, alpha\_stg);}
  \DataTypeTok{vector}\NormalTok{[K\_tox] alpha\_tox\_buff = append\_row([}\DecValTok{0}\NormalTok{]\textquotesingle{}, alpha\_tox);}
\NormalTok{\}}

\KeywordTok{model}\NormalTok{ \{}
  \CommentTok{// Prior model}
  \KeywordTok{target +=}\NormalTok{ beta\_lpdf(q\_C\_0 | }\FloatTok{12.7}\NormalTok{, }\FloatTok{3.7}\NormalTok{); }\CommentTok{// 0.50 \textless{}\textasciitilde{} q\_C\_0 \textless{}\textasciitilde{} 0.95}

  \KeywordTok{target +=}\NormalTok{ normal\_lpdf(alpha\_rel | }\DecValTok{0}\NormalTok{, }\DecValTok{3}\NormalTok{ / }\FloatTok{2.32}\NormalTok{); }\CommentTok{// 0 \textless{}\textasciitilde{} alpha \textless{}\textasciitilde{} {-}log(0.05)}
  \KeywordTok{target +=}\NormalTok{ normal\_lpdf(alpha\_stg | }\DecValTok{0}\NormalTok{, }\DecValTok{3}\NormalTok{ / }\FloatTok{2.32}\NormalTok{); }\CommentTok{// 0 \textless{}\textasciitilde{} alpha \textless{}\textasciitilde{} {-}log(0.05)}
  \KeywordTok{target +=}\NormalTok{ normal\_lpdf(alpha\_tox | }\DecValTok{0}\NormalTok{, }\DecValTok{3}\NormalTok{ / }\FloatTok{2.32}\NormalTok{); }\CommentTok{// 0 \textless{}\textasciitilde{} alpha \textless{}\textasciitilde{} {-}log(0.05)}

  \CommentTok{// Observational model}
  \KeywordTok{target +=}\NormalTok{ bernoulli\_lpmf(y | q\_C\_0 * exp({-}alpha\_rel\_buff[k\_rel]}
\NormalTok{                                           {-}alpha\_stg\_buff[k\_stg]}
\NormalTok{                                           {-}alpha\_tox\_buff[k\_tox]));}
\NormalTok{\}}

\KeywordTok{generated quantities}\NormalTok{ \{}
  \CommentTok{// Proportional decreases in conception probability}
  \DataTypeTok{vector}\NormalTok{[K\_rel] gamma\_rel\_buff = exp({-}alpha\_rel\_buff);}
  \DataTypeTok{vector}\NormalTok{[K\_stg] gamma\_stg\_buff = exp({-}alpha\_stg\_buff);}
  \DataTypeTok{vector}\NormalTok{[K\_tox] gamma\_tox\_buff = exp({-}alpha\_tox\_buff);}

  \CommentTok{// Posterior predictive data}
  \DataTypeTok{array}\NormalTok{[N] }\DataTypeTok{int}\NormalTok{\textless{}}\KeywordTok{lower}\NormalTok{=}\DecValTok{0}\NormalTok{, }\KeywordTok{upper}\NormalTok{=}\DecValTok{1}\NormalTok{\textgreater{} y\_pred;}

  \ControlFlowTok{for}\NormalTok{ (n }\ControlFlowTok{in} \DecValTok{1}\NormalTok{:N) \{}
\NormalTok{    y\_pred[n] = bernoulli\_rng(q\_C\_0 * exp({-}alpha\_rel\_buff[k\_rel[n]]}
\NormalTok{                                          {-}alpha\_stg\_buff[k\_stg[n]]}
\NormalTok{                                          {-}alpha\_tox\_buff[k\_tox[n]]));}
\NormalTok{  \}}
\NormalTok{\}}
\end{Highlighting}
\end{Shaded}

\end{codelisting}

\begin{Shaded}
\begin{Highlighting}[]
\NormalTok{fit }\OtherTok{\textless{}{-}} \FunctionTok{stan}\NormalTok{(}\AttributeTok{file=}\StringTok{"stan\_programs/model2.stan"}\NormalTok{,}
            \AttributeTok{data=}\NormalTok{data, }\AttributeTok{seed=}\DecValTok{8438339}\NormalTok{,}
            \AttributeTok{warmup=}\DecValTok{1000}\NormalTok{, }\AttributeTok{iter=}\DecValTok{2024}\NormalTok{, }\AttributeTok{refresh=}\DecValTok{0}\NormalTok{)}
\end{Highlighting}
\end{Shaded}

The computational diagnostics don't suggest any problems with our
posterior quantification.

\begin{Shaded}
\begin{Highlighting}[]
\NormalTok{diagnostics }\OtherTok{\textless{}{-}}\NormalTok{ util}\SpecialCharTok{$}\FunctionTok{extract\_hmc\_diagnostics}\NormalTok{(fit)}
\NormalTok{util}\SpecialCharTok{$}\FunctionTok{check\_all\_hmc\_diagnostics}\NormalTok{(diagnostics)}
\end{Highlighting}
\end{Shaded}

\begin{verbatim}
  All Hamiltonian Monte Carlo diagnostics are consistent with reliable
Markov chain Monte Carlo.
\end{verbatim}

\begin{Shaded}
\begin{Highlighting}[]
\NormalTok{samples2 }\OtherTok{\textless{}{-}}\NormalTok{ util}\SpecialCharTok{$}\FunctionTok{extract\_expectand\_vals}\NormalTok{(fit)}
\NormalTok{base\_samples }\OtherTok{\textless{}{-}}\NormalTok{ util}\SpecialCharTok{$}\FunctionTok{filter\_expectands}\NormalTok{(samples2,}
                                       \FunctionTok{c}\NormalTok{(}\StringTok{\textquotesingle{}q\_C\_0\textquotesingle{}}\NormalTok{, }\StringTok{\textquotesingle{}alpha\_rel\textquotesingle{}}\NormalTok{ ,}
                                         \StringTok{\textquotesingle{}alpha\_stg\textquotesingle{}}\NormalTok{, }\StringTok{\textquotesingle{}alpha\_tox\textquotesingle{}}\NormalTok{),}
                                       \AttributeTok{check\_arrays=}\ConstantTok{TRUE}\NormalTok{)}
\NormalTok{util}\SpecialCharTok{$}\FunctionTok{check\_all\_expectand\_diagnostics}\NormalTok{(base\_samples)}
\end{Highlighting}
\end{Shaded}

\begin{verbatim}
All expectands checked appear to be behaving well enough for reliable
Markov chain Monte Carlo estimation.
\end{verbatim}

\subsection{Retrodictive Checks}\label{retrodictive-checks-1}

The retrodictive performance aggregated across the entire population
continues to be strong.

\begin{Shaded}
\begin{Highlighting}[]
\FunctionTok{par}\NormalTok{(}\AttributeTok{mfrow=}\FunctionTok{c}\NormalTok{(}\DecValTok{1}\NormalTok{, }\DecValTok{1}\NormalTok{), }\AttributeTok{mar=}\FunctionTok{c}\NormalTok{(}\DecValTok{5}\NormalTok{, }\DecValTok{5}\NormalTok{, }\DecValTok{1}\NormalTok{, }\DecValTok{1}\NormalTok{))}

\NormalTok{util}\SpecialCharTok{$}\FunctionTok{plot\_hist\_quantiles}\NormalTok{(samples2, }\StringTok{\textquotesingle{}y\_pred\textquotesingle{}}\NormalTok{, }\SpecialCharTok{{-}}\FloatTok{0.5}\NormalTok{, }\FloatTok{1.5}\NormalTok{, }\DecValTok{1}\NormalTok{,}
                         \AttributeTok{baseline\_values=}\NormalTok{data}\SpecialCharTok{$}\NormalTok{y,}
                         \AttributeTok{xlab=}\StringTok{"Observed Conception Status"}\NormalTok{)}
\end{Highlighting}
\end{Shaded}

\includegraphics{analysis_files/figure-pdf/unnamed-chunk-26-1.pdf}

Now, however, the conditional retrodictive checks across the patient
characteristics also look good. This suggests that our model of the
variation is adequate, at least for this particular data set.

\begin{Shaded}
\begin{Highlighting}[]
\FunctionTok{par}\NormalTok{(}\AttributeTok{mfrow=}\FunctionTok{c}\NormalTok{(}\DecValTok{1}\NormalTok{, }\DecValTok{3}\NormalTok{), }\AttributeTok{mar=}\FunctionTok{c}\NormalTok{(}\DecValTok{5}\NormalTok{, }\DecValTok{5}\NormalTok{, }\DecValTok{1}\NormalTok{, }\DecValTok{1}\NormalTok{))}

\NormalTok{pred\_names }\OtherTok{\textless{}{-}} \FunctionTok{sapply}\NormalTok{(}\DecValTok{1}\SpecialCharTok{:}\NormalTok{data}\SpecialCharTok{$}\NormalTok{N, }\ControlFlowTok{function}\NormalTok{(n) }\FunctionTok{paste0}\NormalTok{(}\StringTok{\textquotesingle{}y\_pred[\textquotesingle{}}\NormalTok{, n, }\StringTok{\textquotesingle{}]\textquotesingle{}}\NormalTok{))}

\NormalTok{util}\SpecialCharTok{$}\FunctionTok{plot\_conditional\_mean\_quantiles}\NormalTok{(samples2, pred\_names, data}\SpecialCharTok{$}\NormalTok{k\_rel,}
                                     \FloatTok{0.5}\NormalTok{, data}\SpecialCharTok{$}\NormalTok{K\_rel }\SpecialCharTok{+} \FloatTok{0.5}\NormalTok{, }\DecValTok{1}\NormalTok{, data}\SpecialCharTok{$}\NormalTok{y,}
                                     \AttributeTok{xlab=}\StringTok{"Observed Relationship Status"}\NormalTok{,}
                                     \AttributeTok{ylab=}\StringTok{"Average Conception Status"}\NormalTok{)}

\NormalTok{util}\SpecialCharTok{$}\FunctionTok{plot\_conditional\_mean\_quantiles}\NormalTok{(samples2, pred\_names, data}\SpecialCharTok{$}\NormalTok{k\_stg,}
                                     \FloatTok{0.5}\NormalTok{, data}\SpecialCharTok{$}\NormalTok{K\_stg }\SpecialCharTok{+} \FloatTok{0.5}\NormalTok{, }\DecValTok{1}\NormalTok{, data}\SpecialCharTok{$}\NormalTok{y,}
                                     \AttributeTok{xlab=}\StringTok{"Observed Cancer Stage"}\NormalTok{,}
                                     \AttributeTok{ylab=}\StringTok{"Average Conception Status"}\NormalTok{)}

\NormalTok{util}\SpecialCharTok{$}\FunctionTok{plot\_conditional\_mean\_quantiles}\NormalTok{(samples2, pred\_names, data}\SpecialCharTok{$}\NormalTok{k\_tox,}
                                     \FloatTok{0.5}\NormalTok{, data}\SpecialCharTok{$}\NormalTok{K\_tox }\SpecialCharTok{+} \FloatTok{0.5}\NormalTok{, }\DecValTok{1}\NormalTok{, data}\SpecialCharTok{$}\NormalTok{y,}
                                     \AttributeTok{xlab=}\StringTok{"Observed Toxicity Status"}\NormalTok{,}
                                     \AttributeTok{ylab=}\StringTok{"Average Conception Status"}\NormalTok{)}
\end{Highlighting}
\end{Shaded}

\includegraphics{analysis_files/figure-pdf/unnamed-chunk-27-1.pdf}

\subsection{Posterior Insights}\label{posterior-insights-1}

The expanded model offers a variety of posterior behaviors to consider.
Note that in order to learn the baseline conception probability, and
hence variations away from that baseline, we needed to have patients who
are not diagnosed with cancer in the observed cohort. If such a cohort
is not possible then we would need to inform \(q_{C_{0}}\) using a
strong prior model informed by domain expertise, previous studies, or a
combination of the two.

\begin{Shaded}
\begin{Highlighting}[]
\FunctionTok{par}\NormalTok{(}\AttributeTok{mfrow=}\FunctionTok{c}\NormalTok{(}\DecValTok{1}\NormalTok{, }\DecValTok{1}\NormalTok{), }\AttributeTok{mar=}\FunctionTok{c}\NormalTok{(}\DecValTok{5}\NormalTok{, }\DecValTok{5}\NormalTok{, }\DecValTok{1}\NormalTok{, }\DecValTok{1}\NormalTok{))}
\NormalTok{util}\SpecialCharTok{$}\FunctionTok{plot\_expectand\_pushforward}\NormalTok{(samples2[[}\StringTok{\textquotesingle{}q\_C\_0\textquotesingle{}}\NormalTok{]],}
                                \DecValTok{25}\NormalTok{,}
                                \AttributeTok{display\_name=}\FunctionTok{paste}\NormalTok{(}\StringTok{"Baseline"}\NormalTok{,}
                                                   \StringTok{"Probability"}\NormalTok{,}
                                                   \StringTok{"of Conception"}\NormalTok{))}
\end{Highlighting}
\end{Shaded}

\includegraphics{analysis_files/figure-pdf/unnamed-chunk-28-1.pdf}

Because of their non-linear influence on the patient conception
probabilities the impairment parameters can be tricky to correctly
interpret.

\begin{Shaded}
\begin{Highlighting}[]
\FunctionTok{par}\NormalTok{(}\AttributeTok{mfrow=}\FunctionTok{c}\NormalTok{(}\DecValTok{1}\NormalTok{, }\DecValTok{3}\NormalTok{), }\AttributeTok{mar=}\FunctionTok{c}\NormalTok{(}\DecValTok{5}\NormalTok{, }\DecValTok{5}\NormalTok{, }\DecValTok{1}\NormalTok{, }\DecValTok{1}\NormalTok{))}

\NormalTok{names }\OtherTok{\textless{}{-}} \FunctionTok{sapply}\NormalTok{(}\DecValTok{1}\SpecialCharTok{:}\NormalTok{data}\SpecialCharTok{$}\NormalTok{K\_rel,}
                \ControlFlowTok{function}\NormalTok{(k) }\FunctionTok{paste0}\NormalTok{(}\StringTok{\textquotesingle{}alpha\_rel\_buff[\textquotesingle{}}\NormalTok{, k, }\StringTok{\textquotesingle{}]\textquotesingle{}}\NormalTok{))}
\NormalTok{util}\SpecialCharTok{$}\FunctionTok{plot\_disc\_pushforward\_quantiles}\NormalTok{(samples2, names,}
                                     \AttributeTok{xlab=}\StringTok{"Observed Relationship Status"}\NormalTok{,}
                                     \AttributeTok{ylab=}\StringTok{"Conception Impairment"}\NormalTok{,}
                                     \AttributeTok{display\_ylim=}\FunctionTok{c}\NormalTok{(}\SpecialCharTok{{-}}\FloatTok{0.05}\NormalTok{, }\DecValTok{1}\NormalTok{))}

\NormalTok{names }\OtherTok{\textless{}{-}} \FunctionTok{sapply}\NormalTok{(}\DecValTok{1}\SpecialCharTok{:}\NormalTok{data}\SpecialCharTok{$}\NormalTok{K\_stg,}
                \ControlFlowTok{function}\NormalTok{(k) }\FunctionTok{paste0}\NormalTok{(}\StringTok{\textquotesingle{}alpha\_stg\_buff[\textquotesingle{}}\NormalTok{, k, }\StringTok{\textquotesingle{}]\textquotesingle{}}\NormalTok{))}
\NormalTok{util}\SpecialCharTok{$}\FunctionTok{plot\_disc\_pushforward\_quantiles}\NormalTok{(samples2, names,}
                                     \AttributeTok{xlab=}\StringTok{"Observed Cancer Stage"}\NormalTok{,}
                                     \AttributeTok{ylab=}\StringTok{"Conception Impairment"}\NormalTok{,}
                                     \AttributeTok{display\_ylim=}\FunctionTok{c}\NormalTok{(}\SpecialCharTok{{-}}\FloatTok{0.05}\NormalTok{, }\DecValTok{1}\NormalTok{))}

\NormalTok{names }\OtherTok{\textless{}{-}} \FunctionTok{sapply}\NormalTok{(}\DecValTok{1}\SpecialCharTok{:}\NormalTok{data}\SpecialCharTok{$}\NormalTok{K\_tox,}
                \ControlFlowTok{function}\NormalTok{(k) }\FunctionTok{paste0}\NormalTok{(}\StringTok{\textquotesingle{}alpha\_tox\_buff[\textquotesingle{}}\NormalTok{, k, }\StringTok{\textquotesingle{}]\textquotesingle{}}\NormalTok{))}
\NormalTok{util}\SpecialCharTok{$}\FunctionTok{plot\_disc\_pushforward\_quantiles}\NormalTok{(samples2, names,}
                                     \AttributeTok{xlab=}\StringTok{"Observed Toxicity Status"}\NormalTok{,}
                                     \AttributeTok{ylab=}\StringTok{"Conception Impairment"}\NormalTok{,}
                                     \AttributeTok{display\_ylim=}\FunctionTok{c}\NormalTok{(}\SpecialCharTok{{-}}\FloatTok{0.05}\NormalTok{, }\DecValTok{1}\NormalTok{))}
\end{Highlighting}
\end{Shaded}

\includegraphics{analysis_files/figure-pdf/unnamed-chunk-29-1.pdf}

Fortunately we can always propagate our posterior inferences to the more
interpretable proportional decreases, \[
\gamma_{k} = \exp( - \alpha_{k} ).
\]

\begin{Shaded}
\begin{Highlighting}[]
\FunctionTok{par}\NormalTok{(}\AttributeTok{mfrow=}\FunctionTok{c}\NormalTok{(}\DecValTok{1}\NormalTok{, }\DecValTok{3}\NormalTok{), }\AttributeTok{mar=}\FunctionTok{c}\NormalTok{(}\DecValTok{5}\NormalTok{, }\DecValTok{5}\NormalTok{, }\DecValTok{1}\NormalTok{, }\DecValTok{1}\NormalTok{))}

\NormalTok{names }\OtherTok{\textless{}{-}} \FunctionTok{sapply}\NormalTok{(}\DecValTok{1}\SpecialCharTok{:}\NormalTok{data}\SpecialCharTok{$}\NormalTok{K\_rel,}
                \ControlFlowTok{function}\NormalTok{(k) }\FunctionTok{paste0}\NormalTok{(}\StringTok{\textquotesingle{}gamma\_rel\_buff[\textquotesingle{}}\NormalTok{, k, }\StringTok{\textquotesingle{}]\textquotesingle{}}\NormalTok{))}
\NormalTok{util}\SpecialCharTok{$}\FunctionTok{plot\_disc\_pushforward\_quantiles}\NormalTok{(samples2, names,}
                                     \AttributeTok{xlab=}\StringTok{"Observed Relationship Status"}\NormalTok{,}
                                     \AttributeTok{ylab=}\FunctionTok{paste}\NormalTok{(}\StringTok{"Proportional"}\NormalTok{,}
                                                \StringTok{"Probability Decrease"}\NormalTok{),}
                                     \AttributeTok{display\_ylim=}\FunctionTok{c}\NormalTok{(}\DecValTok{0}\NormalTok{, }\DecValTok{1}\NormalTok{))}

\NormalTok{names }\OtherTok{\textless{}{-}} \FunctionTok{sapply}\NormalTok{(}\DecValTok{1}\SpecialCharTok{:}\NormalTok{data}\SpecialCharTok{$}\NormalTok{K\_stg,}
                \ControlFlowTok{function}\NormalTok{(k) }\FunctionTok{paste0}\NormalTok{(}\StringTok{\textquotesingle{}gamma\_stg\_buff[\textquotesingle{}}\NormalTok{, k, }\StringTok{\textquotesingle{}]\textquotesingle{}}\NormalTok{))}
\NormalTok{util}\SpecialCharTok{$}\FunctionTok{plot\_disc\_pushforward\_quantiles}\NormalTok{(samples2, names,}
                                     \AttributeTok{xlab=}\StringTok{"Observed Cancer Stage"}\NormalTok{,}
                                     \AttributeTok{ylab=}\FunctionTok{paste}\NormalTok{(}\StringTok{"Proportional"}\NormalTok{,}
                                                \StringTok{"Probability Decrease"}\NormalTok{),}
                                     \AttributeTok{display\_ylim=}\FunctionTok{c}\NormalTok{(}\DecValTok{0}\NormalTok{, }\DecValTok{1}\NormalTok{))}

\NormalTok{names }\OtherTok{\textless{}{-}} \FunctionTok{sapply}\NormalTok{(}\DecValTok{1}\SpecialCharTok{:}\NormalTok{data}\SpecialCharTok{$}\NormalTok{K\_tox,}
                \ControlFlowTok{function}\NormalTok{(k) }\FunctionTok{paste0}\NormalTok{(}\StringTok{\textquotesingle{}gamma\_tox\_buff[\textquotesingle{}}\NormalTok{, k, }\StringTok{\textquotesingle{}]\textquotesingle{}}\NormalTok{))}
\NormalTok{util}\SpecialCharTok{$}\FunctionTok{plot\_disc\_pushforward\_quantiles}\NormalTok{(samples2, names,}
                                     \AttributeTok{xlab=}\StringTok{"Observed Toxicity Status"}\NormalTok{,}
                                     \AttributeTok{ylab=}\FunctionTok{paste}\NormalTok{(}\StringTok{"Proportional"}\NormalTok{,}
                                                \StringTok{"Probability Decrease"}\NormalTok{),}
                                     \AttributeTok{display\_ylim=}\FunctionTok{c}\NormalTok{(}\DecValTok{0}\NormalTok{, }\DecValTok{1}\NormalTok{))}
\end{Highlighting}
\end{Shaded}

\includegraphics{analysis_files/figure-pdf/unnamed-chunk-30-1.pdf}

\section{Model 3}\label{model-3}

Because all of the patient characteristics in the observed cohort are
known we did not have to model them to infer heterogeneity in fertility
within the cohort. If patient characteristics are prone to missingness
or we want to inform behaviors for other patient cohorts, however, then
we need to model the patient characteristics. The latter is particularly
important if we want to consider hypothetical, sometimes referred to as
counterfactual or out-of-sample, cohorts subject to potential
interventions.

\subsection{Model 3a}\label{model-3a}

Because consistently modeling all of the patient characteristics at the
same time can be challenging we'll walk through the process as
deliberately as possible.

\subsubsection{Patient Characteristic Observational
Model}\label{patient-characteristic-observational-model}

In general the patient characteristics of any population are defined by
a joint probability distribution. For a population of patients
characterized by relationship status, cancer stage, treatment status,
and treatment toxicity status this means modeling the joint probability
density function \[
p( \text{relationship}, \text{stage},
   \text{treatment}, \text{toxicity} ).
\]

Modeling joint probability distributions, and all of the complex
couplings between the component variables they can manifest, can be
overwhelming. One way to make them more manageable is to decompose them
into lower-dimensional conditional probability distributions. When this
decomposition follows the data generating process these conditional
probability distributions become more interpretable and hence more
straightforward to model.

For example consider the conditional decomposition
(Figure~\ref{fig-model3a}) \begin{align*}
p( \text{relationship}&, \text{stage},
   \text{treatment}, \text{toxicity} )
\\
&=\;\,
p( \text{toxicity} \mid
   \text{relationship}, \text{stage}, \text{treatment} )
\\
&\quad \cdot
p( \text{treatment} \mid \text{relationship}, \text{stage} )
\\
&\quad \cdot
p( \text{relationship} \mid \text{stage} )
\\
&\quad \cdot
p( \text{stage} ).
\end{align*}

\begin{figure}

\centering{

\includegraphics[width=0.5\textwidth,height=\textheight]{figures/model3a/model3.pdf}

}

\caption{\label{fig-model3a}Conditionally decomposing the joint patient
characteristic model allows us to focus on smaller, more interpretable
component models.}

\end{figure}%

Treatment toxicity certainly depends on whether or not a patient is
undergoing treatment in the first place. Moreover it can depend on the
stage of cancer, with more aggressive cancers making a patient more
vulnerable to treatment side effects. On the other hand treatment
toxicity is not directly influenced by the relationship status of a
patient. Consequently the first conditional probability simplifies to \[
p( \text{toxicity} \mid
   \text{relationship}, \text{stage}, \text{treatment} )
=
p( \text{toxicity} \mid \text{stage}, \text{treatment} ).
\]

In theory treatment status could depend on relationship status; for
example a long term partner might increase the probability of seeking
treatment. For this analysis, however, we will assume that treatment
within this particular cohort is determined entirely by clinicians and
hence is largely independent of relationship status, \[
p( \text{treatment} \mid \text{relationship}, \text{stage} )
=
p( \text{treatment} \mid \text{stage} ).
\]

With these simplifications our patient characteristic model becomes
(Figure~\ref{fig-model3b}) \begin{align*}
p( \text{relationship}&, \text{stage},
   \text{treatment}, \text{toxicity} )
\\
&=\;\,
p( \text{toxicity} \mid
   \text{stage}, \text{treatment} )
\\
&\quad \cdot
p( \text{treatment} \mid \text{stage} )
\\
&\quad \cdot
p( \text{relationship} \mid \text{stage} )
\\
&\quad \cdot
p( \text{stage} ).
\end{align*}

\begin{figure}

\centering{

\includegraphics[width=0.5\textwidth,height=\textheight]{figures/model3b/model3.pdf}

}

\caption{\label{fig-model3b}Our domain expertise eliminates some of the
conditional dependencies, simplifying this conditionally decomposition
of the joint patient characteristic model.}

\end{figure}%

Finally some of the conditional probabilities are known precisely. If
treatment always follows a cancer diagnosis then \[
p( \text{treatment} \mid \text{no cancer} )
=
p( k_{\text{trt}} = 2 \mid k_{\text{stg}} = 1 ) = 0,
\] or equivalently \[
p( \text{no treatment} \mid \text{no cancer} )
=
p( k_{\text{trt}} = 1 \mid k_{\text{stg}} = 1 ) = 1.
\] Similarly without an active treatment there cannot be any treatment
toxicity. Consequently \[
p( \text{no toxicity} \mid \text{stage}, \text{treatment} )
=
p( k_{\text{tox}} = 1 \mid k_{\text{stg}}, k_{\text{trt}} = 1)
=
1
\] for all stages.

In practice all of the remaining discrete probabilities can be
implemented as simplices. For example \(p( \text{stage} )\) contains
three probabilities, one for each stage, which can be implemented with a
three-component simplex variable \[
\mathbf{q}_{\text{stg}}
=
( q_{\text{stg} = 1}, q_{\text{stg} = 2}, q_{\text{stg} = 3})
\] with \[
0 \le q_{\text{stg} = k} \le 1
\] and \[
\sum_{k = 1}^{K_{\text{stg}}} q_{\text{stg} = k} = 1.
\] Similarly \(p( \text{relationship} \mid \text{stage} )\) can be
implemented with three, two-component simplex variables \begin{align*}
\mathbf{q}_{\text{rel} \mid \text{stg} = 1}
&=
( q_{\text{rel} = 1 \mid \text{stg} = 1},
  q_{\text{rel} = 2 \mid \text{stg} = 1} )
\\
\mathbf{q}_{\text{rel} \mid \text{stg} = 2}
&=
( q_{\text{rel} = 1 \mid \text{stg} = 2},
  q_{\text{rel} = 2 \mid \text{stg} = 2} )
\\
\mathbf{q}_{\text{rel} \mid \text{stg} = 2}
&=
( q_{\text{rel} = 1 \mid \text{stg} = 3},
  q_{\text{rel} = 2 \mid \text{stg} = 3} ).
\end{align*}

The remaining patient characteristics probabilities have to be inferred
from observed data using an appropriate categorical observational model
(Figure~\ref{fig-model3c}), \begin{align*}
&\text{categorical}(k_{\text{stg}, n} \mid \mathbf{q}_{\text{stg}})
\\
&\text{categorical}(k_{\text{rel}, n} \, \mid
                    \mathbf{q}_{\text{rel} \mid \text{stg} = k_{\text{stg}, n}})
\\
&\text{categorical}(k_{\text{trt}, n} \, \mid
                    \mathbf{q}_{\text{trt} \mid \text{stg} = k_{\text{stg}, n}})
\\
&\text{categorical}(k_{\text{tox}, n} \mid
                    \mathbf{q}_{\text{tox} \mid \text{stg} = k_{\text{stg}, n}, \text{rel} = k_{\text{rel}, n}}).
\end{align*}

\begin{figure}

\centering{

\includegraphics[width=0.75\textwidth,height=\textheight]{figures/model3c/model3.pdf}

}

\caption{\label{fig-model3c}The joint patient model is parameterized by
a collection of conditional probabilities, each of which can be
implemented with simplex variable types.}

\end{figure}%

\subsubsection{Patient Characteristic Prior
Model}\label{patient-characteristic-prior-model}

The
\href{https://en.wikipedia.org/wiki/Dirichlet_distribution}{Dirichlet
family} of probability density functions provides a convenient diversity
of probabilistic models over simplices, which in turn is particularly
useful for developing a principled prior model for our patient
characteristic model.

For example the Dirichlet probability density function \[
\text{Dirichlet}( q_{1}, \ldots, q_{K} \mid
                  \gamma_{1}, \ldots, \gamma_{K})
\] with \(\gamma_{k} = 1\) for all \(k\) is uniform over the
\((K - 1)\)-simplex. This is useful when we have limited domain
expertise.

On the other hand the Dirichlet model with \[
\gamma_{k} = \rho_{k} / \tau + 1
\] concentrates around the baseline probabilities \[
\rho_{1}, \ldots, \rho_{K}
\] with the value \(\tau\) determining the strength of that
concentration. These configurations are useful when our domain expertise
is more informative.

The Dirichlet family can also be configured to concentrate on more
extreme behaviors. For example if any of the \(\gamma_{k}\) are less
than one then the resulting Dirichlet probability density function will
concentrate on at least one simplex boundary.

When the properties of any particular Dirichlet model are not clear from
inspecting the \texttt{gamma} parameters we can always build intuition
by studying samples.

\begin{Shaded}
\begin{Highlighting}[]
\FunctionTok{library}\NormalTok{(colormap)}
\NormalTok{nom\_colors }\OtherTok{\textless{}{-}} \FunctionTok{c}\NormalTok{(}\StringTok{"\#DCBCBC"}\NormalTok{, }\StringTok{"\#C79999"}\NormalTok{, }\StringTok{"\#B97C7C"}\NormalTok{,}
                \StringTok{"\#A25050"}\NormalTok{, }\StringTok{"\#8F2727"}\NormalTok{, }\StringTok{"\#7C0000"}\NormalTok{)}
\NormalTok{line\_colors }\OtherTok{\textless{}{-}} \FunctionTok{colormap}\NormalTok{(}\AttributeTok{colormap=}\NormalTok{nom\_colors, }\AttributeTok{nshades=}\DecValTok{25}\NormalTok{)}

\NormalTok{plot\_simplex\_samples }\OtherTok{\textless{}{-}} \ControlFlowTok{function}\NormalTok{(gammas, }\AttributeTok{baseline=}\ConstantTok{FALSE}\NormalTok{, }\AttributeTok{main=}\StringTok{""}\NormalTok{) \{}
\NormalTok{  K }\OtherTok{\textless{}{-}} \FunctionTok{length}\NormalTok{(gammas)}

  \FunctionTok{plot}\NormalTok{(}\DecValTok{1}\NormalTok{, }\AttributeTok{type=}\StringTok{"n"}\NormalTok{, }\AttributeTok{main=}\NormalTok{main,}
       \AttributeTok{xlim=}\FunctionTok{c}\NormalTok{(}\FloatTok{0.5}\NormalTok{, K }\SpecialCharTok{+} \FloatTok{0.5}\NormalTok{), }\AttributeTok{xlab=}\StringTok{"Component"}\NormalTok{,}
       \AttributeTok{ylim=}\FunctionTok{c}\NormalTok{(}\DecValTok{0}\NormalTok{, }\DecValTok{1}\NormalTok{), }\AttributeTok{ylab=}\StringTok{"Probability"}\NormalTok{)}

\NormalTok{  idxs }\OtherTok{\textless{}{-}} \FunctionTok{rep}\NormalTok{(}\DecValTok{1}\SpecialCharTok{:}\NormalTok{K, }\AttributeTok{each=}\DecValTok{2}\NormalTok{)}

\NormalTok{  xs }\OtherTok{\textless{}{-}} \FunctionTok{sapply}\NormalTok{(}\DecValTok{1}\SpecialCharTok{:}\FunctionTok{length}\NormalTok{(idxs),}
               \ControlFlowTok{function}\NormalTok{(k) }\ControlFlowTok{if}\NormalTok{(k }\SpecialCharTok{\%\%} \DecValTok{2} \SpecialCharTok{==} \DecValTok{1}\NormalTok{) idxs[k] }\SpecialCharTok{{-}} \FloatTok{0.5}
                           \ControlFlowTok{else}\NormalTok{            idxs[k] }\SpecialCharTok{+} \FloatTok{0.5}\NormalTok{)}

  \ControlFlowTok{for}\NormalTok{ (s }\ControlFlowTok{in} \DecValTok{1}\SpecialCharTok{:}\DecValTok{25}\NormalTok{) \{}
\NormalTok{    q }\OtherTok{\textless{}{-}} \FunctionTok{rgamma}\NormalTok{(K, gammas, }\DecValTok{1}\NormalTok{)}
\NormalTok{    q }\OtherTok{\textless{}{-}}\NormalTok{ q }\SpecialCharTok{/} \FunctionTok{sum}\NormalTok{(q)}

    \ControlFlowTok{for}\NormalTok{ (k }\ControlFlowTok{in} \DecValTok{1}\SpecialCharTok{:}\NormalTok{K) \{}
\NormalTok{      idx1 }\OtherTok{\textless{}{-}} \DecValTok{2} \SpecialCharTok{*}\NormalTok{ k }\SpecialCharTok{{-}} \DecValTok{1}
\NormalTok{      idx2 }\OtherTok{\textless{}{-}} \DecValTok{2} \SpecialCharTok{*}\NormalTok{ k}
      \FunctionTok{lines}\NormalTok{(xs[idx1}\SpecialCharTok{:}\NormalTok{idx2], }\FunctionTok{rep}\NormalTok{(q[k], }\DecValTok{2}\NormalTok{), }\AttributeTok{col=}\NormalTok{line\_colors[s], }\AttributeTok{lwd=}\DecValTok{3}\NormalTok{)}
\NormalTok{    \}}
\NormalTok{  \}}

  \ControlFlowTok{if}\NormalTok{ (baseline) \{}
\NormalTok{    rho }\OtherTok{\textless{}{-}}\NormalTok{ (gammas }\SpecialCharTok{{-}} \DecValTok{1}\NormalTok{)}
\NormalTok{    rho }\OtherTok{\textless{}{-}}\NormalTok{ rho }\SpecialCharTok{/} \FunctionTok{sum}\NormalTok{(rho)}
    \ControlFlowTok{for}\NormalTok{ (k }\ControlFlowTok{in} \DecValTok{1}\SpecialCharTok{:}\NormalTok{K) \{}
\NormalTok{      idx1 }\OtherTok{\textless{}{-}} \DecValTok{2} \SpecialCharTok{*}\NormalTok{ k }\SpecialCharTok{{-}} \DecValTok{1}
\NormalTok{      idx2 }\OtherTok{\textless{}{-}} \DecValTok{2} \SpecialCharTok{*}\NormalTok{ k}
      \FunctionTok{lines}\NormalTok{(xs[idx1}\SpecialCharTok{:}\NormalTok{idx2], }\FunctionTok{rep}\NormalTok{(rho[k], }\DecValTok{2}\NormalTok{), }\AttributeTok{col=}\NormalTok{util}\SpecialCharTok{$}\NormalTok{c\_mid\_teal, }\AttributeTok{lwd=}\DecValTok{6}\NormalTok{)}
      \FunctionTok{lines}\NormalTok{(xs[idx1}\SpecialCharTok{:}\NormalTok{idx2], }\FunctionTok{rep}\NormalTok{(rho[k], }\DecValTok{2}\NormalTok{), }\AttributeTok{col=}\NormalTok{util}\SpecialCharTok{$}\NormalTok{c\_mid\_teal, }\AttributeTok{lwd=}\DecValTok{3}\NormalTok{)}
\NormalTok{    \}}
\NormalTok{  \}}
\NormalTok{\}}
\end{Highlighting}
\end{Shaded}

\begin{Shaded}
\begin{Highlighting}[]
\FunctionTok{par}\NormalTok{(}\AttributeTok{mfrow=}\FunctionTok{c}\NormalTok{(}\DecValTok{1}\NormalTok{, }\DecValTok{1}\NormalTok{), }\AttributeTok{mar=}\FunctionTok{c}\NormalTok{(}\DecValTok{5}\NormalTok{, }\DecValTok{5}\NormalTok{, }\DecValTok{5}\NormalTok{, }\DecValTok{1}\NormalTok{))}

\FunctionTok{plot\_simplex\_samples}\NormalTok{(}\FunctionTok{c}\NormalTok{(}\DecValTok{5}\NormalTok{, }\DecValTok{7}\NormalTok{, }\DecValTok{8}\NormalTok{, }\DecValTok{3}\NormalTok{, }\DecValTok{9}\NormalTok{, }\DecValTok{5}\NormalTok{), }\AttributeTok{baseline=}\ConstantTok{TRUE}\NormalTok{)}
\end{Highlighting}
\end{Shaded}

\includegraphics{analysis_files/figure-pdf/unnamed-chunk-32-1.pdf}

What do we know about the cohort?

Because patients were included only from referrals for fertility
preservation consultation, patients with cancer diagnoses should not
dominate. The more domain expertise we have about referral rates the
more precisely we can tune an appropriate prior model; here we will
assume a weak concentration towards no cancer diagnosis, \[
\boldsymbol{\gamma}_{\text{stg}}
=   \frac{ \left( \frac{3}{5}, \frac{2}{5}, 0 \right) }{ \frac{1}{5} }
  + \mathbf{1}
=
\left(4, 3, 1 \right).
\]

\begin{Shaded}
\begin{Highlighting}[]
\FunctionTok{par}\NormalTok{(}\AttributeTok{mfrow=}\FunctionTok{c}\NormalTok{(}\DecValTok{1}\NormalTok{, }\DecValTok{1}\NormalTok{), }\AttributeTok{mar=}\FunctionTok{c}\NormalTok{(}\DecValTok{5}\NormalTok{, }\DecValTok{5}\NormalTok{, }\DecValTok{5}\NormalTok{, }\DecValTok{1}\NormalTok{))}

\FunctionTok{plot\_simplex\_samples}\NormalTok{(}\FunctionTok{c}\NormalTok{(}\DecValTok{4}\NormalTok{, }\DecValTok{3}\NormalTok{, }\DecValTok{1}\NormalTok{), }\AttributeTok{baseline=}\ConstantTok{TRUE}\NormalTok{, }\AttributeTok{main=}\StringTok{"Cancer Stage"}\NormalTok{)}
\end{Highlighting}
\end{Shaded}

\includegraphics{analysis_files/figure-pdf/unnamed-chunk-33-1.pdf}

In general relationships are strained more and more as cancer progresses
to more advanced stages, although here we don't have too much a priori
understanding of just how much, \begin{align*}
\boldsymbol{\gamma}_{\text{rel} \mid \text{stg} = 1}
&=   \frac{ \left( 1, 0 \right) }{ \frac{1}{3} } + \mathbf{1}
= \left(4, 1 \right),
\\
\boldsymbol{\gamma}_{\text{rel} \mid \text{stg} = 2}
&=   \frac{ \left( 1, 0 \right) }{ 1 } + \mathbf{1}
= \left(2, 1 \right),
\\
\boldsymbol{\gamma}_{\text{rel} \mid \text{stg} = 3}
&= \left(1, 1 \right).
\end{align*}

\begin{Shaded}
\begin{Highlighting}[]
\FunctionTok{par}\NormalTok{(}\AttributeTok{mfrow=}\FunctionTok{c}\NormalTok{(}\DecValTok{1}\NormalTok{, }\DecValTok{3}\NormalTok{), }\AttributeTok{mar=}\FunctionTok{c}\NormalTok{(}\DecValTok{5}\NormalTok{, }\DecValTok{5}\NormalTok{, }\DecValTok{5}\NormalTok{, }\DecValTok{1}\NormalTok{))}

\FunctionTok{plot\_simplex\_samples}\NormalTok{(}\FunctionTok{c}\NormalTok{(}\DecValTok{4}\NormalTok{, }\DecValTok{1}\NormalTok{), }\AttributeTok{baseline=}\ConstantTok{TRUE}\NormalTok{,}
                     \AttributeTok{main=}\StringTok{"Relationship Status}\SpecialCharTok{\textbackslash{}n}\StringTok{No Cancer Diagnosis"}\NormalTok{)}

\FunctionTok{plot\_simplex\_samples}\NormalTok{(}\FunctionTok{c}\NormalTok{(}\DecValTok{2}\NormalTok{, }\DecValTok{1}\NormalTok{), }\AttributeTok{baseline=}\ConstantTok{TRUE}\NormalTok{,}
                     \AttributeTok{main=}\StringTok{"Relationship Status}\SpecialCharTok{\textbackslash{}n}\StringTok{Early Stage Cancer"}\NormalTok{)}

\FunctionTok{plot\_simplex\_samples}\NormalTok{(}\FunctionTok{c}\NormalTok{(}\DecValTok{1}\NormalTok{, }\DecValTok{1}\NormalTok{), }\AttributeTok{baseline=}\ConstantTok{FALSE}\NormalTok{,}
                     \AttributeTok{main=}\StringTok{"Relationship Status}\SpecialCharTok{\textbackslash{}n}\StringTok{Advanced Stage Cancer"}\NormalTok{)}
\end{Highlighting}
\end{Shaded}

\includegraphics{analysis_files/figure-pdf/unnamed-chunk-34-1.pdf}

Similarly we expect that treatment becomes more likely as cancer
progresses, \begin{align*}
\boldsymbol{\gamma}_{\text{trt} \mid \text{stg} = 2}
&=   \frac{ \left( 0, 1 \right) }{ \frac{1}{2} } + \mathbf{1}
= \left(1, 3 \right),
\\
\boldsymbol{\gamma}_{\text{trt} \mid \text{stg} = 3}
&= \left(0.5, 4 \right).
\end{align*}

\begin{Shaded}
\begin{Highlighting}[]
\FunctionTok{par}\NormalTok{(}\AttributeTok{mfrow=}\FunctionTok{c}\NormalTok{(}\DecValTok{1}\NormalTok{, }\DecValTok{2}\NormalTok{), }\AttributeTok{mar=}\FunctionTok{c}\NormalTok{(}\DecValTok{5}\NormalTok{, }\DecValTok{5}\NormalTok{, }\DecValTok{5}\NormalTok{, }\DecValTok{1}\NormalTok{))}

\FunctionTok{plot\_simplex\_samples}\NormalTok{(}\FunctionTok{c}\NormalTok{(}\DecValTok{1}\NormalTok{, }\DecValTok{3}\NormalTok{), }\AttributeTok{baseline=}\ConstantTok{TRUE}\NormalTok{,}
                     \AttributeTok{main=}\StringTok{"Relationship Status}\SpecialCharTok{\textbackslash{}n}\StringTok{Early Stage Cancer"}\NormalTok{)}

\FunctionTok{plot\_simplex\_samples}\NormalTok{(}\FunctionTok{c}\NormalTok{(}\FloatTok{0.5}\NormalTok{, }\DecValTok{4}\NormalTok{), }\AttributeTok{baseline=}\ConstantTok{FALSE}\NormalTok{,}
                     \AttributeTok{main=}\StringTok{"Relationship Status}\SpecialCharTok{\textbackslash{}n}\StringTok{Advanced Stage Cancer"}\NormalTok{)}
\end{Highlighting}
\end{Shaded}

\includegraphics{analysis_files/figure-pdf/unnamed-chunk-35-1.pdf}

Finally toxicity should increase with cancer stage, but only if a
patient is being actively treated, \begin{align*}
\boldsymbol{\gamma}_{\text{tox} \mid \text{stg} = 2, \text{trt} = 2}
&=   \frac{ \left( \frac{3}{6}, \frac{2}{6}, \frac{1}{6}, 0 \right) }{ \frac{1}{6} }   + \mathbf{1}
= \left(4, 3, 2, 1 \right),
\\
\boldsymbol{\gamma}_{\text{tox} \mid \text{stg} = 2, \text{trt} = 2}
&=   \frac{ \left( \frac{1}{5}, \frac{2}{5}, \frac{2}{5}, 0 \right) }{ \frac{1}{5} }   + \mathbf{1}
= \left(2, 3, 3, 1 \right),
\\
\boldsymbol{\gamma}_{\text{tox} \mid \text{stg} = 3, \text{trt} = 2}
&=   \frac{ \left( 0, \frac{1}{5}, \frac{2}{5}, \frac{2}{5} \right) }{ \frac{1}{5} }   + \mathbf{1}
= \left(1, 2, 3, 3 \right).
\end{align*}

\begin{Shaded}
\begin{Highlighting}[]
\FunctionTok{par}\NormalTok{(}\AttributeTok{mfrow=}\FunctionTok{c}\NormalTok{(}\DecValTok{1}\NormalTok{, }\DecValTok{1}\NormalTok{), }\AttributeTok{mar=}\FunctionTok{c}\NormalTok{(}\DecValTok{5}\NormalTok{, }\DecValTok{5}\NormalTok{, }\DecValTok{5}\NormalTok{, }\DecValTok{1}\NormalTok{))}

\NormalTok{title }\OtherTok{\textless{}{-}} \StringTok{"Relationship Status}\SpecialCharTok{\textbackslash{}n}\StringTok{No Cancer Diagnosis, No Treatment"}
\FunctionTok{plot\_simplex\_samples}\NormalTok{(}\FunctionTok{c}\NormalTok{(}\DecValTok{4}\NormalTok{, }\DecValTok{3}\NormalTok{, }\DecValTok{2}\NormalTok{, }\DecValTok{1}\NormalTok{), }\AttributeTok{baseline=}\ConstantTok{TRUE}\NormalTok{,}
                     \AttributeTok{main=}\NormalTok{title)}
\end{Highlighting}
\end{Shaded}

\includegraphics{analysis_files/figure-pdf/unnamed-chunk-36-1.pdf}

\begin{Shaded}
\begin{Highlighting}[]
\FunctionTok{par}\NormalTok{(}\AttributeTok{mfrow=}\FunctionTok{c}\NormalTok{(}\DecValTok{1}\NormalTok{, }\DecValTok{1}\NormalTok{), }\AttributeTok{mar=}\FunctionTok{c}\NormalTok{(}\DecValTok{5}\NormalTok{, }\DecValTok{5}\NormalTok{, }\DecValTok{5}\NormalTok{, }\DecValTok{1}\NormalTok{))}

\NormalTok{title }\OtherTok{\textless{}{-}} \StringTok{"Relationship Status}\SpecialCharTok{\textbackslash{}n}\StringTok{Early Stage Cancer, Active Treatment"}
\FunctionTok{plot\_simplex\_samples}\NormalTok{(}\FunctionTok{c}\NormalTok{(}\DecValTok{2}\NormalTok{, }\DecValTok{3}\NormalTok{, }\DecValTok{3}\NormalTok{, }\DecValTok{1}\NormalTok{), }\AttributeTok{baseline=}\ConstantTok{TRUE}\NormalTok{,}
                     \AttributeTok{main=}\NormalTok{title)}
\end{Highlighting}
\end{Shaded}

\includegraphics{analysis_files/figure-pdf/unnamed-chunk-37-1.pdf}

\begin{Shaded}
\begin{Highlighting}[]
\FunctionTok{par}\NormalTok{(}\AttributeTok{mfrow=}\FunctionTok{c}\NormalTok{(}\DecValTok{1}\NormalTok{, }\DecValTok{1}\NormalTok{), }\AttributeTok{mar=}\FunctionTok{c}\NormalTok{(}\DecValTok{5}\NormalTok{, }\DecValTok{5}\NormalTok{, }\DecValTok{5}\NormalTok{, }\DecValTok{1}\NormalTok{))}
\NormalTok{title }\OtherTok{\textless{}{-}} \StringTok{"Relationship Status}\SpecialCharTok{\textbackslash{}n}\StringTok{Advanced Stage Cancer, Active Treatment"}
\FunctionTok{plot\_simplex\_samples}\NormalTok{(}\FunctionTok{c}\NormalTok{(}\DecValTok{1}\NormalTok{, }\DecValTok{2}\NormalTok{, }\DecValTok{3}\NormalTok{, }\DecValTok{3}\NormalTok{), }\AttributeTok{baseline=}\ConstantTok{FALSE}\NormalTok{,}
                     \AttributeTok{main=}\NormalTok{title)}
\end{Highlighting}
\end{Shaded}

\includegraphics{analysis_files/figure-pdf/unnamed-chunk-38-1.pdf}

Overall the full prior model is relatively diffuse, but it does suppress
the more unrealistic patient characteristics.

\subsubsection{Posterior
Quantification}\label{posterior-quantification-2}

If the fertility heterogeneity is independent of how the patient
characteristic groups are populated then our previous model and new
patient characteristic model simply compose together
(Figure~\ref{fig-model3d}). This is by no means a trivial assumption in
practice. For example it would be violated if referrals into the
observed cohort favored patients who suffered from fertility issues.

\begin{figure}

\centering{

\includegraphics[width=1\textwidth,height=\textheight]{figures/model3d/model3.pdf}

}

\caption{\label{fig-model3d}When fertility is independent of how the
cohort was selected we can simply compose a patient fertility model with
a patient characteristic model.}

\end{figure}%

All of these patient characteristic probabilities probabilities can be
conveniently implemented with the \texttt{simplex} variable types in
Stan. The only challenge is organizing all of the simplices effectively.

Also note that when we generate posterior predictive simulations in the
\texttt{generated\ quantities} block we simulate new patient
characteristics as well as new fertility outcomes.

\begin{codelisting}

\caption{\texttt{model3a.stan}}

\begin{Shaded}
\begin{Highlighting}[]
\KeywordTok{data}\NormalTok{ \{}
  \CommentTok{// Number of observations}
  \DataTypeTok{int}\NormalTok{\textless{}}\KeywordTok{lower}\NormalTok{=}\DecValTok{1}\NormalTok{\textgreater{} N;}

  \CommentTok{// Number of predictions}
  \DataTypeTok{int}\NormalTok{\textless{}}\KeywordTok{lower}\NormalTok{=}\DecValTok{1}\NormalTok{\textgreater{} N\_pred;}

  \CommentTok{// Relationship status}
  \CommentTok{// k = 1: Stable partner}
  \CommentTok{// k = 2: No partner}
  \DataTypeTok{int}\NormalTok{\textless{}}\KeywordTok{lower}\NormalTok{=}\DecValTok{1}\NormalTok{\textgreater{} K\_rel;}

  \CommentTok{// Cancer stage}
  \CommentTok{// k = 1: No cancer}
  \CommentTok{// k = 2: Early stage cancer}
  \CommentTok{// k = 3: Advanced stage cancer}
  \DataTypeTok{int}\NormalTok{\textless{}}\KeywordTok{lower}\NormalTok{=}\DecValTok{1}\NormalTok{\textgreater{} K\_stg;}

  \CommentTok{// Treatment status}
  \CommentTok{// k = 1: No treatment}
  \CommentTok{// k = 2: Treatment}
  \DataTypeTok{int}\NormalTok{\textless{}}\KeywordTok{lower}\NormalTok{=}\DecValTok{1}\NormalTok{\textgreater{} K\_trt;}

  \CommentTok{// Toxicity status}
  \CommentTok{// k = 1: None}
  \CommentTok{// k = 2: Low}
  \CommentTok{// k = 3: Medium}
  \CommentTok{// k = 4: High}
  \DataTypeTok{int}\NormalTok{\textless{}}\KeywordTok{lower}\NormalTok{=}\DecValTok{1}\NormalTok{\textgreater{} K\_tox;}

  \CommentTok{// Observed conception status}
  \CommentTok{// y = 0: No conception}
  \CommentTok{// y = 1: Conception}
  \DataTypeTok{array}\NormalTok{[N] }\DataTypeTok{int}\NormalTok{\textless{}}\KeywordTok{lower}\NormalTok{=}\DecValTok{0}\NormalTok{, }\KeywordTok{upper}\NormalTok{=}\DecValTok{1}\NormalTok{\textgreater{} y;}

  \CommentTok{// Observed relationship status;}
  \DataTypeTok{array}\NormalTok{[N] }\DataTypeTok{int}\NormalTok{\textless{}}\KeywordTok{lower}\NormalTok{=}\DecValTok{1}\NormalTok{, }\KeywordTok{upper}\NormalTok{=K\_rel\textgreater{} k\_rel;}

  \CommentTok{// Observed cancer stage;}
  \DataTypeTok{array}\NormalTok{[N] }\DataTypeTok{int}\NormalTok{\textless{}}\KeywordTok{lower}\NormalTok{=}\DecValTok{1}\NormalTok{, }\KeywordTok{upper}\NormalTok{=K\_stg\textgreater{} k\_stg;}

  \CommentTok{// Observed treatment status;}
  \DataTypeTok{array}\NormalTok{[N] }\DataTypeTok{int}\NormalTok{\textless{}}\KeywordTok{lower}\NormalTok{=}\DecValTok{1}\NormalTok{, }\KeywordTok{upper}\NormalTok{=K\_trt\textgreater{} k\_trt;}

  \CommentTok{// Observed toxicity status;}
  \DataTypeTok{array}\NormalTok{[N] }\DataTypeTok{int}\NormalTok{\textless{}}\KeywordTok{lower}\NormalTok{=}\DecValTok{1}\NormalTok{, }\KeywordTok{upper}\NormalTok{=K\_tox\textgreater{} k\_tox;}
\NormalTok{\}}

\KeywordTok{parameters}\NormalTok{ \{}
  \CommentTok{// Marginal probability of cancer stage}
  \DataTypeTok{simplex}\NormalTok{[K\_stg] q\_stg;}

  \CommentTok{// Conditional probability of relationship status given cancer stage}
  \DataTypeTok{array}\NormalTok{[K\_stg] }\DataTypeTok{simplex}\NormalTok{[K\_rel] q\_rel;}

  \CommentTok{// Conditional probability of treatment status given active cancer stage}
  \DataTypeTok{array}\NormalTok{[K\_stg {-} }\DecValTok{1}\NormalTok{] }\DataTypeTok{simplex}\NormalTok{[K\_trt] q\_trt\_active\_stg;}

  \CommentTok{// Conditional probability of toxicity status given cancer stage and}
  \CommentTok{// active treatment status}
  \DataTypeTok{array}\NormalTok{[K\_stg] }\DataTypeTok{simplex}\NormalTok{[K\_tox] q\_tox\_active\_trt;}

  \CommentTok{// Probability of conception for baseline patients in a stable}
  \CommentTok{// relationship, no cancer, and no toxicity}
  \DataTypeTok{real}\NormalTok{\textless{}}\KeywordTok{lower}\NormalTok{=}\DecValTok{0}\NormalTok{, }\KeywordTok{upper}\NormalTok{=}\DecValTok{1}\NormalTok{\textgreater{} q\_C\_0;}

  \CommentTok{// Proportional decreases in conception probability due to}
  \CommentTok{// non{-}baseline relationship status, cancer stage, and toxicity}
  \CommentTok{// status.}
  \DataTypeTok{positive\_ordered}\NormalTok{[K\_rel {-} }\DecValTok{1}\NormalTok{] alpha\_rel;}
  \DataTypeTok{positive\_ordered}\NormalTok{[K\_stg {-} }\DecValTok{1}\NormalTok{] alpha\_stg;}
  \DataTypeTok{positive\_ordered}\NormalTok{[K\_tox {-} }\DecValTok{1}\NormalTok{] alpha\_tox;}
\NormalTok{\}}

\KeywordTok{transformed parameters}\NormalTok{ \{}
  \DataTypeTok{vector}\NormalTok{[K\_rel] alpha\_rel\_buff = append\_row([}\DecValTok{0}\NormalTok{]\textquotesingle{}, alpha\_rel);}
  \DataTypeTok{vector}\NormalTok{[K\_stg] alpha\_stg\_buff = append\_row([}\DecValTok{0}\NormalTok{]\textquotesingle{}, alpha\_stg);}
  \DataTypeTok{vector}\NormalTok{[K\_tox] alpha\_tox\_buff = append\_row([}\DecValTok{0}\NormalTok{]\textquotesingle{}, alpha\_tox);}

  \CommentTok{// Conditional probability of treatment status given cancer stage}
  \DataTypeTok{array}\NormalTok{[K\_stg] }\DataTypeTok{simplex}\NormalTok{[K\_trt] q\_trt = append\_array(\{ [}\FloatTok{1.0}\NormalTok{, }\FloatTok{0.0}\NormalTok{]\textquotesingle{} \},}
\NormalTok{                                                   q\_trt\_active\_stg);}

  \CommentTok{// Conditional probability of toxicity status given cancer stage and}
  \CommentTok{// treatment status}
  \DataTypeTok{array}\NormalTok{[K\_stg, K\_trt] }\DataTypeTok{simplex}\NormalTok{[K\_tox] q\_tox;}
\NormalTok{  q\_tox[, }\DecValTok{1}\NormalTok{] = \{ [}\DecValTok{1}\NormalTok{, }\DecValTok{0}\NormalTok{, }\DecValTok{0}\NormalTok{, }\DecValTok{0}\NormalTok{]\textquotesingle{},}
\NormalTok{                 [}\DecValTok{1}\NormalTok{, }\DecValTok{0}\NormalTok{, }\DecValTok{0}\NormalTok{, }\DecValTok{0}\NormalTok{]\textquotesingle{},}
\NormalTok{                 [}\DecValTok{1}\NormalTok{, }\DecValTok{0}\NormalTok{, }\DecValTok{0}\NormalTok{, }\DecValTok{0}\NormalTok{]\textquotesingle{} \};}
\NormalTok{  q\_tox[, }\DecValTok{2}\NormalTok{] = q\_tox\_active\_trt;}
\NormalTok{\}}

\KeywordTok{model}\NormalTok{ \{}
  \CommentTok{// Prior model}
  \KeywordTok{target +=}\NormalTok{ dirichlet\_lpdf(q\_stg | [}\DecValTok{4}\NormalTok{, }\DecValTok{3}\NormalTok{, }\DecValTok{1}\NormalTok{]\textquotesingle{});}
  \KeywordTok{target +=}\NormalTok{ dirichlet\_lpdf(q\_rel[}\DecValTok{1}\NormalTok{] | [}\DecValTok{4}\NormalTok{, }\DecValTok{1}\NormalTok{]\textquotesingle{});}
  \KeywordTok{target +=}\NormalTok{ dirichlet\_lpdf(q\_rel[}\DecValTok{2}\NormalTok{] | [}\DecValTok{2}\NormalTok{, }\DecValTok{1}\NormalTok{]\textquotesingle{});}
  \KeywordTok{target +=}\NormalTok{ dirichlet\_lpdf(q\_rel[}\DecValTok{3}\NormalTok{] | [}\DecValTok{1}\NormalTok{, }\DecValTok{1}\NormalTok{]\textquotesingle{});}
  \KeywordTok{target +=}\NormalTok{ dirichlet\_lpdf(q\_trt\_active\_stg[}\DecValTok{1}\NormalTok{] | [}\DecValTok{1}\NormalTok{, }\DecValTok{3}\NormalTok{]\textquotesingle{});}
  \KeywordTok{target +=}\NormalTok{ dirichlet\_lpdf(q\_trt\_active\_stg[}\DecValTok{2}\NormalTok{] | [}\FloatTok{0.5}\NormalTok{, }\DecValTok{4}\NormalTok{]\textquotesingle{});}
  \KeywordTok{target +=}\NormalTok{ dirichlet\_lpdf(q\_tox\_active\_trt[}\DecValTok{1}\NormalTok{] | [}\DecValTok{4}\NormalTok{, }\DecValTok{3}\NormalTok{, }\DecValTok{2}\NormalTok{, }\DecValTok{1}\NormalTok{]\textquotesingle{});}
  \KeywordTok{target +=}\NormalTok{ dirichlet\_lpdf(q\_tox\_active\_trt[}\DecValTok{2}\NormalTok{] | [}\DecValTok{2}\NormalTok{, }\DecValTok{3}\NormalTok{, }\DecValTok{3}\NormalTok{, }\DecValTok{1}\NormalTok{]\textquotesingle{});}
  \KeywordTok{target +=}\NormalTok{ dirichlet\_lpdf(q\_tox\_active\_trt[}\DecValTok{3}\NormalTok{] | [}\DecValTok{1}\NormalTok{, }\DecValTok{2}\NormalTok{, }\DecValTok{3}\NormalTok{, }\DecValTok{3}\NormalTok{]\textquotesingle{});}

  \KeywordTok{target +=}\NormalTok{ beta\_lpdf(q\_C\_0 | }\FloatTok{12.7}\NormalTok{, }\FloatTok{3.7}\NormalTok{); }\CommentTok{// 0.50 \textless{}\textasciitilde{} q\_C\_0 \textless{}\textasciitilde{} 0.95}

  \KeywordTok{target +=}\NormalTok{ normal\_lpdf(alpha\_rel | }\DecValTok{0}\NormalTok{, }\DecValTok{3}\NormalTok{ / }\FloatTok{2.32}\NormalTok{); }\CommentTok{// 0 \textless{}\textasciitilde{} alpha \textless{}\textasciitilde{} {-}log(0.05)}
  \KeywordTok{target +=}\NormalTok{ normal\_lpdf(alpha\_stg | }\DecValTok{0}\NormalTok{, }\DecValTok{3}\NormalTok{ / }\FloatTok{2.32}\NormalTok{); }\CommentTok{// 0 \textless{}\textasciitilde{} alpha \textless{}\textasciitilde{} {-}log(0.05)}
  \KeywordTok{target +=}\NormalTok{ normal\_lpdf(alpha\_tox | }\DecValTok{0}\NormalTok{, }\DecValTok{3}\NormalTok{ / }\FloatTok{2.32}\NormalTok{); }\CommentTok{// 0 \textless{}\textasciitilde{} alpha \textless{}\textasciitilde{} {-}log(0.05)}

  \CommentTok{// Observational model}
  \ControlFlowTok{for}\NormalTok{ (n }\ControlFlowTok{in} \DecValTok{1}\NormalTok{:N) \{}
    \KeywordTok{target +=}\NormalTok{ categorical\_lpmf(k\_stg[n] | q\_stg);}
    \KeywordTok{target +=}\NormalTok{ categorical\_lpmf(k\_rel[n] | q\_rel[k\_stg[n]]);}
    \KeywordTok{target +=}\NormalTok{ categorical\_lpmf(k\_trt[n] | q\_trt[k\_stg[n]]);}
    \KeywordTok{target +=}\NormalTok{ categorical\_lpmf(k\_tox[n] | q\_tox[k\_stg[n],}
\NormalTok{                                                k\_trt[n]]);}
\NormalTok{  \}}

  \KeywordTok{target +=}\NormalTok{ bernoulli\_lpmf(y | q\_C\_0 * exp({-}alpha\_rel\_buff[k\_rel]}
\NormalTok{                                           {-}alpha\_stg\_buff[k\_stg]}
\NormalTok{                                           {-}alpha\_tox\_buff[k\_tox]));}
\NormalTok{\}}

\KeywordTok{generated quantities}\NormalTok{ \{}
  \CommentTok{// Posterior predictive data}
  \DataTypeTok{array}\NormalTok{[N\_pred] }\DataTypeTok{real}\NormalTok{\textless{}}\KeywordTok{lower}\NormalTok{=}\DecValTok{0}\NormalTok{, }\KeywordTok{upper}\NormalTok{=}\DecValTok{1}\NormalTok{\textgreater{} q\_pred;}
  \DataTypeTok{array}\NormalTok{[N\_pred] }\DataTypeTok{int}\NormalTok{\textless{}}\KeywordTok{lower}\NormalTok{=}\DecValTok{0}\NormalTok{, }\KeywordTok{upper}\NormalTok{=}\DecValTok{1}\NormalTok{\textgreater{}  y\_pred;}

  \DataTypeTok{array}\NormalTok{[N\_pred] }\DataTypeTok{int}\NormalTok{\textless{}}\KeywordTok{lower}\NormalTok{=}\DecValTok{1}\NormalTok{, }\KeywordTok{upper}\NormalTok{=K\_rel\textgreater{} k\_rel\_pred;}
  \DataTypeTok{array}\NormalTok{[N\_pred] }\DataTypeTok{int}\NormalTok{\textless{}}\KeywordTok{lower}\NormalTok{=}\DecValTok{1}\NormalTok{, }\KeywordTok{upper}\NormalTok{=K\_stg\textgreater{} k\_stg\_pred;}
  \DataTypeTok{array}\NormalTok{[N\_pred] }\DataTypeTok{int}\NormalTok{\textless{}}\KeywordTok{lower}\NormalTok{=}\DecValTok{1}\NormalTok{, }\KeywordTok{upper}\NormalTok{=K\_trt\textgreater{} k\_trt\_pred;}
  \DataTypeTok{array}\NormalTok{[N\_pred] }\DataTypeTok{int}\NormalTok{\textless{}}\KeywordTok{lower}\NormalTok{=}\DecValTok{1}\NormalTok{, }\KeywordTok{upper}\NormalTok{=K\_tox\textgreater{} k\_tox\_pred;}

  \ControlFlowTok{for}\NormalTok{ (n }\ControlFlowTok{in} \DecValTok{1}\NormalTok{:N\_pred) \{}
\NormalTok{    k\_stg\_pred[n] = categorical\_rng(q\_stg);}
\NormalTok{    k\_rel\_pred[n] = categorical\_rng(q\_rel[k\_stg\_pred[n]]);}
\NormalTok{    k\_trt\_pred[n] = categorical\_rng(q\_trt[k\_stg\_pred[n]]);}
\NormalTok{    k\_tox\_pred[n] = categorical\_rng(q\_tox[k\_stg\_pred[n],}
\NormalTok{                                          k\_trt\_pred[n]]);}

\NormalTok{    q\_pred[n] = q\_C\_0 * exp({-}alpha\_rel\_buff[k\_rel\_pred[n]]}
\NormalTok{                            {-}alpha\_stg\_buff[k\_stg\_pred[n]]}
\NormalTok{                            {-}alpha\_tox\_buff[k\_tox\_pred[n]]);}
\NormalTok{    y\_pred[n] = bernoulli\_rng(q\_pred[n]);}
\NormalTok{  \}}
\NormalTok{\}}
\end{Highlighting}
\end{Shaded}

\end{codelisting}

\begin{Shaded}
\begin{Highlighting}[]
\NormalTok{data}\SpecialCharTok{$}\NormalTok{N\_pred }\OtherTok{\textless{}{-}} \DecValTok{5000}

\NormalTok{fit }\OtherTok{\textless{}{-}} \FunctionTok{stan}\NormalTok{(}\AttributeTok{file=}\StringTok{"stan\_programs/model3a.stan"}\NormalTok{,}
            \AttributeTok{data=}\NormalTok{data, }\AttributeTok{seed=}\DecValTok{8438338}\NormalTok{,}
            \AttributeTok{warmup=}\DecValTok{1000}\NormalTok{, }\AttributeTok{iter=}\DecValTok{2024}\NormalTok{, }\AttributeTok{refresh=}\DecValTok{0}\NormalTok{)}
\end{Highlighting}
\end{Shaded}

Although our model is quickly increasing in complexity the computational
diagnostics suggest that our posterior computation is keeping up.

\begin{Shaded}
\begin{Highlighting}[]
\NormalTok{diagnostics }\OtherTok{\textless{}{-}}\NormalTok{ util}\SpecialCharTok{$}\FunctionTok{extract\_hmc\_diagnostics}\NormalTok{(fit)}
\NormalTok{util}\SpecialCharTok{$}\FunctionTok{check\_all\_hmc\_diagnostics}\NormalTok{(diagnostics)}
\end{Highlighting}
\end{Shaded}

\begin{verbatim}
  All Hamiltonian Monte Carlo diagnostics are consistent with reliable
Markov chain Monte Carlo.
\end{verbatim}

\begin{Shaded}
\begin{Highlighting}[]
\NormalTok{samples3a }\OtherTok{\textless{}{-}}\NormalTok{ util}\SpecialCharTok{$}\FunctionTok{extract\_expectand\_vals}\NormalTok{(fit)}
\NormalTok{base\_samples }\OtherTok{\textless{}{-}}\NormalTok{ util}\SpecialCharTok{$}\FunctionTok{filter\_expectands}\NormalTok{(samples3a,}
                                       \FunctionTok{c}\NormalTok{(}\StringTok{\textquotesingle{}q\_stg\textquotesingle{}}\NormalTok{, }\StringTok{\textquotesingle{}q\_trt\_active\_stg\textquotesingle{}}\NormalTok{,}
                                         \StringTok{\textquotesingle{}q\_tox\_active\_trt\textquotesingle{}}\NormalTok{,}\StringTok{\textquotesingle{}q\_C\_0\textquotesingle{}}\NormalTok{,}
                                         \StringTok{\textquotesingle{}alpha\_rel\textquotesingle{}}\NormalTok{, }\StringTok{\textquotesingle{}alpha\_stg\textquotesingle{}}\NormalTok{,}
                                         \StringTok{\textquotesingle{}alpha\_tox\textquotesingle{}}\NormalTok{),}
                                       \AttributeTok{check\_arrays=}\ConstantTok{TRUE}\NormalTok{)}
\NormalTok{util}\SpecialCharTok{$}\FunctionTok{check\_all\_expectand\_diagnostics}\NormalTok{(base\_samples)}
\end{Highlighting}
\end{Shaded}

\begin{verbatim}
All expectands checked appear to be behaving well enough for reliable
Markov chain Monte Carlo estimation.
\end{verbatim}

\subsubsection{Retrodictive Checks}\label{retrodictive-checks-2}

Posterior retrodictive checks for the patient characteristics all look
good, suggesting that the assumptions underlying our patient
characteristic model is adequately capturing the details of the observed
data.

\begin{Shaded}
\begin{Highlighting}[]
\FunctionTok{par}\NormalTok{(}\AttributeTok{mfrow=}\FunctionTok{c}\NormalTok{(}\DecValTok{2}\NormalTok{, }\DecValTok{2}\NormalTok{), }\AttributeTok{mar=}\FunctionTok{c}\NormalTok{(}\DecValTok{5}\NormalTok{, }\DecValTok{5}\NormalTok{, }\DecValTok{1}\NormalTok{, }\DecValTok{1}\NormalTok{))}

\NormalTok{util}\SpecialCharTok{$}\FunctionTok{plot\_hist\_quantiles}\NormalTok{(samples3a, }\StringTok{\textquotesingle{}k\_rel\_pred\textquotesingle{}}\NormalTok{,}
                         \FloatTok{0.5}\NormalTok{, data}\SpecialCharTok{$}\NormalTok{K\_rel }\SpecialCharTok{+} \FloatTok{0.5}\NormalTok{, }\DecValTok{1}\NormalTok{,}
                         \AttributeTok{baseline\_values=}\NormalTok{data}\SpecialCharTok{$}\NormalTok{k\_rel,}
                         \AttributeTok{xlab=}\StringTok{"Relationship Status"}\NormalTok{)}

\NormalTok{util}\SpecialCharTok{$}\FunctionTok{plot\_hist\_quantiles}\NormalTok{(samples3a, }\StringTok{\textquotesingle{}k\_stg\_pred\textquotesingle{}}\NormalTok{,}
                         \FloatTok{0.5}\NormalTok{, data}\SpecialCharTok{$}\NormalTok{K\_stg }\SpecialCharTok{+} \FloatTok{0.5}\NormalTok{, }\DecValTok{1}\NormalTok{,}
                         \AttributeTok{baseline\_values=}\NormalTok{data}\SpecialCharTok{$}\NormalTok{k\_stg,}
                         \AttributeTok{xlab=}\StringTok{"Cancer Stage"}\NormalTok{)}

\NormalTok{util}\SpecialCharTok{$}\FunctionTok{plot\_hist\_quantiles}\NormalTok{(samples3a, }\StringTok{\textquotesingle{}k\_trt\_pred\textquotesingle{}}\NormalTok{,}
                         \FloatTok{0.5}\NormalTok{, data}\SpecialCharTok{$}\NormalTok{K\_trt }\SpecialCharTok{+} \FloatTok{0.5}\NormalTok{, }\DecValTok{1}\NormalTok{,}
                         \AttributeTok{baseline\_values=}\NormalTok{data}\SpecialCharTok{$}\NormalTok{k\_trt,}
                         \AttributeTok{xlab=}\StringTok{"Treatment Status"}\NormalTok{)}

\NormalTok{util}\SpecialCharTok{$}\FunctionTok{plot\_hist\_quantiles}\NormalTok{(samples3a, }\StringTok{\textquotesingle{}k\_tox\_pred\textquotesingle{}}\NormalTok{,}
                         \FloatTok{0.5}\NormalTok{, data}\SpecialCharTok{$}\NormalTok{K\_tox }\SpecialCharTok{+} \FloatTok{0.5}\NormalTok{, }\DecValTok{1}\NormalTok{,}
                         \AttributeTok{baseline\_values=}\NormalTok{data}\SpecialCharTok{$}\NormalTok{k\_tox,}
                         \AttributeTok{xlab=}\StringTok{"Toxicity Status"}\NormalTok{)}
\end{Highlighting}
\end{Shaded}

\includegraphics{analysis_files/figure-pdf/unnamed-chunk-41-1.pdf}

Moreover the observed conception status is consistent with the full
posterior predictive conception status, where we consider variation in
the clinical, demographic, and fertility behaviors.

\begin{Shaded}
\begin{Highlighting}[]
\FunctionTok{par}\NormalTok{(}\AttributeTok{mfrow=}\FunctionTok{c}\NormalTok{(}\DecValTok{1}\NormalTok{, }\DecValTok{1}\NormalTok{), }\AttributeTok{mar=}\FunctionTok{c}\NormalTok{(}\DecValTok{5}\NormalTok{, }\DecValTok{5}\NormalTok{, }\DecValTok{1}\NormalTok{, }\DecValTok{1}\NormalTok{))}

\NormalTok{util}\SpecialCharTok{$}\FunctionTok{plot\_hist\_quantiles}\NormalTok{(samples3a, }\StringTok{\textquotesingle{}y\_pred\textquotesingle{}}\NormalTok{, }\SpecialCharTok{{-}}\FloatTok{0.5}\NormalTok{, }\FloatTok{1.5}\NormalTok{, }\DecValTok{1}\NormalTok{,}
                         \AttributeTok{baseline\_values=}\NormalTok{data}\SpecialCharTok{$}\NormalTok{y,}
                         \AttributeTok{xlab=}\StringTok{"Observed Conception Status"}\NormalTok{)}
\end{Highlighting}
\end{Shaded}

\includegraphics{analysis_files/figure-pdf/unnamed-chunk-42-1.pdf}

Note that we cannot consider conditional retrodictive checks here as we
did previously because the observed data and posterior predictions are
now based on \emph{different} realizations of the patient characteristic
variables.

\subsubsection{Posterior Insights}\label{posterior-insights-2}

Content with the adequacy of our model we can examine the patient
characteristic inferences one by one. About half of the observed cohort
suffers from cancer.

\begin{Shaded}
\begin{Highlighting}[]
\FunctionTok{par}\NormalTok{(}\AttributeTok{mfrow=}\FunctionTok{c}\NormalTok{(}\DecValTok{1}\NormalTok{, }\DecValTok{1}\NormalTok{), }\AttributeTok{mar=}\FunctionTok{c}\NormalTok{(}\DecValTok{5}\NormalTok{, }\DecValTok{5}\NormalTok{, }\DecValTok{1}\NormalTok{, }\DecValTok{1}\NormalTok{))}

\NormalTok{names }\OtherTok{\textless{}{-}} \FunctionTok{sapply}\NormalTok{(}\DecValTok{1}\SpecialCharTok{:}\NormalTok{data}\SpecialCharTok{$}\NormalTok{K\_stg,}
                \ControlFlowTok{function}\NormalTok{(k) }\FunctionTok{paste0}\NormalTok{(}\StringTok{\textquotesingle{}q\_stg[\textquotesingle{}}\NormalTok{, k, }\StringTok{\textquotesingle{}]\textquotesingle{}}\NormalTok{))}
\NormalTok{util}\SpecialCharTok{$}\FunctionTok{plot\_disc\_pushforward\_quantiles}\NormalTok{(samples3a, names,}
                                     \AttributeTok{xlab=}\StringTok{"Observed Cancer Stage"}\NormalTok{,}
                                     \AttributeTok{ylab=}\StringTok{"Probability"}\NormalTok{,}
                                     \AttributeTok{display\_ylim=}\FunctionTok{c}\NormalTok{(}\DecValTok{0}\NormalTok{, }\DecValTok{1}\NormalTok{))}
\end{Highlighting}
\end{Shaded}

\includegraphics{analysis_files/figure-pdf/unnamed-chunk-43-1.pdf}

Stable relationships become less common as cancer progresses.

\begin{Shaded}
\begin{Highlighting}[]
\FunctionTok{par}\NormalTok{(}\AttributeTok{mfrow=}\FunctionTok{c}\NormalTok{(}\DecValTok{1}\NormalTok{, }\DecValTok{3}\NormalTok{), }\AttributeTok{mar=}\FunctionTok{c}\NormalTok{(}\DecValTok{5}\NormalTok{, }\DecValTok{5}\NormalTok{, }\DecValTok{1}\NormalTok{, }\DecValTok{1}\NormalTok{))}

\ControlFlowTok{for}\NormalTok{ (ks }\ControlFlowTok{in} \DecValTok{1}\SpecialCharTok{:}\NormalTok{data}\SpecialCharTok{$}\NormalTok{K\_stg) \{}
\NormalTok{  names }\OtherTok{\textless{}{-}} \FunctionTok{sapply}\NormalTok{(}\DecValTok{1}\SpecialCharTok{:}\NormalTok{data}\SpecialCharTok{$}\NormalTok{K\_rel,}
                  \ControlFlowTok{function}\NormalTok{(k) }\FunctionTok{paste0}\NormalTok{(}\StringTok{\textquotesingle{}q\_rel[\textquotesingle{}}\NormalTok{, ks, }\StringTok{\textquotesingle{},\textquotesingle{}}\NormalTok{, k, }\StringTok{\textquotesingle{}]\textquotesingle{}}\NormalTok{))}
\NormalTok{  util}\SpecialCharTok{$}\FunctionTok{plot\_disc\_pushforward\_quantiles}\NormalTok{(samples3a, names,}
                                       \AttributeTok{xlab=}\StringTok{"Observed Relationship Status"}\NormalTok{,}
                                       \AttributeTok{ylab=}\StringTok{"Probability"}\NormalTok{,}
                                       \AttributeTok{display\_ylim=}\FunctionTok{c}\NormalTok{(}\DecValTok{0}\NormalTok{, }\DecValTok{1}\NormalTok{),}
                                       \AttributeTok{main=}\FunctionTok{paste}\NormalTok{(}\StringTok{"Cancer Stage "}\NormalTok{, ks))}
\NormalTok{\}}
\end{Highlighting}
\end{Shaded}

\includegraphics{analysis_files/figure-pdf/unnamed-chunk-44-1.pdf}

Conversely treatment becomes more common as cancer progresses.

\begin{Shaded}
\begin{Highlighting}[]
\FunctionTok{par}\NormalTok{(}\AttributeTok{mfrow=}\FunctionTok{c}\NormalTok{(}\DecValTok{1}\NormalTok{, }\DecValTok{3}\NormalTok{), }\AttributeTok{mar=}\FunctionTok{c}\NormalTok{(}\DecValTok{5}\NormalTok{, }\DecValTok{5}\NormalTok{, }\DecValTok{1}\NormalTok{, }\DecValTok{1}\NormalTok{))}

\ControlFlowTok{for}\NormalTok{ (ks }\ControlFlowTok{in} \DecValTok{1}\SpecialCharTok{:}\NormalTok{data}\SpecialCharTok{$}\NormalTok{K\_stg) \{}
\NormalTok{  names }\OtherTok{\textless{}{-}} \FunctionTok{sapply}\NormalTok{(}\DecValTok{1}\SpecialCharTok{:}\NormalTok{data}\SpecialCharTok{$}\NormalTok{K\_trt,}
                  \ControlFlowTok{function}\NormalTok{(k) }\FunctionTok{paste0}\NormalTok{(}\StringTok{\textquotesingle{}q\_trt[\textquotesingle{}}\NormalTok{, ks, }\StringTok{\textquotesingle{},\textquotesingle{}}\NormalTok{, k, }\StringTok{\textquotesingle{}]\textquotesingle{}}\NormalTok{))}
\NormalTok{  util}\SpecialCharTok{$}\FunctionTok{plot\_disc\_pushforward\_quantiles}\NormalTok{(samples3a, names,}
                                       \AttributeTok{xlab=}\StringTok{"Observed Treatment Status"}\NormalTok{,}
                                       \AttributeTok{ylab=}\StringTok{"Probability"}\NormalTok{,}
                                       \AttributeTok{display\_ylim=}\FunctionTok{c}\NormalTok{(}\DecValTok{0}\NormalTok{, }\DecValTok{1}\NormalTok{),}
                                       \AttributeTok{main=}\FunctionTok{paste}\NormalTok{(}\StringTok{"Cancer Stage "}\NormalTok{, ks))}
\NormalTok{\}}
\end{Highlighting}
\end{Shaded}

\includegraphics{analysis_files/figure-pdf/unnamed-chunk-45-1.pdf}

Treatment toxicity becomes more severe as cancer progresses, at least
for patients actively undergoing treatment.

\begin{Shaded}
\begin{Highlighting}[]
\FunctionTok{par}\NormalTok{(}\AttributeTok{mfrow=}\FunctionTok{c}\NormalTok{(}\DecValTok{2}\NormalTok{, }\DecValTok{3}\NormalTok{), }\AttributeTok{mar=}\FunctionTok{c}\NormalTok{(}\DecValTok{5}\NormalTok{, }\DecValTok{5}\NormalTok{, }\DecValTok{1}\NormalTok{, }\DecValTok{1}\NormalTok{))}

\ControlFlowTok{for}\NormalTok{ (kt }\ControlFlowTok{in} \DecValTok{1}\SpecialCharTok{:}\NormalTok{data}\SpecialCharTok{$}\NormalTok{K\_trt) \{}
  \ControlFlowTok{for}\NormalTok{ (ks }\ControlFlowTok{in} \DecValTok{1}\SpecialCharTok{:}\NormalTok{data}\SpecialCharTok{$}\NormalTok{K\_stg) \{}
\NormalTok{    names }\OtherTok{\textless{}{-}} \FunctionTok{sapply}\NormalTok{(}\DecValTok{1}\SpecialCharTok{:}\NormalTok{data}\SpecialCharTok{$}\NormalTok{K\_tox,}
                    \ControlFlowTok{function}\NormalTok{(k) }\FunctionTok{paste0}\NormalTok{(}\StringTok{\textquotesingle{}q\_tox[\textquotesingle{}}\NormalTok{, ks, }\StringTok{\textquotesingle{},\textquotesingle{}}\NormalTok{, kt, }\StringTok{\textquotesingle{},\textquotesingle{}}\NormalTok{, k, }\StringTok{\textquotesingle{}]\textquotesingle{}}\NormalTok{))}
\NormalTok{    util}\SpecialCharTok{$}\FunctionTok{plot\_disc\_pushforward\_quantiles}\NormalTok{(samples3a, names,}
                                       \AttributeTok{xlab=}\StringTok{"Observed Toxicity Status"}\NormalTok{,}
                                       \AttributeTok{ylab=}\StringTok{"Probability"}\NormalTok{,}
                                       \AttributeTok{display\_ylim=}\FunctionTok{c}\NormalTok{(}\DecValTok{0}\NormalTok{, }\DecValTok{1}\NormalTok{),}
                                       \AttributeTok{main=}\FunctionTok{paste}\NormalTok{(}\StringTok{"Cancer Stage "}\NormalTok{, ks,}
                                                  \StringTok{", Treatment "}\NormalTok{,  kt))}
\NormalTok{  \}}
\NormalTok{\}}
\end{Highlighting}
\end{Shaded}

\includegraphics{analysis_files/figure-pdf/unnamed-chunk-46-1.pdf}

Note that \texttt{k\_stg\ =\ 1} and \texttt{k\_trt\ =\ 2} corresponds to
cancer treatment without a cancer diagnosis, which is not possible in
this scenario. Consequently we have no observations with these
particular patient characteristics, and the corresponding toxicity
probabilities are informed by only the prior model. This is why these
inferences exhibit the largest uncertainties.

In order to quantify the effect of a potential interventions we need to
compare the fertility of the entire observed cohort to some new,
hypothetical cohort. This is straightforward when fertility in both
cohorts behaves homogeneously across patients but substantially more
subtle when patient fertility is heterogeneous. In the heterogeneous
case we to reduce the population behavior into a summarize amenable to
direct comparison.

For example we might summarize the cohort population behavior with a
probability distribution of histograms. This summary convolves the
variation in behavior across individual patients and the inferential
uncertainties of those behaviors together into a single probabilistic
prediction.

\begin{Shaded}
\begin{Highlighting}[]
\FunctionTok{par}\NormalTok{(}\AttributeTok{mfrow=}\FunctionTok{c}\NormalTok{(}\DecValTok{1}\NormalTok{, }\DecValTok{1}\NormalTok{), }\AttributeTok{mar=}\FunctionTok{c}\NormalTok{(}\DecValTok{5}\NormalTok{, }\DecValTok{5}\NormalTok{, }\DecValTok{1}\NormalTok{, }\DecValTok{1}\NormalTok{))}

\NormalTok{util}\SpecialCharTok{$}\FunctionTok{plot\_hist\_quantiles}\NormalTok{(samples3a, }\StringTok{\textquotesingle{}q\_pred\textquotesingle{}}\NormalTok{, }\SpecialCharTok{{-}}\FloatTok{0.05}\NormalTok{, }\FloatTok{1.05}\NormalTok{, }\FloatTok{0.1}\NormalTok{,}
                         \AttributeTok{xlab=}\StringTok{"Conception Probability"}\NormalTok{)}
\end{Highlighting}
\end{Shaded}

\includegraphics{analysis_files/figure-pdf/unnamed-chunk-47-1.pdf}

Notice the peak at high conception probabilities here; this is due to
the baseline individuals in the population. If useful we can always
stratify these histograms to isolate particular patient sub-populations
of interest, such as the baseline patients.

Another approach is to reduce the entire population to a scalar summary
and then visualize the posterior predictive distribution of that
summary.

For instance we might consider ensemble, or population, average of the
individual patient fertilities, \[
\left< q \right> = \frac{1}{N} \sum_{n = 1}^{N} q_{C, n},
\] and then analyze the posterior distribution of \(\left< q \right>\).
This is particularly useful for emphasizing the centrality of the
population.

\begin{Shaded}
\begin{Highlighting}[]
\NormalTok{var\_repl }\OtherTok{\textless{}{-}} \FunctionTok{list}\NormalTok{(}\StringTok{\textquotesingle{}p\textquotesingle{}}\OtherTok{=}\NormalTok{util}\SpecialCharTok{$}\FunctionTok{name\_array}\NormalTok{(}\StringTok{\textquotesingle{}q\_pred\textquotesingle{}}\NormalTok{, }\FunctionTok{c}\NormalTok{(data}\SpecialCharTok{$}\NormalTok{N\_pred)))}

\NormalTok{pop\_ave\_samples }\OtherTok{\textless{}{-}}
\NormalTok{  util}\SpecialCharTok{$}\FunctionTok{eval\_expectand\_pushforward}\NormalTok{(samples3a,}
                                  \ControlFlowTok{function}\NormalTok{(p) }\FunctionTok{mean}\NormalTok{(p),}
\NormalTok{                                  var\_repl)}

\FunctionTok{par}\NormalTok{(}\AttributeTok{mfrow=}\FunctionTok{c}\NormalTok{(}\DecValTok{1}\NormalTok{, }\DecValTok{1}\NormalTok{), }\AttributeTok{mar=}\FunctionTok{c}\NormalTok{(}\DecValTok{5}\NormalTok{, }\DecValTok{5}\NormalTok{, }\DecValTok{1}\NormalTok{, }\DecValTok{1}\NormalTok{))}

\NormalTok{name }\OtherTok{\textless{}{-}} \StringTok{"Conception Probability}\SpecialCharTok{\textbackslash{}n}\StringTok{Population Average"}
\NormalTok{util}\SpecialCharTok{$}\FunctionTok{plot\_expectand\_pushforward}\NormalTok{(pop\_ave\_samples,}
                                \DecValTok{150}\NormalTok{, }\AttributeTok{flim=}\FunctionTok{c}\NormalTok{(}\DecValTok{0}\NormalTok{, }\DecValTok{1}\NormalTok{),}
                                \AttributeTok{display\_name=}\NormalTok{name)}
\end{Highlighting}
\end{Shaded}

\includegraphics{analysis_files/figure-pdf/unnamed-chunk-48-1.pdf}

In cases where we are more interested in extremes of the population we
might consider ensemble/population quantiles such as the lower quartile
\(Q\) defined implicitly as \[
\frac{ \sum_{n = 1}^{N} I[ Q > q_{C, n} ] }
{ N } \approx 0.25.
\]

\begin{Shaded}
\begin{Highlighting}[]
\NormalTok{var\_repl }\OtherTok{\textless{}{-}} \FunctionTok{list}\NormalTok{(}\StringTok{\textquotesingle{}p\textquotesingle{}}\OtherTok{=}\NormalTok{util}\SpecialCharTok{$}\FunctionTok{name\_array}\NormalTok{(}\StringTok{\textquotesingle{}q\_pred\textquotesingle{}}\NormalTok{, }\FunctionTok{c}\NormalTok{(data}\SpecialCharTok{$}\NormalTok{N\_pred)))}

\NormalTok{pop\_quant\_samples }\OtherTok{\textless{}{-}}
\NormalTok{  util}\SpecialCharTok{$}\FunctionTok{eval\_expectand\_pushforward}\NormalTok{(samples3a,}
                                  \ControlFlowTok{function}\NormalTok{(p) }\FunctionTok{quantile}\NormalTok{(p, }\AttributeTok{prob=}\FunctionTok{c}\NormalTok{(}\FloatTok{0.25}\NormalTok{)),}
\NormalTok{                                  var\_repl)}

\FunctionTok{par}\NormalTok{(}\AttributeTok{mfrow=}\FunctionTok{c}\NormalTok{(}\DecValTok{1}\NormalTok{, }\DecValTok{1}\NormalTok{), }\AttributeTok{mar=}\FunctionTok{c}\NormalTok{(}\DecValTok{5}\NormalTok{, }\DecValTok{5}\NormalTok{, }\DecValTok{1}\NormalTok{, }\DecValTok{1}\NormalTok{))}

\NormalTok{name }\OtherTok{\textless{}{-}} \StringTok{"Conception Probability}\SpecialCharTok{\textbackslash{}n}\StringTok{Population Lower Quartile"}
\NormalTok{util}\SpecialCharTok{$}\FunctionTok{plot\_expectand\_pushforward}\NormalTok{(pop\_quant\_samples,}
                                \DecValTok{150}\NormalTok{, }\AttributeTok{flim=}\FunctionTok{c}\NormalTok{(}\DecValTok{0}\NormalTok{, }\DecValTok{1}\NormalTok{),}
                                \AttributeTok{display\_name=}\NormalTok{name)}
\end{Highlighting}
\end{Shaded}

\includegraphics{analysis_files/figure-pdf/unnamed-chunk-49-1.pdf}

\subsection{Model 3b}\label{model-3b}

Now that we've modeled the patient characteristic of the observed cohort
we can consider what behaviors might manifest in other, hypothetical
populations of practical interest.

For example what would happen to the population fertility with the
introduction of a new, less toxic treatment? Because none of the other
patient characteristics depend on toxicity we can model this
intervention by changing only
\(\mathbf{q}_{\text{tox} \mid \text{trt, stg}}\).

\begin{Shaded}
\begin{Highlighting}[]
\NormalTok{q\_tox\_hyp }\OtherTok{\textless{}{-}} \FunctionTok{list}\NormalTok{(}\FunctionTok{c}\NormalTok{(}\DecValTok{0}\NormalTok{, }\FloatTok{0.69}\NormalTok{, }\FloatTok{0.30}\NormalTok{, }\FloatTok{0.01}\NormalTok{),}
                  \FunctionTok{c}\NormalTok{(}\DecValTok{0}\NormalTok{, }\FloatTok{0.35}\NormalTok{, }\FloatTok{0.55}\NormalTok{, }\FloatTok{0.20}\NormalTok{) )}

\FunctionTok{par}\NormalTok{(}\AttributeTok{mfrow=}\FunctionTok{c}\NormalTok{(}\DecValTok{2}\NormalTok{, }\DecValTok{2}\NormalTok{), }\AttributeTok{mar=}\FunctionTok{c}\NormalTok{(}\DecValTok{5}\NormalTok{, }\DecValTok{5}\NormalTok{, }\DecValTok{3}\NormalTok{, }\DecValTok{1}\NormalTok{))}

\NormalTok{kt }\OtherTok{\textless{}{-}} \DecValTok{2}
\ControlFlowTok{for}\NormalTok{ (ks }\ControlFlowTok{in} \DecValTok{2}\SpecialCharTok{:}\NormalTok{data}\SpecialCharTok{$}\NormalTok{K\_stg) \{}
\NormalTok{  names }\OtherTok{\textless{}{-}} \FunctionTok{sapply}\NormalTok{(}\DecValTok{1}\SpecialCharTok{:}\NormalTok{data}\SpecialCharTok{$}\NormalTok{K\_tox,}
                  \ControlFlowTok{function}\NormalTok{(k) }\FunctionTok{paste0}\NormalTok{(}\StringTok{\textquotesingle{}q\_tox[\textquotesingle{}}\NormalTok{, ks, }\StringTok{\textquotesingle{},\textquotesingle{}}\NormalTok{, kt, }\StringTok{\textquotesingle{},\textquotesingle{}}\NormalTok{, k, }\StringTok{\textquotesingle{}]\textquotesingle{}}\NormalTok{))}
\NormalTok{  util}\SpecialCharTok{$}\FunctionTok{plot\_disc\_pushforward\_quantiles}\NormalTok{(samples3a, names,}
                                       \AttributeTok{xlab=}\StringTok{"Observed Toxicity Status"}\NormalTok{,}
                                       \AttributeTok{ylab=}\StringTok{"Probability"}\NormalTok{,}
                                       \AttributeTok{display\_ylim=}\FunctionTok{c}\NormalTok{(}\DecValTok{0}\NormalTok{, }\DecValTok{1}\NormalTok{),}
                                       \AttributeTok{main=}\FunctionTok{paste}\NormalTok{(}\StringTok{"Observed}\SpecialCharTok{\textbackslash{}n}\StringTok{"}\NormalTok{,}
                                                  \StringTok{"Cancer Stage "}\NormalTok{, ks,}
                                                  \StringTok{", Treatment "}\NormalTok{,  kt))}

  \FunctionTok{plot}\NormalTok{(}\DecValTok{0}\NormalTok{, }\AttributeTok{type=}\StringTok{\textquotesingle{}n\textquotesingle{}}\NormalTok{,}
       \AttributeTok{main=}\FunctionTok{paste}\NormalTok{(}\StringTok{"Hypothetical}\SpecialCharTok{\textbackslash{}n}\StringTok{"}\NormalTok{,}
                  \StringTok{"Cancer Stage "}\NormalTok{, ks,}
                  \StringTok{", Treatment "}\NormalTok{,  kt),}
       \AttributeTok{xlim=}\FunctionTok{c}\NormalTok{(}\FloatTok{0.5}\NormalTok{, data}\SpecialCharTok{$}\NormalTok{K\_tox }\SpecialCharTok{+} \FloatTok{0.5}\NormalTok{),}
       \AttributeTok{xlab=}\StringTok{"Observed Toxicity Status"}\NormalTok{,}
       \AttributeTok{ylim=}\FunctionTok{c}\NormalTok{(}\DecValTok{0}\NormalTok{, }\DecValTok{1}\NormalTok{), }\AttributeTok{ylab=}\StringTok{"Probability"}\NormalTok{)}
  \ControlFlowTok{for}\NormalTok{ (k }\ControlFlowTok{in} \DecValTok{1}\SpecialCharTok{:}\NormalTok{data}\SpecialCharTok{$}\NormalTok{K\_tox) \{}
    \FunctionTok{lines}\NormalTok{(}\FunctionTok{c}\NormalTok{(k }\SpecialCharTok{{-}} \FloatTok{0.5}\NormalTok{, k }\SpecialCharTok{+} \FloatTok{0.5}\NormalTok{),}
          \FunctionTok{rep}\NormalTok{(q\_tox\_hyp[[ks }\SpecialCharTok{{-}} \DecValTok{1}\NormalTok{]][k], }\DecValTok{2}\NormalTok{),}
          \AttributeTok{lwd=}\DecValTok{2}\NormalTok{, }\AttributeTok{col=}\NormalTok{util}\SpecialCharTok{$}\NormalTok{c\_mid\_teal)}
\NormalTok{  \}}
\NormalTok{\}}
\end{Highlighting}
\end{Shaded}

\includegraphics{analysis_files/figure-pdf/unnamed-chunk-50-1.pdf}

Because we have embraced the probabilistic glory of Bayesian inference
there's no reason why we have to be limited to point hypotheticals. We
can also use a probability distribution to quantify uncertainty in the
potential behaviors instead of having to rely on precise values.

\begin{Shaded}
\begin{Highlighting}[]
\NormalTok{tau\_hyp }\OtherTok{\textless{}{-}} \FloatTok{0.01}

\NormalTok{alpha\_tox\_hyp }\OtherTok{\textless{}{-}} \FunctionTok{list}\NormalTok{()}
\ControlFlowTok{for}\NormalTok{ (k }\ControlFlowTok{in} \DecValTok{1}\SpecialCharTok{:}\DecValTok{2}\NormalTok{) \{}
\NormalTok{  alpha\_tox\_hyp[[k]] }\OtherTok{\textless{}{-}}\NormalTok{ q\_tox\_hyp[[k]] }\SpecialCharTok{/}\NormalTok{ tau\_hyp }\SpecialCharTok{+} \FunctionTok{rep}\NormalTok{(}\DecValTok{1}\NormalTok{, data}\SpecialCharTok{$}\NormalTok{K\_tox)}
\NormalTok{\}}
\end{Highlighting}
\end{Shaded}

\begin{Shaded}
\begin{Highlighting}[]
\NormalTok{S }\OtherTok{\textless{}{-}} \DecValTok{1000}
\NormalTok{samples\_hyp }\OtherTok{\textless{}{-}} \FunctionTok{list}\NormalTok{()}

\ControlFlowTok{for}\NormalTok{ (ks }\ControlFlowTok{in} \DecValTok{2}\SpecialCharTok{:}\NormalTok{data}\SpecialCharTok{$}\NormalTok{K\_stg) \{}
  \ControlFlowTok{for}\NormalTok{ (k }\ControlFlowTok{in} \DecValTok{1}\SpecialCharTok{:}\NormalTok{data}\SpecialCharTok{$}\NormalTok{K\_tox) \{}
\NormalTok{    name }\OtherTok{\textless{}{-}} \FunctionTok{paste0}\NormalTok{(}\StringTok{\textquotesingle{}q\_tox[\textquotesingle{}}\NormalTok{, ks, }\StringTok{\textquotesingle{},\textquotesingle{}}\NormalTok{, k, }\StringTok{\textquotesingle{}]\textquotesingle{}}\NormalTok{)}
\NormalTok{    samples\_hyp[[name]] }\OtherTok{\textless{}{-}} \FunctionTok{matrix}\NormalTok{(}\DecValTok{0}\NormalTok{, }\AttributeTok{nrow=}\DecValTok{1}\NormalTok{, }\AttributeTok{ncol=}\NormalTok{S)}
\NormalTok{  \}}
  \ControlFlowTok{for}\NormalTok{ (s }\ControlFlowTok{in} \DecValTok{1}\SpecialCharTok{:}\NormalTok{S) \{}
\NormalTok{    q }\OtherTok{\textless{}{-}} \FunctionTok{rgamma}\NormalTok{(data}\SpecialCharTok{$}\NormalTok{K\_tox, alpha\_tox\_hyp[[ks }\SpecialCharTok{{-}} \DecValTok{1}\NormalTok{]], }\DecValTok{1}\NormalTok{)}
\NormalTok{    q }\OtherTok{\textless{}{-}}\NormalTok{ q }\SpecialCharTok{/} \FunctionTok{sum}\NormalTok{(q)}
    \ControlFlowTok{for}\NormalTok{ (k }\ControlFlowTok{in} \DecValTok{1}\SpecialCharTok{:}\NormalTok{data}\SpecialCharTok{$}\NormalTok{K\_tox) \{}
\NormalTok{      name }\OtherTok{\textless{}{-}} \FunctionTok{paste0}\NormalTok{(}\StringTok{\textquotesingle{}q\_tox[\textquotesingle{}}\NormalTok{, ks, }\StringTok{\textquotesingle{},\textquotesingle{}}\NormalTok{, k, }\StringTok{\textquotesingle{}]\textquotesingle{}}\NormalTok{)}
\NormalTok{      samples\_hyp[[name]][}\DecValTok{1}\NormalTok{, s]  }\OtherTok{\textless{}{-}}\NormalTok{ q[k]}
\NormalTok{    \}}
\NormalTok{  \}}
\NormalTok{\}}
\end{Highlighting}
\end{Shaded}

\begin{Shaded}
\begin{Highlighting}[]
\FunctionTok{par}\NormalTok{(}\AttributeTok{mfrow=}\FunctionTok{c}\NormalTok{(}\DecValTok{2}\NormalTok{, }\DecValTok{2}\NormalTok{), }\AttributeTok{mar=}\FunctionTok{c}\NormalTok{(}\DecValTok{5}\NormalTok{, }\DecValTok{5}\NormalTok{, }\DecValTok{3}\NormalTok{, }\DecValTok{1}\NormalTok{))}

\NormalTok{kt }\OtherTok{\textless{}{-}} \DecValTok{2}
\ControlFlowTok{for}\NormalTok{ (ks }\ControlFlowTok{in} \DecValTok{2}\SpecialCharTok{:}\NormalTok{data}\SpecialCharTok{$}\NormalTok{K\_stg) \{}
\NormalTok{  names }\OtherTok{\textless{}{-}} \FunctionTok{sapply}\NormalTok{(}\DecValTok{1}\SpecialCharTok{:}\NormalTok{data}\SpecialCharTok{$}\NormalTok{K\_tox,}
                  \ControlFlowTok{function}\NormalTok{(k) }\FunctionTok{paste0}\NormalTok{(}\StringTok{\textquotesingle{}q\_tox[\textquotesingle{}}\NormalTok{, ks, }\StringTok{\textquotesingle{},\textquotesingle{}}\NormalTok{, kt, }\StringTok{\textquotesingle{},\textquotesingle{}}\NormalTok{, k, }\StringTok{\textquotesingle{}]\textquotesingle{}}\NormalTok{))}
\NormalTok{  util}\SpecialCharTok{$}\FunctionTok{plot\_disc\_pushforward\_quantiles}\NormalTok{(samples3a, names,}
                                       \AttributeTok{xlab=}\StringTok{"Observed Toxicity Status"}\NormalTok{,}
                                       \AttributeTok{ylab=}\StringTok{"Probability"}\NormalTok{,}
                                       \AttributeTok{display\_ylim=}\FunctionTok{c}\NormalTok{(}\DecValTok{0}\NormalTok{, }\DecValTok{1}\NormalTok{),}
                                       \AttributeTok{main=}\FunctionTok{paste}\NormalTok{(}\StringTok{"Observed}\SpecialCharTok{\textbackslash{}n}\StringTok{"}\NormalTok{,}
                                                  \StringTok{"Cancer Stage "}\NormalTok{, ks,}
                                                  \StringTok{", Treatment "}\NormalTok{,  kt))}

\NormalTok{  names }\OtherTok{\textless{}{-}} \FunctionTok{sapply}\NormalTok{(}\DecValTok{1}\SpecialCharTok{:}\NormalTok{data}\SpecialCharTok{$}\NormalTok{K\_tox,}
                  \ControlFlowTok{function}\NormalTok{(k) }\FunctionTok{paste0}\NormalTok{(}\StringTok{\textquotesingle{}q\_tox[\textquotesingle{}}\NormalTok{, ks, }\StringTok{\textquotesingle{},\textquotesingle{}}\NormalTok{, k, }\StringTok{\textquotesingle{}]\textquotesingle{}}\NormalTok{))}
\NormalTok{  util}\SpecialCharTok{$}\FunctionTok{plot\_disc\_pushforward\_quantiles}\NormalTok{(samples\_hyp, names,}
                                       \AttributeTok{xlab=}\StringTok{"Observed Toxicity Status"}\NormalTok{,}
                                       \AttributeTok{ylab=}\StringTok{"Probability"}\NormalTok{,}
                                       \AttributeTok{display\_ylim=}\FunctionTok{c}\NormalTok{(}\DecValTok{0}\NormalTok{, }\DecValTok{1}\NormalTok{),}
                                       \AttributeTok{main=}\FunctionTok{paste}\NormalTok{(}\StringTok{"Hypothetical}\SpecialCharTok{\textbackslash{}n}\StringTok{"}\NormalTok{,}
                                                  \StringTok{"Cancer Stage "}\NormalTok{, ks,}
                                                  \StringTok{", Treatment "}\NormalTok{,  kt))}
\NormalTok{\}}
\end{Highlighting}
\end{Shaded}

\includegraphics{analysis_files/figure-pdf/unnamed-chunk-53-1.pdf}

To propagate these changes forward to updated conception inferences we
just need to run the previous model with a modified
\texttt{generated\ quantities} block implementing the new inferential
quantities.

\begin{codelisting}

\caption{\texttt{model3b.stan}}

\begin{Shaded}
\begin{Highlighting}[]
\KeywordTok{data}\NormalTok{ \{}
  \CommentTok{// Number of observations}
  \DataTypeTok{int}\NormalTok{\textless{}}\KeywordTok{lower}\NormalTok{=}\DecValTok{1}\NormalTok{\textgreater{} N;}

  \CommentTok{// Number of predictions}
  \DataTypeTok{int}\NormalTok{\textless{}}\KeywordTok{lower}\NormalTok{=}\DecValTok{1}\NormalTok{\textgreater{} N\_pred;}

  \CommentTok{// Relationship status}
  \CommentTok{// k = 1: Stable partner}
  \CommentTok{// k = 2: No partner}
  \DataTypeTok{int}\NormalTok{\textless{}}\KeywordTok{lower}\NormalTok{=}\DecValTok{1}\NormalTok{\textgreater{} K\_rel;}

  \CommentTok{// Cancer stage}
  \CommentTok{// k = 1: No cancer}
  \CommentTok{// k = 2: Early stage cancer}
  \CommentTok{// k = 3: Advanced stage cancer}
  \DataTypeTok{int}\NormalTok{\textless{}}\KeywordTok{lower}\NormalTok{=}\DecValTok{1}\NormalTok{\textgreater{} K\_stg;}

  \CommentTok{// Treatment status}
  \CommentTok{// k = 1: No treatment}
  \CommentTok{// k = 2: Treatment}
  \DataTypeTok{int}\NormalTok{\textless{}}\KeywordTok{lower}\NormalTok{=}\DecValTok{1}\NormalTok{\textgreater{} K\_trt;}

  \CommentTok{// Toxicity status}
  \CommentTok{// k = 1: None}
  \CommentTok{// k = 2: Low}
  \CommentTok{// k = 3: Medium}
  \CommentTok{// k = 4: High}
  \DataTypeTok{int}\NormalTok{\textless{}}\KeywordTok{lower}\NormalTok{=}\DecValTok{1}\NormalTok{\textgreater{} K\_tox;}

  \CommentTok{// Observed conception status}
  \CommentTok{// y = 0: No conception}
  \CommentTok{// y = 1: Conception}
  \DataTypeTok{array}\NormalTok{[N] }\DataTypeTok{int}\NormalTok{\textless{}}\KeywordTok{lower}\NormalTok{=}\DecValTok{0}\NormalTok{, }\KeywordTok{upper}\NormalTok{=}\DecValTok{1}\NormalTok{\textgreater{} y;}

  \CommentTok{// Observed relationship status;}
  \DataTypeTok{array}\NormalTok{[N] }\DataTypeTok{int}\NormalTok{\textless{}}\KeywordTok{lower}\NormalTok{=}\DecValTok{1}\NormalTok{, }\KeywordTok{upper}\NormalTok{=K\_rel\textgreater{} k\_rel;}

  \CommentTok{// Observed cancer stage;}
  \DataTypeTok{array}\NormalTok{[N] }\DataTypeTok{int}\NormalTok{\textless{}}\KeywordTok{lower}\NormalTok{=}\DecValTok{1}\NormalTok{, }\KeywordTok{upper}\NormalTok{=K\_stg\textgreater{} k\_stg;}

  \CommentTok{// Observed treatment status;}
  \DataTypeTok{array}\NormalTok{[N] }\DataTypeTok{int}\NormalTok{\textless{}}\KeywordTok{lower}\NormalTok{=}\DecValTok{1}\NormalTok{, }\KeywordTok{upper}\NormalTok{=K\_trt\textgreater{} k\_trt;}

  \CommentTok{// Observed toxicity status;}
  \DataTypeTok{array}\NormalTok{[N] }\DataTypeTok{int}\NormalTok{\textless{}}\KeywordTok{lower}\NormalTok{=}\DecValTok{1}\NormalTok{, }\KeywordTok{upper}\NormalTok{=K\_tox\textgreater{} k\_tox;}

  \CommentTok{// Hypothetical toxicity distribution configurations}
  \DataTypeTok{vector}\NormalTok{[K\_tox] alpha\_tox\_hyp1;}
  \DataTypeTok{vector}\NormalTok{[K\_tox] alpha\_tox\_hyp2;}
\NormalTok{\}}

\KeywordTok{parameters}\NormalTok{ \{}
  \CommentTok{// Marginal probability of cancer stage}
  \DataTypeTok{simplex}\NormalTok{[K\_stg] q\_stg;}

  \CommentTok{// Conditional probability of relationship status given cancer stage}
  \DataTypeTok{array}\NormalTok{[K\_stg] }\DataTypeTok{simplex}\NormalTok{[K\_rel] q\_rel;}

  \CommentTok{// Conditional probability of treatment status given active cancer stage}
  \DataTypeTok{array}\NormalTok{[K\_stg {-} }\DecValTok{1}\NormalTok{] }\DataTypeTok{simplex}\NormalTok{[K\_trt] q\_trt\_active\_stg;}

  \CommentTok{// Conditional probability of toxicity status given cancer stage and}
  \CommentTok{// active treatment status}
  \DataTypeTok{array}\NormalTok{[K\_stg] }\DataTypeTok{simplex}\NormalTok{[K\_tox] q\_tox\_active\_trt;}

  \CommentTok{// Probability of conception for baseline patients in a stable}
  \CommentTok{// relationship, no cancer, and no toxicity}
  \DataTypeTok{real}\NormalTok{\textless{}}\KeywordTok{lower}\NormalTok{=}\DecValTok{0}\NormalTok{, }\KeywordTok{upper}\NormalTok{=}\DecValTok{1}\NormalTok{\textgreater{} q\_C\_0;}

  \CommentTok{// Proportional decreases in conception probability due to}
  \CommentTok{// non{-}baseline relationship status, cancer stage, and toxicity}
  \CommentTok{// status.}
  \DataTypeTok{positive\_ordered}\NormalTok{[K\_rel {-} }\DecValTok{1}\NormalTok{] alpha\_rel;}
  \DataTypeTok{positive\_ordered}\NormalTok{[K\_stg {-} }\DecValTok{1}\NormalTok{] alpha\_stg;}
  \DataTypeTok{positive\_ordered}\NormalTok{[K\_tox {-} }\DecValTok{1}\NormalTok{] alpha\_tox;}
\NormalTok{\}}

\KeywordTok{transformed parameters}\NormalTok{ \{}
  \DataTypeTok{vector}\NormalTok{[K\_rel] alpha\_rel\_buff = append\_row([}\DecValTok{0}\NormalTok{]\textquotesingle{}, alpha\_rel);}
  \DataTypeTok{vector}\NormalTok{[K\_stg] alpha\_stg\_buff = append\_row([}\DecValTok{0}\NormalTok{]\textquotesingle{}, alpha\_stg);}
  \DataTypeTok{vector}\NormalTok{[K\_tox] alpha\_tox\_buff = append\_row([}\DecValTok{0}\NormalTok{]\textquotesingle{}, alpha\_tox);}

  \CommentTok{// Conditional probability of treatment status given cancer stage}
  \DataTypeTok{array}\NormalTok{[K\_stg] }\DataTypeTok{simplex}\NormalTok{[K\_trt] q\_trt = append\_array(\{ [}\FloatTok{1.0}\NormalTok{, }\FloatTok{0.0}\NormalTok{]\textquotesingle{} \},}
\NormalTok{                                                   q\_trt\_active\_stg);}

  \CommentTok{// Conditional probability of toxicity status given cancer stage and}
  \CommentTok{// treatment status}
  \DataTypeTok{array}\NormalTok{[K\_stg, K\_trt] }\DataTypeTok{simplex}\NormalTok{[K\_tox] q\_tox;}
\NormalTok{  q\_tox[, }\DecValTok{1}\NormalTok{] = \{ [}\DecValTok{1}\NormalTok{, }\DecValTok{0}\NormalTok{, }\DecValTok{0}\NormalTok{, }\DecValTok{0}\NormalTok{]\textquotesingle{},}
\NormalTok{                 [}\DecValTok{1}\NormalTok{, }\DecValTok{0}\NormalTok{, }\DecValTok{0}\NormalTok{, }\DecValTok{0}\NormalTok{]\textquotesingle{},}
\NormalTok{                 [}\DecValTok{1}\NormalTok{, }\DecValTok{0}\NormalTok{, }\DecValTok{0}\NormalTok{, }\DecValTok{0}\NormalTok{]\textquotesingle{} \};}
\NormalTok{  q\_tox[, }\DecValTok{2}\NormalTok{] = q\_tox\_active\_trt;}
\NormalTok{\}}

\KeywordTok{model}\NormalTok{ \{}
  \CommentTok{// Prior model}
  \KeywordTok{target +=}\NormalTok{ dirichlet\_lpdf(q\_stg | [}\DecValTok{4}\NormalTok{, }\DecValTok{3}\NormalTok{, }\DecValTok{1}\NormalTok{]\textquotesingle{});}
  \KeywordTok{target +=}\NormalTok{ dirichlet\_lpdf(q\_rel[}\DecValTok{1}\NormalTok{] | [}\DecValTok{4}\NormalTok{, }\DecValTok{1}\NormalTok{]\textquotesingle{});}
  \KeywordTok{target +=}\NormalTok{ dirichlet\_lpdf(q\_rel[}\DecValTok{2}\NormalTok{] | [}\DecValTok{2}\NormalTok{, }\DecValTok{1}\NormalTok{]\textquotesingle{});}
  \KeywordTok{target +=}\NormalTok{ dirichlet\_lpdf(q\_rel[}\DecValTok{3}\NormalTok{] | [}\DecValTok{1}\NormalTok{, }\DecValTok{1}\NormalTok{]\textquotesingle{});}
  \KeywordTok{target +=}\NormalTok{ dirichlet\_lpdf(q\_trt\_active\_stg[}\DecValTok{1}\NormalTok{] | [}\DecValTok{1}\NormalTok{, }\DecValTok{3}\NormalTok{]\textquotesingle{});}
  \KeywordTok{target +=}\NormalTok{ dirichlet\_lpdf(q\_trt\_active\_stg[}\DecValTok{2}\NormalTok{] | [}\FloatTok{0.5}\NormalTok{, }\DecValTok{4}\NormalTok{]\textquotesingle{});}
  \KeywordTok{target +=}\NormalTok{ dirichlet\_lpdf(q\_tox\_active\_trt[}\DecValTok{1}\NormalTok{] | [}\DecValTok{4}\NormalTok{, }\DecValTok{3}\NormalTok{, }\DecValTok{2}\NormalTok{, }\DecValTok{1}\NormalTok{]\textquotesingle{});}
  \KeywordTok{target +=}\NormalTok{ dirichlet\_lpdf(q\_tox\_active\_trt[}\DecValTok{2}\NormalTok{] | [}\DecValTok{2}\NormalTok{, }\DecValTok{3}\NormalTok{, }\DecValTok{3}\NormalTok{, }\DecValTok{1}\NormalTok{]\textquotesingle{});}
  \KeywordTok{target +=}\NormalTok{ dirichlet\_lpdf(q\_tox\_active\_trt[}\DecValTok{3}\NormalTok{] | [}\DecValTok{1}\NormalTok{, }\DecValTok{2}\NormalTok{, }\DecValTok{3}\NormalTok{, }\DecValTok{3}\NormalTok{]\textquotesingle{});}

  \KeywordTok{target +=}\NormalTok{ beta\_lpdf(q\_C\_0 | }\FloatTok{12.7}\NormalTok{, }\FloatTok{3.7}\NormalTok{); }\CommentTok{// 0.50 \textless{}\textasciitilde{} q\_C\_0 \textless{}\textasciitilde{} 0.95}

  \KeywordTok{target +=}\NormalTok{ normal\_lpdf(alpha\_rel | }\DecValTok{0}\NormalTok{, }\DecValTok{3}\NormalTok{ / }\FloatTok{2.32}\NormalTok{); }\CommentTok{// 0 \textless{}\textasciitilde{} alpha \textless{}\textasciitilde{} {-}log(0.05)}
  \KeywordTok{target +=}\NormalTok{ normal\_lpdf(alpha\_stg | }\DecValTok{0}\NormalTok{, }\DecValTok{3}\NormalTok{ / }\FloatTok{2.32}\NormalTok{); }\CommentTok{// 0 \textless{}\textasciitilde{} alpha \textless{}\textasciitilde{} {-}log(0.05)}
  \KeywordTok{target +=}\NormalTok{ normal\_lpdf(alpha\_tox | }\DecValTok{0}\NormalTok{, }\DecValTok{3}\NormalTok{ / }\FloatTok{2.32}\NormalTok{); }\CommentTok{// 0 \textless{}\textasciitilde{} alpha \textless{}\textasciitilde{} {-}log(0.05)}

  \CommentTok{// Observational model}
  \ControlFlowTok{for}\NormalTok{ (n }\ControlFlowTok{in} \DecValTok{1}\NormalTok{:N) \{}
    \KeywordTok{target +=}\NormalTok{ categorical\_lpmf(k\_stg[n] | q\_stg);}
    \KeywordTok{target +=}\NormalTok{ categorical\_lpmf(k\_rel[n] | q\_rel[k\_stg[n]]);}
    \KeywordTok{target +=}\NormalTok{ categorical\_lpmf(k\_trt[n] | q\_trt[k\_stg[n]]);}
    \KeywordTok{target +=}\NormalTok{ categorical\_lpmf(k\_tox[n] | q\_tox[k\_stg[n],}
\NormalTok{                                                k\_trt[n]]);}
\NormalTok{  \}}

  \KeywordTok{target +=}\NormalTok{ bernoulli\_lpmf(y | q\_C\_0 * exp({-}alpha\_rel\_buff[k\_rel]}
\NormalTok{                                           {-}alpha\_stg\_buff[k\_stg]}
\NormalTok{                                           {-}alpha\_tox\_buff[k\_tox]));}
\NormalTok{\}}

\KeywordTok{generated quantities}\NormalTok{ \{}
  \CommentTok{// Posterior predictive data}
  \DataTypeTok{array}\NormalTok{[N\_pred] }\DataTypeTok{real}\NormalTok{\textless{}}\KeywordTok{lower}\NormalTok{=}\DecValTok{0}\NormalTok{, }\KeywordTok{upper}\NormalTok{=}\DecValTok{1}\NormalTok{\textgreater{} q\_pred;}
  \DataTypeTok{array}\NormalTok{[N\_pred] }\DataTypeTok{real}\NormalTok{\textless{}}\KeywordTok{lower}\NormalTok{=}\DecValTok{0}\NormalTok{, }\KeywordTok{upper}\NormalTok{=}\DecValTok{1}\NormalTok{\textgreater{} q\_hyp\_pred;}

  \ControlFlowTok{for}\NormalTok{ (n }\ControlFlowTok{in} \DecValTok{1}\NormalTok{:N\_pred) \{}
    \DataTypeTok{int}\NormalTok{ k\_stg\_pred = categorical\_rng(q\_stg);}
    \DataTypeTok{int}\NormalTok{ k\_rel\_pred = categorical\_rng(q\_rel[k\_stg\_pred]);}
    \DataTypeTok{int}\NormalTok{ k\_trt\_pred = categorical\_rng(q\_trt[k\_stg\_pred]);}
    \DataTypeTok{int}\NormalTok{ k\_tox\_pred = categorical\_rng(q\_tox[k\_stg\_pred,}
\NormalTok{                                           k\_trt\_pred]);}

\NormalTok{    q\_pred[n] = q\_C\_0 * exp({-}alpha\_rel\_buff[k\_rel\_pred]}
\NormalTok{                            {-}alpha\_stg\_buff[k\_stg\_pred]}
\NormalTok{                            {-}alpha\_tox\_buff[k\_tox\_pred]);}
\NormalTok{  \}}

\NormalTok{  \{}
    \DataTypeTok{array}\NormalTok{[K\_stg, K\_trt] }\DataTypeTok{vector}\NormalTok{[K\_tox] q\_tox\_hyp;}
\NormalTok{    q\_tox\_hyp[, }\DecValTok{1}\NormalTok{] = q\_tox[, }\DecValTok{1}\NormalTok{];}
\NormalTok{    q\_tox\_hyp[, }\DecValTok{2}\NormalTok{] = \{ [}\DecValTok{1}\NormalTok{, }\DecValTok{0}\NormalTok{, }\DecValTok{0}\NormalTok{, }\DecValTok{0}\NormalTok{]\textquotesingle{},}
\NormalTok{                       dirichlet\_rng(alpha\_tox\_hyp1),}
\NormalTok{                       dirichlet\_rng(alpha\_tox\_hyp2) \};}

    \ControlFlowTok{for}\NormalTok{ (n }\ControlFlowTok{in} \DecValTok{1}\NormalTok{:N\_pred) \{}
      \DataTypeTok{int}\NormalTok{ k\_stg\_pred = categorical\_rng(q\_stg);}
      \DataTypeTok{int}\NormalTok{ k\_rel\_pred = categorical\_rng(q\_rel[k\_stg\_pred]);}
      \DataTypeTok{int}\NormalTok{ k\_trt\_pred = categorical\_rng(q\_trt[k\_stg\_pred]);}
      \DataTypeTok{int}\NormalTok{ k\_tox\_pred = categorical\_rng(q\_tox\_hyp[k\_stg\_pred,}
\NormalTok{                                                 k\_trt\_pred]);}

\NormalTok{      q\_hyp\_pred[n] = q\_C\_0 * exp({-}alpha\_rel\_buff[k\_rel\_pred]}
\NormalTok{                                  {-}alpha\_stg\_buff[k\_stg\_pred]}
\NormalTok{                                  {-}alpha\_tox\_buff[k\_tox\_pred]);}
\NormalTok{    \}}
\NormalTok{  \}}
\NormalTok{\}}
\end{Highlighting}
\end{Shaded}

\end{codelisting}

\begin{Shaded}
\begin{Highlighting}[]
\NormalTok{data}\SpecialCharTok{$}\NormalTok{alpha\_tox\_hyp1 }\OtherTok{\textless{}{-}}\NormalTok{ q\_tox\_hyp[[}\DecValTok{1}\NormalTok{]]}
\NormalTok{data}\SpecialCharTok{$}\NormalTok{alpha\_tox\_hyp2 }\OtherTok{\textless{}{-}}\NormalTok{ q\_tox\_hyp[[}\DecValTok{2}\NormalTok{]]}

\NormalTok{fit }\OtherTok{\textless{}{-}} \FunctionTok{stan}\NormalTok{(}\AttributeTok{file=}\StringTok{"stan\_programs/model3b.stan"}\NormalTok{,}
            \AttributeTok{data=}\NormalTok{data, }\AttributeTok{seed=}\DecValTok{8438338}\NormalTok{,}
            \AttributeTok{warmup=}\DecValTok{1000}\NormalTok{, }\AttributeTok{iter=}\DecValTok{2024}\NormalTok{, }\AttributeTok{refresh=}\DecValTok{0}\NormalTok{)}
\end{Highlighting}
\end{Shaded}

The changes to the \texttt{generated\ quantities} block do alter how
pseudo-random numbers are used by Stan. This, in turn, can affect the
stochastic posterior computation. Consequently we have to double check
the computational diagnostics to ensure that no new issues have arisen.
Fortunately everything still looks okay.

\begin{Shaded}
\begin{Highlighting}[]
\NormalTok{diagnostics }\OtherTok{\textless{}{-}}\NormalTok{ util}\SpecialCharTok{$}\FunctionTok{extract\_hmc\_diagnostics}\NormalTok{(fit)}
\NormalTok{util}\SpecialCharTok{$}\FunctionTok{check\_all\_hmc\_diagnostics}\NormalTok{(diagnostics)}
\end{Highlighting}
\end{Shaded}

\begin{verbatim}
  All Hamiltonian Monte Carlo diagnostics are consistent with reliable
Markov chain Monte Carlo.
\end{verbatim}

\begin{Shaded}
\begin{Highlighting}[]
\NormalTok{samples3b }\OtherTok{\textless{}{-}}\NormalTok{ util}\SpecialCharTok{$}\FunctionTok{extract\_expectand\_vals}\NormalTok{(fit)}
\NormalTok{base\_samples }\OtherTok{\textless{}{-}}\NormalTok{ util}\SpecialCharTok{$}\FunctionTok{filter\_expectands}\NormalTok{(samples3b,}
                                       \FunctionTok{c}\NormalTok{(}\StringTok{\textquotesingle{}q\_stg\textquotesingle{}}\NormalTok{, }\StringTok{\textquotesingle{}q\_trt\_active\_stg\textquotesingle{}}\NormalTok{,}
                                         \StringTok{\textquotesingle{}q\_tox\_active\_trt\textquotesingle{}}\NormalTok{,}\StringTok{\textquotesingle{}q\_C\_0\textquotesingle{}}\NormalTok{,}
                                         \StringTok{\textquotesingle{}alpha\_rel\textquotesingle{}}\NormalTok{, }\StringTok{\textquotesingle{}alpha\_stg\textquotesingle{}}\NormalTok{,}
                                         \StringTok{\textquotesingle{}alpha\_tox\textquotesingle{}}\NormalTok{),}
                                       \AttributeTok{check\_arrays=}\ConstantTok{TRUE}\NormalTok{)}
\NormalTok{util}\SpecialCharTok{$}\FunctionTok{check\_all\_expectand\_diagnostics}\NormalTok{(base\_samples)}
\end{Highlighting}
\end{Shaded}

\begin{verbatim}
All expectands checked appear to be behaving well enough for reliable
Markov chain Monte Carlo estimation.
\end{verbatim}

Because the model and data are the same the retrodictive checks will all
be equivalent, and we can jump directly to analyzing the posterior
inferences. In particular comparing inferences for the population
summaries from the observed and hypothetical cohorts allows us to
quantify the efficacy of the new treatment.

The behavior of the baseline sub-population remains the same, but the
rest of the population shifts to higher conception probabilities after
the hypothetical intervention. This makes sense given that the baseline
patients are not undergoing treatments and hence are not influenced by
any changes in those treatments.

\begin{Shaded}
\begin{Highlighting}[]
\FunctionTok{par}\NormalTok{(}\AttributeTok{mfrow=}\FunctionTok{c}\NormalTok{(}\DecValTok{1}\NormalTok{, }\DecValTok{2}\NormalTok{), }\AttributeTok{mar=}\FunctionTok{c}\NormalTok{(}\DecValTok{5}\NormalTok{, }\DecValTok{5}\NormalTok{, }\DecValTok{1}\NormalTok{, }\DecValTok{1}\NormalTok{))}

\NormalTok{util}\SpecialCharTok{$}\FunctionTok{plot\_hist\_quantiles}\NormalTok{(samples3b, }\StringTok{\textquotesingle{}q\_pred\textquotesingle{}}\NormalTok{, }\SpecialCharTok{{-}}\FloatTok{0.05}\NormalTok{, }\FloatTok{1.05}\NormalTok{, }\FloatTok{0.1}\NormalTok{,}
                         \AttributeTok{xlab=}\StringTok{"Conception Probability"}\NormalTok{,}
                         \AttributeTok{main=}\StringTok{"Observed Cohort"}\NormalTok{)}

\NormalTok{util}\SpecialCharTok{$}\FunctionTok{plot\_hist\_quantiles}\NormalTok{(samples3b, }\StringTok{\textquotesingle{}q\_hyp\_pred\textquotesingle{}}\NormalTok{, }\SpecialCharTok{{-}}\FloatTok{0.05}\NormalTok{, }\FloatTok{1.05}\NormalTok{, }\FloatTok{0.1}\NormalTok{,}
                         \AttributeTok{xlab=}\StringTok{"Conception Probability"}\NormalTok{,}
                         \AttributeTok{main=}\StringTok{"Hypothetical Cohort"}\NormalTok{)}
\end{Highlighting}
\end{Shaded}

\includegraphics{analysis_files/figure-pdf/unnamed-chunk-56-1.pdf}

Reducing the two cohorts to average conception probabilities
demonstrates the benefit of the new treatment more clearly.

\begin{Shaded}
\begin{Highlighting}[]
\NormalTok{var\_repl }\OtherTok{\textless{}{-}} \FunctionTok{list}\NormalTok{(}\StringTok{\textquotesingle{}p\textquotesingle{}}\OtherTok{=}\NormalTok{util}\SpecialCharTok{$}\FunctionTok{name\_array}\NormalTok{(}\StringTok{\textquotesingle{}q\_pred\textquotesingle{}}\NormalTok{, }\FunctionTok{c}\NormalTok{(data}\SpecialCharTok{$}\NormalTok{N\_pred)))}
\NormalTok{pop\_ave\_samples }\OtherTok{\textless{}{-}}
\NormalTok{  util}\SpecialCharTok{$}\FunctionTok{eval\_expectand\_pushforward}\NormalTok{(samples3b,}
                                  \ControlFlowTok{function}\NormalTok{(p) }\FunctionTok{mean}\NormalTok{(p),}
\NormalTok{                                  var\_repl)}

\NormalTok{var\_repl }\OtherTok{\textless{}{-}} \FunctionTok{list}\NormalTok{(}\StringTok{\textquotesingle{}p\textquotesingle{}}\OtherTok{=}\NormalTok{util}\SpecialCharTok{$}\FunctionTok{name\_array}\NormalTok{(}\StringTok{\textquotesingle{}q\_hyp\_pred\textquotesingle{}}\NormalTok{, }\FunctionTok{c}\NormalTok{(data}\SpecialCharTok{$}\NormalTok{N\_pred)))}
\NormalTok{hyp\_pop\_ave\_samples }\OtherTok{\textless{}{-}}
\NormalTok{  util}\SpecialCharTok{$}\FunctionTok{eval\_expectand\_pushforward}\NormalTok{(samples3b,}
                                  \ControlFlowTok{function}\NormalTok{(p) }\FunctionTok{mean}\NormalTok{(p),}
\NormalTok{                                  var\_repl)}
\end{Highlighting}
\end{Shaded}

\begin{Shaded}
\begin{Highlighting}[]
\FunctionTok{par}\NormalTok{(}\AttributeTok{mfrow=}\FunctionTok{c}\NormalTok{(}\DecValTok{1}\NormalTok{, }\DecValTok{1}\NormalTok{), }\AttributeTok{mar=}\FunctionTok{c}\NormalTok{(}\DecValTok{5}\NormalTok{, }\DecValTok{5}\NormalTok{, }\DecValTok{1}\NormalTok{, }\DecValTok{1}\NormalTok{))}

\NormalTok{name }\OtherTok{\textless{}{-}} \StringTok{"Conception Probability}\SpecialCharTok{\textbackslash{}n}\StringTok{Population Average"}
\NormalTok{util}\SpecialCharTok{$}\FunctionTok{plot\_expectand\_pushforward}\NormalTok{(pop\_ave\_samples,}
                                \DecValTok{25}\NormalTok{, }\AttributeTok{flim=}\FunctionTok{c}\NormalTok{(}\FloatTok{0.6}\NormalTok{, }\FloatTok{0.72}\NormalTok{),}
                                \AttributeTok{display\_name=}\NormalTok{name,}
                                \AttributeTok{col=}\NormalTok{util}\SpecialCharTok{$}\NormalTok{c\_light)}
\FunctionTok{text}\NormalTok{(}\FloatTok{0.62}\NormalTok{, }\DecValTok{40}\NormalTok{, }\StringTok{"Observed}\SpecialCharTok{\textbackslash{}n}\StringTok{Cohort"}\NormalTok{, }\AttributeTok{col=}\NormalTok{util}\SpecialCharTok{$}\NormalTok{c\_light)}

\NormalTok{util}\SpecialCharTok{$}\FunctionTok{plot\_expectand\_pushforward}\NormalTok{(hyp\_pop\_ave\_samples,}
                                \DecValTok{25}\NormalTok{, }\AttributeTok{flim=}\FunctionTok{c}\NormalTok{(}\FloatTok{0.6}\NormalTok{, }\FloatTok{0.72}\NormalTok{),}
                                \AttributeTok{border=}\StringTok{"\#BBBBBB88"}\NormalTok{,}
                                \AttributeTok{add=}\ConstantTok{TRUE}\NormalTok{)}
\FunctionTok{text}\NormalTok{(}\FloatTok{0.69}\NormalTok{, }\DecValTok{20}\NormalTok{, }\StringTok{"Hypothetical}\SpecialCharTok{\textbackslash{}n}\StringTok{Cohort"}\NormalTok{, }\AttributeTok{col=}\NormalTok{util}\SpecialCharTok{$}\NormalTok{c\_dark)}
\end{Highlighting}
\end{Shaded}

\includegraphics{analysis_files/figure-pdf/unnamed-chunk-58-1.pdf}

We can even directly calculate the posterior probability that the
average conception probability is higher in the hypothetical cohort.

\begin{Shaded}
\begin{Highlighting}[]
\NormalTok{ave\_samples }\OtherTok{\textless{}{-}} \FunctionTok{list}\NormalTok{(}\StringTok{"obs"} \OtherTok{=}\NormalTok{ pop\_ave\_samples,}
                    \StringTok{"hyp"} \OtherTok{=}\NormalTok{ hyp\_pop\_ave\_samples)}
\NormalTok{p\_est }\OtherTok{\textless{}{-}}
\NormalTok{  util}\SpecialCharTok{$}\FunctionTok{implicit\_subset\_prob}\NormalTok{(ave\_samples,}
                            \ControlFlowTok{function}\NormalTok{(obs, hyp) hyp }\SpecialCharTok{\textgreater{}}\NormalTok{ obs)}

\NormalTok{format\_string }\OtherTok{\textless{}{-}} \FunctionTok{paste}\NormalTok{(}\StringTok{"Posterior probability that hypothetical"}\NormalTok{,}
                       \StringTok{"population average}\SpecialCharTok{\textbackslash{}n}\StringTok{is greater than observed"}\NormalTok{,}
                       \StringTok{"population average = \%.3f +/{-} \%.3f."}\NormalTok{)}
\FunctionTok{cat}\NormalTok{(}\FunctionTok{sprintf}\NormalTok{(format\_string, p\_est[}\DecValTok{1}\NormalTok{], }\DecValTok{2} \SpecialCharTok{*}\NormalTok{ p\_est[}\DecValTok{2}\NormalTok{]))}
\end{Highlighting}
\end{Shaded}

\begin{verbatim}
Posterior probability that hypothetical population average
is greater than observed population average = 0.787 +/- 0.013.
\end{verbatim}

Interestingly the lower quantile summary exhibits a smaller benefit to
the new treatment. This suggests that improvements to the lowest
fertility patients are weaker than improvements to the population as a
whole.

\begin{Shaded}
\begin{Highlighting}[]
\NormalTok{var\_repl }\OtherTok{\textless{}{-}} \FunctionTok{list}\NormalTok{(}\StringTok{\textquotesingle{}p\textquotesingle{}}\OtherTok{=}\NormalTok{util}\SpecialCharTok{$}\FunctionTok{name\_array}\NormalTok{(}\StringTok{\textquotesingle{}q\_pred\textquotesingle{}}\NormalTok{, }\FunctionTok{c}\NormalTok{(data}\SpecialCharTok{$}\NormalTok{N\_pred)))}
\NormalTok{pop\_quant\_samples }\OtherTok{\textless{}{-}}
\NormalTok{  util}\SpecialCharTok{$}\FunctionTok{eval\_expectand\_pushforward}\NormalTok{(samples3b,}
                                  \ControlFlowTok{function}\NormalTok{(p) }\FunctionTok{quantile}\NormalTok{(p, }\AttributeTok{prob=}\FunctionTok{c}\NormalTok{(}\FloatTok{0.25}\NormalTok{)),}
\NormalTok{                                  var\_repl)}

\NormalTok{var\_repl }\OtherTok{\textless{}{-}} \FunctionTok{list}\NormalTok{(}\StringTok{\textquotesingle{}p\textquotesingle{}}\OtherTok{=}\NormalTok{util}\SpecialCharTok{$}\FunctionTok{name\_array}\NormalTok{(}\StringTok{\textquotesingle{}q\_hyp\_pred\textquotesingle{}}\NormalTok{, }\FunctionTok{c}\NormalTok{(data}\SpecialCharTok{$}\NormalTok{N\_pred)))}
\NormalTok{hyp\_pop\_quant\_samples }\OtherTok{\textless{}{-}}
\NormalTok{  util}\SpecialCharTok{$}\FunctionTok{eval\_expectand\_pushforward}\NormalTok{(samples3b,}
                                  \ControlFlowTok{function}\NormalTok{(p) }\FunctionTok{quantile}\NormalTok{(p, }\AttributeTok{prob=}\FunctionTok{c}\NormalTok{(}\FloatTok{0.25}\NormalTok{)),}
\NormalTok{                                  var\_repl)}
\end{Highlighting}
\end{Shaded}

\begin{Shaded}
\begin{Highlighting}[]
\FunctionTok{par}\NormalTok{(}\AttributeTok{mfrow=}\FunctionTok{c}\NormalTok{(}\DecValTok{1}\NormalTok{, }\DecValTok{1}\NormalTok{), }\AttributeTok{mar=}\FunctionTok{c}\NormalTok{(}\DecValTok{5}\NormalTok{, }\DecValTok{5}\NormalTok{, }\DecValTok{1}\NormalTok{, }\DecValTok{1}\NormalTok{))}

\NormalTok{name }\OtherTok{\textless{}{-}} \StringTok{"Conception Probability}\SpecialCharTok{\textbackslash{}n}\StringTok{Population Lower Quantile"}
\NormalTok{util}\SpecialCharTok{$}\FunctionTok{plot\_expectand\_pushforward}\NormalTok{(pop\_quant\_samples,}
                                \DecValTok{25}\NormalTok{, }\AttributeTok{flim=}\FunctionTok{c}\NormalTok{(}\FloatTok{0.3}\NormalTok{, }\FloatTok{0.8}\NormalTok{),}
                                \AttributeTok{display\_name=}\NormalTok{name,}
                                \AttributeTok{col=}\NormalTok{util}\SpecialCharTok{$}\NormalTok{c\_light)}
\FunctionTok{text}\NormalTok{(}\FloatTok{0.45}\NormalTok{, }\DecValTok{5}\NormalTok{, }\StringTok{"Observed}\SpecialCharTok{\textbackslash{}n}\StringTok{Cohort"}\NormalTok{, }\AttributeTok{col=}\NormalTok{util}\SpecialCharTok{$}\NormalTok{c\_light)}

\NormalTok{util}\SpecialCharTok{$}\FunctionTok{plot\_expectand\_pushforward}\NormalTok{(hyp\_pop\_quant\_samples,}
                                \DecValTok{25}\NormalTok{, }\AttributeTok{flim=}\FunctionTok{c}\NormalTok{(}\FloatTok{0.3}\NormalTok{, }\FloatTok{0.8}\NormalTok{),}
                                \AttributeTok{border=}\StringTok{"\#BBBBBB88"}\NormalTok{,}
                                \AttributeTok{add=}\ConstantTok{TRUE}\NormalTok{)}
\FunctionTok{text}\NormalTok{(}\FloatTok{0.675}\NormalTok{, }\DecValTok{5}\NormalTok{, }\StringTok{"Hypothetical}\SpecialCharTok{\textbackslash{}n}\StringTok{Cohort"}\NormalTok{, }\AttributeTok{col=}\NormalTok{util}\SpecialCharTok{$}\NormalTok{c\_dark)}
\end{Highlighting}
\end{Shaded}

\includegraphics{analysis_files/figure-pdf/unnamed-chunk-61-1.pdf}

\begin{Shaded}
\begin{Highlighting}[]
\NormalTok{quant\_samples }\OtherTok{\textless{}{-}} \FunctionTok{list}\NormalTok{(}\StringTok{"obs"} \OtherTok{=}\NormalTok{ pop\_quant\_samples,}
                      \StringTok{"hyp"} \OtherTok{=}\NormalTok{ hyp\_pop\_quant\_samples)}
\NormalTok{p\_est }\OtherTok{\textless{}{-}}
\NormalTok{  util}\SpecialCharTok{$}\FunctionTok{implicit\_subset\_prob}\NormalTok{(quant\_samples,}
                            \ControlFlowTok{function}\NormalTok{(obs, hyp) hyp }\SpecialCharTok{\textgreater{}}\NormalTok{ obs)}

\NormalTok{format\_string }\OtherTok{\textless{}{-}} \FunctionTok{paste}\NormalTok{(}\StringTok{"Posterior probability that hypothetical"}\NormalTok{,}
                       \StringTok{"population lower quantile}\SpecialCharTok{\textbackslash{}n}\StringTok{is greater than"}\NormalTok{,}
                       \StringTok{"observed population lower quantile"}\NormalTok{,}
                       \StringTok{"= \%.3f +/{-} \%.3f."}\NormalTok{)}
\FunctionTok{cat}\NormalTok{(}\FunctionTok{sprintf}\NormalTok{(format\_string, p\_est[}\DecValTok{1}\NormalTok{], }\DecValTok{2} \SpecialCharTok{*}\NormalTok{ p\_est[}\DecValTok{2}\NormalTok{]))}
\end{Highlighting}
\end{Shaded}

\begin{verbatim}
Posterior probability that hypothetical population lower quantile
is greater than observed population lower quantile = 0.570 +/- 0.015.
\end{verbatim}

\section{Model 4}\label{model-4}

At this point we have done some pretty cool Bayesian modeling and
inference, but we have yet to consider ART. This is concerning given
that ART can drastically increase fertility regardless of the direct and
indirect cancer effects.

Unfortunately the posterior predictive conception behavior of our
previous model very poorly matches the observed conception behavior once
we take ART into account. Model critique is always fundamentally limited
by the criteria we consider!

\begin{Shaded}
\begin{Highlighting}[]
\FunctionTok{par}\NormalTok{(}\AttributeTok{mfrow=}\FunctionTok{c}\NormalTok{(}\DecValTok{1}\NormalTok{, }\DecValTok{1}\NormalTok{), }\AttributeTok{mar=}\FunctionTok{c}\NormalTok{(}\DecValTok{5}\NormalTok{, }\DecValTok{5}\NormalTok{, }\DecValTok{1}\NormalTok{, }\DecValTok{1}\NormalTok{))}

\NormalTok{pred\_names }\OtherTok{\textless{}{-}} \FunctionTok{sapply}\NormalTok{(}\DecValTok{1}\SpecialCharTok{:}\NormalTok{data}\SpecialCharTok{$}\NormalTok{N, }\ControlFlowTok{function}\NormalTok{(n) }\FunctionTok{paste0}\NormalTok{(}\StringTok{\textquotesingle{}y\_pred[\textquotesingle{}}\NormalTok{, n, }\StringTok{\textquotesingle{}]\textquotesingle{}}\NormalTok{))}

\NormalTok{util}\SpecialCharTok{$}\FunctionTok{plot\_conditional\_mean\_quantiles}\NormalTok{(samples2, pred\_names, data}\SpecialCharTok{$}\NormalTok{k\_art,}
                                     \SpecialCharTok{{-}}\FloatTok{0.5}\NormalTok{, }\FloatTok{1.5}\NormalTok{, }\DecValTok{1}\NormalTok{, data}\SpecialCharTok{$}\NormalTok{y,}
                                     \AttributeTok{xlab=}\StringTok{"Observed ART Status"}\NormalTok{,}
                                     \AttributeTok{ylab=}\StringTok{"Average Conception Status"}\NormalTok{)}
\end{Highlighting}
\end{Shaded}

\includegraphics{analysis_files/figure-pdf/unnamed-chunk-63-1.pdf}

Upon reflection this discrepancy isn't too surprising. The higher
fertility of ART patients will bias the overall fertility of the
observed cohort to larger values than what we would see from patients
attempting natural conception alone.

\subsection{Model 4a}\label{model-4a}

In order to make faithful inferences for both the observed cohort and
any hypothetical cohorts we need to account for ART by quantifying the
conception probability of ART and non-ART patients separately.

\subsubsection{ART Observational Model}\label{art-observational-model}

Patients who have not undergone ART can be modeled as we did in the
previous model, with the baseline probability of conceptual now more
precisely interpreted as the baseline probability of \emph{natural}
conception. The challenge with modeling ART patients is that successful
conceptions could be a result of either natural or ART methods but
\emph{we do not know which}. Consequently we have to take into account
the potential success of both methods of conception at the same time.

ART conception and natural conception are likely to be coupled at some
level. For example patients in stable, monogamous relations with female
partners are unlikely to attempt one methods after conception is
achieved with the other. In theory we could attempt to model these
competing events, but here we will make a simpler assumption that the
two methods of conception are approximately independent so that an
observed conception is given by a success with either method or both.

Given this assumptions there are a few ways to calculate the overall
conception probability for a given patient.

One way is to recognize that the probability of no conception is equal
to the probability of both methods failing to conceive, \begin{align*}
p( \text{no conception} )
&=
p( \text{no conception} )
\\
1 - p(\text{conception})
&=\;\,
p(\text{no natural conception})
\\
&\quad \cdot
p(\text{no ART conception})
\\
1 - p(\text{conception})
&=\;\,
(1 - p(\text{natural conception}))
\\
&\quad \cdot
(1 - p(\text{ART conception}))
\\
p(\text{conception})
&=\quad\!
  p(\text{ART conception}) \, (1 - p(\text{natural conception}))
\\
&\quad+
p(\text{natural conception}).
\end{align*}

Another way is to decompose the probability of conception into
conditional probabilities, \begin{align*}
p( \text{conception} )
&=\hphantom{+\cdot}\,
  p( \text{conception} \mid \text{No natural conception})
\\
&\hphantom{=+}\cdot
  p( \text{No natural conception} )
\\
&\hphantom{=} + \hphantom{\cdot}\,
p( \text{conception} \mid \text{natural conception})
\\
&\hphantom{=+}\cdot
  p( \text{natural conception} )
\\
p( \text{conception} )
&=\hphantom{+\cdot}\,
 p( \text{ART conception} )
\\
&\hphantom{=+}\cdot
  (1 - p( \text{natural conception} ) )
\\
&\hphantom{=} + \hphantom{\cdot}\,
1
\\
&\hphantom{=+}\cdot
  p( \text{natural conception} )
\\
p( \text{conception} )
&=\hphantom{+}\!
  p( \text{ART conception} ) \, (1 - p(\text{natural conception}))
\\
&\hphantom{=} +
p( \text{natural conception} ).
\end{align*}

The probability of natural conception \(p( \text{natural conception} )\)
for an individual patient is given by the previous model, \[
q_{NC, n}
=
q_{NC_{0}} \,
\exp(-\alpha_{\text{stg}, n}
     -\alpha_{\text{rel}, n}
     -\alpha_{\text{tox}, n} ).
\]

In general \(p( \text{ART conception} )\) can also vary across the
various patient categories, but here will assume that it is at least
approximately homogeneous so that it can be modeled with a single
parameter \(q_{AC}\).

\subsubsection{ART Prior Model}\label{art-prior-model}

In our previous elicitation of reasonable behaviors for the baseline
conception probability we did not explicitly consider ART. Consequently
our previous prior model for \(q_{C_{0}}\) is now somewhat ambiguous.

If we were implicitly considering domain expertise for natural
conception then we can apply the same prior model to \(q_{NC_{0}}\). On
the other hand if were implicitly considering domain expertise for any
method of conception then we would need to update the prior model to
incorporate any information we have about natural conception in
particular. Here we will assume the former and use the same prior model
for \(q_{NC_{0}}\) as we did for \(q_{C_{0}}\).

This leaves a prior model for \(q_{\text{art}}\). Here let's say that
our clinical domain expertise disfavors ART conception probabilities
below \(0.4\) and above \(0.8\).

\begin{Shaded}
\begin{Highlighting}[]
\NormalTok{q\_low }\OtherTok{\textless{}{-}} \FloatTok{0.4}
\NormalTok{q\_high }\OtherTok{\textless{}{-}} \FloatTok{0.8}

\FunctionTok{stan}\NormalTok{(}\AttributeTok{file=}\StringTok{\textquotesingle{}stan\_programs/prior\_tune\_beta.stan\textquotesingle{}}\NormalTok{,}
     \AttributeTok{data=}\FunctionTok{list}\NormalTok{(}\StringTok{\textquotesingle{}q\_low\textquotesingle{}} \OtherTok{=}\NormalTok{ q\_low, }\StringTok{\textquotesingle{}q\_high\textquotesingle{}} \OtherTok{=}\NormalTok{ q\_high),}
     \AttributeTok{iter=}\DecValTok{1}\NormalTok{, }\AttributeTok{warmup=}\DecValTok{0}\NormalTok{, }\AttributeTok{chains=}\DecValTok{1}\NormalTok{,}
     \AttributeTok{seed=}\DecValTok{4838282}\NormalTok{, }\AttributeTok{algorithm=}\StringTok{"Fixed\_param"}\NormalTok{)}
\end{Highlighting}
\end{Shaded}

\begin{verbatim}
alpha = 18.2301
beta = 11.6206

SAMPLING FOR MODEL 'anon_model' NOW (CHAIN 1).
Chain 1: Iteration: 1 / 1 [100%]  (Sampling)
Chain 1: 
Chain 1:  Elapsed Time: 0 seconds (Warm-up)
Chain 1:                0 seconds (Sampling)
Chain 1:                0 seconds (Total)
Chain 1: 
\end{verbatim}

\begin{verbatim}
Inference for Stan model: anon_model.
1 chains, each with iter=1; warmup=0; thin=1; 
post-warmup draws per chain=1, total post-warmup draws=1.

       mean se_mean sd  2.5%   25%   50%   75% 97.5% n_eff Rhat
alpha 18.23      NA NA 18.23 18.23 18.23 18.23 18.23     0  NaN
beta  11.62      NA NA 11.62 11.62 11.62 11.62 11.62     0  NaN
lp__   0.00      NA NA  0.00  0.00  0.00  0.00  0.00     0  NaN

Samples were drawn using (diag_e) at Tue Apr 29 21:56:11 2025.
For each parameter, n_eff is a crude measure of effective sample size,
and Rhat is the potential scale reduction factor on split chains (at 
convergence, Rhat=1).
\end{verbatim}

\begin{Shaded}
\begin{Highlighting}[]
\FunctionTok{par}\NormalTok{(}\AttributeTok{mfrow=}\FunctionTok{c}\NormalTok{(}\DecValTok{1}\NormalTok{, }\DecValTok{1}\NormalTok{), }\AttributeTok{mar=}\FunctionTok{c}\NormalTok{(}\DecValTok{5}\NormalTok{, }\DecValTok{5}\NormalTok{, }\DecValTok{3}\NormalTok{, }\DecValTok{1}\NormalTok{))}

\NormalTok{qs }\OtherTok{\textless{}{-}} \FunctionTok{seq}\NormalTok{(}\DecValTok{0}\NormalTok{, }\DecValTok{1}\NormalTok{, }\FloatTok{0.001}\NormalTok{)}
\NormalTok{dens }\OtherTok{\textless{}{-}} \FunctionTok{dbeta}\NormalTok{(qs, }\FloatTok{18.2}\NormalTok{, }\FloatTok{11.6}\NormalTok{)}
\FunctionTok{plot}\NormalTok{(qs, dens, }\AttributeTok{type=}\StringTok{"l"}\NormalTok{, }\AttributeTok{col=}\NormalTok{util}\SpecialCharTok{$}\NormalTok{c\_dark, }\AttributeTok{lwd=}\DecValTok{2}\NormalTok{,}
     \AttributeTok{xlab=}\StringTok{"Conception Probability"}\NormalTok{,}
     \AttributeTok{ylab=}\StringTok{"Prior Density"}\NormalTok{, }\AttributeTok{yaxt=}\StringTok{\textquotesingle{}n\textquotesingle{}}\NormalTok{)}

\NormalTok{q98 }\OtherTok{\textless{}{-}} \FunctionTok{seq}\NormalTok{(q\_low, q\_high, }\FloatTok{0.001}\NormalTok{)}
\NormalTok{dens }\OtherTok{\textless{}{-}} \FunctionTok{dbeta}\NormalTok{(q98, }\FloatTok{18.2}\NormalTok{, }\FloatTok{11.6}\NormalTok{)}
\NormalTok{q98 }\OtherTok{\textless{}{-}} \FunctionTok{c}\NormalTok{(q98, q\_high, q\_low)}
\NormalTok{dens }\OtherTok{\textless{}{-}} \FunctionTok{c}\NormalTok{(dens, }\DecValTok{0}\NormalTok{, }\DecValTok{0}\NormalTok{)}

\FunctionTok{polygon}\NormalTok{(q98, dens, }\AttributeTok{col=}\NormalTok{util}\SpecialCharTok{$}\NormalTok{c\_dark, }\AttributeTok{border=}\ConstantTok{NA}\NormalTok{)}

\FunctionTok{abline}\NormalTok{(}\AttributeTok{v=}\NormalTok{q\_low,  }\AttributeTok{lwd=}\DecValTok{3}\NormalTok{, }\AttributeTok{lty=}\DecValTok{2}\NormalTok{, }\AttributeTok{col=}\StringTok{\textquotesingle{}\#DDDDDD\textquotesingle{}}\NormalTok{)}
\FunctionTok{abline}\NormalTok{(}\AttributeTok{v=}\NormalTok{q\_high, }\AttributeTok{lwd=}\DecValTok{3}\NormalTok{, }\AttributeTok{lty=}\DecValTok{2}\NormalTok{, }\AttributeTok{col=}\StringTok{\textquotesingle{}\#DDDDDD\textquotesingle{}}\NormalTok{)}
\end{Highlighting}
\end{Shaded}

\includegraphics{analysis_files/figure-pdf/unnamed-chunk-65-1.pdf}

\subsubsection{Posterior
Quantification}\label{posterior-quantification-3}

Incorporating ART results in an expanded model that exhausts all of the
features of the observed data (Figure~\ref{fig-model4}).

\begin{figure}

\centering{

\includegraphics[width=1\textwidth,height=\textheight]{figures/model4/model4.pdf}

}

\caption{\label{fig-model4}Our final model takes into account the
influence of assisted reproduction technologies on patient conception.}

\end{figure}%

In order to apply all of the retrodictive checks that have previously
considered we will need to simulate posterior predictive patient
characteristics but then simulate posterior predictive conception
statuses from the \emph{observed} patient characteristics and not the
newly simulated ones.

\begin{codelisting}

\caption{\texttt{model4a.stan}}

\begin{Shaded}
\begin{Highlighting}[]
\KeywordTok{data}\NormalTok{ \{}
  \CommentTok{// Number of observations}
  \DataTypeTok{int}\NormalTok{\textless{}}\KeywordTok{lower}\NormalTok{=}\DecValTok{1}\NormalTok{\textgreater{} N;}

  \CommentTok{// Relationship status}
  \CommentTok{// k = 1: Stable partner}
  \CommentTok{// k = 2: No partner}
  \DataTypeTok{int}\NormalTok{\textless{}}\KeywordTok{lower}\NormalTok{=}\DecValTok{1}\NormalTok{\textgreater{} K\_rel;}

  \CommentTok{// Cancer stage}
  \CommentTok{// k = 1: No cancer}
  \CommentTok{// k = 2: Early stage cancer}
  \CommentTok{// k = 3: Advanced stage cancer}
  \DataTypeTok{int}\NormalTok{\textless{}}\KeywordTok{lower}\NormalTok{=}\DecValTok{1}\NormalTok{\textgreater{} K\_stg;}

  \CommentTok{// Treatment status}
  \CommentTok{// k = 1: No treatment}
  \CommentTok{// k = 2: Treatment}
  \DataTypeTok{int}\NormalTok{\textless{}}\KeywordTok{lower}\NormalTok{=}\DecValTok{1}\NormalTok{\textgreater{} K\_trt;}

  \CommentTok{// Toxicity status}
  \CommentTok{// k = 1: None}
  \CommentTok{// k = 2: Low}
  \CommentTok{// k = 3: Medium}
  \CommentTok{// k = 4: High}
  \DataTypeTok{int}\NormalTok{\textless{}}\KeywordTok{lower}\NormalTok{=}\DecValTok{1}\NormalTok{\textgreater{} K\_tox;}

  \CommentTok{// Observed conception status}
  \CommentTok{// y = 0: No conception}
  \CommentTok{// y = 1: Conception}
  \DataTypeTok{array}\NormalTok{[N] }\DataTypeTok{int}\NormalTok{\textless{}}\KeywordTok{lower}\NormalTok{=}\DecValTok{0}\NormalTok{, }\KeywordTok{upper}\NormalTok{=}\DecValTok{1}\NormalTok{\textgreater{} y;}

  \CommentTok{// Observed relationship status;}
  \DataTypeTok{array}\NormalTok{[N] }\DataTypeTok{int}\NormalTok{\textless{}}\KeywordTok{lower}\NormalTok{=}\DecValTok{1}\NormalTok{, }\KeywordTok{upper}\NormalTok{=K\_rel\textgreater{} k\_rel;}

  \CommentTok{// Observed cancer stage;}
  \DataTypeTok{array}\NormalTok{[N] }\DataTypeTok{int}\NormalTok{\textless{}}\KeywordTok{lower}\NormalTok{=}\DecValTok{1}\NormalTok{, }\KeywordTok{upper}\NormalTok{=K\_stg\textgreater{} k\_stg;}

  \CommentTok{// Observed treatment status;}
  \DataTypeTok{array}\NormalTok{[N] }\DataTypeTok{int}\NormalTok{\textless{}}\KeywordTok{lower}\NormalTok{=}\DecValTok{1}\NormalTok{, }\KeywordTok{upper}\NormalTok{=K\_trt\textgreater{} k\_trt;}

  \CommentTok{// Observed toxicity status;}
  \DataTypeTok{array}\NormalTok{[N] }\DataTypeTok{int}\NormalTok{\textless{}}\KeywordTok{lower}\NormalTok{=}\DecValTok{1}\NormalTok{, }\KeywordTok{upper}\NormalTok{=K\_tox\textgreater{} k\_tox;}

  \CommentTok{// Observed assistive reproductive technology (ART) status}
  \CommentTok{// k = 0: No ART}
  \CommentTok{// k = 1: ART}
  \DataTypeTok{array}\NormalTok{[N] }\DataTypeTok{int}\NormalTok{\textless{}}\KeywordTok{lower}\NormalTok{=}\DecValTok{0}\NormalTok{, }\KeywordTok{upper}\NormalTok{=}\DecValTok{1}\NormalTok{\textgreater{} k\_art;}
\NormalTok{\}}

\KeywordTok{parameters}\NormalTok{ \{}
  \CommentTok{// Marginal probability of cancer stage}
  \DataTypeTok{simplex}\NormalTok{[K\_stg] q\_stg;}

  \CommentTok{// Conditional probability of relationship status given cancer stage}
  \DataTypeTok{array}\NormalTok{[K\_stg] }\DataTypeTok{simplex}\NormalTok{[K\_rel] q\_rel;}

  \CommentTok{// Conditional probability of treatment status given active cancer stage}
  \DataTypeTok{array}\NormalTok{[K\_stg {-} }\DecValTok{1}\NormalTok{] }\DataTypeTok{simplex}\NormalTok{[K\_trt] q\_trt\_active\_stg;}

  \CommentTok{// Conditional probability of toxicity status given cancer stage and}
  \CommentTok{// active treatment status}
  \DataTypeTok{array}\NormalTok{[K\_stg] }\DataTypeTok{simplex}\NormalTok{[K\_tox] q\_tox\_active\_trt;}

  \CommentTok{// Probability of natural conception for baseline patients in a stable}
  \CommentTok{// relationship, no cancer, and no toxicity}
  \DataTypeTok{real}\NormalTok{\textless{}}\KeywordTok{lower}\NormalTok{=}\DecValTok{0}\NormalTok{, }\KeywordTok{upper}\NormalTok{=}\DecValTok{1}\NormalTok{\textgreater{} q\_NC\_0;}

  \CommentTok{// Proportional decreases in conception probability due to}
  \CommentTok{// non{-}baseline relationship status, cancer stage, and toxicity}
  \CommentTok{// status.}
  \DataTypeTok{positive\_ordered}\NormalTok{[K\_rel {-} }\DecValTok{1}\NormalTok{] alpha\_rel;}
  \DataTypeTok{positive\_ordered}\NormalTok{[K\_stg {-} }\DecValTok{1}\NormalTok{] alpha\_stg;}
  \DataTypeTok{positive\_ordered}\NormalTok{[K\_tox {-} }\DecValTok{1}\NormalTok{] alpha\_tox;}

  \CommentTok{// Probability of ART conception}
  \DataTypeTok{real}\NormalTok{\textless{}}\KeywordTok{lower}\NormalTok{=}\DecValTok{0}\NormalTok{, }\KeywordTok{upper}\NormalTok{=}\DecValTok{1}\NormalTok{\textgreater{} q\_AC;}
\NormalTok{\}}

\KeywordTok{transformed parameters}\NormalTok{ \{}
  \DataTypeTok{vector}\NormalTok{[K\_rel] alpha\_rel\_buff = append\_row([}\DecValTok{0}\NormalTok{]\textquotesingle{}, alpha\_rel);}
  \DataTypeTok{vector}\NormalTok{[K\_stg] alpha\_stg\_buff = append\_row([}\DecValTok{0}\NormalTok{]\textquotesingle{}, alpha\_stg);}
  \DataTypeTok{vector}\NormalTok{[K\_tox] alpha\_tox\_buff = append\_row([}\DecValTok{0}\NormalTok{]\textquotesingle{}, alpha\_tox);}

  \CommentTok{// Conditional probability of treatment status given cancer stage}
  \DataTypeTok{array}\NormalTok{[K\_stg] }\DataTypeTok{simplex}\NormalTok{[K\_trt] q\_trt = append\_array(\{ [}\FloatTok{1.0}\NormalTok{, }\FloatTok{0.0}\NormalTok{]\textquotesingle{} \},}
\NormalTok{                                                   q\_trt\_active\_stg);}

  \CommentTok{// Conditional probability of toxicity status given cancer stage and}
  \CommentTok{// treatment status}
  \DataTypeTok{array}\NormalTok{[K\_stg, K\_trt] }\DataTypeTok{simplex}\NormalTok{[K\_tox] q\_tox;}
\NormalTok{  q\_tox[, }\DecValTok{1}\NormalTok{] = \{ [}\DecValTok{1}\NormalTok{, }\DecValTok{0}\NormalTok{, }\DecValTok{0}\NormalTok{, }\DecValTok{0}\NormalTok{]\textquotesingle{},}
\NormalTok{                 [}\DecValTok{1}\NormalTok{, }\DecValTok{0}\NormalTok{, }\DecValTok{0}\NormalTok{, }\DecValTok{0}\NormalTok{]\textquotesingle{},}
\NormalTok{                 [}\DecValTok{1}\NormalTok{, }\DecValTok{0}\NormalTok{, }\DecValTok{0}\NormalTok{, }\DecValTok{0}\NormalTok{]\textquotesingle{} \};}
\NormalTok{  q\_tox[, }\DecValTok{2}\NormalTok{] = q\_tox\_active\_trt;}
\NormalTok{\}}

\KeywordTok{model}\NormalTok{ \{}
  \CommentTok{// Prior model}
  \KeywordTok{target +=}\NormalTok{ dirichlet\_lpdf(q\_stg | [}\DecValTok{4}\NormalTok{, }\DecValTok{3}\NormalTok{, }\DecValTok{1}\NormalTok{]\textquotesingle{});}
  \KeywordTok{target +=}\NormalTok{ dirichlet\_lpdf(q\_rel[}\DecValTok{1}\NormalTok{] | [}\DecValTok{4}\NormalTok{, }\DecValTok{1}\NormalTok{]\textquotesingle{});}
  \KeywordTok{target +=}\NormalTok{ dirichlet\_lpdf(q\_rel[}\DecValTok{2}\NormalTok{] | [}\DecValTok{2}\NormalTok{, }\DecValTok{1}\NormalTok{]\textquotesingle{});}
  \KeywordTok{target +=}\NormalTok{ dirichlet\_lpdf(q\_rel[}\DecValTok{3}\NormalTok{] | [}\DecValTok{1}\NormalTok{, }\DecValTok{1}\NormalTok{]\textquotesingle{});}
  \KeywordTok{target +=}\NormalTok{ dirichlet\_lpdf(q\_trt\_active\_stg[}\DecValTok{1}\NormalTok{] | [}\DecValTok{1}\NormalTok{, }\DecValTok{3}\NormalTok{]\textquotesingle{});}
  \KeywordTok{target +=}\NormalTok{ dirichlet\_lpdf(q\_trt\_active\_stg[}\DecValTok{2}\NormalTok{] | [}\FloatTok{0.5}\NormalTok{, }\DecValTok{4}\NormalTok{]\textquotesingle{});}
  \KeywordTok{target +=}\NormalTok{ dirichlet\_lpdf(q\_tox\_active\_trt[}\DecValTok{1}\NormalTok{] | [}\DecValTok{4}\NormalTok{, }\DecValTok{3}\NormalTok{, }\DecValTok{2}\NormalTok{, }\DecValTok{1}\NormalTok{]\textquotesingle{});}
  \KeywordTok{target +=}\NormalTok{ dirichlet\_lpdf(q\_tox\_active\_trt[}\DecValTok{2}\NormalTok{] | [}\DecValTok{2}\NormalTok{, }\DecValTok{3}\NormalTok{, }\DecValTok{3}\NormalTok{, }\DecValTok{1}\NormalTok{]\textquotesingle{});}
  \KeywordTok{target +=}\NormalTok{ dirichlet\_lpdf(q\_tox\_active\_trt[}\DecValTok{3}\NormalTok{] | [}\DecValTok{1}\NormalTok{, }\DecValTok{2}\NormalTok{, }\DecValTok{3}\NormalTok{, }\DecValTok{3}\NormalTok{]\textquotesingle{});}

  \KeywordTok{target +=}\NormalTok{ beta\_lpdf(q\_NC\_0 | }\FloatTok{12.7}\NormalTok{, }\FloatTok{3.7}\NormalTok{); }\CommentTok{// 0.50 \textless{}\textasciitilde{} q\_NC\_0 \textless{}\textasciitilde{} 0.95}

  \KeywordTok{target +=}\NormalTok{ normal\_lpdf(alpha\_rel | }\DecValTok{0}\NormalTok{, }\DecValTok{3}\NormalTok{ / }\FloatTok{2.32}\NormalTok{); }\CommentTok{// 0 \textless{}\textasciitilde{} alpha \textless{}\textasciitilde{} {-}log(0.05)}
  \KeywordTok{target +=}\NormalTok{ normal\_lpdf(alpha\_stg | }\DecValTok{0}\NormalTok{, }\DecValTok{3}\NormalTok{ / }\FloatTok{2.32}\NormalTok{); }\CommentTok{// 0 \textless{}\textasciitilde{} alpha \textless{}\textasciitilde{} {-}log(0.05)}
  \KeywordTok{target +=}\NormalTok{ normal\_lpdf(alpha\_tox | }\DecValTok{0}\NormalTok{, }\DecValTok{3}\NormalTok{ / }\FloatTok{2.32}\NormalTok{); }\CommentTok{// 0 \textless{}\textasciitilde{} alpha \textless{}\textasciitilde{} {-}log(0.05)}

  \KeywordTok{target +=}\NormalTok{ beta\_lpdf(q\_AC | }\FloatTok{18.2}\NormalTok{, }\FloatTok{11.6}\NormalTok{); }\CommentTok{// 0.4 \textless{}\textasciitilde{} q\_AC\_0 \textless{}\textasciitilde{} 0.8}

  \CommentTok{// Observational model}
  \ControlFlowTok{for}\NormalTok{ (n }\ControlFlowTok{in} \DecValTok{1}\NormalTok{:N) \{}
    \DataTypeTok{real}\NormalTok{ q\_NC = q\_NC\_0 * exp({-}alpha\_rel\_buff[k\_rel[n]]}
\NormalTok{                             {-}alpha\_stg\_buff[k\_stg[n]]}
\NormalTok{                             {-}alpha\_tox\_buff[k\_tox[n]]);}

    \KeywordTok{target +=}\NormalTok{ categorical\_lpmf(k\_stg[n] | q\_stg);}
    \KeywordTok{target +=}\NormalTok{ categorical\_lpmf(k\_rel[n] | q\_rel[k\_stg[n]]);}
    \KeywordTok{target +=}\NormalTok{ categorical\_lpmf(k\_trt[n] | q\_trt[k\_stg[n]]);}
    \KeywordTok{target +=}\NormalTok{ categorical\_lpmf(k\_tox[n] | q\_tox[k\_stg[n],}
\NormalTok{                                                k\_trt[n]]);}

    \ControlFlowTok{if}\NormalTok{ (k\_art[n]) \{}
      \KeywordTok{target +=}\NormalTok{ bernoulli\_lpmf(y[n] | q\_AC * (}\DecValTok{1}\NormalTok{ {-} q\_NC) + q\_NC);}
\NormalTok{    \} }\ControlFlowTok{else}\NormalTok{ \{}
      \KeywordTok{target +=}\NormalTok{ bernoulli\_lpmf(y[n] | q\_NC);}
\NormalTok{    \}}
\NormalTok{  \}}
\NormalTok{\}}

\KeywordTok{generated quantities}\NormalTok{ \{}
  \CommentTok{// Posterior predictive data}
  \DataTypeTok{array}\NormalTok{[N] }\DataTypeTok{int}\NormalTok{\textless{}}\KeywordTok{lower}\NormalTok{=}\DecValTok{1}\NormalTok{, }\KeywordTok{upper}\NormalTok{=K\_rel\textgreater{} k\_rel\_pred;}
  \DataTypeTok{array}\NormalTok{[N] }\DataTypeTok{int}\NormalTok{\textless{}}\KeywordTok{lower}\NormalTok{=}\DecValTok{1}\NormalTok{, }\KeywordTok{upper}\NormalTok{=K\_stg\textgreater{} k\_stg\_pred;}
  \DataTypeTok{array}\NormalTok{[N] }\DataTypeTok{int}\NormalTok{\textless{}}\KeywordTok{lower}\NormalTok{=}\DecValTok{1}\NormalTok{, }\KeywordTok{upper}\NormalTok{=K\_trt\textgreater{} k\_trt\_pred;}
  \DataTypeTok{array}\NormalTok{[N] }\DataTypeTok{int}\NormalTok{\textless{}}\KeywordTok{lower}\NormalTok{=}\DecValTok{1}\NormalTok{, }\KeywordTok{upper}\NormalTok{=K\_tox\textgreater{} k\_tox\_pred;}

  \DataTypeTok{array}\NormalTok{[N] }\DataTypeTok{int}\NormalTok{\textless{}}\KeywordTok{lower}\NormalTok{=}\DecValTok{0}\NormalTok{, }\KeywordTok{upper}\NormalTok{=}\DecValTok{1}\NormalTok{\textgreater{} y\_pred;}

  \ControlFlowTok{for}\NormalTok{ (n }\ControlFlowTok{in} \DecValTok{1}\NormalTok{:N) \{}
    \CommentTok{// Posterior predictive patient categorizations}
\NormalTok{    k\_stg\_pred[n] = categorical\_rng(q\_stg);}
\NormalTok{    k\_rel\_pred[n] = categorical\_rng(q\_rel[k\_stg\_pred[n]]);}
\NormalTok{    k\_trt\_pred[n] = categorical\_rng(q\_trt[k\_stg\_pred[n]]);}
\NormalTok{    k\_tox\_pred[n] = categorical\_rng(q\_tox[k\_stg\_pred[n],}
\NormalTok{                                          k\_trt\_pred[n]]);}

    \CommentTok{// Posterior predictive patient conception behavior}
    \CommentTok{// conditioned on the observed patient categorizations,}
    \CommentTok{// not the patient predictions that we just predicted.}
\NormalTok{    \{}
      \DataTypeTok{real}\NormalTok{ q\_NC = q\_NC\_0 * exp({-}alpha\_rel\_buff[k\_rel[n]]}
\NormalTok{                         {-}alpha\_stg\_buff[k\_stg[n]]}
\NormalTok{                         {-}alpha\_tox\_buff[k\_tox[n]]);}
      \ControlFlowTok{if}\NormalTok{ (k\_art[n] == }\DecValTok{1}\NormalTok{) \{}
\NormalTok{        y\_pred[n] = bernoulli\_rng(q\_AC * (}\DecValTok{1}\NormalTok{ {-} q\_NC) + q\_NC);}
\NormalTok{      \} }\ControlFlowTok{else}\NormalTok{ \{}
\NormalTok{        y\_pred[n] = bernoulli\_rng(q\_NC);}
\NormalTok{      \}}
\NormalTok{    \}}
\NormalTok{  \}}
\NormalTok{\}}
\end{Highlighting}
\end{Shaded}

\end{codelisting}

\begin{Shaded}
\begin{Highlighting}[]
\NormalTok{fit }\OtherTok{\textless{}{-}} \FunctionTok{stan}\NormalTok{(}\AttributeTok{file=}\StringTok{"stan\_programs/model4a.stan"}\NormalTok{,}
            \AttributeTok{data=}\NormalTok{data, }\AttributeTok{seed=}\DecValTok{8438338}\NormalTok{,}
            \AttributeTok{warmup=}\DecValTok{1000}\NormalTok{, }\AttributeTok{iter=}\DecValTok{2024}\NormalTok{, }\AttributeTok{refresh=}\DecValTok{0}\NormalTok{)}
\end{Highlighting}
\end{Shaded}

The posterior computation continues to prove robust.

\begin{Shaded}
\begin{Highlighting}[]
\NormalTok{diagnostics }\OtherTok{\textless{}{-}}\NormalTok{ util}\SpecialCharTok{$}\FunctionTok{extract\_hmc\_diagnostics}\NormalTok{(fit)}
\NormalTok{util}\SpecialCharTok{$}\FunctionTok{check\_all\_hmc\_diagnostics}\NormalTok{(diagnostics)}
\end{Highlighting}
\end{Shaded}

\begin{verbatim}
  All Hamiltonian Monte Carlo diagnostics are consistent with reliable
Markov chain Monte Carlo.
\end{verbatim}

\begin{Shaded}
\begin{Highlighting}[]
\NormalTok{samples4a }\OtherTok{\textless{}{-}}\NormalTok{ util}\SpecialCharTok{$}\FunctionTok{extract\_expectand\_vals}\NormalTok{(fit)}
\NormalTok{base\_samples }\OtherTok{\textless{}{-}}\NormalTok{ util}\SpecialCharTok{$}\FunctionTok{filter\_expectands}\NormalTok{(samples4a,}
                                       \FunctionTok{c}\NormalTok{(}\StringTok{\textquotesingle{}q\_stg\textquotesingle{}}\NormalTok{, }\StringTok{\textquotesingle{}q\_trt\_active\_stg\textquotesingle{}}\NormalTok{,}
                                         \StringTok{\textquotesingle{}q\_tox\_active\_trt\textquotesingle{}}\NormalTok{,}\StringTok{\textquotesingle{}q\_NC\_0\textquotesingle{}}\NormalTok{,}
                                         \StringTok{\textquotesingle{}alpha\_rel\textquotesingle{}}\NormalTok{, }\StringTok{\textquotesingle{}alpha\_stg\textquotesingle{}}\NormalTok{,}
                                         \StringTok{\textquotesingle{}alpha\_tox\textquotesingle{}}\NormalTok{, }\StringTok{\textquotesingle{}q\_AC\textquotesingle{}}\NormalTok{),}
                                       \AttributeTok{check\_arrays=}\ConstantTok{TRUE}\NormalTok{)}
\NormalTok{util}\SpecialCharTok{$}\FunctionTok{check\_all\_expectand\_diagnostics}\NormalTok{(base\_samples)}
\end{Highlighting}
\end{Shaded}

\begin{verbatim}
All expectands checked appear to be behaving well enough for reliable
Markov chain Monte Carlo estimation.
\end{verbatim}

\subsubsection{Retrodictive Checks}\label{retrodictive-checks-3}

With all of the checks we have introduced at this point we have to be
careful to go through the retrodictive behavior of each summary one by
one.

First we'll see how well the model recovers the patient characteristic
behavior. Fortunately there are no signs of problems.

\begin{Shaded}
\begin{Highlighting}[]
\FunctionTok{par}\NormalTok{(}\AttributeTok{mfrow=}\FunctionTok{c}\NormalTok{(}\DecValTok{2}\NormalTok{, }\DecValTok{2}\NormalTok{), }\AttributeTok{mar=}\FunctionTok{c}\NormalTok{(}\DecValTok{5}\NormalTok{, }\DecValTok{5}\NormalTok{, }\DecValTok{1}\NormalTok{, }\DecValTok{1}\NormalTok{))}

\NormalTok{util}\SpecialCharTok{$}\FunctionTok{plot\_hist\_quantiles}\NormalTok{(samples4a, }\StringTok{\textquotesingle{}k\_rel\_pred\textquotesingle{}}\NormalTok{,}
                         \FloatTok{0.5}\NormalTok{, data}\SpecialCharTok{$}\NormalTok{K\_rel }\SpecialCharTok{+} \FloatTok{0.5}\NormalTok{, }\DecValTok{1}\NormalTok{,}
                         \AttributeTok{baseline\_values=}\NormalTok{data}\SpecialCharTok{$}\NormalTok{k\_rel,}
                         \AttributeTok{xlab=}\StringTok{"Relationship Status"}\NormalTok{)}

\NormalTok{util}\SpecialCharTok{$}\FunctionTok{plot\_hist\_quantiles}\NormalTok{(samples4a, }\StringTok{\textquotesingle{}k\_stg\_pred\textquotesingle{}}\NormalTok{,}
                         \FloatTok{0.5}\NormalTok{, data}\SpecialCharTok{$}\NormalTok{K\_stg }\SpecialCharTok{+} \FloatTok{0.5}\NormalTok{, }\DecValTok{1}\NormalTok{,}
                         \AttributeTok{baseline\_values=}\NormalTok{data}\SpecialCharTok{$}\NormalTok{k\_stg,}
                         \AttributeTok{xlab=}\StringTok{"Cancer Stage"}\NormalTok{)}

\NormalTok{util}\SpecialCharTok{$}\FunctionTok{plot\_hist\_quantiles}\NormalTok{(samples4a, }\StringTok{\textquotesingle{}k\_trt\_pred\textquotesingle{}}\NormalTok{,}
                         \FloatTok{0.5}\NormalTok{, data}\SpecialCharTok{$}\NormalTok{K\_trt }\SpecialCharTok{+} \FloatTok{0.5}\NormalTok{, }\DecValTok{1}\NormalTok{,}
                         \AttributeTok{baseline\_values=}\NormalTok{data}\SpecialCharTok{$}\NormalTok{k\_trt,}
                         \AttributeTok{xlab=}\StringTok{"Treatment Status"}\NormalTok{)}

\NormalTok{util}\SpecialCharTok{$}\FunctionTok{plot\_hist\_quantiles}\NormalTok{(samples4a, }\StringTok{\textquotesingle{}k\_tox\_pred\textquotesingle{}}\NormalTok{,}
                         \FloatTok{0.5}\NormalTok{, data}\SpecialCharTok{$}\NormalTok{K\_tox }\SpecialCharTok{+} \FloatTok{0.5}\NormalTok{, }\DecValTok{1}\NormalTok{,}
                         \AttributeTok{baseline\_values=}\NormalTok{data}\SpecialCharTok{$}\NormalTok{k\_tox,}
                         \AttributeTok{xlab=}\StringTok{"Toxicity Status"}\NormalTok{)}
\end{Highlighting}
\end{Shaded}

\includegraphics{analysis_files/figure-pdf/unnamed-chunk-68-1.pdf}

Next we have the posterior retrodictive check sensitive to the aggregate
conception behavior. Again all looks good.

\begin{Shaded}
\begin{Highlighting}[]
\FunctionTok{par}\NormalTok{(}\AttributeTok{mfrow=}\FunctionTok{c}\NormalTok{(}\DecValTok{1}\NormalTok{, }\DecValTok{1}\NormalTok{), }\AttributeTok{mar=}\FunctionTok{c}\NormalTok{(}\DecValTok{5}\NormalTok{, }\DecValTok{5}\NormalTok{, }\DecValTok{1}\NormalTok{, }\DecValTok{1}\NormalTok{))}

\NormalTok{util}\SpecialCharTok{$}\FunctionTok{plot\_hist\_quantiles}\NormalTok{(samples4a, }\StringTok{\textquotesingle{}y\_pred\textquotesingle{}}\NormalTok{, }\SpecialCharTok{{-}}\FloatTok{0.5}\NormalTok{, }\FloatTok{1.5}\NormalTok{, }\DecValTok{1}\NormalTok{,}
                         \AttributeTok{baseline\_values=}\NormalTok{data}\SpecialCharTok{$}\NormalTok{y,}
                         \AttributeTok{xlab=}\StringTok{"Observed Conception Status"}\NormalTok{)}
\end{Highlighting}
\end{Shaded}

\includegraphics{analysis_files/figure-pdf/unnamed-chunk-69-1.pdf}

Finally we can compare the observed and posterior predictive conception
behavior stratified by all of the clinical and demographic categories,
including the use of ART. Looks like this expanded model has resolved
the retrodictive tension of the previous model.

\begin{Shaded}
\begin{Highlighting}[]
\FunctionTok{par}\NormalTok{(}\AttributeTok{mfrow=}\FunctionTok{c}\NormalTok{(}\DecValTok{2}\NormalTok{, }\DecValTok{1}\NormalTok{), }\AttributeTok{mar=}\FunctionTok{c}\NormalTok{(}\DecValTok{5}\NormalTok{, }\DecValTok{5}\NormalTok{, }\DecValTok{1}\NormalTok{, }\DecValTok{1}\NormalTok{))}

\NormalTok{pred\_names }\OtherTok{\textless{}{-}} \FunctionTok{sapply}\NormalTok{(}\DecValTok{1}\SpecialCharTok{:}\NormalTok{data}\SpecialCharTok{$}\NormalTok{N, }\ControlFlowTok{function}\NormalTok{(n) }\FunctionTok{paste0}\NormalTok{(}\StringTok{\textquotesingle{}y\_pred[\textquotesingle{}}\NormalTok{, n, }\StringTok{\textquotesingle{}]\textquotesingle{}}\NormalTok{))}

\NormalTok{util}\SpecialCharTok{$}\FunctionTok{plot\_conditional\_mean\_quantiles}\NormalTok{(samples4a, pred\_names, data}\SpecialCharTok{$}\NormalTok{k\_rel,}
                                     \FloatTok{0.5}\NormalTok{, data}\SpecialCharTok{$}\NormalTok{K\_rel }\SpecialCharTok{+} \FloatTok{0.5}\NormalTok{, }\DecValTok{1}\NormalTok{, data}\SpecialCharTok{$}\NormalTok{y,}
                                     \AttributeTok{xlab=}\StringTok{"Observed Relationship Status"}\NormalTok{,}
                                     \AttributeTok{ylab=}\StringTok{"Average Conception Status"}\NormalTok{)}

\NormalTok{util}\SpecialCharTok{$}\FunctionTok{plot\_conditional\_mean\_quantiles}\NormalTok{(samples4a, pred\_names, data}\SpecialCharTok{$}\NormalTok{k\_stg,}
                                     \FloatTok{0.5}\NormalTok{, data}\SpecialCharTok{$}\NormalTok{K\_stg }\SpecialCharTok{+} \FloatTok{0.5}\NormalTok{, }\DecValTok{1}\NormalTok{, data}\SpecialCharTok{$}\NormalTok{y,}
                                     \AttributeTok{xlab=}\StringTok{"Observed Cancer Stage"}\NormalTok{,}
                                     \AttributeTok{ylab=}\StringTok{"Average Conception Status"}\NormalTok{)}
\end{Highlighting}
\end{Shaded}

\includegraphics{analysis_files/figure-pdf/unnamed-chunk-70-1.pdf}

\begin{Shaded}
\begin{Highlighting}[]
\FunctionTok{par}\NormalTok{(}\AttributeTok{mfrow=}\FunctionTok{c}\NormalTok{(}\DecValTok{2}\NormalTok{, }\DecValTok{1}\NormalTok{), }\AttributeTok{mar=}\FunctionTok{c}\NormalTok{(}\DecValTok{5}\NormalTok{, }\DecValTok{5}\NormalTok{, }\DecValTok{1}\NormalTok{, }\DecValTok{1}\NormalTok{))}

\NormalTok{util}\SpecialCharTok{$}\FunctionTok{plot\_conditional\_mean\_quantiles}\NormalTok{(samples4a, pred\_names, data}\SpecialCharTok{$}\NormalTok{k\_tox,}
                                     \FloatTok{0.5}\NormalTok{, data}\SpecialCharTok{$}\NormalTok{K\_tox }\SpecialCharTok{+} \FloatTok{0.5}\NormalTok{, }\DecValTok{1}\NormalTok{, data}\SpecialCharTok{$}\NormalTok{y,}
                                     \AttributeTok{xlab=}\StringTok{"Observed Toxicity Status"}\NormalTok{,}
                                     \AttributeTok{ylab=}\StringTok{"Average Conception Status"}\NormalTok{)}

\NormalTok{util}\SpecialCharTok{$}\FunctionTok{plot\_conditional\_mean\_quantiles}\NormalTok{(samples4a, pred\_names, data}\SpecialCharTok{$}\NormalTok{k\_art,}
                                     \SpecialCharTok{{-}}\FloatTok{0.5}\NormalTok{, }\FloatTok{1.5}\NormalTok{, }\DecValTok{1}\NormalTok{, data}\SpecialCharTok{$}\NormalTok{y,}
                                     \AttributeTok{xlab=}\StringTok{"Observed ART Status"}\NormalTok{,}
                                     \AttributeTok{ylab=}\StringTok{"Average Conception Status"}\NormalTok{)}
\end{Highlighting}
\end{Shaded}

\includegraphics{analysis_files/figure-pdf/unnamed-chunk-71-1.pdf}

\subsubsection{Posterior Insights}\label{posterior-insights-3}

We can analyze the posterior inferences from this new model directly.

\begin{Shaded}
\begin{Highlighting}[]
\FunctionTok{par}\NormalTok{(}\AttributeTok{mfrow=}\FunctionTok{c}\NormalTok{(}\DecValTok{1}\NormalTok{, }\DecValTok{1}\NormalTok{), }\AttributeTok{mar=}\FunctionTok{c}\NormalTok{(}\DecValTok{5}\NormalTok{, }\DecValTok{5}\NormalTok{, }\DecValTok{1}\NormalTok{, }\DecValTok{1}\NormalTok{))}

\NormalTok{name }\OtherTok{\textless{}{-}} \StringTok{"Probability of ART Conception"}
\NormalTok{util}\SpecialCharTok{$}\FunctionTok{plot\_expectand\_pushforward}\NormalTok{(samples4a[[}\StringTok{\textquotesingle{}q\_AC\textquotesingle{}}\NormalTok{]], }\DecValTok{25}\NormalTok{,}
                                \AttributeTok{display\_name=}\NormalTok{name)}
\end{Highlighting}
\end{Shaded}

\includegraphics{analysis_files/figure-pdf/unnamed-chunk-72-1.pdf}

That said we can better understand the impact of explicitly modeling ART
by comparing the posterior inferences from this new model to those of
our last model that did not consider ART.

Inferences for the baseline conception probability are similar, but the
inferences for the fertility impairment parameters change substantially.
The previous model had to contort itself to fit the observed data as
well as possible, resulting in misleading inferences and predictions
that will not generalize well to other cohorts.

\begin{Shaded}
\begin{Highlighting}[]
\FunctionTok{par}\NormalTok{(}\AttributeTok{mfrow=}\FunctionTok{c}\NormalTok{(}\DecValTok{2}\NormalTok{, }\DecValTok{2}\NormalTok{), }\AttributeTok{mar=}\FunctionTok{c}\NormalTok{(}\DecValTok{5}\NormalTok{, }\DecValTok{5}\NormalTok{, }\DecValTok{1}\NormalTok{, }\DecValTok{1}\NormalTok{))}

\NormalTok{name }\OtherTok{\textless{}{-}} \StringTok{"Baseline Probability of Conception"}
\NormalTok{util}\SpecialCharTok{$}\FunctionTok{plot\_expectand\_pushforward}\NormalTok{(samples3a[[}\StringTok{\textquotesingle{}q\_C\_0\textquotesingle{}}\NormalTok{]],}
                                \DecValTok{25}\NormalTok{, }\AttributeTok{flim=}\FunctionTok{c}\NormalTok{(}\FloatTok{0.75}\NormalTok{, }\FloatTok{0.85}\NormalTok{),}
                                \AttributeTok{display\_name=}\NormalTok{name,}
                                \AttributeTok{main=}\StringTok{"Not Modeling ART"}\NormalTok{)}

\NormalTok{name }\OtherTok{\textless{}{-}} \StringTok{"Baseline Probability of Natural Conception"}
\NormalTok{util}\SpecialCharTok{$}\FunctionTok{plot\_expectand\_pushforward}\NormalTok{(samples4a[[}\StringTok{\textquotesingle{}q\_NC\_0\textquotesingle{}}\NormalTok{]],}
                                \DecValTok{25}\NormalTok{, }\AttributeTok{flim=}\FunctionTok{c}\NormalTok{(}\FloatTok{0.75}\NormalTok{, }\FloatTok{0.85}\NormalTok{),}
                                \AttributeTok{display\_name=}\NormalTok{name,}
                                \AttributeTok{main=}\StringTok{"Modeling ART"}\NormalTok{)}

\NormalTok{names }\OtherTok{\textless{}{-}} \FunctionTok{sapply}\NormalTok{(}\DecValTok{1}\SpecialCharTok{:}\NormalTok{data}\SpecialCharTok{$}\NormalTok{K\_rel,}
                \ControlFlowTok{function}\NormalTok{(k) }\FunctionTok{paste0}\NormalTok{(}\StringTok{\textquotesingle{}alpha\_rel\_buff[\textquotesingle{}}\NormalTok{, k, }\StringTok{\textquotesingle{}]\textquotesingle{}}\NormalTok{))}
\NormalTok{util}\SpecialCharTok{$}\FunctionTok{plot\_disc\_pushforward\_quantiles}\NormalTok{(samples3a, names,}
                                     \AttributeTok{xlab=}\StringTok{"Observed Relationship Status"}\NormalTok{,}
                                     \AttributeTok{ylab=}\StringTok{"Conception Impairment"}\NormalTok{,}
                                     \AttributeTok{display\_ylim=}\FunctionTok{c}\NormalTok{(}\SpecialCharTok{{-}}\FloatTok{0.05}\NormalTok{, }\FloatTok{2.5}\NormalTok{))}

\NormalTok{names }\OtherTok{\textless{}{-}} \FunctionTok{sapply}\NormalTok{(}\DecValTok{1}\SpecialCharTok{:}\NormalTok{data}\SpecialCharTok{$}\NormalTok{K\_rel,}
                \ControlFlowTok{function}\NormalTok{(k) }\FunctionTok{paste0}\NormalTok{(}\StringTok{\textquotesingle{}alpha\_rel\_buff[\textquotesingle{}}\NormalTok{, k, }\StringTok{\textquotesingle{}]\textquotesingle{}}\NormalTok{))}
\NormalTok{util}\SpecialCharTok{$}\FunctionTok{plot\_disc\_pushforward\_quantiles}\NormalTok{(samples4a, names,}
                                     \AttributeTok{xlab=}\StringTok{"Observed Relationship Status"}\NormalTok{,}
                                     \AttributeTok{ylab=}\StringTok{"Conception Impairment"}\NormalTok{,}
                                     \AttributeTok{display\_ylim=}\FunctionTok{c}\NormalTok{(}\SpecialCharTok{{-}}\FloatTok{0.05}\NormalTok{, }\FloatTok{2.5}\NormalTok{))}
\end{Highlighting}
\end{Shaded}

\includegraphics{analysis_files/figure-pdf/unnamed-chunk-73-1.pdf}

\begin{Shaded}
\begin{Highlighting}[]
\FunctionTok{par}\NormalTok{(}\AttributeTok{mfrow=}\FunctionTok{c}\NormalTok{(}\DecValTok{2}\NormalTok{, }\DecValTok{2}\NormalTok{), }\AttributeTok{mar=}\FunctionTok{c}\NormalTok{(}\DecValTok{5}\NormalTok{, }\DecValTok{5}\NormalTok{, }\DecValTok{1}\NormalTok{, }\DecValTok{1}\NormalTok{))}
\NormalTok{names }\OtherTok{\textless{}{-}} \FunctionTok{sapply}\NormalTok{(}\DecValTok{1}\SpecialCharTok{:}\NormalTok{data}\SpecialCharTok{$}\NormalTok{K\_stg,}
                \ControlFlowTok{function}\NormalTok{(k) }\FunctionTok{paste0}\NormalTok{(}\StringTok{\textquotesingle{}alpha\_stg\_buff[\textquotesingle{}}\NormalTok{, k, }\StringTok{\textquotesingle{}]\textquotesingle{}}\NormalTok{))}
\NormalTok{util}\SpecialCharTok{$}\FunctionTok{plot\_disc\_pushforward\_quantiles}\NormalTok{(samples3a, names,}
                                     \AttributeTok{xlab=}\StringTok{"Observed Cancer Stage"}\NormalTok{,}
                                     \AttributeTok{ylab=}\StringTok{"Conception Impairment"}\NormalTok{,}
                                     \AttributeTok{display\_ylim=}\FunctionTok{c}\NormalTok{(}\SpecialCharTok{{-}}\FloatTok{0.05}\NormalTok{, }\FloatTok{1.5}\NormalTok{),}
                                     \AttributeTok{main=}\StringTok{"Not Modeling ART"}\NormalTok{)}

\NormalTok{names }\OtherTok{\textless{}{-}} \FunctionTok{sapply}\NormalTok{(}\DecValTok{1}\SpecialCharTok{:}\NormalTok{data}\SpecialCharTok{$}\NormalTok{K\_stg,}
                \ControlFlowTok{function}\NormalTok{(k) }\FunctionTok{paste0}\NormalTok{(}\StringTok{\textquotesingle{}alpha\_stg\_buff[\textquotesingle{}}\NormalTok{, k, }\StringTok{\textquotesingle{}]\textquotesingle{}}\NormalTok{))}
\NormalTok{util}\SpecialCharTok{$}\FunctionTok{plot\_disc\_pushforward\_quantiles}\NormalTok{(samples4a, names,}
                                     \AttributeTok{xlab=}\StringTok{"Observed Cancer Stage"}\NormalTok{,}
                                     \AttributeTok{ylab=}\StringTok{"Conception Impairment"}\NormalTok{,}
                                     \AttributeTok{display\_ylim=}\FunctionTok{c}\NormalTok{(}\SpecialCharTok{{-}}\FloatTok{0.05}\NormalTok{, }\FloatTok{1.5}\NormalTok{),}
                                     \AttributeTok{main=}\StringTok{"Modeling ART"}\NormalTok{)}

\NormalTok{names }\OtherTok{\textless{}{-}} \FunctionTok{sapply}\NormalTok{(}\DecValTok{1}\SpecialCharTok{:}\NormalTok{data}\SpecialCharTok{$}\NormalTok{K\_tox,}
                \ControlFlowTok{function}\NormalTok{(k) }\FunctionTok{paste0}\NormalTok{(}\StringTok{\textquotesingle{}alpha\_tox\_buff[\textquotesingle{}}\NormalTok{, k, }\StringTok{\textquotesingle{}]\textquotesingle{}}\NormalTok{))}
\NormalTok{util}\SpecialCharTok{$}\FunctionTok{plot\_disc\_pushforward\_quantiles}\NormalTok{(samples3a, names,}
                                     \AttributeTok{xlab=}\StringTok{"Observed Toxicity Status"}\NormalTok{,}
                                     \AttributeTok{ylab=}\StringTok{"Conception Impairment"}\NormalTok{,}
                                     \AttributeTok{display\_ylim=}\FunctionTok{c}\NormalTok{(}\SpecialCharTok{{-}}\FloatTok{0.05}\NormalTok{, }\FloatTok{2.5}\NormalTok{))}

\NormalTok{names }\OtherTok{\textless{}{-}} \FunctionTok{sapply}\NormalTok{(}\DecValTok{1}\SpecialCharTok{:}\NormalTok{data}\SpecialCharTok{$}\NormalTok{K\_tox,}
                \ControlFlowTok{function}\NormalTok{(k) }\FunctionTok{paste0}\NormalTok{(}\StringTok{\textquotesingle{}alpha\_tox\_buff[\textquotesingle{}}\NormalTok{, k, }\StringTok{\textquotesingle{}]\textquotesingle{}}\NormalTok{))}
\NormalTok{util}\SpecialCharTok{$}\FunctionTok{plot\_disc\_pushforward\_quantiles}\NormalTok{(samples4a, names,}
                                     \AttributeTok{xlab=}\StringTok{"Observed Toxicity Status"}\NormalTok{,}
                                     \AttributeTok{ylab=}\StringTok{"Conception Impairment"}\NormalTok{,}
                                     \AttributeTok{display\_ylim=}\FunctionTok{c}\NormalTok{(}\SpecialCharTok{{-}}\FloatTok{0.05}\NormalTok{, }\FloatTok{2.5}\NormalTok{))}
\end{Highlighting}
\end{Shaded}

\includegraphics{analysis_files/figure-pdf/unnamed-chunk-74-1.pdf}

\subsection{Model 4b}\label{model-4b}

With a model that can distinguish between natural and ART conceptions we
can make much more accurate inferences for the hypothetical treatment
that we considered above. Here we'll assume that the main epidemiology
focus is the difference in only natural conceptions. In this case the
ART conceptions in the observed data are something of a contamination
that we need to model in order to isolate the desired behavior.

\begin{codelisting}

\caption{\texttt{model4b.stan}}

\begin{Shaded}
\begin{Highlighting}[]
\KeywordTok{data}\NormalTok{ \{}
  \CommentTok{// Number of observations}
  \DataTypeTok{int}\NormalTok{\textless{}}\KeywordTok{lower}\NormalTok{=}\DecValTok{1}\NormalTok{\textgreater{} N;}

  \CommentTok{// Number of predictions}
  \DataTypeTok{int}\NormalTok{\textless{}}\KeywordTok{lower}\NormalTok{=}\DecValTok{1}\NormalTok{\textgreater{} N\_pred;}

  \CommentTok{// Relationship status}
  \CommentTok{// k = 1: Stable partner}
  \CommentTok{// k = 2: No partner}
  \DataTypeTok{int}\NormalTok{\textless{}}\KeywordTok{lower}\NormalTok{=}\DecValTok{1}\NormalTok{\textgreater{} K\_rel;}

  \CommentTok{// Cancer stage}
  \CommentTok{// k = 1: No cancer}
  \CommentTok{// k = 2: Early stage cancer}
  \CommentTok{// k = 3: Advanced stage cancer}
  \DataTypeTok{int}\NormalTok{\textless{}}\KeywordTok{lower}\NormalTok{=}\DecValTok{1}\NormalTok{\textgreater{} K\_stg;}

  \CommentTok{// Treatment status}
  \CommentTok{// k = 1: No treatment}
  \CommentTok{// k = 2: Treatment}
  \DataTypeTok{int}\NormalTok{\textless{}}\KeywordTok{lower}\NormalTok{=}\DecValTok{1}\NormalTok{\textgreater{} K\_trt;}

  \CommentTok{// Toxicity status}
  \CommentTok{// k = 1: None}
  \CommentTok{// k = 2: Low}
  \CommentTok{// k = 3: Medium}
  \CommentTok{// k = 4: High}
  \DataTypeTok{int}\NormalTok{\textless{}}\KeywordTok{lower}\NormalTok{=}\DecValTok{1}\NormalTok{\textgreater{} K\_tox;}

  \CommentTok{// Observed conception status}
  \CommentTok{// y = 0: No conception}
  \CommentTok{// y = 1: Conception}
  \DataTypeTok{array}\NormalTok{[N] }\DataTypeTok{int}\NormalTok{\textless{}}\KeywordTok{lower}\NormalTok{=}\DecValTok{0}\NormalTok{, }\KeywordTok{upper}\NormalTok{=}\DecValTok{1}\NormalTok{\textgreater{} y;}

  \CommentTok{// Observed relationship status;}
  \DataTypeTok{array}\NormalTok{[N] }\DataTypeTok{int}\NormalTok{\textless{}}\KeywordTok{lower}\NormalTok{=}\DecValTok{1}\NormalTok{, }\KeywordTok{upper}\NormalTok{=K\_rel\textgreater{} k\_rel;}

  \CommentTok{// Observed cancer stage;}
  \DataTypeTok{array}\NormalTok{[N] }\DataTypeTok{int}\NormalTok{\textless{}}\KeywordTok{lower}\NormalTok{=}\DecValTok{1}\NormalTok{, }\KeywordTok{upper}\NormalTok{=K\_stg\textgreater{} k\_stg;}

  \CommentTok{// Observed treatment status;}
  \DataTypeTok{array}\NormalTok{[N] }\DataTypeTok{int}\NormalTok{\textless{}}\KeywordTok{lower}\NormalTok{=}\DecValTok{1}\NormalTok{, }\KeywordTok{upper}\NormalTok{=K\_trt\textgreater{} k\_trt;}

  \CommentTok{// Observed toxicity status;}
  \DataTypeTok{array}\NormalTok{[N] }\DataTypeTok{int}\NormalTok{\textless{}}\KeywordTok{lower}\NormalTok{=}\DecValTok{1}\NormalTok{, }\KeywordTok{upper}\NormalTok{=K\_tox\textgreater{} k\_tox;}

  \CommentTok{// Observed assistive reproductive technology (ART) status}
  \CommentTok{// k = 0: No ART}
  \CommentTok{// k = 1: ART}
  \DataTypeTok{array}\NormalTok{[N] }\DataTypeTok{int}\NormalTok{\textless{}}\KeywordTok{lower}\NormalTok{=}\DecValTok{0}\NormalTok{, }\KeywordTok{upper}\NormalTok{=}\DecValTok{1}\NormalTok{\textgreater{} k\_art;}

  \CommentTok{// Hypothetical toxicity distribution configurations}
  \DataTypeTok{vector}\NormalTok{[K\_tox] alpha\_tox\_hyp1;}
  \DataTypeTok{vector}\NormalTok{[K\_tox] alpha\_tox\_hyp2;}
\NormalTok{\}}

\KeywordTok{parameters}\NormalTok{ \{}
  \CommentTok{// Marginal probability of cancer stage}
  \DataTypeTok{simplex}\NormalTok{[K\_stg] q\_stg;}

  \CommentTok{// Conditional probability of relationship status given cancer stage}
  \DataTypeTok{array}\NormalTok{[K\_stg] }\DataTypeTok{simplex}\NormalTok{[K\_rel] q\_rel;}

  \CommentTok{// Conditional probability of treatment status given active cancer stage}
  \DataTypeTok{array}\NormalTok{[K\_stg {-} }\DecValTok{1}\NormalTok{] }\DataTypeTok{simplex}\NormalTok{[K\_trt] q\_trt\_active\_stg;}

  \CommentTok{// Conditional probability of toxicity status given cancer stage and}
  \CommentTok{// active treatment status}
  \DataTypeTok{array}\NormalTok{[K\_stg] }\DataTypeTok{simplex}\NormalTok{[K\_tox] q\_tox\_active\_trt;}

  \CommentTok{// Probability of natural conception for baseline patients in a stable}
  \CommentTok{// relationship, no cancer, and no toxicity}
  \DataTypeTok{real}\NormalTok{\textless{}}\KeywordTok{lower}\NormalTok{=}\DecValTok{0}\NormalTok{, }\KeywordTok{upper}\NormalTok{=}\DecValTok{1}\NormalTok{\textgreater{} q\_NC\_0;}

  \CommentTok{// Proportional decreases in conception probability due to}
  \CommentTok{// non{-}baseline relationship status, cancer stage, and toxicity}
  \CommentTok{// status.}
  \DataTypeTok{positive\_ordered}\NormalTok{[K\_rel {-} }\DecValTok{1}\NormalTok{] alpha\_rel;}
  \DataTypeTok{positive\_ordered}\NormalTok{[K\_stg {-} }\DecValTok{1}\NormalTok{] alpha\_stg;}
  \DataTypeTok{positive\_ordered}\NormalTok{[K\_tox {-} }\DecValTok{1}\NormalTok{] alpha\_tox;}

  \CommentTok{// Probability of ART conception}
  \DataTypeTok{real}\NormalTok{\textless{}}\KeywordTok{lower}\NormalTok{=}\DecValTok{0}\NormalTok{, }\KeywordTok{upper}\NormalTok{=}\DecValTok{1}\NormalTok{\textgreater{} q\_AC;}
\NormalTok{\}}

\KeywordTok{transformed parameters}\NormalTok{ \{}
  \DataTypeTok{vector}\NormalTok{[K\_rel] alpha\_rel\_buff = append\_row([}\DecValTok{0}\NormalTok{]\textquotesingle{}, alpha\_rel);}
  \DataTypeTok{vector}\NormalTok{[K\_stg] alpha\_stg\_buff = append\_row([}\DecValTok{0}\NormalTok{]\textquotesingle{}, alpha\_stg);}
  \DataTypeTok{vector}\NormalTok{[K\_tox] alpha\_tox\_buff = append\_row([}\DecValTok{0}\NormalTok{]\textquotesingle{}, alpha\_tox);}

  \CommentTok{// Conditional probability of treatment status given cancer stage}
  \DataTypeTok{array}\NormalTok{[K\_stg] }\DataTypeTok{simplex}\NormalTok{[K\_trt] q\_trt = append\_array(\{ [}\FloatTok{1.0}\NormalTok{, }\FloatTok{0.0}\NormalTok{]\textquotesingle{} \},}
\NormalTok{                                                   q\_trt\_active\_stg);}

  \CommentTok{// Conditional probability of toxicity status given cancer stage and}
  \CommentTok{// treatment status}
  \DataTypeTok{array}\NormalTok{[K\_stg, K\_trt] }\DataTypeTok{simplex}\NormalTok{[K\_tox] q\_tox;}
\NormalTok{  q\_tox[, }\DecValTok{1}\NormalTok{] = \{ [}\DecValTok{1}\NormalTok{, }\DecValTok{0}\NormalTok{, }\DecValTok{0}\NormalTok{, }\DecValTok{0}\NormalTok{]\textquotesingle{},}
\NormalTok{                 [}\DecValTok{1}\NormalTok{, }\DecValTok{0}\NormalTok{, }\DecValTok{0}\NormalTok{, }\DecValTok{0}\NormalTok{]\textquotesingle{},}
\NormalTok{                 [}\DecValTok{1}\NormalTok{, }\DecValTok{0}\NormalTok{, }\DecValTok{0}\NormalTok{, }\DecValTok{0}\NormalTok{]\textquotesingle{} \};}
\NormalTok{  q\_tox[, }\DecValTok{2}\NormalTok{] = q\_tox\_active\_trt;}
\NormalTok{\}}

\KeywordTok{model}\NormalTok{ \{}
  \CommentTok{// Prior model}
  \KeywordTok{target +=}\NormalTok{ dirichlet\_lpdf(q\_stg | [}\DecValTok{4}\NormalTok{, }\DecValTok{3}\NormalTok{, }\DecValTok{1}\NormalTok{]\textquotesingle{});}
  \KeywordTok{target +=}\NormalTok{ dirichlet\_lpdf(q\_rel[}\DecValTok{1}\NormalTok{] | [}\DecValTok{4}\NormalTok{, }\DecValTok{1}\NormalTok{]\textquotesingle{});}
  \KeywordTok{target +=}\NormalTok{ dirichlet\_lpdf(q\_rel[}\DecValTok{2}\NormalTok{] | [}\DecValTok{2}\NormalTok{, }\DecValTok{1}\NormalTok{]\textquotesingle{});}
  \KeywordTok{target +=}\NormalTok{ dirichlet\_lpdf(q\_rel[}\DecValTok{3}\NormalTok{] | [}\DecValTok{1}\NormalTok{, }\DecValTok{1}\NormalTok{]\textquotesingle{});}
  \KeywordTok{target +=}\NormalTok{ dirichlet\_lpdf(q\_trt\_active\_stg[}\DecValTok{1}\NormalTok{] | [}\DecValTok{1}\NormalTok{, }\DecValTok{3}\NormalTok{]\textquotesingle{});}
  \KeywordTok{target +=}\NormalTok{ dirichlet\_lpdf(q\_trt\_active\_stg[}\DecValTok{2}\NormalTok{] | [}\FloatTok{0.5}\NormalTok{, }\DecValTok{4}\NormalTok{]\textquotesingle{});}
  \KeywordTok{target +=}\NormalTok{ dirichlet\_lpdf(q\_tox\_active\_trt[}\DecValTok{1}\NormalTok{] | [}\DecValTok{4}\NormalTok{, }\DecValTok{3}\NormalTok{, }\DecValTok{2}\NormalTok{, }\DecValTok{1}\NormalTok{]\textquotesingle{});}
  \KeywordTok{target +=}\NormalTok{ dirichlet\_lpdf(q\_tox\_active\_trt[}\DecValTok{2}\NormalTok{] | [}\DecValTok{2}\NormalTok{, }\DecValTok{3}\NormalTok{, }\DecValTok{3}\NormalTok{, }\DecValTok{1}\NormalTok{]\textquotesingle{});}
  \KeywordTok{target +=}\NormalTok{ dirichlet\_lpdf(q\_tox\_active\_trt[}\DecValTok{3}\NormalTok{] | [}\DecValTok{1}\NormalTok{, }\DecValTok{2}\NormalTok{, }\DecValTok{3}\NormalTok{, }\DecValTok{3}\NormalTok{]\textquotesingle{});}

  \KeywordTok{target +=}\NormalTok{ beta\_lpdf(q\_NC\_0 | }\FloatTok{12.7}\NormalTok{, }\FloatTok{3.7}\NormalTok{); }\CommentTok{// 0.50 \textless{}\textasciitilde{} q\_NC\_0 \textless{}\textasciitilde{} 0.95}

  \KeywordTok{target +=}\NormalTok{ normal\_lpdf(alpha\_rel | }\DecValTok{0}\NormalTok{, }\DecValTok{3}\NormalTok{ / }\FloatTok{2.32}\NormalTok{); }\CommentTok{// 0 \textless{}\textasciitilde{} alpha \textless{}\textasciitilde{} {-}log(0.05)}
  \KeywordTok{target +=}\NormalTok{ normal\_lpdf(alpha\_stg | }\DecValTok{0}\NormalTok{, }\DecValTok{3}\NormalTok{ / }\FloatTok{2.32}\NormalTok{); }\CommentTok{// 0 \textless{}\textasciitilde{} alpha \textless{}\textasciitilde{} {-}log(0.05)}
  \KeywordTok{target +=}\NormalTok{ normal\_lpdf(alpha\_tox | }\DecValTok{0}\NormalTok{, }\DecValTok{3}\NormalTok{ / }\FloatTok{2.32}\NormalTok{); }\CommentTok{// 0 \textless{}\textasciitilde{} alpha \textless{}\textasciitilde{} {-}log(0.05)}

  \KeywordTok{target +=}\NormalTok{ beta\_lpdf(q\_AC | }\FloatTok{18.2}\NormalTok{, }\FloatTok{11.6}\NormalTok{); }\CommentTok{// 0.4 \textless{}\textasciitilde{} q\_AC\_0 \textless{}\textasciitilde{} 0.8}

  \CommentTok{// Observational model}
  \ControlFlowTok{for}\NormalTok{ (n }\ControlFlowTok{in} \DecValTok{1}\NormalTok{:N) \{}
    \DataTypeTok{real}\NormalTok{ q\_NC = q\_NC\_0 * exp({-}alpha\_rel\_buff[k\_rel[n]]}
\NormalTok{                             {-}alpha\_stg\_buff[k\_stg[n]]}
\NormalTok{                             {-}alpha\_tox\_buff[k\_tox[n]]);}

    \KeywordTok{target +=}\NormalTok{ categorical\_lpmf(k\_stg[n] | q\_stg);}
    \KeywordTok{target +=}\NormalTok{ categorical\_lpmf(k\_rel[n] | q\_rel[k\_stg[n]]);}
    \KeywordTok{target +=}\NormalTok{ categorical\_lpmf(k\_trt[n] | q\_trt[k\_stg[n]]);}
    \KeywordTok{target +=}\NormalTok{ categorical\_lpmf(k\_tox[n] | q\_tox[k\_stg[n],}
\NormalTok{                                                k\_trt[n]]);}

    \ControlFlowTok{if}\NormalTok{ (k\_art[n]) \{}
      \KeywordTok{target +=}\NormalTok{ bernoulli\_lpmf(y[n] | q\_AC * (}\DecValTok{1}\NormalTok{ {-} q\_NC) + q\_NC);}
\NormalTok{    \} }\ControlFlowTok{else}\NormalTok{ \{}
      \KeywordTok{target +=}\NormalTok{ bernoulli\_lpmf(y[n] | q\_NC);}
\NormalTok{    \}}
\NormalTok{  \}}
\NormalTok{\}}

\KeywordTok{generated quantities}\NormalTok{ \{}
  \CommentTok{// Posterior predictive data}
  \DataTypeTok{array}\NormalTok{[N\_pred] }\DataTypeTok{real}\NormalTok{\textless{}}\KeywordTok{lower}\NormalTok{=}\DecValTok{0}\NormalTok{, }\KeywordTok{upper}\NormalTok{=}\DecValTok{1}\NormalTok{\textgreater{} q\_pred;}
  \DataTypeTok{array}\NormalTok{[N\_pred] }\DataTypeTok{real}\NormalTok{\textless{}}\KeywordTok{lower}\NormalTok{=}\DecValTok{0}\NormalTok{, }\KeywordTok{upper}\NormalTok{=}\DecValTok{1}\NormalTok{\textgreater{} q\_hyp\_pred;}

  \ControlFlowTok{for}\NormalTok{ (n }\ControlFlowTok{in} \DecValTok{1}\NormalTok{:N\_pred) \{}
    \DataTypeTok{int}\NormalTok{ k\_stg\_pred = categorical\_rng(q\_stg);}
    \DataTypeTok{int}\NormalTok{ k\_rel\_pred = categorical\_rng(q\_rel[k\_stg\_pred]);}
    \DataTypeTok{int}\NormalTok{ k\_trt\_pred = categorical\_rng(q\_trt[k\_stg\_pred]);}
    \DataTypeTok{int}\NormalTok{ k\_tox\_pred = categorical\_rng(q\_tox[k\_stg\_pred,}
\NormalTok{                                           k\_trt\_pred]);}

\NormalTok{    q\_pred[n] = q\_NC\_0 * exp({-}alpha\_rel\_buff[k\_rel\_pred]}
\NormalTok{                             {-}alpha\_stg\_buff[k\_stg\_pred]}
\NormalTok{                             {-}alpha\_tox\_buff[k\_tox\_pred]);}
\NormalTok{  \}}

\NormalTok{  \{}
    \DataTypeTok{array}\NormalTok{[K\_stg, K\_trt] }\DataTypeTok{vector}\NormalTok{[K\_tox] q\_tox\_hyp;}
\NormalTok{    q\_tox\_hyp[, }\DecValTok{1}\NormalTok{] = q\_tox[, }\DecValTok{1}\NormalTok{];}
\NormalTok{    q\_tox\_hyp[, }\DecValTok{2}\NormalTok{] = \{ [}\DecValTok{1}\NormalTok{, }\DecValTok{0}\NormalTok{, }\DecValTok{0}\NormalTok{, }\DecValTok{0}\NormalTok{]\textquotesingle{},}
\NormalTok{                       dirichlet\_rng(alpha\_tox\_hyp1),}
\NormalTok{                       dirichlet\_rng(alpha\_tox\_hyp2) \};}

    \ControlFlowTok{for}\NormalTok{ (n }\ControlFlowTok{in} \DecValTok{1}\NormalTok{:N\_pred) \{}
      \DataTypeTok{int}\NormalTok{ k\_stg\_pred = categorical\_rng(q\_stg);}
      \DataTypeTok{int}\NormalTok{ k\_rel\_pred = categorical\_rng(q\_rel[k\_stg\_pred]);}
      \DataTypeTok{int}\NormalTok{ k\_trt\_pred = categorical\_rng(q\_trt[k\_stg\_pred]);}
      \DataTypeTok{int}\NormalTok{ k\_tox\_pred = categorical\_rng(q\_tox\_hyp[k\_stg\_pred,}
\NormalTok{                                                 k\_trt\_pred]);}

\NormalTok{      q\_hyp\_pred[n] = q\_NC\_0 * exp({-}alpha\_rel\_buff[k\_rel\_pred]}
\NormalTok{                                   {-}alpha\_stg\_buff[k\_stg\_pred]}
\NormalTok{                                   {-}alpha\_tox\_buff[k\_tox\_pred]);}
\NormalTok{    \}}
\NormalTok{  \}}
\NormalTok{\}}
\end{Highlighting}
\end{Shaded}

\end{codelisting}

\begin{Shaded}
\begin{Highlighting}[]
\NormalTok{fit }\OtherTok{\textless{}{-}} \FunctionTok{stan}\NormalTok{(}\AttributeTok{file=}\StringTok{"stan\_programs/model4b.stan"}\NormalTok{,}
            \AttributeTok{data=}\NormalTok{data, }\AttributeTok{seed=}\DecValTok{8438338}\NormalTok{,}
            \AttributeTok{warmup=}\DecValTok{1000}\NormalTok{, }\AttributeTok{iter=}\DecValTok{2024}\NormalTok{, }\AttributeTok{refresh=}\DecValTok{0}\NormalTok{)}
\end{Highlighting}
\end{Shaded}

One last check of the computational diagnostics; one more breath of
appreciation for the robustness and scalability of Hamiltonian Monte
Carlo.

\begin{Shaded}
\begin{Highlighting}[]
\NormalTok{diagnostics }\OtherTok{\textless{}{-}}\NormalTok{ util}\SpecialCharTok{$}\FunctionTok{extract\_hmc\_diagnostics}\NormalTok{(fit)}
\NormalTok{util}\SpecialCharTok{$}\FunctionTok{check\_all\_hmc\_diagnostics}\NormalTok{(diagnostics)}
\end{Highlighting}
\end{Shaded}

\begin{verbatim}
  All Hamiltonian Monte Carlo diagnostics are consistent with reliable
Markov chain Monte Carlo.
\end{verbatim}

\begin{Shaded}
\begin{Highlighting}[]
\NormalTok{samples4b }\OtherTok{\textless{}{-}}\NormalTok{ util}\SpecialCharTok{$}\FunctionTok{extract\_expectand\_vals}\NormalTok{(fit)}
\NormalTok{base\_samples }\OtherTok{\textless{}{-}}\NormalTok{ util}\SpecialCharTok{$}\FunctionTok{filter\_expectands}\NormalTok{(samples4b,}
                                       \FunctionTok{c}\NormalTok{(}\StringTok{\textquotesingle{}q\_stg\textquotesingle{}}\NormalTok{, }\StringTok{\textquotesingle{}q\_trt\_active\_stg\textquotesingle{}}\NormalTok{,}
                                         \StringTok{\textquotesingle{}q\_tox\_active\_trt\textquotesingle{}}\NormalTok{,}\StringTok{\textquotesingle{}q\_NC\_0\textquotesingle{}}\NormalTok{,}
                                         \StringTok{\textquotesingle{}alpha\_rel\textquotesingle{}}\NormalTok{, }\StringTok{\textquotesingle{}alpha\_stg\textquotesingle{}}\NormalTok{,}
                                         \StringTok{\textquotesingle{}alpha\_tox\textquotesingle{}}\NormalTok{, }\StringTok{\textquotesingle{}q\_AC\textquotesingle{}}\NormalTok{),}
                                       \AttributeTok{check\_arrays=}\ConstantTok{TRUE}\NormalTok{)}
\NormalTok{util}\SpecialCharTok{$}\FunctionTok{check\_all\_expectand\_diagnostics}\NormalTok{(base\_samples)}
\end{Highlighting}
\end{Shaded}

\begin{verbatim}
All expectands checked appear to be behaving well enough for reliable
Markov chain Monte Carlo estimation.
\end{verbatim}

What do our more accurate inferences now have to say about the behavior
of the hypothetical cohort?

The histogram of conception probabilities exhibits similar behavior to
before, with the baseline peak mostly the same across the two cohorts
but the remaining bulk shifting slightly towards larger conception
probabilities in the hypothetical treatment.

\begin{Shaded}
\begin{Highlighting}[]
\FunctionTok{par}\NormalTok{(}\AttributeTok{mfrow=}\FunctionTok{c}\NormalTok{(}\DecValTok{1}\NormalTok{, }\DecValTok{2}\NormalTok{), }\AttributeTok{mar=}\FunctionTok{c}\NormalTok{(}\DecValTok{5}\NormalTok{, }\DecValTok{5}\NormalTok{, }\DecValTok{1}\NormalTok{, }\DecValTok{1}\NormalTok{))}

\NormalTok{util}\SpecialCharTok{$}\FunctionTok{plot\_hist\_quantiles}\NormalTok{(samples4b, }\StringTok{\textquotesingle{}q\_pred\textquotesingle{}}\NormalTok{, }\SpecialCharTok{{-}}\FloatTok{0.05}\NormalTok{, }\FloatTok{1.05}\NormalTok{, }\FloatTok{0.1}\NormalTok{,}
                         \AttributeTok{xlab=}\StringTok{"Natural Conception Probability"}\NormalTok{,}
                         \AttributeTok{main=}\StringTok{"Observed Cohort"}\NormalTok{)}

\NormalTok{util}\SpecialCharTok{$}\FunctionTok{plot\_hist\_quantiles}\NormalTok{(samples4b, }\StringTok{\textquotesingle{}q\_hyp\_pred\textquotesingle{}}\NormalTok{, }\SpecialCharTok{{-}}\FloatTok{0.05}\NormalTok{, }\FloatTok{1.05}\NormalTok{, }\FloatTok{0.1}\NormalTok{,}
                         \AttributeTok{xlab=}\StringTok{"Natural Conception Probability"}\NormalTok{,}
                         \AttributeTok{main=}\StringTok{"Hypothetical Cohort"}\NormalTok{)}
\end{Highlighting}
\end{Shaded}

\includegraphics{analysis_files/figure-pdf/unnamed-chunk-77-1.pdf}

With the ART conceptions no longer biasing our inferences the
hypothetical treatment appears to have similar benefits for both the
ensemble average conception probability and the ensemble lower quantile
conception probability.

\begin{Shaded}
\begin{Highlighting}[]
\NormalTok{var\_repl }\OtherTok{\textless{}{-}} \FunctionTok{list}\NormalTok{(}\StringTok{\textquotesingle{}p\textquotesingle{}}\OtherTok{=}\NormalTok{util}\SpecialCharTok{$}\FunctionTok{name\_array}\NormalTok{(}\StringTok{\textquotesingle{}q\_pred\textquotesingle{}}\NormalTok{, }\FunctionTok{c}\NormalTok{(data}\SpecialCharTok{$}\NormalTok{N\_pred)))}
\NormalTok{pop\_ave\_samples }\OtherTok{\textless{}{-}}
\NormalTok{  util}\SpecialCharTok{$}\FunctionTok{eval\_expectand\_pushforward}\NormalTok{(samples4b,}
                                  \ControlFlowTok{function}\NormalTok{(p) }\FunctionTok{mean}\NormalTok{(p),}
\NormalTok{                                  var\_repl)}

\NormalTok{var\_repl }\OtherTok{\textless{}{-}} \FunctionTok{list}\NormalTok{(}\StringTok{\textquotesingle{}p\textquotesingle{}}\OtherTok{=}\NormalTok{util}\SpecialCharTok{$}\FunctionTok{name\_array}\NormalTok{(}\StringTok{\textquotesingle{}q\_hyp\_pred\textquotesingle{}}\NormalTok{, }\FunctionTok{c}\NormalTok{(data}\SpecialCharTok{$}\NormalTok{N\_pred)))}
\NormalTok{hyp\_pop\_ave\_samples }\OtherTok{\textless{}{-}}
\NormalTok{  util}\SpecialCharTok{$}\FunctionTok{eval\_expectand\_pushforward}\NormalTok{(samples4b,}
                                  \ControlFlowTok{function}\NormalTok{(p) }\FunctionTok{mean}\NormalTok{(p),}
\NormalTok{                                  var\_repl)}
\end{Highlighting}
\end{Shaded}

\begin{Shaded}
\begin{Highlighting}[]
\FunctionTok{par}\NormalTok{(}\AttributeTok{mfrow=}\FunctionTok{c}\NormalTok{(}\DecValTok{1}\NormalTok{, }\DecValTok{1}\NormalTok{), }\AttributeTok{mar=}\FunctionTok{c}\NormalTok{(}\DecValTok{5}\NormalTok{, }\DecValTok{5}\NormalTok{, }\DecValTok{1}\NormalTok{, }\DecValTok{1}\NormalTok{))}

\NormalTok{name }\OtherTok{\textless{}{-}} \StringTok{"Natural Conception Probability}\SpecialCharTok{\textbackslash{}n}\StringTok{Population Average"}
\NormalTok{util}\SpecialCharTok{$}\FunctionTok{plot\_expectand\_pushforward}\NormalTok{(pop\_ave\_samples,}
                                \DecValTok{25}\NormalTok{, }\AttributeTok{flim=}\FunctionTok{c}\NormalTok{(}\FloatTok{0.45}\NormalTok{, }\FloatTok{0.65}\NormalTok{),}
                                \AttributeTok{display\_name=}\NormalTok{name,}
                                \AttributeTok{col=}\NormalTok{util}\SpecialCharTok{$}\NormalTok{c\_light)}
\FunctionTok{text}\NormalTok{(}\FloatTok{0.500}\NormalTok{, }\DecValTok{20}\NormalTok{, }\StringTok{"Observed}\SpecialCharTok{\textbackslash{}n}\StringTok{Cohort"}\NormalTok{, }\AttributeTok{col=}\NormalTok{util}\SpecialCharTok{$}\NormalTok{c\_light)}

\NormalTok{util}\SpecialCharTok{$}\FunctionTok{plot\_expectand\_pushforward}\NormalTok{(hyp\_pop\_ave\_samples,}
                                \DecValTok{25}\NormalTok{, }\AttributeTok{flim=}\FunctionTok{c}\NormalTok{(}\FloatTok{0.45}\NormalTok{, }\FloatTok{0.65}\NormalTok{),}
                                \AttributeTok{border=}\StringTok{"\#BBBBBB88"}\NormalTok{,}
                                \AttributeTok{add=}\ConstantTok{TRUE}\NormalTok{)}
\FunctionTok{text}\NormalTok{(}\FloatTok{0.605}\NormalTok{, }\DecValTok{15}\NormalTok{, }\StringTok{"Hypothetical}\SpecialCharTok{\textbackslash{}n}\StringTok{Cohort"}\NormalTok{, }\AttributeTok{col=}\NormalTok{util}\SpecialCharTok{$}\NormalTok{c\_dark)}
\end{Highlighting}
\end{Shaded}

\includegraphics{analysis_files/figure-pdf/unnamed-chunk-79-1.pdf}

\begin{Shaded}
\begin{Highlighting}[]
\NormalTok{ave\_samples }\OtherTok{\textless{}{-}} \FunctionTok{list}\NormalTok{(}\StringTok{"obs"} \OtherTok{=}\NormalTok{ pop\_ave\_samples,}
                    \StringTok{"hyp"} \OtherTok{=}\NormalTok{ hyp\_pop\_ave\_samples)}
\NormalTok{p\_est }\OtherTok{\textless{}{-}}
\NormalTok{  util}\SpecialCharTok{$}\FunctionTok{implicit\_subset\_prob}\NormalTok{(ave\_samples,}
                            \ControlFlowTok{function}\NormalTok{(obs, hyp) hyp }\SpecialCharTok{\textgreater{}}\NormalTok{ obs)}

\NormalTok{format\_string }\OtherTok{\textless{}{-}} \FunctionTok{paste}\NormalTok{(}\StringTok{"Posterior probability that hypothetical"}\NormalTok{,}
                       \StringTok{"population average}\SpecialCharTok{\textbackslash{}n}\StringTok{is greater than observed"}\NormalTok{,}
                       \StringTok{"population average = \%.3f +/{-} \%.3f."}\NormalTok{)}
\FunctionTok{cat}\NormalTok{(}\FunctionTok{sprintf}\NormalTok{(format\_string, p\_est[}\DecValTok{1}\NormalTok{], }\DecValTok{2} \SpecialCharTok{*}\NormalTok{ p\_est[}\DecValTok{2}\NormalTok{]))}
\end{Highlighting}
\end{Shaded}

\begin{verbatim}
Posterior probability that hypothetical population average
is greater than observed population average = 0.779 +/- 0.013.
\end{verbatim}

\begin{Shaded}
\begin{Highlighting}[]
\NormalTok{var\_repl }\OtherTok{\textless{}{-}} \FunctionTok{list}\NormalTok{(}\StringTok{\textquotesingle{}p\textquotesingle{}}\OtherTok{=}\NormalTok{util}\SpecialCharTok{$}\FunctionTok{name\_array}\NormalTok{(}\StringTok{\textquotesingle{}q\_pred\textquotesingle{}}\NormalTok{, }\FunctionTok{c}\NormalTok{(data}\SpecialCharTok{$}\NormalTok{N\_pred)))}
\NormalTok{pop\_quant\_samples }\OtherTok{\textless{}{-}}
\NormalTok{  util}\SpecialCharTok{$}\FunctionTok{eval\_expectand\_pushforward}\NormalTok{(samples4b,}
                                  \ControlFlowTok{function}\NormalTok{(p) }\FunctionTok{quantile}\NormalTok{(p, }\AttributeTok{prob=}\FunctionTok{c}\NormalTok{(}\FloatTok{0.25}\NormalTok{)),}
\NormalTok{                                  var\_repl)}

\NormalTok{var\_repl }\OtherTok{\textless{}{-}} \FunctionTok{list}\NormalTok{(}\StringTok{\textquotesingle{}p\textquotesingle{}}\OtherTok{=}\NormalTok{util}\SpecialCharTok{$}\FunctionTok{name\_array}\NormalTok{(}\StringTok{\textquotesingle{}q\_hyp\_pred\textquotesingle{}}\NormalTok{, }\FunctionTok{c}\NormalTok{(data}\SpecialCharTok{$}\NormalTok{N\_pred)))}
\NormalTok{hyp\_pop\_quant\_samples }\OtherTok{\textless{}{-}}
\NormalTok{  util}\SpecialCharTok{$}\FunctionTok{eval\_expectand\_pushforward}\NormalTok{(samples4b,}
                                  \ControlFlowTok{function}\NormalTok{(p) }\FunctionTok{quantile}\NormalTok{(p, }\AttributeTok{prob=}\FunctionTok{c}\NormalTok{(}\FloatTok{0.25}\NormalTok{)),}
\NormalTok{                                  var\_repl)}
\end{Highlighting}
\end{Shaded}

\begin{Shaded}
\begin{Highlighting}[]
\FunctionTok{par}\NormalTok{(}\AttributeTok{mfrow=}\FunctionTok{c}\NormalTok{(}\DecValTok{1}\NormalTok{, }\DecValTok{1}\NormalTok{), }\AttributeTok{mar=}\FunctionTok{c}\NormalTok{(}\DecValTok{5}\NormalTok{, }\DecValTok{5}\NormalTok{, }\DecValTok{1}\NormalTok{, }\DecValTok{1}\NormalTok{))}

\NormalTok{name }\OtherTok{\textless{}{-}} \StringTok{"Natural Conception Probability}\SpecialCharTok{\textbackslash{}n}\StringTok{Population Lower Quantile"}
\NormalTok{util}\SpecialCharTok{$}\FunctionTok{plot\_expectand\_pushforward}\NormalTok{(pop\_quant\_samples,}
                                \DecValTok{25}\NormalTok{, }\AttributeTok{flim=}\FunctionTok{c}\NormalTok{(}\FloatTok{0.0}\NormalTok{, }\FloatTok{0.7}\NormalTok{),}
                                \AttributeTok{display\_name=}\NormalTok{name,}
                                \AttributeTok{col=}\NormalTok{util}\SpecialCharTok{$}\NormalTok{c\_light)}
\FunctionTok{text}\NormalTok{(}\FloatTok{0.2}\NormalTok{, }\DecValTok{10}\NormalTok{, }\StringTok{"Observed}\SpecialCharTok{\textbackslash{}n}\StringTok{Cohort"}\NormalTok{, }\AttributeTok{col=}\NormalTok{util}\SpecialCharTok{$}\NormalTok{c\_light)}

\NormalTok{util}\SpecialCharTok{$}\FunctionTok{plot\_expectand\_pushforward}\NormalTok{(hyp\_pop\_quant\_samples,}
                                \DecValTok{25}\NormalTok{, }\AttributeTok{flim=}\FunctionTok{c}\NormalTok{(}\FloatTok{0.0}\NormalTok{, }\FloatTok{0.7}\NormalTok{),}
                                \AttributeTok{border=}\StringTok{"\#BBBBBB88"}\NormalTok{,}
                                \AttributeTok{add=}\ConstantTok{TRUE}\NormalTok{)}
\FunctionTok{text}\NormalTok{(}\FloatTok{0.4}\NormalTok{, }\DecValTok{5}\NormalTok{, }\StringTok{"Hypothetical}\SpecialCharTok{\textbackslash{}n}\StringTok{Cohort"}\NormalTok{, }\AttributeTok{col=}\NormalTok{util}\SpecialCharTok{$}\NormalTok{c\_dark)}
\end{Highlighting}
\end{Shaded}

\includegraphics{analysis_files/figure-pdf/unnamed-chunk-82-1.pdf}

\begin{Shaded}
\begin{Highlighting}[]
\NormalTok{quant\_samples }\OtherTok{\textless{}{-}} \FunctionTok{list}\NormalTok{(}\StringTok{"obs"} \OtherTok{=}\NormalTok{ pop\_quant\_samples,}
                      \StringTok{"hyp"} \OtherTok{=}\NormalTok{ hyp\_pop\_quant\_samples)}
\NormalTok{p\_est }\OtherTok{\textless{}{-}}
\NormalTok{  util}\SpecialCharTok{$}\FunctionTok{implicit\_subset\_prob}\NormalTok{(quant\_samples,}
                            \ControlFlowTok{function}\NormalTok{(obs, hyp) hyp }\SpecialCharTok{\textgreater{}}\NormalTok{ obs)}

\NormalTok{format\_string }\OtherTok{\textless{}{-}} \FunctionTok{paste}\NormalTok{(}\StringTok{"Posterior probability that hypothetical"}\NormalTok{,}
                       \StringTok{"population lower quantile}\SpecialCharTok{\textbackslash{}n}\StringTok{is greater than"}\NormalTok{,}
                       \StringTok{"observed population lower quantile"}\NormalTok{,}
                       \StringTok{"= \%.3f +/{-} \%.3f."}\NormalTok{)}
\FunctionTok{cat}\NormalTok{(}\FunctionTok{sprintf}\NormalTok{(format\_string, p\_est[}\DecValTok{1}\NormalTok{], }\DecValTok{2} \SpecialCharTok{*}\NormalTok{ p\_est[}\DecValTok{2}\NormalTok{]))}
\end{Highlighting}
\end{Shaded}

\begin{verbatim}
Posterior probability that hypothetical population lower quantile
is greater than observed population lower quantile = 0.680 +/- 0.015.
\end{verbatim}

\section{Conclusion}\label{conclusion}

Acknowledging an underlying data generating process, and using any
available domain expertise to model it, is a powerful way to guide
statistical analyses. In this case study we were able to model not only
the variation in natural conception probability across patient
characteristics but also the patient characteristics themselves and the
potential contamination from alternative conception methods.

This analyses has only scratched the surface of what Bayesian modeling
techniques can accomplish. For example the patient characteristic model
we built can be used as a basis for inferring partially missing clinical
and demographic information, a process also known as \emph{imputation}.
This can be especially useful when dealing with complications like
patient dropout across the observation interval.

At the same time we can model not only the conception behavior within
the ART and non-ART groups but also the probability of any particular
patient using ART. In general this probability will be coupled with the
other patient characteristic behaviors but modeling that coupling would
allow us to construct even more nuanced hypothetical cohorts, and more
precise generalized predictions.

\section*{Acknowledgements}\label{acknowledgements}
\addcontentsline{toc}{section}{Acknowledgements}

I thank Simon Steiger , Peter Alping, Joshua Entrop, and Paulina Sell
for helpful comments.

\section*{License}\label{license}
\addcontentsline{toc}{section}{License}

The code in this case study is copyrighted by Michael Betancourt and
licensed under the new BSD (3-clause) license:

\url{https://opensource.org/licenses/BSD-3-Clause}

The text and figures in this chapter are copyrighted by Michael
Betancourt and licensed under the CC BY-NC 4.0 license:

\url{https://creativecommons.org/licenses/by-nc/4.0/}

\section*{Original Computing
Environment}\label{original-computing-environment}
\addcontentsline{toc}{section}{Original Computing Environment}

\begin{Shaded}
\begin{Highlighting}[]
\FunctionTok{writeLines}\NormalTok{(}\FunctionTok{readLines}\NormalTok{(}\FunctionTok{file.path}\NormalTok{(}\FunctionTok{Sys.getenv}\NormalTok{(}\StringTok{"HOME"}\NormalTok{), }\StringTok{".R/Makevars"}\NormalTok{)))}
\end{Highlighting}
\end{Shaded}

\begin{verbatim}
CC=clang

CXXFLAGS=-O3 -mtune=native -march=native -Wno-unused-variable -Wno-unused-function -Wno-macro-redefined -Wno-unneeded-internal-declaration
CXX=clang++ -arch x86_64 -ftemplate-depth-256

CXX14FLAGS=-O3 -mtune=native -march=native -Wno-unused-variable -Wno-unused-function -Wno-macro-redefined -Wno-unneeded-internal-declaration -Wno-unknown-pragmas
CXX14=clang++ -arch x86_64 -ftemplate-depth-256
\end{verbatim}

\begin{Shaded}
\begin{Highlighting}[]
\FunctionTok{sessionInfo}\NormalTok{()}
\end{Highlighting}
\end{Shaded}

\begin{verbatim}
R version 4.3.2 (2023-10-31)
Platform: x86_64-apple-darwin20 (64-bit)
Running under: macOS Sonoma 14.4.1

Matrix products: default
BLAS:   /Library/Frameworks/R.framework/Versions/4.3-x86_64/Resources/lib/libRblas.0.dylib 
LAPACK: /Library/Frameworks/R.framework/Versions/4.3-x86_64/Resources/lib/libRlapack.dylib;  LAPACK version 3.11.0

locale:
[1] en_US.UTF-8/en_US.UTF-8/en_US.UTF-8/C/en_US.UTF-8/en_US.UTF-8

time zone: America/New_York
tzcode source: internal

attached base packages:
[1] stats     graphics  grDevices utils     datasets  methods   base     

other attached packages:
[1] colormap_0.1.4     rstan_2.32.6       StanHeaders_2.32.7

loaded via a namespace (and not attached):
 [1] gtable_0.3.4       jsonlite_1.8.8     compiler_4.3.2     Rcpp_1.0.11       
 [5] stringr_1.5.1      parallel_4.3.2     gridExtra_2.3      scales_1.3.0      
 [9] yaml_2.3.8         fastmap_1.1.1      ggplot2_3.4.4      R6_2.5.1          
[13] curl_5.2.0         knitr_1.45         tibble_3.2.1       munsell_0.5.0     
[17] pillar_1.9.0       rlang_1.1.2        utf8_1.2.4         V8_4.4.1          
[21] stringi_1.8.3      inline_0.3.19      xfun_0.41          RcppParallel_5.1.7
[25] cli_3.6.2          magrittr_2.0.3     digest_0.6.33      grid_4.3.2        
[29] lifecycle_1.0.4    vctrs_0.6.5        evaluate_0.23      glue_1.6.2        
[33] QuickJSR_1.0.8     codetools_0.2-19   stats4_4.3.2       pkgbuild_1.4.3    
[37] fansi_1.0.6        colorspace_2.1-0   rmarkdown_2.25     matrixStats_1.2.0 
[41] tools_4.3.2        loo_2.6.0          pkgconfig_2.0.3    htmltools_0.5.7   
\end{verbatim}



\end{document}
